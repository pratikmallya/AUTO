\chapter{ Running \AUTO using Command Mode.} \label{sec:command_mode}
\AUTO can be run with the interface described in Chapter~\ref{ch:CLUI} 
or with the commands described below.
The \AUTO aliases must have been activated; see Section~\ref{sec:Installation}; 
and an equations-file \filef{ xxx.c} 
and a corresponding constants-file \filef{ c.xxx} 
(see Section~\ref{sec: User_supplied_files})
must be in the current user directory.
\\
\emp{ Do not run \AUTO in the directory \filef{ auto/2000} 
or in any of its subdirectories.}

\subsection{ Basic commands.} 

\begin{itemize}
\item[\commandf{@r}]:
  Type \commandf{ @r xxx} to run \AUTO.
  Restart data, if needed, are expected in \filef{ s.xxx},
  and \AUTO-constants in \filef{ c.xxx}.
  This is the simplest way to run \AUTO.
\item[-]
  Type \commandf{ @r xxx yyy} to run \AUTO
  with equations-file \filef{ xxx.c} and restart data-file \filef{ s.yyy}.
  \AUTO-constants must be in \filef{ c.xxx}.
\item[-]
  Type \commandf{ @r xxx yyy zzz} to run \AUTO
  with equations-file \filef{ xxx.c}, restart data-file \filef{ s.yyy}
  and constants-file \filef{ c.zzz}.

\item[\commandf{@R}]~:
  The command \commandf{ @R xxx} is equivalent to the command \commandf{ @r xxx} above.
\item[-]
  Type \commandf{ @R xxx i}  to run \AUTO with equations-file \filef{ xxx.c},
  constants-file \filef{ c.xxx.i}
  and, if needed, restart data-file \filef{ s.xxx}. 
\item[-]
  Type \commandf{ @R xxx i yyy} to run \AUTO
  with equations-file \filef{ xxx.c}, 
  constants-file \filef{ c.xxx.i}
  and restart data-file \filef{ s.yyy}.

\item[\commandf{@sv}]~:
  Type \commandf{ @sv xxx} to save the output-files 
  \filef{ fort.7}, \filef{ fort.8}, \filef{ fort.9},
  as \filef{ b.xxx}, \filef{ s.xxx}, \filef{ d.xxx}, respectively.
  Existing files by these names will be deleted.

\item[\commandf{@ap}]~:
  Type \commandf{ @ap xxx} to append the output-files 
  \filef{ fort.7}, \filef{ fort.8}, \filef{ fort.9}, 
  to existing data-files 
  \filef{ b.xxx}, \filef{ s.xxx}, \filef{ d.xxx}, resp.
\item[-]
  Type \commandf{ @ap xxx yyy} 
  to append 
  \filef{ b.xxx}, \filef{ s.xxx}, \filef{ d.xxx}, to
  \filef{ b.yyy}, \filef{ s.yyy}, \filef{ d.yyy}, resp.
\end{itemize}

\subsection{ Plotting commands.} 

\begin{itemize}

\item[\commandf{@p}]~:
  Type \commandf{ @p xxx} to run the graphics program {\cal PLAUT}
  (See Chapter~\ref{ch:PLAUT})
  for the graphical inspection of the data-files 
  \filef{ b.xxx} and \filef{ s.xxx}. 
\item[-]
  Type \commandf{ @p} to run the graphics program {\cal PLAUT}
  for the graphical inspection of the output-files 
  \filef{ fort.7} and \filef{ fort.8}.

\item[\commandf{@ps}]~:
  Type \commandf{ @ps fig.x} to convert a saved {\cal PLAUT} figure \filef{ fig.x}
  from compact {\cal PLOT10} format to {\cal PostScript} format.
  The converted file is called \filef{ fig.x.ps}. 
  The original file is left unchanged.

\item[\commandf{@pr}]~:
  Type \commandf{ @pr fig.x} to convert a saved {\cal PLAUT} figure \filef{ fig.x}
  from compact {\cal PLOT10} format to {\cal PostScript} format and send it to the
  printer.
  The converted file is called \filef{ fig.x.ps}. 
  The original file is left unchanged.
\end{itemize}

\subsection{ File-manipulation.} 

\begin{itemize}

\item[\commandf{@cp}]~:
  Type \commandf{ @cp xxx yyy} 
  to copy the data-files 
  \filef{ b.xxx}, \filef{ s.xxx}, \filef{ d.xxx}, \filef{ c.xxx} to
  \filef{ b.yyy}, \filef{ s.yyy}, \filef{ d.yyy}, \filef{ c.yyy}, respectively.

\item[\commandf{@mv}]~:
  Type \commandf{ @mv xxx yyy} 
  to move the data-files 
  \filef{ b.xxx}, \filef{ s.xxx}, \filef{ d.xxx}, \filef{ c.xxx}, to
  \filef{ b.yyy}, \filef{ s.yyy}, \filef{ d.yyy}, \filef{ c.yyy}, respectively.

\item[\commandf{@df}]~:
  Type \commandf{ @df} 
  to delete the output-files 
  \filef{ fort.7}, \filef{ fort.8}, \filef{ fort.9}.

\item[\commandf{@cl}]~:
  Type \commandf{ @cl} 
  to clean the current directory.
  This command will delete  all files of the form
  \filef{ fort.*}, \filef{ *.o}, and \filef{ *.exe}.

\item[\commandf{@dl}]~:
  Type \commandf{ @dl xxx} 
 to delete the data-files 
  \filef{ b.xxx}, \filef{ s.xxx}, \filef{ d.xxx}.
\end{itemize}

\subsection{ Diagnostics.} 

\begin{itemize}
\item[\commandf{@lp}]~:
  Type \commandf{ @lp} to list the value of the ``limit point function'' 
  in the output-file \filef{ fort.9}. This function
  vanishes at a limit point (fold).
  \item[-]
  Type \commandf{ @lp xxx} to list the value of the ``limit point function'' 
  in the data-file \filef{ d.xxx}. This function
  vanishes at a limit point (fold).
\item[\commandf{@bp}]~:
  Type \commandf{ @bp} to list the value of the ``branch-point function'' 
  in the output-file \filef{ fort.9}. This function
  vanishes at a branch point.
  \item[-]
  Type \commandf{ @bp xxx} to list the value of the ``branch-point function''
  in the data-file \filef{ d.xxx}. This function
  vanishes at a branch point.
\item[\commandf{@hb}]~:
  Type \commandf{ @hb} to list the value of the ``Hopf function'' 
  in the output-file \filef{ fort.9}. This function
  vanishes at a Hopf bifurcation point.
  \item[-]
  Type \commandf{ @hb xxx} to list the value of the ``Hopf function''
  in the data-file \filef{ d.xxx}. This function
  vanishes at a  Hopf bifurcation point.
\item[\commandf{@sp}]~:
  Type \commandf{ @sp} to list the value of the 
  ``secondary-periodic bifurcation function'' 
  in the output-file \filef{ fort.9}. This function
  vanishes at period-doubling and torus bifurcations.
  \item[-]
  Type \commandf{ @sp xxx} to list the value of the
   ``secondary-periodic bifurcation function''
  in the data-file \filef{ d.xxx}. This function
  vanishes at period-doubling and torus bifurcations.
\item[\commandf{@it}]~:
  Type \commandf{ @it} to list the number of Newton iterations per
  continuation step in \filef{ fort.9}. 
  \item[-]
   Type \commandf{ @it xxx} to list the number of Newton iterations per
  continuation step in \filef{ d.xxx}. 
\item[\commandf{@st}]~:
  Type \commandf{ @st} to list the continuation step size for each
  continuation step in  \filef{ fort.9}. 
  \item[-]
   Type \commandf{ @st xxx} to list the continuation step size for each
  continuation step in \filef{ d.xxx}. 
\item[\commandf{@ev}]~:
  Type \commandf{ @ev} to list the eigenvalues of the Jacobian 
  in \filef{ fort.9}. 
  (Algebraic problems.)
  \item[-]
   Type \commandf{ @ev xxx} to list the eigenvalues of the Jacobian 
  in \filef{ d.xxx}. 
  (Algebraic problems.)
\item[\commandf{@fl}]~:
  Type \commandf{ @fl} to list the Floquet multipliers
  in the output-file \filef{ fort.9}. 
  (Differential equations.)
  \item[-]
   Type \commandf{ @fl xxx} to list the Floquet multipliers 
  in the data-file \filef{ d.xxx}. 
  (Differential equations.)
\end{itemize}

\subsection{ File-editing.} 

\begin{itemize}

\item[\commandf{@e7}]~:
  To use the vi editor to edit the output-file \filef{ fort.7}.
\item[\commandf{@e8}]~:
  To use the vi editor to edit the output-file \filef{ fort.8}.
\item[\commandf{@e9}]~:
  To use the vi editor to edit the output-file \filef{ fort.9}.
\item[\commandf{@j7}]~:
  To use the SGI jot editor to edit the output-file \filef{ fort.7}.
\item[\commandf{@j8}]~:
  To use the SGI jot editor to edit the output-file \filef{ fort.8}.
\item[\commandf{@j9}]~:
  To use the SGI jot editor to edit the output-file \filef{ fort.9}.
  
\end{itemize}

\subsection{ File-maintenance.} 

\begin{itemize}
\item[\commandf{@lb}]~:
  Type \commandf{ @lb} to run an interactive utility program
  for listing, deleting and relabeling solutions 
  in the output-files \filef{ fort.7} and \filef{ fort.8}.
  The original files are backed up as
\filef{ $\sim$fort.7} and \filef{ $\sim$fort.8}. 
  \item[-]
  Type \commandf{ @lb xxx} to list, delete and relabel solutions
  in the data-files \filef{ b.xxx} and \filef{ s.xxx}.
  The original files are backed up as \filef{ $\sim$b.xxx} and \filef{ $\sim$s.xxx}. 
\item[-]
  Type \commandf{ @lb xxx yyy} to list, delete and relabel solutions
  in the data-files \filef{ b.xxx} and \filef{ s.xxx}.
  The modified files are written as \filef{ b.yyy} and \filef{ s.yyy}. 

\item[\commandf{@fc}]~:
  Type \commandf{ @fc xxx} to convert a user-supplied data file \filef{ xxx.dat}
  to \AUTO format. The converted file is called \filef{ s.dat}.
  The original file is left unchanged.
  \AUTO automatically sets the period in \filef{ PAR(11)}.
  Other parameter values must be set in \filef{ stpnt}. (When necessary,
  PAR(11) may also be redefined there.) 
  The constants-file file \filef{ c.xxx} must be present, as the 
  \AUTO-constants \filef{ NTST} and \filef{ NCOL} 
  (Sections~\ref{sec:NTST} and \ref{sec:NCOL}) are used to define the new mesh.
  For examples of using the \commandf{ @fc} command see demos \filef{ lor} and \filef{ pen}.

\item[\commandf{@94to97}]~:
  Type \commandf{ @94to97 xxx} to convert an old \AUTOolder data-file \filef{ s.xxx}
  to new \AUTOold format. The original file is backed up as \filef{ $\sim$s.xxx}.
  This conversion is only necessary for files from early versions 
  of \AUTOolder.
\end{itemize}

\subsection{ HomCont commands.} 

\begin{itemize}
\item[\commandf{@h}]~:
  Use \commandf{ @h} instead of \commandf{ @r} when using {\cal HomCont}, i.e., when \parf{ IPS}=9
  (see Chapter~\ref{ch:HomCont}).
  Type \commandf{ @h xxx} to run \AUTO/{\cal HomCont}.
  Restart data, if needed, are expected in \filef{ s.xxx},
  \AUTO-constants in \filef{ c.xxx} and {\cal HomCont}-constants in \filef{ h.xxx}.
\item[-]
  Type \commandf{ @h xxx yyy} to run \AUTO/{\cal HomCont}
  with equations-file \filef{ xxx.c} and restart data-file \filef{ s.yyy}.
  \AUTO-constants must be in \filef{ c.xxx} and {\cal HomCont}-constants in \filef{ h.xxx}.
\item[-]
  Type \commandf{ @h xxx yyy zzz} to run \AUTO/{\cal HomCont}
  with equations-file \filef{ xxx.c}, restart data-file \filef{ s.yyy}
  and constants-files \filef{ c.zzz} and \filef{ h.zzz}.

\item[\commandf{@H}]~:
  The command \commandf{ @H xxx} is equivalent to the command \commandf{ @h xxx} above.
\item[-]
  Type \commandf{ @H xxx i} in order to run \AUTO/{\cal HomCont} with equations-file \filef{ xxx.c}
  and constants-files \filef{ c.xxx.i} and \filef{ h.xxx.i}
  and, if needed, restart data-file \filef{ s.xxx}. 
\item[-]
  Type \commandf{ @H xxx i yyy} to run \AUTO/{\cal HomCont}
  with equations-file \filef{ xxx.c}, 
  constants-files \filef{ c.xxx.i} and \filef{ h.xxx.i},
  and restart data-file \filef{ s.yyy}.
\end{itemize}

\subsection{ Copying a demo.} 

\begin{itemize}

\item[\commandf{@dm}]~:
  Type \commandf{ @dm xxx} 
  to copy all files 
  from \filef{ auto/2000/demos/xxx}
  to the current user directory.
  Here \filef{ xxx} denotes a demo name; e.g., \filef{ abc}.
  Note that the \commandf{@dm} command also copies a  \filef{Makefile}
  to the current user directory. To avoid the overwriting of
  existing files, always run demos in a clean work directory.
\end{itemize}

\subsection{ Pendula animation.} 

\begin{itemize}
\item[\commandf{@pn}]~:
  Type \commandf{ @pn xxx} to run the pendula animation program
  with data-file \filef{ s.xxx}. (On SGI machine only; see demo \filef{ pen}
  in Section~\ref{sec:Demos_pen} and the file \filef{ auto/2000/pendula/README}.)
\end{itemize}

\subsection{ Viewing the manual.} 

\begin{itemize}

\item[\commandf{@mn}]~: Use {\cal Ghostview} to view the PostScript version of this manual.
\end{itemize}

\newpage
