%==============================================================================
%==============================================================================
\chapter{ {\cal HomCont} Demo : kpr.} \label{ch:HomCont_kpr}
%==============================================================================
%==============================================================================

%==============================================================================
%DEMO=kpr======================================================================
%==============================================================================
\section{ Koper's Extended Van der Pol Model.}
%
The equation-file \filef{ kpr.c} contains the equations
\begin{equation} \label{ko} \begin{array}{rcl}
\dot{x} & = & \eps_1^{-1}\:(k\: y - x^3 +3\:x - \lambda) \\
\dot{y} & = & x - 2\: y + z \\
\dot{z} & = & \eps_2(y-z), 
\end{array} \end{equation}
with $\eps_1 =0.1$ and $\eps_2=1$ \cite{Ko:95}.

To copy across the demo \filef{ kpr} and compile we type
\begin{center}
\commandf{ @dm kpr} \\
\end{center}

\section{The Primary Branch of Homoclinics.}
First, we locate a homoclinic orbit using 
the homotopy method. The file \filef{ kpr.c} 
already contains 
approximate parameter values for a homoclinic orbit, 
namely $\lambda=$\parf{ PAR(1)=-1.851185}, $k=$\parf{ PAR(2)=-0.15}. 
The files \filef{ c.kpr.1} and \filef{ h.kpr.1} specify the appropriate
constants for continuation in $2T$\parf{ =PAR(11)} (also referred
to as \parf{ PERIOD}) and the dummy parameter $\omega_1$=\parf{ PAR(17)}
starting
from a small solution in the local unstable manifold; 
\begin{center}
\commandf{ make first}
\end{center}
Among the output there is the line
\begin{verbatim}
     BR    PT  TY LAB    PERIOD        L2-NORM     ...    PAR(17)    ...
      1    29  UZ   2  1.900184E+01  1.693817E+00  ...  4.433433E-09 ... 
\end{verbatim}
which indicates that a zero of the artificial parameter $\omega_1$
has been located. This means that the right-hand end point of the solution
belongs to the plane that is tangent to the stable manifold at the saddle. 
The output is stored in files \filef{ b.1, s.1, d.1}. 
Upon plotting the data at label \parf{ 2} (see Figure \ref{kf.1a})
it can be noted that although the right-hand projection boundary
condition is satisfied, the solution is still quite away from the
equilibrium. 

The right-hand endpoint can be made to approach the
equilibrium by performing a further continuation in $T$ with the
right-hand projection condition satisfied (\parf{ PAR(17)} fixed) but
with $\lambda$ allowed to vary. 
%
\begin{figure}[p]
\epsfysize 9.0cm
\centerline{\epsffile{include/kpr1.ps}}
\caption{Projection on the $(x,y)$-plane of solutions 
of the boundary value 
problem with $2T=19.08778$.}
\label{kf.1a}
\end{figure}
\begin{figure}[p]
\epsfysize 9.0cm
\centerline{\epsffile{include/kpr2.ps}}
\caption{Projection on the $(x,y)$-plane of solutions of the 
boundary value problem with $2T = 60.0$.}
\label{kf.1b}
\end{figure}
%
\begin{center}
\commandf{ make second}
\end{center}
the output at label \parf{ 4}, stored in \filef{ kpr.2},
\begin{verbatim}
   BR   PT TY   LAB    PERIOD       L2-NORM     ...    PAR(1)     ...
   1    35  UZ   4  6.000000E+01  1.672806E+00  ... -1.851185E+00 ...
\end{verbatim}
provides a good approximation to a homoclinic solution (see Figure
\ref{kf.1b}). 

The second stage to obtain a starting solution 
is to add a solution to the modified adjoint
variational equation. This is achieved by setting both 
\parf{ ITWIST} and \parf{ ISTART} to 1 (in \filef{ h.kpr.3}), which generates
a trivial guess for the adjoint equations. Because the adjoint
equations are linear, only a single
Newton step (by continuation in a trivial parameter) 
is required to provide a solution.
Rather than choose a parameter that might be used internally
by \AUTO, in \filef{ c.kpr.3} we take the continuation parameter
to be \parf{ PAR(11)}, which is not quite a trivial parameter
but whose affect upon the solution is mild.
\begin{center}
\commandf{ make third}
\end{center}
The output at the second point (label \parf{ 6}) 
contains the converged homoclinic
solution (variables (\parf{ U(1), U(2), U(3)}) and the adjoint (\parf{
U(4), U(5), U(6)})). We now have a starting solution 
and are ready to perform two-parameter continuation.

The fourth run
\begin{center}
\commandf{ make fourth}
\end{center}
continues the homoclinic orbit in \parf{ PAR(1)} and \parf{ PAR(2)}. 
%
\begin{figure}[p]
\epsfysize 9.0cm
\centerline{\epsffile{include/kpr4.ps}}
\caption{Projection on the $(x,y)$-plane of solutions $\phi(t)$
at \parf{ 1} ($\lambda=-1.825470, k=-0.1760749$) and
\parf{ 2} ($\lambda=-1.686154, k=-0.3183548$).}
\label{kf.2a}
\end{figure}
\begin{figure}[p]
\epsfysize 8.0cm
\centerline{\epsffile{include/kpr5.ps}}
\caption{Three-dimensional blow-up of the solution curves
 $\phi(t)$ 
at labels \parf{ 1} (dotted) and \parf{ 2} (solid line) from Figure 3.8.}
\label{kf.2b}
\end{figure}
%
Note that several other parameters appear in
the output. \parf{ PAR(10)} is a dummy parameter
that should be zero when the adjoint is being computed correctly;
\parf{ PAR(29)}, \parf{ PAR(30)}, \parf{ PAR(33)} correspond to the
test functions $\psi_9$,$\psi_{10}$ and $\psi_{13}$. 
That these test functions were activated is specified
in three places in \filef{ c.kpr.4} and \filef{ h.kpr.4} 
as described in Section~\ref{sec:HomCont_Test_functions}.  

Note that at the end-point of
the branch (reached when after \parf{ NMX=50} steps) \parf{ PAR(29)} is
approximately zero which corresponds to a zero of $\psi_9$, a 
non-central saddle-node homoclinic orbit. We shall return to the computation of
this codimension-two point later. Before reaching this point,
among the output we find two zeroes of \parf{ PAR(33)}
(test function $\psi_{13}$) which gives the accurate
location of two inclination-flip bifurcations,
\begin{verbatim}
 BR  PT  TY LAB    PAR(1)     ...     PAR(2)        PAR(10)   ...    PAR(33)  
  1   6  UZ  10 -1.801662E+00 ... -2.002660E-01 -7.255434E-07 ... -1.425714E-04
  1  12  UZ  11 -1.568756E+00 ... -4.395468E-01 -2.156353E-07 ...  4.514073E-07
\end{verbatim}
That the test function really does have a regular zero at this point can
be checked from the data saved in \filef{ b.3}, plotting \parf{ PAR(33)} as
a function of \parf{ PAR(1)} or \parf{ PAR(2)}. 
Figure \ref{kf.2a} presents solutions $\phi(t)$ of the modified adjoint 
variational equation (for details see \citeasnoun{ChKuSa:95})
at parameter values on the homoclinic 
branch before and after the first detected inclination flip. 
Note that these solutions were obtained by choosing a smaller
step \parf{ DS} and more output (smaller \parf{ NPR}) in
\filef{ c.kpr.4}.
A blow-up of the region close to the origin of this 
figure is shown in Figure \ref{kf.2b}.
It illustrates the flip of the solutions of the adjoint equation while
moving through the bifurcation point. Note that the data in this
figure were plotted after first performing an additional
continuation of the solutions with respect to \parf{ PAR(11)}. 

Continuing in the other direction 
\begin{center}
\commandf{ make fifth}
\end{center}
we approach a Bogdanov-Takens point
\begin{verbatim}  
 BR    PT  TY LAB    PAR(1)     ...    PAR(10)    ...    PAR(33)    
  1    50  EP  13 -1.938276E+00 ... -7.523344E+00 ...  6.310810E+01
\end{verbatim}
\begin{figure}[t]
\epsfysize 9.0cm
\centerline{\epsffile{include/kpr6.ps}}
\caption{Computed homoclinic orbits approaching the BT point}
\label{kp.6}
\end{figure}
Note that the numerical approximation has ceased to become reliable, since 
\parf{ PAR(10)} has now become large. 
Phase portraits of homoclinic orbits between the BT point and the first
inclination flip 
are depicted in Figure \ref{kp.6}. Note how the computed homoclinic orbits
approaching the BT point have their endpoints well away from the equilibrium.
To follow the homoclinic orbit to 
the BT point with more precision, we would need to first perform continuation 
in $T$ (\parf{ PAR(11)}) to obtain a more accurate homoclinic solution.


\section{More Accuracy and Saddle-Node Homoclinic Orbits.}
Continuation in $T$ 
in order to obtain an approximation of the homoclinic orbit over a
longer interval is necessary for parameter values near a non-hyperbolic
equilibrium (either a saddle-node or BT) where the convergence
to the equilibrium is slower. 
First, we start from the original homoclinic orbit computed
via the homotopy method, label \parf{ 4}, which is well away from
the non-hyperbolic equilibrium.
Also, we shall no longer be interested in
in inclination flips so we set \parf{ ITWIST=0} in \filef{ c.kpr.6},
and in order to compute up to \parf{ PAR(11)=1000}, we set up a
user-defined function for this. Running \AUTO with \parf{ PAR(11)} and 
\parf{ PAR(2)} as free parameters
\begin{center}
\commandf{ make sixth}
\end{center}
we obtain among the output
\begin{verbatim}
  BR    PT  TY LAB     PERIOD       L2-NORM    ...    PAR(2)     
   1    35  UZ   6  1.000000E+03  1.661910E+00 ... -1.500000E-01
\end{verbatim}

We can now repeat the computation of the branch of saddle homoclinic
orbits in \parf{ PAR(1)} and \parf{ PAR(2)} from this point with
the test functions $\psi_9$ and $\psi_{10}$ for non-central
saddle-node homoclinic orbits activated 
\begin{center}
\commandf{ make seventh}
\end{center}
The saddle-node point is now detected at 
\begin{verbatim} 
  BR    PT  TY LAB    PAR(1)     ...    PAR(2)        PAR(29)       PAR(30)
   1    30  UZ   8  1.765003E-01 ... -2.405345E+00  2.743361E-06  2.309317E+01
\end{verbatim}
which is stored in \filef{ s.7}.
That \parf{ PAR(29)} ($\psi_9$) is zeroed shows that this
is a non-central saddle-node connecting the centre manifold to the strong stable
manifold. Note that all output beyond this point, although a well-posed
solution to the boundary-value problem, is spurious in that it no longer
represents a homoclinic orbit to a saddle equilibrium (see
\citeasnoun{ChKuSa:95}). If we had chosen
to, we could continue in the other direction in order to
approach the BT point more accurately by reversing the sign of
\parf{ DS} in \filef{ c.kpr.7}.
 
The files \filef{ c.kpr.9} and \filef{ h.kpr.9} contain the constants necessary 
for switching to continuation of the central saddle-node homoclinic curve 
in two parameters starting from the non-central saddle-node homoclinic orbit
stored as label \parf{ 8} in \filef{ s.7}.
\begin{center}
\commandf{ make eighth}
\end{center}
In this run we have activated the test functions for saddle to saddle-node
transition points along curves of saddle homoclinic orbits ($\psi_{15}$ and 
$\psi_{16}$). Among the output we find
\begin{verbatim}
  BR    PT  TY LAB    PAR(1)     ...    PAR(2)        PAR(35)       PAR(36)    
   1    38  UZ  13  1.765274E-01 ... -2.405284E+00  9.705426E-03 -5.464784E-07
\end{verbatim}
\begin{figure}[p]
\epsfysize 9.0cm
\centerline{\epsffile{include/kpr7.ps}}
\caption{Two non-central saddle-node homoclinic orbits, \parf{ 1} and \parf{ 3};
and, \parf{ 2}, a central saddle-node homoclinic orbit between
these two points \label{kf.7}}
\end{figure}
\begin{figure}[p]
\epsfysize 9.0cm
\centerline{\epsffile{include/kpr8.ps}}
\caption{The big homoclinic orbit approaching a figure-of-eight}
\label{kp.8}
\end{figure}
%
which corresponds to the branch of homoclinic orbits leaving
the locus of saddle-nodes in a second non-central saddle-node
homoclinic bifurcation (a zero of $\psi_{16}$). 

Note that the parameter values do not vary much between the
two codimension-two non-central saddle-node points (labels \parf{ 8} and \parf{ 13}).
However, Figure \ref{kf.7} shows clearly that between the two
codimension-two points 
the homoclinic orbit
rotates between the two components of the 1D stable manifold, i.e.\
between the two boundaries of the center-stable manifold of the saddle
node. The overall effect of this process is the transformation of a
nearby ``small'' saddle homoclinic orbit to a ``big'' saddle
homoclinic orbit (i.e.\ with two extra turning points in phase space).  

Finally, we can switch to continuation of the big saddle homoclinic orbit 
from the new codim 2 point at label \parf{ 13}. 
\begin{center}
\commandf{ make ninth}
\end{center}
Note that \AUTO takes a large number of steps near the line  
\parf{ PAR(1)=0}, while \parf{ PAR(2)} approaches $-2.189\ldots$  
(which is why we chose such a large value \parf{ NMX=500} in \filef{ c.kpr.9}). This
particular computation ends at 
\begin{verbatim}  
  BR    PT  TY LAB    PAR(1)        L2-NORM    ...    PAR(2)   
   1   500  EP  24 -1.218988E-05  2.181205E-01 ... -2.189666E+00
\end{verbatim}
By plotting phase portraits of orbits approaching this end point (see Figure
\ref{kp.8}) we see a ``canard-like'' like transformation of the big homoclinic
orbit to a pair of homoclinic orbits in a figure-of-eight configuration.
That we get a figure-of-eight is not a surprise because \parf{ PAR(1)=0}
corresponds to a symmetry in the differential equations \cite{Ko:94};
note also that the equilibrium, stored as (\parf{ PAR(12), PAR(13), PAR(14)}) in
\filef{ d.9}, approaches the origin as we approach the figure-of-eight homoclinic.

\section{Three-Parameter Continuation.}
We now consider curves in three parameters of each of
the codimension-two points encountered in this model, by
freeing the parameter $\eps=$ \parf{ PAR(3)}.
First we continue the first inclination flip stored at label
\parf{ 7} in \filef{ s.3}
\begin{center}
\it make tenth
\end{center}
Note that \parf{ ITWIST=1} in \filef{ h.kpr.10}, so that the adjoint is also
continued, and there is one fixed condition \parf{ IFIXED(1)=13} so that
test function $\psi_{13}$ has been frozen.
Among the output there is a codimension-three point (zero of $\psi_9$)
where the neutrally twisted homoclinic orbit collides with the saddle-node
curve
\begin{verbatim}
 BR  PT  TY LAB    PAR(1)     ...   PAR(2)        PAR(3)        PAR(29)    ...     
  1  28  UZ  14  1.282702E-01 ... -2.519325E+00 5.744770E-01 -4.347113E-09 ...
\end{verbatim}
The other detected inclination flip (at label \parf{ 8} in \filef{ s.3}) is continued
similarly
\begin{center}
\it make eleventh
\end{center}
giving among its output another codim 3 saddle-node inclination-flip point
\begin{verbatim} 
 BR  PT  TY LAB    PAR(1)     ...   PAR(2)        PAR(3)        PAR(29)    ...  
  1  27  UZ  14  1.535420E-01 ... -2.458100E+00 1.171705E+00 -1.933188E-07 ... 
\end{verbatim}
Output beyond both of these codim 3 points is spurious and both computations end in
an \parf{ MX} point (no convergence).

To continue the non-central saddle-node homoclinic orbits it is
necessary to work on the data without the solution $\phi(t)$. We
restart from the data saved at \parf{ LAB=8} and \parf{ LAB=13} in
\filef{ s.7} and \filef{ s.8} respectively. We could continue these codim 2 points in two
ways, either by appending the defining condition $\psi_{16} =0$ to
the continuation of saddle-node homoclinic orbits (with \parf{ IEQUIB=2},
etc.), or by appending $\psi_{9} =0$ to the continuation 
of a saddle homoclinic orbit (with \parf{ IEQUIB=1}. 
The first approach is used in the example \filef{ mtn},  
for contrast we shall adopt the second approach here.
\begin{figure}[p]
\epsfysize 9.0cm
\centerline{\epsffile{include/kpr10.ps}}
\caption{Projection onto the \parf{ (PAR(3),PAR(2))}-plane of the non-central
saddle-node homoclinic orbit curves (labeled \parf{ 1} and \parf{ 2}) and the 
inclination-flip curves (labeled \parf{ 3} and \parf{ 4})}.
\label{kp.10}
\end{figure}
%
\begin{center}
\commandf{ make twelfth}\\
\commandf{ make thirteenth}
\end{center}
The projection onto the $(\eps,k)$-plane of all four of these
codimension-two curves is given in Figure \ref{kp.10}. 
The intersection of the inclination-flip lines with one of the
non-central saddle-node homoclinic lines is apparent. Note that the two
non-central saddle-node homoclinic orbit curves are almost overlaid, but
that as in Figure \ref{kf.7} the orbits look quite distinct in phase space.

\section{ Detailed \AUTO-Commands.}
\begin{table}[htbp]
\begin{center}
\begin{tabular}{| l | l |}
\hline
  \AUTO-COMMAND  & ACTION \\
\hline
%==============================================================================
  \commandf{ ! mkdir kpr} & create an empty work directory \\ 
  \commandf{ cd kpr} & change directory \\
  \commandf{ demo('kpr')} & copy the demo files to the work directory \\
\hline
%==============================================================================
  \commandf{ run(c='kpr.1',h='kpr.1')} &  continuation in the time-length parameter \parf{ PAR(11)} \\ 
  \commandf{ sv('1')} & save output-files as \filef{ b.1, s.1, d.1} \\ 
\hline
%==============================================================================
  \commandf{ run(c='kpr.2',h='kpr.2',s='1')} & locate the homoclinic orbit; restart from \filef{ s.1} \\ 
  \commandf{ sv('2')} & save output-files as \filef{ b.2, s.2, d.2} \\ 
\hline
%=============================================================================
  \commandf{ run(c='kpr.3',h='kpr.3',s='2')} & generate adjoint variables  ; restart from \filef{ s.2} \\ 
  \commandf{ sv('3')} & save output-files as \filef{ b.3, s.3, d.3} \\ 
\hline
%==============================================================================
  \commandf{ run(c='kpr.4',h='kpr.4',s='3')} & continue the homoclinic orbit; restart from \filef{ s.3} \\ 
  \commandf{ ap('3')} & append output-files to \filef{ b.3, s.3, d.3} \\ 
\hline
%=============================================================================
  \commandf{ run(c='kpr.5',h='kpr.5',s='3')} & continue in reverse direction; restart from \filef{ s.3} \\
  \commandf{ ap('3')} & append output-files to \filef{ b.3, s.3, d.3} \\ 
\hline
%==============================================================================
  \commandf{ run(c='kpr.6',h='kpr.6',s='2')} & increase the period; restart from \filef{ s.2} \\ 
  \commandf{ sv('6')} & save output-files as \filef{ b.6, s.6, d.6} \\ 
\hline
%==============================================================================
\end{tabular}
\caption{Detailed \AUTO-Commands for running demo \filef{ kpr}.}
\label{tbl:demo_kpr_1}
\end{center}
\end{table}


\begin{table}[htbp]
\begin{center}
\begin{tabular}{| l | l |}
\hline
  \AUTO-COMMAND  & ACTION \\
\hline
%==============================================================================
  \commandf{ run(c='kpr.7',h='kpr.7',s='6')} & recompute the branch of homoclinic orbits; restart from \filef{ s.6} \\ 
  \commandf{ sv('7')} & save output-files as \filef{ b.7, s.7, d.7} \\ 
\hline
%==============================================================================
  \commandf{ run(c='kpr.8',h='kpr.8',s='7')} & continue central saddle-node homoclinics; restart from \filef{ s.7} \\ 
  \commandf{ sv('8')} & save output-files as \filef{ b.8, s.8, d.8} \\ 
\hline
%==============================================================================
  \commandf{ run(c='kpr.9',h='kpr.9',s='8')} & continue homoclinics from codim-2 point; restart from \filef{ s.8} \\ 
  \commandf{ sv('9')} & save output-files as \filef{ b.9, s.9, d.9} \\ 
\hline
%==============================================================================
  \commandf{ run(c='kpr.10',h='kpr.10',s='3')} & 3-parameter curve of inclination-flips; restart from \filef{ s.3} \\ 
  \commandf{ sv('10')} & save output-files as \filef{ b.10, s.10, d.10} \\ 
\hline
%==============================================================================
  \commandf{ run(c='kpr.11',h='kpr.11',s='3')} & another curve of inclination-flips; restart from \filef{ s.3} \\ 
  \commandf{ sv('11')} & save output-files as \filef{ b.11, s.11, d.11} \\ 
\hline
%==============================================================================
  \commandf{ run(c='kpr.12',h='kpr.12',s='7')} & continue non-central saddle-node homoclinics; restart from \filef{ s.7} \\ 
  \commandf{ sv('12')} & save output-files as \filef{ b.12, s.12, d.12} \\ 
\hline
%==============================================================================
  \commandf{ run(c='kpr.13',h='kpr.13',s='8')} & continue non-central saddle-node homoclinics; restart from \filef{ s.8} \\ 
  \commandf{ ap('12')} & append output-files to \filef{ b.12, s.12, d.12} \\ 
\hline
%==============================================================================
\end{tabular}
\caption{Detailed \AUTO-Commands for running demo \filef{ kpr}.}
\label{tbl:demo_kpr_2}
\end{center}
\end{table}

