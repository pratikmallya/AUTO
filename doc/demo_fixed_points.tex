%==============================================================================
%==============================================================================
\chapter{ \AUTO Demos : Fixed points.} \label{ch:Demos_Fixed_points}
%==============================================================================
%==============================================================================

%==============================================================================
%DEMO=enz======================================================================
%==============================================================================
\section{ enz : Stationary Solutions of an Enzyme Model.} \label{sec:Demos_enz}
The equations, that model a two-compartment enzyme system 
(\citename{JPK:80} \citeyear{JPK:80}),
are given by
\begin{equation} \label{2'} \begin{array}{cl}
 s_1 '&=
 (s_0 - s_1) + (s_2 - s_1) - \rho R (s_1), \\
 s_2 '&=
 (s_0 +\mu - s_2) + (s_1 - s_2) - \rho R (s_2), \\\end{array} \end{equation}
where
$$ R (s)={s \over 1+s+ \kappa s^{2} }.$$
The free parameter is $s_0$. Other parameters are fixed.
This equation is also considered in 
\citename{DoKeKe:91a} \citeyear{DoKeKe:91a}.

\begin{table}[htbp]
\begin{center}
\begin{tabular}{| l | l |}
\hline
  \AUTO-COMMAND  & ACTION \\
\hline
%==============================================================================
  \commandf{ ! mkdir enz} & create an empty work directory \\ 
  \commandf{ cd enz} & change directory \\
  \commandf{ demo('enz')} & copy the demo files to the work directory \\
\hline
%==============================================================================
  \commandf{ ld('enz')} & load the problem definition \\
  \commandf{ run(c='enz.1')} & compute stationary solution branches \\ 
  \commandf{ sv('enz')} & save output-files as \filef{ b.enz, s.enz, d.enz} \\ 
\hline
\end{tabular}
\caption{Commands for running demo \filef{ enz}.}
\label{tbl:demo_enz}
\end{center}
\end{table}

\newpage
%==============================================================================
%DEMO=dd2======================================================================
%==============================================================================
\section{ dd2 : Fixed Points of a Discrete Dynamical System.} \label{sec:Demos_dd2}
This demo illustrates the computation of a solution branch and
its bifurcating branches for a discrete dynamical system.
Also illustrated is the continuation of 
Naimark-Sacker (or Hopf) bifurcations
The equations, a discrete predator-prey system, are
\begin{equation} \begin{array}{cl}
 u_1^{k+1} &=p_1
 u_1^{k}(1-u_1^{k})-p_2u_1^{k} u_2^{k},\\
 u_2^{k+1}&=(1-p_3)u_2^{k}+p_2u_1^{k}u_2^{k}.\\
\end{array} \end{equation}
In the first run $p_1$ is free.
In the second run, both $p_1$ and $p_2$ are free.
The remaining equation parameter, $p_3$, is fixed in both runs.

\begin{table}[htbp]
\begin{center}
\begin{tabular}{| l | l |}
\hline
  \AUTO-COMMAND  & ACTION \\
\hline
%==============================================================================
  \commandf{ ! mkdir dd2 } & create an empty work directory \\ 
  \commandf{ cd dd2 } & change directory \\
  \commandf{ demo('dd2') } & copy the demo files to the work directory \\
\hline
%==============================================================================
 
  \commandf{ ld('dd2')} & load the problem definition \\ 
  \commandf{ run(c='dd2.1')} & 1st run; fixed point solution branches \\ 
  \commandf{ sv('dd2')} & save output-files as \filef{ b.dd2, s.dd2, d.dd2} \\ 
\hline
%==============================================================================
  \commandf{ run(c='dd2.2',s='dd2')} & \parbox[t]{3in}{2nd run; a locus of Naimark-Sacker bifurcations.  Constants changed : {\tt IRS, ISW} \vspace{0.2cm}}\\ 
  \commandf{ sv('ns')} & save output-files as \filef{ b.ns, s.ns, d.ns} \\ 
\hline
\end{tabular}
\caption{Commands for running demo \filef{ dd2}.}
\label{tbl:demo_dd2}
\end{center}
\end{table}





