%==============================================================================
%==============================================================================
\chapter{ \AUTO Demos : Optimization.} \label{ch:Demos_Opt}
%==============================================================================
%==============================================================================

\newpage
%==============================================================================
%DEMO=opt======================================================================
%==============================================================================
\section{ opt : A Model Algebraic Optimization Problem.} \label{sec:Demos_opt}
This demo illustrates the method of successive continuation 
for constrained optimization problems 
 by applying it to the following
simple problem~:~
Find the
maximum sum of coordinates on the unit sphere in $R^{5}$.
Coordinate 1 is treated as the state variable.
Coordinates 2-5 are treated as control parameters.
For details on the successive continuation procedure
see  \citename{DoKeKe:91a} \citeyear{DoKeKe:91a},
\citename{DoKeKe:91b} \citeyear{DoKeKe:91b}.

\begin{table}[htbp]
\begin{center}
\begin{tabular}{| l | l |}
\hline
  \AUTO-COMMAND  & ACTION \\
\hline
%==============================================================================
  \commandf{ ! mkdir opt} & create an empty work directory \\ 
  \commandf{ cd opt} & change directory \\
  \commandf{ demo('opt')} & copy the demo files to the work directory \\
\hline
%==============================================================================
  \commandf{ run(c='opt.1')} & one free equation parameter \\ 
  \commandf{ sv('1')} & save output-files as \filef{ b.1, s.1, d.1} \\ 
\hline
%==============================================================================
  \commandf{ run(c='opt.2',s='1')} & \parbox[t]{3in}{two free equation parameters; read restart data from \filef{ s.1}.  Constants changed : \parf{ IRS} \vspace{0.2cm}}\\ 
  \commandf{ sv('2')} & save output-files as \filef{ b.2, s.2, d.2} \\ 
\hline
%============================================================================== 
  \commandf{ run(c='opt.3',s='2')} & \parbox[t]{3in}{three free equation parameters; read restart data from \filef{ s.2}.  Constants changed : \parf{ IRS} \vspace{0.2cm}}\\ 
  \commandf{ sv('3')} & save output-files as \filef{ b.3, s.3, d.3} \\ 
\hline
%==============================================================================
  \commandf{ run(c='opt.4',s='3')} & \parbox[t]{3in}{four free equation parameters; read restart data from \filef{ s.3}.  Constants changed : \parf{ IRS} \vspace{0.2cm}}\\ 
  \commandf{ sv('4')} & save output-files as \filef{ b.4, s.4, d.4} \\ 
\hline
%==============================================================================
\end{tabular}
\caption{Commands for running demo \filef{ opt}.}
\label{tbl:demo_opt}
\end{center}
\end{table}

\newpage
%==============================================================================
%DEMO=ops======================================================================
%==============================================================================
\section{ ops : Optimization of Periodic Solutions.} \label{sec:Demos_ops}
This demo illustrates the method of successive continuation
for the optimization of periodic solutions.
For a detailed description of the basic method see
\citename{DoKeKe:91b} \citeyear{DoKeKe:91b}.
The illustrative system of autonomous ODEs, taken from 
\citename{Alej:91} \citeyear{Alej:91}, is
\begin{equation} \begin{array}{cl}
  x'(t) & = [-\lambda_4(x^3/3-x) + (z-x)/\lambda_2 - y]/\lambda_1, \\
  y'(t) &= x-\lambda_3, \\
  z'(t) &= -(z-x)/\lambda_2,
\end{array} \end{equation}
with objective functional
$$ \omega = 
  \int_0^{1} g(x,y,z;\lambda_1,\lambda_2,\lambda_3,\lambda_4) ~ dt, $$
where $g(x,y,z;\lambda_1,\lambda_2,\lambda_3,\lambda_4) \equiv \lambda_3$.
Thus, in this application, a one-parameter extremum of $g$ corresponds
to a fold with respect to the problem parameter $\lambda_3$, 
and multi-parameter extrema correspond to generalized folds.
Note that, in general, the objective functional is an integral along 
the periodic orbit, so that a variety of optimization problems
can be addressed.

For the case of periodic solutions, the extended optimality system
can be generated automatically, i.e., one need only define the vector field 
and the objective functional, as in done in the file \filef{ ops.c}.
For reference purpose it is convenient here to write down
the full extended system in its general form~:

\begin{equation} \begin{array}{cl}
  &u'(t)  = T f \bigl( u(t),\lambda \bigr) ,
  \qquad T\in \R {\rm ~(period)},~ u(\cdot),f(\cdot,\cdot) \in \Rn, 
  ~ \lambda \in \R^{n_{\lambda}},  \\
  & \cr
  &w'(t)  = -Tf_u\bigl( u(t),\lambda \bigr)^{*} w(t) 
  + \kappa u_0'(t) 
  + \gamma g_u\bigl( u(t),\lambda \bigr)^{*}, 
  \qquad w(\cdot) \in \Rn,~ \kappa, \gamma \in \R, \\
  & \cr
  &u(1) - u(0) = 0, \qquad w(1) - w(0) = 0,  \\
  & \cr  
  &\int_{0}^{1} u(t)^{*} u_0'(t)~ dt = 0,  \\
  & \cr
  &\int_{0}^{1}  \omega - g\bigl(u(t),\lambda\bigr)  ~dt = 0,  \\
  & \cr
  &\int_0^{1}  w(t)^{*}w(t)
  + \kappa^2 + \gamma^{2} - \alpha ~ dt = 0, 
  \qquad \alpha \in \R,  \\ 
  & \cr
  &\int_0^{1}  f\bigl( u(t),\lambda \bigr)^{*}w(t) 
  - \gamma g_{T}\bigl( u(t),\lambda \bigr)
  - \tau_0  ~ dt= 0, \qquad \tau_0 \in \R,  \\
  & \cr
  &\int_0^{1}  T f_{\lambda_i}\bigl( u(t),\lambda \bigr)^{*}w(t)
  - \gamma g_{\lambda_i}\bigl( u(t),\lambda \bigr)
  - \tau_i  ~dt= 0, 
  \qquad \tau_i \in \R, \quad i=1, \cdots, n_{\lambda}.\\
\end{array} \end{equation}
Above  $u_0$ is a reference solution, namely, the previous solution along 
a solution branch.  

\newpage
In the computations below, the two preliminary runs, with \parf{ IPS}=1 and \parf{ IPS}=2,
respectively, locate periodic solutions. 
The subsequent runs are with \parf{ IPS}=15 and hence use the automatically
 generated extended system.

\begin{itemize}
\item[-] 
  \commandf{ Run 1.}~ Locate a Hopf bifurcation. 
  The free system parameter is $\lambda_3$. 
\item[-]\commandf{ Run 2.}~ 
  Compute a branch of periodic solutions from the Hopf bifurcation.
\item[-]\commandf{ Run 3.}~ 
  This run retraces part of the periodic solution branch, 
  using the full optimality system,
  but with all adjoint variables, $w(\cdot), \kappa, \gamma$, 
  and hence $\alpha$, equal to zero.
  The optimality parameters $\tau_0$ and $\tau_3$ are zero throughout.
  An extremum of the objective functional with respect to $\lambda_3$
  is located.
  Such a point corresponds to a branch point of the extended system. 
  Given the choice of objective functional in this demo, 
  this extremum is also a fold with respect to $\lambda_3$.
\item[-]\commandf{ Run 4.}~
  Branch switching at the above-found branch point yields nonzero
  values of the adjoint variables.
  Any point on the bifurcating branch away from the branch point
  can serve as starting solution for the next run.
  In fact, the branch-switching can be viewed as generating
  a nonzero eigenvector in an eigenvalue-eigenvector relation.
  Apart from the adjoint variables, all other variables remain
  unchanged along the bifurcating branch.
\item[-]\commandf{ Run 5.}~ 
  The above-found starting solution is continued in two system parameters, 
  here $\lambda_3$ and $\lambda_2$; i.e., a two-parameter branch 
  of extrema with respect to $\lambda_3$ is computed.
  Along this branch the value of the optimality parameter $\tau_2$ 
  is monitored, i.e., the value of the functional that vanishes 
  at an extremum with respect to the system parameter $\lambda_2$.
  Such a zero of $\tau_2$ is, in fact, located, and hence an extremum 
  of the objective functional with respect to both $\lambda_2$ and 
  $\lambda_3$ has been found.
  Note that, in general, $\tau_i$ is the value of the
  functional that vanishes at an extremum with respect to the system
  parameter $\lambda_i$.
\item[-]\commandf{ Run 6.}~ 
  In the final run, the above-found two-parameter extremum is continued
  in three system parameters, here $\lambda_1$, $\lambda_2$, 
  and $\lambda_3$, toward $\lambda_1=0$.
  Again, given the particular choice of objective functional,
  this final continuation has an alternate significance here~:
  it also represents a three-parameter branch of transcritical
  secondary periodic bifurcations points.
\end{itemize}

Although not illustrated here, one can restart an ordinary
continuation of periodic solutions, using \parf{ IPS}=2 or \parf{ IPS}=3,
from a labeled solution point on a branch computed with \parf{ IPS}=15.

\newpage
The free scalar variables specified in the \AUTO constants-files
for Run~3 and Run~4 are shown in  Table~\ref{tbl:demo_ops_1}.

\begin{table}[htbp]
\begin{center}
\begin{tabular}{| l | r | r | r | r | r | r | r |}
\hline
  Index& 3 & 11 & 12 & 22  & -22 & -23 & -31 \\
\hline
  Variable& $\lambda_3$ & $T$ &  $\alpha$ & $\tau_2$  
  & $[\lambda_2]$ & $[\lambda_3]$ & $[T]$ \\
\hline
\end{tabular}
\caption{\commandf{ Runs 3 and 4}~ (files \filef{ c.ops.3} and \filef{ c.ops.4}).}
\label{tbl:demo_ops_1}
\end{center}
\end{table}

The parameter $\alpha$, which is the norm of the adjoint variables,
becomes nonzero after branch switching in Run~4.
The negative indices (-22, -23, and -31) set the active optimality 
functionals, namely for $\lambda_2$, $\lambda_3$, and $T$, respectively,
with corresponding variables $\tau_2$, $\tau_3$, and $\tau_0$,
respectively.
These should be set in the first run with \parf{ IPS}=15 and remain unchanged
in all subsequent runs.


\begin{table}[htbp]
\begin{center}
\begin{tabular}{| l | r | r | r | r | r | r | r |}
\hline
  Index& 3 & ~2 & 11 & 22  & -22 & -23 & -31 \\
\hline
  Variable& $\lambda_3$ & $\lambda_2$ & $T$ & $\tau_2$  
  & $[\lambda_2]$ & $[\lambda_3]$ & $[T]$ \\
\hline
\end{tabular}
\caption{\commandf{ Run 5}~ (file \filef{ c.ops.5}).}
\label{tbl:demo_ops_2}
\end{center}
\end{table}

In Run~5 the parameter $\alpha$, which has been replaced by $\lambda_2$,
remains fixed and nonzero.
The variable $\tau_2$ monitors the value of the optimality functional 
associated with $\lambda_2$.
The zero of $\tau_2$ located in this run signals an extremum  
with respect to $\lambda_2$.


\begin{table}[htbp]
\begin{center}
\begin{tabular}{| l | r | r | r | r | r | r | r |}
\hline
  Index& 3 & ~2 & ~1 & 11  & -22 & -23 & -31 \\
\hline
  Variable& $\lambda_3$ & $\lambda_2$ & $\lambda_1$ & $T$  
  & $[\lambda_2]$ & $[\lambda_3]$ & $[T]$ \\
\hline
\end{tabular}
\caption{\commandf{ Run 6}~ (file \filef{ c.ops.6}).}
\label{tbl:demo_ops_3}
\end{center}
\end{table}


In Run~6 $\tau_2$, which has been replaced by $\lambda_1$, remains zero.


Note that $\tau_0$ and $\tau_3$ are not used as variables in any
of the runs; in fact, their values remain zero throughout.
Also note that the optimality functionals corresponding to 
$\tau_0$ and $\tau_3$ (or, equivalently, to $T$ and $\lambda_3$) 
\emp{ are} active in all runs.
This set-up allows the detection of the extremum of the objective functional,
with $T$ and $\lambda_3$ as scalar equation parameters,
as a bifurcation in the third run.

The parameter $\lambda_4$, and its corresponding optimality variable $\tau_4$,
are not used in this demo.
Also, $\lambda_1$ is used in the last run only, and its corresponding 
optimality variable $\tau_1$ is never used.



\begin{table}[htbp]
\begin{center}
\begin{tabular}{| l | l |}
\hline
  \AUTO-COMMAND  & ACTION \\
\hline
%==============================================================================
  \commandf{ ! mkdir ops} & create an empty work directory \\ 
  \commandf{ cd ops} & change directory \\
  \commandf{ demo('ops')} & copy the demo files to the work directory \\
\hline
%==============================================================================
  \commandf{ run(c='ops.1')} & locate a Hopf bifurcation \\ 
  \commandf{ sv('0')} & save output-files as \filef{ b.0, s.0, d.0} \\ 
\hline
%==============================================================================
  \commandf{ run(c='ops.2',s='0')} & \parbox[t]{3in}{compute a branch of periodic solutions;  restart from \filef{ s.0}. Constants changed : \parf{ IPS, IRS, NMX, NUZR}  \vspace{0.2cm}}\\ 
  \commandf{ ap('0')} & append the output-files to \filef{ b.0, s.0, d.0} \\ 
\hline
%==============================================================================
  \commandf{ run(c='ops.3',s='0')} & \parbox[t]{3in}{locate a 1-parameter extremum as a bifurcation; restart from \filef{ s.0}.  Constants changed : \parf{ IPS, IRS, ICP}, $\cdots$ \vspace{0.2cm}}\\ 
  \commandf{ sv('1')} & save the output-files as \filef{ b.1, s.1, d.1} \\ 
\hline
%==============================================================================
  \commandf{ run(c='ops.4',s='1')} & \parbox[t]{3in}{switch branches to generate optimality starting data; restart from \filef{ s.1}.  Constants changed : \parf{ IRS, ISP, ISW, NMX} \vspace{0.2cm}}\\ 
  \commandf{ ap('1')} & append the output-files to \filef{ b.1, s.1, d.1} \\ 
\hline
%==============================================================================
  \commandf{ run(c='ops.5',s='1')} & \parbox[t]{3in}{compute 2-parameter branch of 1-parameter extrema; restart from \filef{ s.1}.  Constants changed : \parf{ IRS, ISW, ICP, ISW}, $\cdots$\vspace{0.2cm}}\\ 
  \commandf{ sv('2')} & save the output-files as \filef{ b.2, s.2, d.2} \\ 
\hline
%==============================================================================
  \commandf{ run(c='ops.6',s='2')} & \parbox[t]{3in}{compute 3-parameter branch of 2-parameter extrema; restart from \filef{ s.2}.   Constants changed : \parf{ IRS, ICP, EPSL, EPSU, NUZR} \vspace{0.2cm}}\\ 
  \commandf{ sv('3')} & save the output-files as \filef{ b.3, s.3, d.3} \\ 
\hline
%==============================================================================
\end{tabular}
\caption{Commands for running demo \filef{ ops}.}
\label{tbl:demo_ops_4}
\end{center}
\end{table}

\newpage
%==============================================================================
%DEMO=obv======================================================================
%==============================================================================
\section{ obv : Optimization for a BVP.} \label{sec:Demos_obv}
This demo illustrates use of the method of successive continuation
for a  boundary value optimization problem.
A detailed description of the basic method, as well as a discussion
of the specific application considered here, is given in 
\citename{DoKeKe:91b} \citeyear{DoKeKe:91b}.
The required extended system is fully programmed here in the user-supplied
subroutines in \filef{ obv.c}.
For the case of periodic solutions the optimality system can be generated
automatically; see the demo \filef{ ops}.

Consider the system
\begin{equation} \begin{array}{cl}
  u_1'(t) & = u_2(t), \\
  u_2'(t) &=
  -\lambda_1 e^{p(u_1,\lambda_2,\lambda_3)},
\end{array} \end{equation}
where
$ p(u_1,\lambda_2,\lambda_3) \equiv
  u_1 + \lambda_2 u_1^{2} + \lambda_3 u_1^{4},$
with boundary conditions
\begin{equation} \begin{array}{cl}
  u_1(0) &= 0, \\
  u_1(1) &= 0. \\
\end{array} \end{equation}
The objective functional is
$$ \omega = \int_0^{1} (u_1(t)-1)^{2}~ dt
  +  {1\over{10}} \sum_{k=1}^{3} \lambda_{k}^{2}.  $$
The  successive continuation equations are given by
\begin{equation} \begin{array}{cl}
  u_1'(t) &= u_2(t), \\
  u_2'(t) &=
  -\lambda_1 e^{p(u_1,\lambda_2,\lambda_3)}, \\
  w_1'(t) &=
  \lambda_1 e^{p(u_1,\lambda_2,\lambda_3)} p_{u_1} w_2(t)
  + 2 \gamma(u_1(t)-1), \\
  w_2'(t) &= -w_1(t), \\
\end{array} \end{equation}
where
$$ p_{u_1} \equiv
  {{\partial p} \over {\partial u_1}} =
  1 + 2\lambda_2 u_1 + 4\lambda_3 u_1^{3},$$
with 
\begin{equation} \begin{array}{cl}
  u_1(0) = 0,\qquad  &w_1(0) - \beta_1 = 0,\qquad  w_2(0) = 0, \\
  u_1(1) = 0,\qquad  &w_1(1) + \beta_2 = 0,\qquad  w_2(1) = 0, \\\end{array} \end{equation}

$$ \int_0^{1} \bigl[ \omega - (u_1(t)-1)^{2}
  - {1\over{10}} \sum_{k=1}^{3} \lambda_{k}^{2} \bigr]~ dt = 0, $$

$$ \int_0^{1} \bigl[w_1^{2}(t) - \alpha_0 \bigr]~ dt = 0, $$
 
\begin{equation} \begin{array}{cl}
  &\int_0^{1} \bigl[
  -e^{p(u_1,\lambda_2,\lambda_3)} w_2(t)
  - {1\over 5}\gamma \lambda_1
  \bigr]~ dt = 0, \\
  &\int_0^{1} \bigl[
  -\lambda_1 e^{p(u_1,\lambda_2,\lambda_3)} u_1(t)^{2} w_2(t)
  - {1\over 5}\gamma \lambda_2
  - \tau_2  \bigr]~ dt = 0, \\
  &\int_0^{1} \bigl[
  -\lambda_1 e^{p(u_1,\lambda_2,\lambda_3)} u_1(t)^{4} w_2(t)
  - {1\over 5}\gamma \lambda_3
  - \tau_3 \bigr]~ dt = 0. \\\end{array} \end{equation}

In the first run the free equation parameter is $\lambda_1$.
All adjoint variables are zero.
Three extrema of the objective function are located.
These correspond to branch points and, in the second run,
branch switching is done at one of these.
Along the bifurcating branch the adjoint variables become nonzero,
while state variables and $\lambda_1$ remain constant.
Any such non-trivial solution point can be used for continuation 
in two equation parameters, after fixing the $L_2$-norm of one of 
the adjoint variables. This is done in the third run.
Along the resulting branch several two-parameter extrema are located 
by monotoring certain inner products.
One of these is further continued in three equation parameters in the final run,
where a three-parameter extremum is located.


\begin{table}[htbp]
\begin{center}
\begin{tabular}{| l | l |}
\hline
  \AUTO-COMMAND  & ACTION \\
\hline
%==============================================================================
  \commandf{ ! mkdir obv} & create an empty work directory \\ 
  \commandf{ cd obv} & change directory \\
  \commandf{ demo('obv')} & copy the demo files to the work directory \\
\hline
%==============================================================================
  \commandf{ run(c='obv.1')} & locate 1-parameter extrema as branch points \\ 
  \commandf{ sv('obv')} & save output-files as \filef{ b.obv, s.obv, d.obv} \\ 
\hline
%==============================================================================
  \commandf{ run(c='obv.2',s='obv')} & \parbox[t]{3in}{compute a few step on the first bifurcating branch.  Constants changed : \parf{ IRS, ISW, NMX} \vspace{0.2cm}}\\ 
  \commandf{ sv('1')} & save the output-files as \filef{ b.1, s.1, d.1} \\ 
\hline
%==============================================================================
  \commandf{ run(c='obv.3',s='1')} & \parbox[t]{3in}{locate 2-parameter extremum; restart from \filef{ s.1}.  Constants changed : \parf{ IRS, ISW, NMX, ICP(3)} \vspace{0.2cm}}\\ 
  \commandf{ sv('2')} & save the output-files as \filef{ b.2, s.2, d.2} \\ 
\hline
%==============================================================================
  \commandf{ run(c='obv.4',s='2')} & \parbox[t]{3in}{locate 3-parameter extremum; restart from \filef{ s.2}.  Constants changed : \parf{ IRS, ICP(4)} \vspace{0.2cm}}\\ 
  \commandf{ sv('3')} & save the output-files as \filef{ b.3, s.3, d.3} \\ 
\hline
%==============================================================================
\end{tabular}
\caption{Commands for running demo \filef{ obv}.}
\label{tbl:demo_obv}
\end{center}
\end{table}








