

%==============================================================================
%==============================================================================
\chapter{ {\cal HomCont} Demo : she.} \label{ch:HomCont_she}
%==============================================================================
%==============================================================================

%==============================================================================
%DEMO=she======================================================================
%==============================================================================
\section{ A Heteroclinic Example.}
The following system of five equations \citeasnoun{RuMa:95}
\begin{equation} \label{sh1} \begin{array}{rcl}
\dot{x} & = & \mu \, x + x\, y - z\, u, \\
\dot{y} & = & -y - x^2, \\
\dot{z} & = & (4\sigma\, x\, u + 4\sigma\, \mu\, z -9\sigma\, z 
+ 4 x\, u + 4\mu\, z) / 4(1+\sigma)  \\
\dot{u} & = & - \sigma u / 4 - \sigma Q v / 4\pi^2
+ 3(1 + \sigma) x z / 4\sigma \\
\dot{v} & = & \zeta u / 4  - \zeta v / 4
\end{array} 
\end{equation}
has been used to describe shearing instabilities in fluid convection.
The equations possess a rich structure of local and global bifurcations.
Here we shall reproduce a single curve in the $(\sigma,\mu)$-plane
of codimension-one heteroclinic orbits connecting a non-trivial 
equilibrium to the origin for $Q=0$ and $\zeta=4$. The defining
problem is contained in equation-file 
\filef{ she.c}\footnote{The last parameter used to store the equilibria (\parf{ PAR(21)}) is
overlaped here with the first test-function. In this example, it is harmless since the test functions are 
irrelevant for heteroclinic continuation.}, and starting data for the orbit at 
$(\sigma,\mu)=(0.5,0.163875)$ are stored in \filef{ she.dat},
with a truncation interval of \parf{ PAR(11)=85.07}.

We begin by computing towards $\mu=0$ with the option \parf{ IEQUIB=-2}
which means that both equilibria are solved for as part of
the continuation process.
\begin{center}
\commandf{ @dm she} \\
\commandf{ make first}
\end{center} 
This yields the output
\begin{verbatim}
  BR    PT  TY LAB    PAR(3)        L2-NORM    ...    PAR(1)     
   1     5       2  4.528332E-01  3.726787E-01 ...  1.364973E-01
   1    10       3  3.943370E-01  3.303798E-01 ...  1.044119E-01
   1    15       4  3.358942E-01  2.873213E-01 ...  7.515570E-02
   1    20       5  2.772726E-01  2.433403E-01 ...  4.952636E-02
   1    25       6  2.181955E-01  1.981358E-01 ...  2.845849E-02
   1    30  EP   7  1.581633E-01  1.512340E-01 ...  1.292975E-02
\end{verbatim}
Alternatively, for this problem there exists an analytic expression for
the two equilibria. This is specified in the subroutine \funcf{pvls} of
\filef{ she.c}. Re-running with \parf{ IEQUIB=-1}
\begin{center}
\it make second
\end{center} 
we obtain the output
\begin{verbatim}
  BR    PT  TY LAB    PAR(3)        L2-NORM    ...    PAR(1)     
   1     5       2  4.432015E-01  3.657716E-01 ...  1.310559E-01
   1    10       3  3.723085E-01  3.142439E-01 ...  9.300982E-02
   1    15       4  3.008842E-01  2.611556E-01 ...  5.933966E-02
   1    20       5  2.286652E-01  2.062194E-01 ...  3.179939E-02
   1    25       6  1.555409E-01  1.491652E-01 ...  1.239897E-02
   1    30  EP   7  8.107462E-02  9.143108E-02 ...  2.386616E-03
\end{verbatim}
This output is similar to that above, but note that it is obtained slightly
more efficiently because the extra parameters \parf{ PAR(12-21)} representing the
coordinates of the equilibria are no longer
part of the continuation problem. Also note that \AUTO has chosen to take
slightly larger steps along the branch. Finally, we can continue in the opposite
direction along the branch from the original starting point (again with \parf{ IEQUIB=-1}).
\begin{center}
\it make third
\end{center}
%
\begin{figure}[b]
\epsfysize 9.0cm
\centerline{\epsffile{include/she1.ps}}
\caption{Projections into $(x,y,z)$-space of the family of heteroclinic
orbits.}
\label{Fshear}
\end{figure}
%
\begin{verbatim}
  BR    PT  TY LAB    PAR(3)        L2-NORM    ...    PAR(1)     
   1     5       8  4.997590E-01  4.060153E-01 ...  1.637322E-01
   1    10       9  5.705299E-01  4.551872E-01 ...  2.065264E-01
   1    15      10  6.416439E-01  5.031844E-01 ...  2.507829E-01
   1    20      11  7.133301E-01  5.500668E-01 ...  2.959336E-01
   1    25      12  7.857688E-01  5.958712E-01 ...  3.415492E-01
   1    30      13  8.590970E-01  6.406182E-01 ...  3.872997E-01
   1    35  EP  14  9.334159E-01  6.843173E-01 ...  4.329270E-01
\end{verbatim}
The results of both computations are presented in Figure \ref{Fshear}, 
from which we see that the orbit shrinks to zero as
\parf{ PAR(1)=}$\mu \to 0$.

\newpage   
\section{ Detailed \AUTO-Commands.}
\begin{table}[htbp]
\begin{center}
\begin{tabular}{| l | l |}
\hline
  \AUTO-COMMAND  & ACTION \\
\hline
%==============================================================================
  \commandf{ ! mkdir she} & create an empty work directory \\ 
  \commandf{ cd she} & change directory \\
  \commandf{ demo('she')} & copy the demo files to the work directory \\
\hline
%==============================================================================
  \commandf{ us('she')} & use the starting data in \filef{ she.dat} to create \filef{ s.dat} \\ 
  \commandf{ run(c='she.1',h='she.1',s='dat')} &  continue heteroclinic orbit; restart from \filef{ s.dat}\\ 
  \commandf{ sv('1')} & save output-files as \filef{ b.1, s.1, d.1} \\ 
\hline
%==============================================================================
  \commandf{ run(c='she.2',h='she.2',s='dat')} &  repeat with \parf{IEQUIB=-1} \\ 
  \commandf{ sv('2')} & save output-files as \filef{ b.2, s.2, d.2} \\ 
\hline
%=============================================================================
  \commandf{ run(c='she.3',h='she.3',s='2')} & continue in reverse direction ; restart from \filef{ s.2} \\ 
  \commandf{ ap('2')} & append output-files to \filef{ b.2, s.2, d.2} \\ 
\hline
%=============================================================================
\end{tabular}
\caption{Detailed \AUTO-Commands for running demo \filef{ she}.}
\label{tbl:demo_she_1}
\end{center}
\end{table}



