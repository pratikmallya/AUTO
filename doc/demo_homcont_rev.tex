%==============================================================================
%==============================================================================
\chapter{ {\cal HomCont} Demo : rev.} \label{ch:HomCont_rev}
%==============================================================================
%==============================================================================

%==============================================================================
%DEMO=rev======================================================================
%==============================================================================
\section{ A Reversible System.}
The fourth-order differential equation
$$
u'''' + P u'' + u -u^3 =0
$$
arises in a number of contexts, e.g., as the travelling-wave
equation for a nonlinear-Schr\"{o}dinger equation with fourth-order
dissipation \cite{BuAk:95} and as a model of a strut on a symmetric 
nonlinear elastic foundation \cite{HuBoTh:89}. It may be expressed as
a system
\beq
\left \{ 
\begin{array}{rcl}
\dot{u_1} & = & u_2 \\
\dot{u_2} & = & u_3 \\
\dot{u_3} & = & u_4 \\
\dot{u_4} & = & -P u_3 - u_1 + u_1^3
\end{array}
\right.  
\label{6.1}
\eeq
Note that (\ref{6.1}) is invariant under two separate reversibilities
\beq
R_1: (u_1,u_2,u_3,u_4,t) \mapsto (u_1,-u_2,u_3,-u_4,-t)  
\label{6.R1}
\eeq
and 
\beq
R_2: (u_1,u_2,u_3,u_4,t) \mapsto (-u_1,u_2,-u_3,u_4,-t)  
\label{6.R2}
\eeq
First, we copy the demo into a new directory 
\begin{center}
\commandf{ @dm rev }
\end{center}
For this example, we shall make two separate starts
from data stored in equation and data files \filef{ rev.c.1,
rev.dat.1} and \filef{ rev.c.3, rev.dat.3} respectively. The first
of these contains initial data for a solution that is reversible
under $R_1$ and the second for data that is reversible under $R_2$. 
%
%Note that \commandf{ make} or \commandf{ make all} will only run the
%first of these. To make the output starting from the
%$R_2$-reversible solution we need to \commandf{ make run2}. As before,
%though we illustrate here the step by step approach.


\section{An $R_1$-Reversible Homoclinic Solution.}

The first run
\begin{center}
\it make first
\end{center}
starts by 
copying the files \filef{ rev.c.1} and \filef{ rev.dat.1} to 
\filef{ rev.c} and \filef{ rev.dat}. The orbit contained in
the data file is a ``primary'' homoclinic solution for $P=1.6$, with
truncation (half-)interval \parf{ PAR(11) = 39.0448429}.
which is reversible under $R_1$. Note that this reversibility is
specified in \filef{ h.rev.1} via \parf{ NREV=1}, 
\parf{ (IREV(I), I=1,NDIM)} \parf{ = 0 1 0 1}. Note also, from
\filef{ c.rev.1} that we only have one free parameter \parf{ PAR(1)}
because symmetric homoclinic orbits in reversible systems are
generic rather than of codimension one.
The first run  results in the output
\begin{verbatim}
  BR    PT  TY LAB    PAR(1)        L2-NORM       MAX U(1)   ...   
   1     7  UZ   2  1.700002E+00  2.633353E-01  4.179794E-01
   1    12  UZ   3  1.800000E+00  2.682659E-01  4.806063E-01
   1    15  UZ   4  1.900006E+00  2.493415E-01  4.429364E-01
   1    20  EP   5  1.996247E+00  1.111306E-01  1.007111E-01
\end{verbatim}
which is consistent with the theoretical result that the solution
tends uniformly to zero as $P\to 0$. Note, by plotting the data
saved in \filef{ s.1} that only ``half'' of the 
homoclinic orbit is computed up to its point of symmetry. See Figure
\ref{Frev1}.

\begin{figure}[p]
\epsfysize 9.0cm
\centerline{\epsffile{include/rev1.ps}}
\caption{$R_1$-Reversible homoclinic solutions on the half-interval
$x/T \in [0,1]$ where $T=39.0448429$ for $P$ approaching $2$ (solutions
with labels \parf{ 1-5} respectively have decreasing amplitude)}
\label{Frev1}
\end{figure}
\begin{figure}[p]
\epsfysize 9.0cm
\centerline{\epsffile{include/rev2.ps}}
\caption{$R_1$-reversible homoclinic orbits with oscillatory decay 
as $x \to -\infty$ (corresponding to label \parf{ 6}) and monotone decay 
(at label \parf{ 10})}
\label{Frev2}
\end{figure}

The second run continues in the other direction of \parf{ PAR(1)}, with
the test function $\psi_2$ activated 
for the detection of saddle to saddle-focus transition points
\begin{center}
\it make second
\end{center}
The output
\begin{verbatim}
 BR  PT  TY LAB    PAR(1)        L2-NORM       MAX U(1)   ...    PAR(22)    
  1  11  UZ   6  1.000005E+00  2.555446E-01  1.767149E-01 ... -3.000005E+00
  1  22  UZ   7 -1.198325E-07  2.625491E-01  4.697314E-02 ... -2.000000E+00
  1  33  UZ   8 -1.000000E+00  2.741483E-01  4.316007E-03 ... -1.000000E+00
  1  44  UZ   9 -2.000000E+00  2.873838E-01  1.245735E-11 ...  2.318248E-08
  1  55  EP  10 -3.099341E+00  3.020172E-01 -2.749454E-11 ...  1.099341E+00
\end{verbatim}
shows a saddle to saddle-focus transition 
(indicated by a zero of \parf{ PAR(22)}) at \parf{ PAR(1)=-2}. Beyond
that label the first component of the solution is negative and (up to the
point of symmetry) monotone decreasing. See Figure \ref{Frev2}.

\section{An $R_2$-Reversible Homoclinic Solution.}

\begin{center}
\commandf{ make third}
\end{center}
Copies the files \filef{ rev.c.3} and \filef{ rev.dat.3} to 
\filef{ rev.c} and \filef{ rev.dat}, and runs them with the
constants stored in \filef{ c.rev.3} and \filef{ h.rev.3}. 
The orbit contained in
the data file is a ``multi-pulse'' homoclinic solution for $P=1.6$, with
truncation (half-)interval \parf{ PAR(11) = 47.4464189}.
which is reversible under $R_2$. This reversibility is
specified in \filef{ h.rev.1} via \parf{ NREV=1}, 
\parf{ (IREV(I), I=1,NDIM)} \parf{ = 1 0 1 0}. The output 
\begin{verbatim}
  BR    PT  TY LAB    PAR(1)        L2-NORM       MAX U(1)   ...
   1    15  UZ   2  1.700000E+00  3.836401E-01  4.890015E-01  
   1    16  LP   3  1.711574E+00  3.922135E-01  5.442385E-01  
   1    19  UZ   4  1.600000E+00  4.329404E-01  7.769491E-01  
   1    31  UZ   5  1.000000E+00  4.808488E-01  1.083298E+00  
   1    86  UZ   6 -9.664802E-10  5.158463E-01  1.258650E+00  
\end{verbatim}
contains the label of a limit point (\parf{ ILP} was set to \parf{ 1} in
\filef{ c.rev.3}, which corresponds to a ``coalescence'' of two reversible
homoclinic orbits. The two solutions on either side of this limit point are
displayed in Figure \ref{Frev3}. The computation ends in a no-convergence
point. The solution here is depicted in Figure \ref{Frev4}. The lack of
convergence is due to the large peak and trough of the solution rapidly
moving to the left as $P \to -2$ (cf. \citeasnoun{ChSp:93}).

\begin{figure}[p]
\epsfysize 9.0cm
\centerline{\epsffile{include/rev3.ps}}
\caption{Two $R_2$-reversible homoclinic orbits at $P=1.6$ 
corresponding to labels \parf{ 1} (smaller amplitude) and \parf{ 5} (larger amplitude)}
\label{Frev3}
\end{figure}
\begin{figure}[p]
\epsfysize 9.0cm
\centerline{\epsffile{include/rev4.ps}}
\caption{An $R_2$-reversible homoclinic orbit at label \parf{ 8}}
\label{Frev4}
\end{figure}

Continuing from the initial solution in the other parameter direction
\begin{center}
\it make fourth
\end{center}
we obtain the output
\begin{verbatim}
  BR    PT  TY LAB    PAR(1)        L2-NORM       MAX U(1)   ...
   1     7  UZ   8  1.600000E+00  3.701709E-01  3.836833E-01  
   1    33  UZ   9  9.999980E-01  3.614405E-01  1.775035E-01  
   1    93  UZ  10 -7.819855E-06  3.713007E-01  4.698309E-02  
\end{verbatim}
which again ends at a no convergence error for similar reasons.




\newpage
\section{ Detailed \AUTO-Commands.}
\begin{table}[htbp]
\begin{center}
\begin{tabular}{| l | l |}
\hline
  \AUTO-COMMAND  & ACTION \\
\hline
%==============================================================================
  \commandf{ ! mkdir rev} & create an empty work directory \\ 
  \commandf{ cd rev} & change directory \\
  \commandf{ demo('rev')} & copy the demo files to the work directory \\
\hline
%==============================================================================
  \commandf{ cp rev.c.1 rev.c} &  get equations file to \filef{ rev.c}\\
  \commandf{ cp rev.dat.1 rev.dat} & get the starting data to \filef{ rev.dat} \\ 
  \commandf{ us('rev')} & use the starting data in \filef{ rev.dat} to create \filef{ s.dat} \\ 
  \commandf{ run(c='rev.1',h='rev.1',s='dat')} &  increase \parf{ PAR(1)} \\ 
  \commandf{ sv('1')} & save output-files as \filef{ b.1, s.1, d.1} \\ 
\hline
%==============================================================================
  \commandf{ run(c='rev.2',h='rev.2',s='1')} &  continue in reverse direction; restart from \filef{ s.1} \\ 
  \commandf{ ap('1')} & append output-files to \filef{ b.1, s.1, d.1} \\ 
\hline
%=============================================================================
  \commandf{ cp rev.c.3 rev.c} & get equations file with new value of \parf{ PAR(11)}\\
  \commandf{ cp rev.dat.3 rev.dat} & get starting data with different reversibility\\
  \commandf{ us('rev')} & use the starting data in \filef{ rev.dat} to create \filef{ s.dat} \\ 
  \commandf{ run(c='rev.3',h='rev.3',s='dat')} & restart with different reversibility \\ 
  \commandf{ sv('3')} & save output-files as \filef{ b.3, s.3, d.3} \\ 
\hline
%==============================================================================
  \commandf{ run(c='rev.4',h='rev.4',s='3')} & continue in reverse direction; restart from \filef{ s.3} \\ 
  \commandf{ ap('3')} & append output-files to \filef{ b.3, s.3, d.3} \\ 
\hline
%=============================================================================
\end{tabular}
\caption{Detailed \AUTO-Commands for running demo \filef{ rev}.}
\label{tbl:demo_rev_1}
\end{center}
\end{table}

