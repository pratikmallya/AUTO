
%==============================================================================
%==============================================================================
\chapter{ \AUTO Demos : Miscellaneous.} \label{ch:Demos_Misc}
%==============================================================================
%==============================================================================

\newpage
%==============================================================================
%DEMO=pvl======================================================================
%==============================================================================
\section{ pvl : Use of the Subroutine \funcf{ pvls}.} \label{sec:Demos_pvl}

Consider Bratu's equation
\begin{equation} \begin{array}{cl}
  u_1 ' &= u_2  ,  \\
  u_2 ' &= -p_1  e^{u_1} , \\ 
\end{array} \end{equation}
with boundary conditions $ u_1(0)=0$, $u_1(1)=0.$
As in demo \filef{ exp}, a solution curve requires one free parameter;
here $p_1$.

Note that additional parameters are specified in the user-supplied subroutine 
\funcf{ pvls} in file \filef{ pvls.c}, namely,
$p_2$ (the $L_2$-norm of $u_1$),
$p_3$ (the minimum of $u_2$ on the space-interval $[0,1]$~),
$p_4$ (the boundary value $u_2(0)$~).
These additional parameters should be considered as ``solution measures''
for output purposes; they should not be treated as true
continuation parameters.

Note also that four free parameters are specified in the \AUTO-constants file 
\filef{ c.pvl.1}, namely, $p_1$, $p_2$, $p_3$, and $p_4$.
The first one in this list, $p_1$, is the true continuation parameter. 
The parameters $p_2$, $p_3$, and $p_4$ are \emp{ overspecified}
so that their values will appear in the output.
However, 
\emp{ it is essential that the true continuation parameter appear first.}
For example, it would be an error to specify the parameters
in the following order~: $p_2$, $p_1$, $p_3$, $p_4$.

In general, true continuation parameters must appear first in the
parameter-specification in the \AUTO constants-file.
Overspecified parameters will be printed, and can be
defined in \funcf{ pvls}, but they are not part of the intrinsic continuation
procedure.

As this demo also illustrates (see the \parf{ UZR} values in \filef{ c.pvl.1}),
labeled solutions can also be output at selected values 
of the overspecified parameters.

\begin{table}[htbp]
\begin{center}
\begin{tabular}{| l | l |}
\hline
  \AUTO-COMMAND  & ACTION \\
\hline
%==============================================================================
  \commandf{ ! mkdir pvl} & create an empty work directory \\ 
  \commandf{ cd pvl} & change directory \\
  \commandf{ demo('pvl')} & copy the demo files to the work directory \\
\hline
%==============================================================================
  \commandf{ run(c='pvl.1')} & compute a solution branch \\ 
  \commandf{ sv('pvl')} & save output-files as \filef{ b.pvl, s.pvl, d.pvl} \\ 
\hline
%==============================================================================
\end{tabular}
\caption{Commands for running demo \filef{ pvl}.}
\label{tbl:demo_pvl}
\end{center}
\end{table}

\newpage
%==============================================================================
%DEMO=ext======================================================================
%==============================================================================
\section{ ext : Spurious Solutions to BVB.} \label{sec:Demos_ext}

This demo illustrates the computation of spurious solutions
to the boundary value problem
\begin{equation} \begin{array}{cl}
& u_1' - u_2 = 0 , \\
& u_2' + \lambda^2 \pi^2 \sin( u_1 + u_1^2 + u_1^3 ) = 0,
  \qquad t \in [0,1], \\ 
& u_1(0) = 0, \quad u_1(1) = 0. \\
\end{array} \end{equation}
Here the differential equation is discretized using a fixed uniform mesh.
This results in spurious solutions that disappear when an adaptive mesh is used.
See the \AUTO-constant \parf{ IAD} in Section~\ref{sec:Discretization_constants}.
This example is also considered in
\citename{BeDo:81} \citeyear{BeDo:81}
and
\citename{DoKeKe:91b} \citeyear{DoKeKe:91b}.

\begin{table}[htbp]
\begin{center}
\begin{tabular}{| l | l |}
\hline
  \AUTO-COMMAND  & ACTION \\
\hline
%==============================================================================
  \commandf{ ! mkdir ext} & create an empty work directory \\ 
  \commandf{ cd ext} & change directory \\
  \commandf{ demo('ext')} & copy the demo files to the work directory \\
\hline
%==============================================================================
  \commandf{ run(c='ext.1')} & detect bifurcations from the trivial solution branch \\ 
  \commandf{ sv('ext')} & save output-files as \filef{ b.ext, s.ext, d.ext} \\ 
\hline
%==============================================================================
  \commandf{ run(c='ext.2',s='ext')} & \parbox[t]{3in}{compute a bifurcating branch containing spurious bifurcations.  Constants changed : \parf{ IRS, ISW, NUZR} \vspace{0.2cm}}\\ 
  \commandf{ ap('ext')} & append output-files to \filef{ b.ext, s.ext, d.ext} \\ 
\hline
%==============================================================================
\end{tabular}
\caption{Commands for running demo \filef{ ext}.}
\label{tbl:demo_ext}
\end{center}
\end{table}

\newpage
%==============================================================================
%DEMO=tim======================================================================
%==============================================================================
\section{ tim : A Test Problem for Timing \AUTO.} \label{sec:Demos_tim}
This demo is a boundary value problem with variable dimension \parf{ NDIM}. 
It can be used to time the performance of \AUTO 
for various choices of \parf{ NDIM} (which must be even), \parf{ NTST}, and \parf{ NCOL}.
The equations are
\begin{equation} \begin{array}{cl}
  u_i ' &= u_i  ,  \\
  v_i ' &= -p_1 ~  e(u_i) , \\
\end{array} \end{equation}
$i=1,\cdots$,\parf{ NDIM}/2,
with boundary conditions $ u_i(0)=0$, $u_i(1)=0.$
Here 
$$ e(u) = \sum_{k=0}^{n} ~ {u^k \over k!} ~ , $$
with $n=25$.
The computation requires 10 full $LU$-decompositions of the linearized system
that arises from Newton's method for solving the collocation equations.
The commands for running the timing problem for a particular choice 
of \parf{ NDIM}, \parf{ NTST}, and \parf{ NCOL} are given below.
(Note that if \parf{ NDIM} is changed then \parf{ NBC} must be changed accordingly.)

\begin{table}[htbp]
\begin{center}
\begin{tabular}{| l | l |}
\hline
  \AUTO-COMMAND  & ACTION \\
\hline
%==============================================================================
  \commandf{ ! mkdir tim} & create an empty work directory \\ 
  \commandf{ cd tim} & change directory \\
  \commandf{ demo('tim')} & copy the demo files to the work directory \\
\hline
%==============================================================================
  \commandf{ run(c='tim.1')} & Timing run \\ 
  \commandf{ sv('tim')} & save output-files as \filef{ b.tim, s.tim, d.tim} \\ 
\hline
%==============================================================================
\end{tabular}
\caption{Commands for running demo \filef{ tim}.}
\label{tbl:demo_tim}
\end{center}
\end{table}

