
%==============================================================================
%==============================================================================
\chapter{ \AUTO Demos : BVP.} \label{ch:Demos_BVP}
%==============================================================================
%==============================================================================

%==============================================================================
%DEMO=exp======================================================================
%==============================================================================
\section{ exp : Bratu's Equation.} \label{sec:Demos_exp}
This demo illustrates the computation of a solution branch to
the boundary value problem


\begin{equation} \begin{array}{cl}
  u_1 ' &= u_2  ,  \\
  u_2 ' &= -p_1  e^{u_1} , \\
\end{array} \end{equation}
with boundary conditions $ u_1(0)=0 ,  \quad  u_1(1)=0.$
This equation is also considered in 
\citename{DoKeKe:91a} \citeyear{DoKeKe:91a}.
\begin{table}[htbp]
\begin{center}
\begin{tabular}{| l | l |}
\hline
  \AUTO-COMMAND  & ACTION \\
\hline
%==============================================================================
  \commandf{ ! mkdir exp} & create an empty work directory \\ 
  \commandf{ cd exp} & change directory \\
  \commandf{ demo('exp')} & copy the demo files to the work directory \\
\hline
%==============================================================================
  \commandf{ run(c='exp.1')} & 1st run; compute solution branch containing fold \\ 
  \commandf{ sv('exp')} & save output-files as \filef{ b.exp, s.exp, d.exp} \\ 
\hline
%==============================================================================
  \commandf{ run(c='exp.2',s='exp')} & \parbox[t]{3in}{2nd run; restart at a labeled solution, using increased accuracy.  Constants changed : \parf{ IRS, NTST, A1, DSMAX} vspace{0.2cm}}\\ 
  \commandf{ ap('exp')} & append output-files to \filef{ b.exp, s.exp, d.exp} \\ 
\hline
%==============================================================================
\end{tabular}
\caption{Commands for running demo \filef{ exp}.}
\label{tbl:demo_exp}
\end{center}
\end{table}

\newpage
%==============================================================================
%DEMO=int======================================================================
%==============================================================================
\section{ int : Boundary and Integral Constraints.} \label{sec:Demos_int}
This demo illustrates the computation of a solution branch to
the equation

\begin{equation} \begin{array}{cl}
 u_1 ' &= u_2 , \\
  u_2 ' &= -p_1  e^{u_1} , \\\end{array} \end{equation}
with a non-separated boundary condition and an integral constraint:

$$ u_1(0)-u_1(1)-p_2=0 ,\qquad \int_0^{1}u(t)dt-p_3=0 . $$
The solution branch contains a fold, which, in the second run, is  
continued in two equation parameters.

\begin{table}[htbp]
\begin{center}
\begin{tabular}{| l | l |}
\hline
  \AUTO-COMMAND  & ACTION \\
\hline
%==============================================================================
  \commandf{ ! mkdir int} & create an empty work directory \\ 
  \commandf{ cd int} & change directory \\
  \commandf{ demo('int')} & copy the demo files to the work directory \\
\hline
%==============================================================================
  \commandf{ run(c='int.1')} & 1st run; detection of a fold \\ 
  \commandf{ sv('int')} & save output-files as \filef{ b.int, s.int, d.int} \\ 
\hline
%==============================================================================
  \commandf{ run(c='int.2',s='int')} & 2nd run; generate starting data for a curve of folds.  \parbox[t]{3in}{Constants changed : \parf{ IRS, ISW } vspace{0.2cm}}\\ 
  \commandf{ sv('t')} & save the output-files as \filef{ b.t, s.t, d.t} \\ 
\hline
%==============================================================================

  \commandf{ run(c='int.3',s='t')} & \parbox[t]{3in}{3rd run; compute a curve of folds; restart from \filef{ s.t}.  Constants changed : \parf{ IRS} vspace{0.2cm}}\\ 
  \commandf{ sv('lp')} & save the output-files as \filef{ b.lp, s.lp, d.lp} \\ 
\hline
%==============================================================================
\end{tabular}
\caption{Commands for running demo \filef{ int}.}
\label{tbl:demo_int}
\end{center}
\end{table}

\newpage
%==============================================================================
%DEMO=bvp======================================================================
%==============================================================================
\section{ bvp : A Nonlinear ODE Eigenvalue Problem.} \label{sec:Demos_bvp}
This demo illustrates the location of eigenvalues of a nonlinear ODE
boundary value problem as bifurcations from the trivial solution branch.
The branch of solutions that bifurcates at the first eigenvalue
is computed in both directions.
The equations are
\begin{equation} \begin{array}{cl}
 u_1 ' &= u_2  ,  \\
  u_2 ' &=-(p_1 \pi)^{2}u_1 + u_1^{2} ,\end{array} \end{equation}
with boundary conditions $ u_1(0)=0 ,  \quad  u_1(1)=0.$~~~

\begin{table}[htbp]
\begin{center}
\begin{tabular}{| l | l |}
\hline
  \AUTO-COMMAND  & ACTION \\
\hline
%==============================================================================
  \commandf{ ! mkdir bvp} & create an empty work directory \\ 
  \commandf{ cd bvp} & change directory \\
  \commandf{ demo('bvp')} & copy the demo files to the work directory \\
\hline
%==============================================================================
  \commandf{ run(c='bvp.1')} &  compute the trivial solution branch and locate eigenvalues \\ 
  \commandf{ sv('bvp')} & save output-files as \filef{ b.bvp, s.bvp, d.bvp} \\ 
\hline
%==============================================================================
  \commandf{ run(c='bvp.2',s='bvp')} &  \parbox[t]{3in}{compute the first bifurcating branch.  Constants changed : \parf{ IRS, ISW, NPR, DSMAX} \vspace{0.2cm}}\\ 
  \commandf{ ap('bvp')} & append output-files to \filef{ b.bvp, s.bvp, d.bvp} \\ 
\hline
%==============================================================================
  \commandf{ run(c='bvp.3',s='bvp')} &  \parbox[t]{3in}{compute the first bifurcating branch in opposite direction.  Constants changed : \parf{ DS} \vspace{0.2cm}}\\ 
  \commandf{ ap('bvp')} & append output-files to \filef{ b.bvp, s.bvp, d.bvp} \\ 
\hline
%==============================================================================
\end{tabular}
\caption{Commands for running demo \filef{ bvp}.}
\label{tbl:demo_bvp}
\end{center}
\end{table}

\newpage
%==============================================================================
%DEMO=lin======================================================================
%==============================================================================
\section{ lin : A Linear ODE Eigenvalue Problem.} \label{sec:Demos_lin}
This demo illustrates the location of eigenvalues of a linear ODE
boundary value problem as bifurcations from the trivial solution branch.
By means of branch switching an eigenfunction is computed,
as is illustrated for the first eigenvalue. 
This eigenvalue is then continued in two parameters
by fixing the $L_2$-norm of the first solution component.
The eigenvalue problem is given by the equations

\begin{equation} \begin{array}{cl}
  u_1 ' &= u_2  ,  \\
  u_2 ' &= (p_1 \pi)^{2} u_1 , \end{array} \end{equation}
with boundary conditions $ u_1(0)-p_2=0 $ and $  u_1(1)=0.$
We add the integral constraint
 $$ \int_0^{1} u_1(t)^{2} dt - p_3 = 0. $$
Then $p_3$ is simply the $L_2$-norm of the first solution component.
In the first two runs $p_2$ is fixed, while $p_1$ and $p_3$ are free.
In the third run  $p_3$ is fixed, while $p_1$ and $p_2$ are free.

\begin{table}[htbp]
\begin{center}
\begin{tabular}{| l | l |}
\hline
  \AUTO-COMMAND  & ACTION \\
\hline
%==============================================================================
  \commandf{ ! mkdir lin} & create an empty work directory \\ 
  \commandf{ cd lin} & change directory \\
  \commandf{ demo('lin')} & copy the demo files to the work directory \\
\hline
%==============================================================================
  \commandf{ run(c='lin.1')} & 1st run; compute the trivial solution branch and locate eigenvalues \\ 
  \commandf{ sv('lin')} & save output-files as \filef{ b.lin, s.lin, d.lin} \\ 
\hline
%==============================================================================
  \commandf{ run(c='lin.2',s='lin')} & \parbox[t]{3in}{2nd run; compute a few steps along the bifurcating branch.  Constants changed : \parf{ IRS, ISW, DSMAX} \vspace{0.2cm}}\\ 
  \commandf{ ap('lin')} & append output-files to \filef{ b.lin, s.lin, d.lin} \\ 
\hline
%==============================================================================
  \commandf{ run(c='lin.3',s='lin')} & \parbox[t]{3in}{3rd run; compute a two-parameter curve of eigenvalues. Constants changed : \parf{ IRS, ISW, ICP(2)} \vspace{0.2cm}} \\ 
  \commandf{ sv('2p')} & save the output-files as \filef{ b.2p, s.2p, d.2p} \\ 
\hline
%==============================================================================
\end{tabular}
\caption{Commands for running demo \filef{ lin}.}
\label{tbl:demo_lin}
\end{center}
\end{table}

\newpage
%==============================================================================
%DEMO=non======================================================================
%==============================================================================
\section{ non : A Non-Autonomous BVP.} \label{sec:Demos_non}
This demo illustrates the continuation of solutions to
the non-autonomous boundary value problem

\begin{equation} \begin{array}{cl}
  u_1 ' &= u_2  ,  \\
  u_2 ' &= -p_1  e^{x^3 u_1} , \\\end{array} \end{equation}
with boundary conditions $ u_1(0)=0 ,  \quad  u_1(1)=0.$
Here $x$ is the independent variable.
This system is first converted to the following equivalent
autonomous system~:
\begin{equation} \begin{array}{cl}
  u_1 ' &= u_2  ,  \\
  u_2 ' &= -p_1  e^{u_3^3 u_1} ,  \\  
  u_3 ' &= 1 ,  \\
\end{array} \end{equation}
 with boundary conditions $ u_1(0)=0 ,  \quad  u_1(1)=0, \quad u_3(0)=0.$
(For a periodically forced system see demo \filef{ frc}).

\begin{table}[htbp]
\begin{center}
\begin{tabular}{| l | l |}
\hline
  \AUTO-COMMAND  & ACTION \\
\hline
%==============================================================================
  \commandf{ ! mkdir non} & create an empty work directory \\ 
  \commandf{ cd non} & change directory \\
  \commandf{ demo('non')} & copy the demo files to the work directory \\
\hline
%==============================================================================

  \commandf{ run(c='non.1')} & compute the solution branch \\ 
  \commandf{ sv('non')} & save output-files as \filef{ b.non, s.non, d.non} \\ 
\hline
%==============================================================================
\end{tabular}
\caption{Commands for running demo \filef{ non}.}
\label{tbl:demo_non}
\end{center}
\end{table}

\newpage
%==============================================================================
%DEMO=kar======================================================================
%==============================================================================
\section{ kar : The Von Karman Swirling Flows.} \label{sec:Demos_kar}
The steady axi-symmetric flow of a viscous incompressible fluid
above an infinite rotating disk is modeled by the following 
ODE boundary value problem (Equation (11) in
\citename{LeKe:80} \citeyear{LeKe:80}~:
\begin{equation} \begin{array}{cl}
  u_1' &= T u_2,  \\
  u_2' &= T u_3,  \\
  u_3' &= T \bigl[ -2 \gamma u_4 + u_2^2 - 2 u_1 u_3 - u_4^2 \bigr], \\
  u_4' &= T u_5, \\
  u_5' &= T \bigl[ 2 \gamma u_2 + 2 u_2 u_4 - 2 u_1 u_5 \bigr], \\
\end{array} \end{equation}
with left boundary conditions
$$ u_1(0)=0, \qquad u_2(0)=0, \qquad u_4(0)=1-\gamma, $$
and (asymptotic) right boundary conditions
\begin{equation} \begin{array}{cl}
  & \bigl[ f_\infty + a(f_\infty,\gamma) \bigr] ~ u_2(1) + u_3(1)
  - \gamma ~ { u_4(1) \over a(f_\infty,\gamma) } = 0,  \\
  & a(f_\infty,\gamma)~ { b^2(f_\infty,\gamma) \over \gamma } ~u_2(1)
  + \bigl[ f_\infty + a(f_\infty,\gamma) \bigr] ~u_4(1) 
  + u_5(1) = 0, \\
 & u_1(1) = f_\infty,
 \end{array} \end{equation}
where
\begin{equation} \begin{array}{cl}
 & a(f_\infty,\gamma) = {1 \over \sqrt{2} }
  \bigl[ (f_\infty^4 + 4 \gamma^2)^{1/2} + f_\infty^2 \bigr]^{1/2}, \\
 & b(f_\infty,\gamma) = {1 \over \sqrt{2} }
  \bigl[ (f_\infty^4 + 4 \gamma^2)^{1/2} - f_\infty^2 \bigr]^{1/2}.  \\
\end{array} \end{equation}
Note that there are five differential equations and six boundary conditions.
Correspondingly, there are two free parameters in the computation of a 
solution branch, namely $\gamma$ and $f_\infty$.
The ``period'' $T$ is fixed; $T=500$.
The starting solution is $u_i=0$, $i=1,\cdots,5$, 
at $\gamma=1$, $f_\infty=0$.

\begin{table}[htbp]
\begin{center}
\begin{tabular}{| l | l |}
\hline
  \AUTO-COMMAND  & ACTION \\
\hline
%==============================================================================
  \commandf{ ! mkdir kar} & create an empty work directory \\ 
  \commandf{ cd kar} & change directory \\
  \commandf{ demo('kar')} & copy the demo files to the work directory \\
\hline
%==============================================================================
  \commandf{ run(c='kar.1')} & computation of the solution branch \\ 
  \commandf{ sv('kar')} & save output-files as \filef{ b.kar, s.kar, d.kar} \\ 
\hline
%==============================================================================
\end{tabular}
\caption{Commands for running demo \filef{ kar}.}
\label{tbl:demo_kar}
\end{center}
\end{table}

\newpage
%==============================================================================
%DEMO=spb======================================================================
%==============================================================================
\section{ spb : A Singularly-Perturbed BVP.} \label{sec:Demos_spb}
This demo illustrates the use of continuation to compute 
solutions to the singularly perturbed boundary value problem
\begin{equation} \begin{array}{cl}
  u_1 ' &= u_2  ,  \\
  u_2 ' &= {\lambda \over \eps} \bigl(
  u_1 u_2 (u_1^2 - 1) + u_1
  \bigr)  , \\ \end{array} \end{equation}
with boundary conditions $u_1(0)=3/2$,  $u_1(1)=\gamma.$
The parameter $\lambda$ has been introduced into the equations in order
to allow a homotopy from a simple equation with known exact solution
to the actual equation. This is done in the first run.
In the second run $\eps$ is decreased by continuation.
In the third run $\eps$ is fixed at $\eps=.001$ and the solution is continued 
in $\gamma$.
This run takes more than 1500 continuation steps.
For a detailed analysis of the solution behavior see 
\citename{JL:82} \citeyear{JL:82}.
\begin{table}[htbp]
\begin{center}
\begin{tabular}{| l | l |}
\hline
  \AUTO-COMMAND  & ACTION \\
\hline
%==============================================================================
  \commandf{ ! mkdir spb} & create an empty work directory \\ 
  \commandf{ cd spb} & change directory \\
  \commandf{ demo('spb')} & copy the demo files to the work directory \\
\hline
%==============================================================================
  \commandf{ run(c='spb.1')} & 1st run; homotopy from $\lambda=0$ to $\lambda=1$ \\ 
  \commandf{ sv('1')} & save output-files as \filef{ b.1, s.1, d.1} \\ 
\hline
%==============================================================================
  \commandf{ run(c='spb.2',s='1')} & \parbox[t]{3in}{2nd run; let $\eps$ tend to zero; restart from \filef{ s.1}.  constants changed : \parf{ IRS, ICP(1), NTST, DS} \vspace{0.2cm}}\\ 
  \commandf{ sv('2')} & save the output-files as \filef{ b.2, s.2, d.2} \\ 
\hline
%==============================================================================
  \commandf{ run(c='spb.3',s='2')} & \parbox[t]{3in}{3rd run; continuation in $\gamma$; $\eps=0.001$; restart from \filef{ s.2}.  Constants changed : \parf{ IRS, ICP(1), RL0, ITNW, EPSL, EPSU, NUZR} \vspace{0.2cm}} \\ 
  \commandf{ sv('3')} & save the output-files as \filef{ b.3, s.3, d.3} \\ 
\hline
%==============================================================================
\end{tabular}
\caption{Commands for running demo \filef{ spb}.}
\label{tbl:demo_spb}
\end{center}
\end{table}

\newpage
%==============================================================================
%DEMO=ezp======================================================================
%==============================================================================
\section{ ezp : Complex Bifurcation in a BVP.} \label{sec:Demos_ezp}
This demo illustrates the computation of a solution branch to
the the complex boundary value problem

\begin{equation} \begin{array}{cl}
  u_1 ' &= u_2  ,  \\
  u_2 ' &= -p_1  e^{u_1} , \\
\end{array} \end{equation}
with boundary conditions $ u_1(0)=0 , ~u_1(1)=0.$
Here $u_1$ and $u_2$ are allowed to be complex, 
while the parameter $p_1$ can only take real values.
In the real case, this is Bratu's equation, whose solution branch 
contains a fold; see the demo \filef{ exp}.
It is known 
(\citename{HeKe:90} \citeyear{HeKe:90}) that a simple quadratic fold gives rise to a pitch fork
bifurcation in the complex equation.
This bifurcation is located in the first computation below.
In the second and third run, both legs of the bifurcating solution branch
are computed.
On it, both solution components $u_1$ and $u_2$ have nontrivial 
imaginary part.



\begin{table}[htbp]
\begin{center}
\begin{tabular}{| l | l |}
\hline
  \AUTO-COMMAND  & ACTION \\
\hline
%==============================================================================
  \commandf{ ! mkdir ezp} & create an empty work directory \\ 
  \commandf{ cd ezp} & change directory \\
  \commandf{ demo('ezp')} & copy the demo files to the work directory \\
\hline
%==============================================================================
  \commandf{ run(c='ezp.1')} & 1st run; compute solution branch containing fold \\ 
  \commandf{ sv('ezp')} & save output-files as \filef{ p.ezp, s.ezp, d.ezp} \\ 
\hline
%==============================================================================
  \commandf{ run(c='ezp.2',s='ezp')} & \parbox[t]{3in}{2nd run; compute bifurcating complex solution branch.  Constants changed : \parf{ IRS, ISW} \vspace{0.2cm}}\\ 
  \commandf{ ap('ezp')} & append output-files to \filef{ p.ezp, s.ezp, d.ezp} \\ 
\hline
%==============================================================================
  \commandf{ run(c='ezp.3',s='ezp')} & \parbox[t]{3in}{3rd run; compute 2nd leg of bifurcating branch.  constant changed : \parf{ DS} \vspace{0.2cm}}\\ 
  \commandf{ ap('ezp')} & append output-files to \filef{ p.ezp, s.ezp, d.ezp} \\
\hline
%==============================================================================
\end{tabular}
\caption{Commands for running demo \filef{ ezp}.}
\label{tbl:demo_ezp}
\end{center}
\end{table}
%==============================================================================
