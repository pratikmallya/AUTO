
%==============================================================================
%==============================================================================
\chapter{ \AUTO Demos : Parabolic PDEs.} \label{ch:Demos_PDE}
%==============================================================================
%==============================================================================

\newpage
%==============================================================================
%DEMO=pd1======================================================================
%==============================================================================
\section{ pd1 : Stationary States (1D Problem).} \label{sec:Demos_pd1}
This demo uses Euler's method to locate a stationary solution of
a nonlinear parabolic PDE, followed by continuation of this stationary
state in a free problem parameter. The equation is
 $$ {\partial u \over \partial t} 
  = D~{\partial^2 u \over \partial x^2} ~+~  p_1~ u ~( 1-u) , $$
on the space interval $[0,L]$, where $L=$~\parf{ PAR(11)}~$=10$ is fixed throughout,
as is the diffusion constant $D=$~\parf{ PAR(15)}~$=0.1$.
The boundary conditions are $u(0) = u(L) = 0$ for all time.

In the first run the continuation parameter is the independent time variable,
namely \parf{ PAR(14)}, while $p_1=1$ is fixed.
The \AUTO-constants \parf{ DS}, \parf{ DSMIN}, and \parf{ DSMAX} then control the step size
in space-time, here consisting of \parf{ PAR(14)} and  $u(x)$.
Initial data are $u(x)=\sin(\pi x/L)$ at time zero.
Note that in the subroutine \funcf{ stpnt} the initial data must be scaled to 
the unit interval, and that the scaled derivative must also be provided; 
see the equations-file \filef{ pv1.c}.
In the second run the continuation parameter is $p_1$.

Euler time integration is only first order accurate, so that
the time step must be sufficiently small to ensure correct results.
Indeed, this option has been added only as a convenience, and should 
generally be used only to locate stationary states.

\begin{table}[htbp]
\begin{center}
\begin{tabular}{| l | l |}
\hline
  \AUTO-COMMAND  & ACTION \\
\hline
%==============================================================================
  \commandf{ ! mkdir pd1} & create an empty work directory \\ 
  \commandf{ cd pd1} & change directory \\
  \commandf{ demo('pd1') } & copy the demo files to the work directory \\
\hline
%==============================================================================
  \commandf{ run(c='pd1.1') } & time integration towards stationary state \\ 
  \commandf{ sv('1') } & save output-files as \filef{ b.1, s.1, d.1} \\ 
\hline
%==============================================================================
  \commandf{ run(c='pd1.2',s='1')} & \parbox[t]{3in}{continuation of stationary states; read restart data from \filef{ s.1}. constants changed : \parf{ IPS, IRS, ICP}, etc.\vspace{0.2cm}} \\ 
  \commandf{ sv('2') } & save output-files as \filef{ b.2, s.2, d.2} \\ 
\hline
%==============================================================================
\end{tabular}
\caption{Commands for running demo \filef{ pd1}.}
\label{tbl:demo_pd1}
\end{center}
\end{table}

\newpage
%==============================================================================
%DEMO=pd2======================================================================
%==============================================================================
\section{ pd2 : Stationary States (2D Problem).} \label{sec:Demos_pd2}
This demo uses Euler's method to locate a stationary solution of
a nonlinear parabolic PDE, followed by continuation of this stationary
state in a free problem parameter. The equations are
\begin{equation} \begin{array}{cl}
  {\partial u_1 / \partial t} &= D_1~{\partial^2 u_1 / \partial x^2}
  ~+~  p_1~ u ~( 1-u) ~-~ u_1 u_2 , \\
  {\partial u_2 / \partial t} 
  &= D_2~{\partial^2 u_2 / \partial x^2} ~-~ u_2 ~+~ u_1 u_2 , \\
\end{array} \end{equation}
on the space interval $[0,L]$, where $L=$~\parf{ PAR(11)}~$=1$ is fixed throughout,
as are the diffusion constants $D_1=$~\parf{ PAR(15)}~$=1$ and $D_2=$~\parf{ PAR(16)}~$=1$.
The boundary conditions are $u_1(0) = u_1(L) = 0$ and $u_2(0) = u_2(L) = 1$,
for all time.

In the first run the continuation parameter is the independent time variable,
namely \parf{ PAR(14)}, while $p_1=12$ is fixed.
The \AUTO-constants \parf{ DS}, \parf{ DSMIN}, and \parf{ DSMAX} then control the step size
in space-time, here consisting of \parf{ PAR(14)} and $(u_1(x),u_2(x))$.
Initial data at time zero are $u_1(x)=\sin(\pi x/L)$ and $u_2(x)=1$.
Note that in the subroutine \funcf{ stpnt} the initial data must be scaled to 
the unit interval, and that the scaled derivatives must also be provided; 
see the equations-file \filef{ pv2.c}.
In the second run the continuation parameter is $p_1$.
A branch point is located during this run.

Euler time integration is only first order accurate, so that
the time step must be sufficiently small to ensure correct results.
Indeed, this option has been added only as a convenience, and should 
generally be used only to locate stationary states.


\begin{table}[htbp]
\begin{center}
\begin{tabular}{| l | l |}
\hline
  \AUTO-COMMAND  & ACTION \\
\hline
%==============================================================================
  \commandf{ ! mkdir pd2} & create an empty work directory \\ 
  \commandf{ cd pd2} & change directory \\
  \commandf{ demo('pd2') } & copy the demo files to the work directory \\
\hline
%==============================================================================
  \commandf{ run(c='pd2.1') } & time integration towards stationary state \\ 
  \commandf{ sv('1') } & save output-files as \filef{ b.1, s.1, d.1} \\ 
\hline
%==============================================================================
  \commandf{ run(c='pd2.2',s='1')} & \parbox[t]{3in}{continuation of stationary states; read restart data from \filef{ s.1}.  constants changed : \parf{ IPS, IRS, ICP}, etc. \vspace{0.2cm}}\\ 
  \commandf{ sv('2') } & save output-files as \filef{ b.2, s.2, d.2} \\ 
\hline
%==============================================================================
\end{tabular}
\caption{Commands for running demo \filef{ pd2}.}
\label{tbl:demo_pd2}
\end{center}
\end{table}

\newpage
%==============================================================================
%DEMO=wav======================================================================
%==============================================================================
\section{ wav : Periodic Waves.} \label{sec:Demos_wav}
This demo illustrates the computation of various periodic wave solutions
to a system of coupled parabolic partial differential equations
on the spatial interval $[0,1]$.
The equations, that model an enzyme catalyzed reaction 
(\citename{DoKe:86a} \citeyear{DoKe:86a}) are~:
\begin{equation} \begin{array}{cl}
 {\partial u_1 / \partial t}
  &=
  ~{\partial^{2} u_1 / \partial x^{2}}
  -p_1 \bigl[p_4 R(u_1,u_2) - (p_2 - u_1) \bigr] ,\\
 {\partial u_2 / \partial t}
  &=
  \beta {\partial^{2} u_2 / \partial x^{2}}
  -p_1 \bigl[p_4 R(u_1,u_2) - p_7 (p_3 - u_2) \bigr].\\
\end{array} \end{equation}
All equation parameters, except $p_3$, are fixed throughout.

\begin{table}[htbp]
\begin{center}
\begin{tabular}{| l | l |}
\hline
  \AUTO-COMMAND  & ACTION \\
\hline
%==============================================================================
  \commandf{ ! mkdir wav} & create an empty work directory \\ 
  \commandf{ cd wav} & change directory \\
  \commandf{ demo('wav')} & copy the demo files to the work directory \\
\hline
%==============================================================================
  \commandf{ run(c='wav.1') } & 1st run; stationary solutions of the system without diffusion \\ 
  \commandf{ sv('ode') } & save output-files as \filef{ b.ode, s.ode, d.ode} \\ 
\hline
%==============================================================================
  \commandf{ cp c.wav.2 c.wav} & constants changed : \parf{ IPS} \\ 
  \commandf{ run(c='wav.2',s='wav') } & \parbox[t]{3in}{2nd run; detect bifurcations to wave train solutions.  Constants changed : \parf{ IPS} \vspace{0.2cm}}\\ 
  \commandf{ sv('wav') } & save output-files as \filef{ b.wav, s.wav, d.wav} \\ 
\hline
%==============================================================================
  \commandf{ run(c='wav.3',s='wav') } & \parbox[t]{3in}{3rd run; wave train solutions of fixed wave speed.  Constants changed : \parf{ IRS, IPS, NUZR, ILP} \vspace{0.2cm}}\\ 
  \commandf{ ap('wav') } & append output-files to \filef{ b.wav, s.wav, d.wav} \\ 
\hline
%==============================================================================
  \commandf{ run(c='wav.4',s='wav') } & \parbox[t]{3in}{4th run; wave train solutions of fixed wave length.  Constants changed : \parf{ IRS, IPS, NMX, ICP, NUZR} \vspace{0.2cm}}\\ 
  \commandf{ sv('rng') } & save output-files as \filef{ b.rng, s.rng, d.rng} \\ 
\hline
%==============================================================================
  \commandf{ run(c='wav.5',s='wav') } & \parbox[t]{3in}{5th run; time evolution computation.  Constants changed : \parf{ IPS, NMX, NPR, ICP} \vspace{0.2cm}}\\ 
  \commandf{ sv('tim') } & save output-files as \filef{ b.tim, s.tim, d.tim} \\ 
\hline
%==============================================================================
\end{tabular}
\caption{Commands for running demo \filef{ wav}.}
\label{tbl:demo_wav}
\end{center}
\end{table}

\newpage
%==============================================================================
%DEMO=brc======================================================================
%==============================================================================
\section{ brc : Chebyshev Collocation in Space.} \label{sec:Demos_brc}
This demo illustrates the computation of stationary solutions and periodic
solutions to systems of parabolic PDEs in one space variable,
using Chebyshev collocation in space.
More precisely, the approximate solution is assumed of the form
$u(x,t) = \sum_{k=0}^{n+1} u_k(t) \ell_k(x)$.
Here $u_k(t)$ corresponds to $u(x_k,t)$ at the Chebyshev points
$\bigl\{ x_k \bigr\}_{k=1}^{n}$ with respect to the interval $[0,1]$.
The polynomials $\bigl\{ \ell_k(x) \bigr\}_{k=0}^{n+1}$ are the Lagrange
interpolating coefficients with respect to points 
$\bigl\{ x_k \bigr\}_{k=0}^{n+1}$, where $x_0=0$ and $x_{n+1}=1$.
The number of Chebyshev points in $[0,1]$,
as well as the number of equations in the PDE system,
can be set by the user in the file \filef{ brc.inc}.

As an illustrative application we consider the Brusselator
(\citename{HoKnKu:87} \citeyear{HoKnKu:87})
\begin{equation} \begin{array}{cl}
  u_t &= {D_x / L^2} u_{xx} + u^2v - (B+1)u + A,  \\
  v_t &= {D_y / L^2} v_{xx} - u^2v + Bu,  \\
\end{array} \end{equation}
with boundary conditions $u(0,t)=u(1,t)=A$
and $v(0,t)=v(1,t)=B/A$.

Note that, given the non-adaptive spatial discretization,
the computational procedure here is not appropriate for
PDEs with solutions that rapidly vary in space, and care must
be taken to recognize spurious solutions and bifurcations.

\begin{table}[htbp]
\begin{center}
\begin{tabular}{| l | l |}
\hline
  \AUTO-COMMAND  & ACTION \\
\hline
%==============================================================================
  \commandf{ ! mkdir brc} & create an empty work directory \\ 
  \commandf{ cd brc} & change directory \\
  \commandf{ demo('brc')} & copy the demo files to the work directory \\
\hline
%==============================================================================
  \commandf{ run(c='brc.1') } & compute the stationary solution branch with Hopf bifurcations \\ 
  \commandf{ sv('brc') } & save output-files as \filef{ b.brc, s.brc, d.brc} \\ 
\hline
%==============================================================================
  \commandf{ run(c='brc.2',s='brc') } & \parbox[t]{3in}{compute a branch of periodic solutions from the first Hopf point.  Constants changed : \parf{ IRS, IPS} \vspace{0.2cm}}\\ 
  \commandf{ ap('brc') } & append the output-files to \filef{ b.brc, s.brc, d.brc} \\ 
\hline
%==============================================================================
  \commandf{ run(c='brc.3',s='brc') } & \parbox[t]{3in}{compute a solution branch from a secondary periodic bifurcation.  Constants changed : \parf{ IRS, ISW} \vspace{0.2cm}}\\ 
  \commandf{ ap('brc') } & append the output-files to \filef{ b.brc, s.brc, d.brc} \\ 
\hline
%==============================================================================
\end{tabular}
\caption{Commands for running demo \filef{ brc}.}
\label{tbl:demo_brc}
\end{center}
\end{table}


\newpage
%==============================================================================
%DEMO=brf======================================================================
%==============================================================================
\section{ brf : Finite Differences in Space.} \label{sec:Demos_brf}
This demo illustrates the computation of stationary solutions and periodic
solutions to systems of parabolic PDEs in one space variable.
A fourth order accurate finite difference approximation is used to
approximate the second order space derivatives. 
This reduces the PDE to an autonomous ODE of fixed dimension
which \AUTO is capable of treating.
The spatial mesh is uniform; the number of mesh intervals,
as well as the number of equations in the PDE system,
can be set by the user in the file \filef{ brf.inc}.

As an illustrative application we consider the Brusselator
(\citename{HoKnKu:87} \citeyear{HoKnKu:87})
\begin{equation} \begin{array}{cl}
  u_t &= {D_x / L^2} u_{xx} + u^2v - (B+1)u + A,  \\
  v_t &= {D_y / L^2} v_{xx} - u^2v + Bu,  \\
\end{array} \end{equation}
with boundary conditions $u(0,t)=u(1,t)=A$
and $v(0,t)=v(1,t)=B/A$.

Note that, given the non-adaptive spatial discretization,
the computational procedure here is not appropriate for
PDEs with solutions that rapidly vary in space, and care must
be taken to recognize spurious solutions and bifurcations.


\begin{table}[htbp]
\begin{center}
\begin{tabular}{| l | l |}
\hline
  \AUTO-COMMAND  & ACTION \\
\hline
%==============================================================================
  \commandf{ ! mkdir brf} & create an empty work directory \\ 
  \commandf{ cd brf} & change directory \\
  \commandf{ demo('brf') } & copy the demo files to the work directory \\
\hline
%==============================================================================
  \commandf{ run(c='brf.1') } & compute the stationary solution branch with Hopf bifurcations \\ 
  \commandf{ sv('brf') } & save output-files as \filef{ b.brf, s.brf, d.brf} \\ 
\hline
%==============================================================================
  \commandf{ run(c='brf.2',s='brf') } & \parbox[t]{3in}{compute a branch of periodic solutions from the first Hopf point.  Constants changed : \parf{ IRS, IPS}  \vspace{0.2cm}}\\ 
  \commandf{ ap('brf') } & append the output-files to \filef{ b.brf, s.brf, d.brf} \\ 
\hline
%==============================================================================
  \commandf{ run(c='brf.3',s='brf') } & \parbox[t]{3in}{compute a solution branch from a secondary periodic bifurcation.  Constants changed : \parf{ IRS, ISW} \vspace{0.2cm}}\\ 
  \commandf{ ap('brf') } & append the output-files to \filef{ b.brf, s.brf, d.brf} \\ 
\hline
%==============================================================================
\end{tabular}
\caption{Commands for running demo \filef{ brf}.}
\label{tbl:demo_brf}
\end{center}
\end{table}

\newpage
%==============================================================================
%DEMO=bru======================================================================
%==============================================================================
\section{ bru : Euler Time Integration (the Brusselator).} \label{sec:Demos_bru}
This demo illustrates the use of Euler's method for time integration
of a nonlinear parabolic PDE.
The example is the Brusselator
(\citename{HoKnKu:87} \citeyear{HoKnKu:87}), given by
\begin{equation} \begin{array}{cl}
  u_t &= {D_x / L^2} u_{xx} + u^2v - (B+1)u + A,  \\
  v_t &= {D_y / L^2} v_{xx} - u^2v + Bu,  \\
\end{array} \end{equation}
with boundary conditions $u(0,t)=u(1,t)=A$
and $v(0,t)=v(1,t)=B/A$. All parameters are given fixed values
for which a stable periodic solution is known to exist.

The continuation parameter is the independent time variable,
namely \parf{ PAR(14)}.
The \AUTO-constants \parf{ DS}, \parf{ DSMIN}, and \parf{ DSMAX}
then control the step size
in space-time, here consisting of \parf{ PAR(14)} and $(u(x),v(x))$.
Initial data at time zero are 
$u(x)=A - 0.5 \sin(\pi x)$ and 
$v(x)=B/A + 0.7 \sin(\pi x)$.
Note that in the subroutine \funcf{ stpnt} the space derivatives of $u$ and $v$
must also be provided; 
see the equations-file \filef{ bru.c}.

Euler time integration is only first order accurate, so that
the time step must be sufficiently small to ensure correct results.
This option has been added only as a convenience, and should 
generally be used only to locate stationary states.
Indeed, in the case of the asymptotic periodic state of this demo,
the number of required steps is very large and use of a better time 
integrator is advisable.


\begin{table}[htbp]
\begin{center}
\begin{tabular}{| l | l |}
\hline
  \AUTO-COMMAND  & ACTION \\
\hline
%==============================================================================
  \commandf{ ! mkdir bru} & create an empty work directory \\ 
  \commandf{ cd bru} & change directory \\
  \commandf{ demo('bru') } & copy the demo files to the work directory \\
\hline
%==============================================================================
  \commandf{ run(c='bru.1') } & time integration \\ 
  \commandf{ sv('bru') } & save output-files as \filef{ b.bru, s.bru, d.bru} \\ 
\hline
%==============================================================================
\end{tabular}
\caption{Commands for running demo \filef{ bru}.}
\label{tbl:demo_bru}
\end{center}
\end{table}


