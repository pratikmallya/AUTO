%==============================================================================
%==============================================================================
\chapter{ Notes on Using \AUTO.}  \label{ch:Notes_on_Using_AUTO}
%==============================================================================
%==============================================================================
\section{ Restrictions on the Use of \parf{ PAR}.} \label{sec:Restrictions_on_PAR}
The array \parf{ PAR} in the user-supplied subroutines is available
for equation parameters that the user wants to vary at some point
in the computations.
In any particular computation the free parameter(s) must be designated
in \parf{ ICP}; see Section~\ref{sec:Free_parameters}.
The following restrictions apply~:

\begin{itemize}
\item[-]
  The maximum number of parameters, \parf{ NPARX} in \filef{ auto/2000/src/auto\_c.h},
  has pre-defined value \parf{ NPARX}=36.  \parf{ NPARX} should not normally be increased
  and it should never be decreased.
  Any increase of \parf{ NPARX} must be followed by recompilation of \AUTO.
\item[-]
  Generally one should only use \parf{ PAR(1)-PAR(9)} for equation parameters,
  as \AUTO may need the other components internally.  
\end{itemize}

\section{ Efficiency.} \label{sec:Efficiency}
In \AUTO, efficiency has at times been sacrificed for generality of programming.
This applies in particular to computations in which \AUTO generates
an extended system, for example, computations with \parf{ ISW}=2.
However, the user has significant control over computational efficiency,
in particular through judicious choice of the \AUTO-constants  
\parf{ DS}, \parf{ DSMIN}, and \parf{ DSMAX}, and, for ODEs, \parf{ NTST} and \parf{ NCOL}.
Initial experimentation normally suggests appropriate values.

Slowly varying solutions to ODEs can often 
be computed with remarkably small values of \parf{ NTST} and \parf{ NCOL}, 
for example, \parf{ NTST}=5,  \parf{ NCOL}=2.
Generally, however, it is recommended to set \parf{ NCOL}=4,
and then to use the ``smallest'' value of \parf{ NTST} that maintains convergence.

The choice of the pseudo-arclength stepsize parameters
\parf{ DS}, \parf{ DSMIN}, and \parf{ DSMAX}
is highly problem dependent.
Generally, \parf{ DSMIN} should not be taken too small,
in order to prevent excessive step refinement in case of non-convergence.
It should also not be too large, in order to avoid instant non-convergence.
\parf{ DSMAX} should be sufficiently large, in order to reduce computation time
and amount of output data.
On the other hand, it should be sufficiently small, in order to prevent
stepping over bifurcations without detecting them.
For a given equation, appropriate values of these constants 
can normally be found after some initial experimentation.

The constants \parf{ ITNW}, \parf{ NWTN}, \parf{ THL}, \parf{ EPSU}, \parf{ EPSL}, \parf{ EPSS} 
also affect efficiency.
Understanding their significance is therefore useful; 
see Section~\ref{sec:Tolerances} and Section~\ref{sec:step_size}.
Finally, it is recommended that initial computations be done with 
\parf{ ILP=0}; no fold detection;
and \parf{ ISP}=1; no bifurcation detection for ODEs.
 
\section{ Correctness of Results.} \label{sec:Correctness}
\AUTO-computed solutions to ODEs are almost always structurally correct,
because the mesh adaption strategy, if \parf{ IAD}$>$0, safeguards to some extent
against spurious solutions.
If these do occur, possibly near infinite-period orbits,
the unusual appearance of the solution branch typically serves as a warning.
Repeating the computation with increased \parf{ NTST} is then recommended.

\section{ Bifurcation Points and Folds.} \label{sec:Bifurcations}
It is recommended that the detection of folds 
and bifurcation points be initially disabled.
For example, if an equation has a ``vertical'' solution branch
then \AUTO may try to locate one fold after another.

Generally, degenerate bifurcations cannot be detected.
Furthermore, bifurcations that are close to each other may not
be noticed when the pseudo-arclength step size is not sufficiently small.
Hopf bifurcation points may go unnoticed if no clear crossing of
the imaginary axis takes place. This may happen when there are other
real or complex eigenvalues near the imaginary axis and when 
the pseudo-arclength step is large compared to the rate
of change of the critical eigenvalue pair.
A typical  case is a Hopf bifurcation close to a fold.
Similarly, Hopf bifurcations may go undetected if switching from
real to complex conjugate, followed by crossing of the imaginary
axis, occurs rapidly with respect to the pseudo-arclength step size.
Secondary periodic bifurcations may not be detected for similar reasons.
In case of doubt, carefully inspect the contents of the diagnostics file
\filef{ fort.9}.
 
\section{ Floquet Multipliers.} \label{sec:Floquet_multipliers}

\AUTO extracts an approximation to the linearized Poincar\'e map from 
the Jacobian of the linearized collocation system that arises in Newton's method.
This procedure is very efficient; the map is computed at negligible extra cost.
The linear equations solver of \AUTO is described in 
\citename{DoKeKe:91b} \citeyear{DoKeKe:91b}.
The actual Floquet multiplier solver was written by
\citename{Fa:94} \citeyear{Fa:94}.
For a detailed description of the algorithm see 
\citename{FaJe:91} \citeyear{FaJe:91}.

For periodic solutions, the exact linearized Poincar\'e map always has 
a multiplier $z=1$.
A good accuracy check is to inspect this 
multiplier in the diagnostics output-file \filef{ fort.9}.
If this multiplier becomes inaccurate then the automatic detection 
of potential secondary periodic bifurcations (if \parf{ ISP}=2) is discontinued 
and a warning is printed in \filef{ fort.9}.
It is strongly recommended that the contents of this file be habitually inspected,
in particular to verify whether solutions labeled as BP or TR 
(cf.~Table~\ref{tbl:Solution_Types}) have indeed  been correctly classified.
 
\section{ Memory Requirements.} \label{sec:Memory_requirements}
Pre-defined maximum values of certain \AUTO-constants
are in \filef{ auto/2000/src/auto\_c.h}; 
see also Section~\ref{sec:Restrictions}. 
These maxima affect the run-time memory requirements
and should not be set to unnecessarily large values.
If an application only solves algebraic systems and if \parf{ NDIM} is ``large''
then memory requirements can be much reduced by setting each of
\parf{ NTSTX}, \parf{ NCOLX}, \parf{ NBCX}, \parf{ NINTX}, 
equal to $1$ in \filef{ auto/2000/src/auto\_c.h},
followed by recompilation of the \AUTO libraries.

