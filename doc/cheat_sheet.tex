
\documentclass[12pt]{report}

\usepackage{epsf}
\usepackage{include/harvard}
\usepackage{longtable}
\usepackage{float}
\usepackage{xr}
\externaldocument{auto}
\pagestyle{plain}
\topmargin=-45pt
\textwidth=6.85in
\textheight=9in
\evensidemargin=-15pt
\oddsidemargin=-15pt
%
\newcommand{\autoversion}{0.9.9}

\newcommand{\ds}{\displaystyle}
\newcommand{\beq}{\begin{equation}} 
\newcommand{\eeq}{\end{equation}}
\def\norm#1{\parallel#1\parallel}
\def\abs#1{\mid#1\mid}
\def\eps{\epsilon}
\def\R{{\rm R}}
\def\Rn{{\rm R}^n}

\newcommand{\AUTO}{{\cal AUTO}~}
\newcommand{\AUTOc}{\AUTO 2000~}
\newcommand{\AUTOold}{\AUTO 97~}
\newcommand{\AUTOolder}{\AUTO 94~}

\newcommand{\python}{{\cal Python}}

\newcommand{\emp}{\emph}

% A font for AUTO parameters
%\newcommand{\parf}[1]{\filef{#1}}
\newcommand{\parf}[1]{{\tt #1}}

% A font for UNIX commands
%\newcommand{\commandf}[1]{{\it #1}}
\newcommand{\commandf}[1]{{\tt #1}}

% A font for environment variables
%\newcommand{\envf}[1]{\filef{#1}}
\newcommand{\envf}[1]{{\it #1}}

% A font for function names
%\newcommand{\funcf}[1]{\filef{#1}}
\newcommand{\funcf}[1]{{\bf #1}}

% A font for filenames
%\newcommand{\filef}[1]{{\tt #1}}
\newcommand{\filef}[1]{{\sf #1}}

% A font for ftp sites and web pages
%\newcommand{\webf}[1]{\commandf{#1}}
\newcommand{\webf}[1]{{\sl #1}}

% A float type for examples
\newfloat{Example}{htb}{exa}


\begin{document}
{\footnotesize
\begin{tabular}{|l|p{5in}|}
\hline 
A0 & 
 The lower bound on the principal solution measure.
\\ \hline 
A1 & 
 The upper bound on the principal solution measure.
\\ \hline 
DS & 
 The constant DS defines the pseudoarclength stepsize to be 
 used for the first attempted step along any branch.
\\ \hline 
DSMAX & 
 The maximum allowable absolute value of the pseudoarclength stepsize.
\\ \hline 
DSMIN & 
 This is minimum allowable absolute value of the pseudoarclength step
 size. 
\\ \hline 
EPSL & 
 Relative convergence criterion for equations parameters in 
 the Newton/Chord method.
\\ \hline 
EPSS & 
 Relative arclength convergence criterion for the detection of special 
 solutions. 
\\ \hline 
EPSU & 
 Relative convergence criterion for solution components in the 
 Newton/Chord method.
\\ \hline 
IAD & 
 This constant controls the mesh adaption.
\\ \hline 
IADS & 
 This constant controls the frequency of adaption of the pseudoarclength
 stepsize.
\\ \hline 
ICP & 
 ICP designates the free parameters.
\\ \hline 
IID & 
 This constant controls the amount of diagnostic output printed in fort.9.
\\ \hline 
ILP & 
 This constant controls the detection of fold.
\\ \hline 
IPLT & 
 This constant allows redefinition of the principal solution measure.
\\ \hline 
IPS & 
 This constant defines the problem type.
\\ \hline 
IRS & 
 This constant sets the label of the solution where the computation is to
 be restarted.
\\ \hline 
ISP & 
 This constant controls the detection of branch points, perioddoubling
 bifurcations, and torus bifurcations.
\\ \hline 
ISW & 
 This constant controls branch switching at branch points for the case of
 differential equations.
\\ \hline 
ITMX & 
 The maximum number of iterations allowed in the accurate location of
 special solutions.
\\ \hline 
ITNW & 
 The maximum number of combined NewtonChord iterations. 
\\ \hline 
JAC & 
 Used to indicate whether derivatives are supplied by the user or to be
 obtained by differencing.
\\ \hline 
MXBF & 
 This constant, which is effective for algebraic problems only, sets the
 maximum number of bifurcations to be treated.
\\ \hline 
NBC & 
 The number of boundary conditions as specified in the user-supplied 
 subroutine bcnd.
\\ \hline 
NCOL & 
 The number of Gauss collocation points per mesh interval, (2 < NCOL < 7).
\\ \hline 
NDIM & 
 Dimension of the system of equations as specified in the user-supplied
 subroutine func.
\\ \hline 
NICP & 
 The constant NICP indicates how many free parameters have been specified.
\\ \hline 
NINT & 
 The number of integral conditions as specified in the user-supplied sub
 routine icnd.
\\ \hline 
NMX & 
 The maximum number of steps to be taken along any branch.
\\ \hline 
NPR & 
 This constant can be used to regularly write fort.8 plotting and restart
 data. 
\\ \hline 
NTHL & 
 NTHL defines the number of parameters whose weight is to be modified.
\\ \hline 
NTHU & 
 NTHU defines the number of states whose weight is to be modified.
\\ \hline 
NTST & 
 The number of mesh intervals used for discretization. 
\\ \hline 
NUZR & 

\\ \hline 
NWTN & 
 After NWTN Newton iterations the Jacobian is frozen.
\\ \hline 
RL0 & 
 The lower bound on the principal continuation parameter.
\\ \hline 
RL1 & 
 The upper bound on the principal continuation parameter.
\\ \hline 
\end{tabular}


\pagebreak

\begin{longtable}{|l|p{5in}|}
\hline 
A0 & 

 The lower bound on the principal solution measure.
 (By default, if \parf{ IPLT}=0, the principal solution measure
 is the $L_2$-norm of the state vector or state vector function.
 See the \AUTO-constant \parf{ IPLT} in Section~\ref{sec:IPLT} 
 for choosing another principal solution measure.)
\\ \hline 
A1 & 

 The upper bound on the principal solution measure.
\\ \hline 
DS & 

 \AUTO uses pseudo-arclength continuation for following solution branches.
 The pseudo-arclength stepsize is the distance between
 the current solution and the next solution on a branch.
 By default, this distance includes all state variables
 (or state functions) and all free parameters.
 The constant \parf{ DS} defines the pseudo-arclength stepsize to be used for the
 first attempted step along any branch. 
 (Note that if \parf{ IADS}$>$0 then \parf{ DS} will automatically be adapted
 for subsequent steps and for failed steps.)
 \parf{ DS} may be chosen positive or negative; changing its sign 
 reverses the direction of computation.
 The relation \parf{ DSMIN} $\le$ $\abs \parf{ DS}$ $\le$ \parf{ DSMAX} must be satisfied.
 The precise choice of \parf{ DS} is problem-dependent; the demos use a value 
 that was found appropriate after some experimentation.
\\ \hline 
DSMAX & 

 The maximum allowable absolute value of the pseudo-arclength stepsize.
 \parf{ DSMAX} must be positive.
 It is only effective if the pseudo-arclength step is adaptive,
 i.e., if \parf{ IADS}$>$0.
 The choice of \parf{ DSMAX} is highly problem-dependent; 
 most demos use a value that was found appropriate after some
 experimentation.
 See also the discussion in Section~\ref{sec:Efficiency}.
\\ \hline 
DSMIN & 

 This is minimum allowable absolute value of the pseudo-arclength 
 stepsize. \parf{ DSMIN} must be positive.
 It is only effective if the pseudo-arclength step is adaptive,
 i.e., if \parf{ IADS}$>$0.
 The choice of \parf{ DSMIN} is highly problem-dependent;
 most demos use a value that was found appropriate after some
 experimentation.
 See also the discussion in Section~\ref{sec:Efficiency}.
\\ \hline 
EPSL & 

 Relative convergence criterion for equation parameters in the Newton/Chord 
 method. Most demos use \parf{ EPSL}=$10^{-6}$ or \parf{ EPSL}=$10^{-7}$,
 which is the recommended value range.
\\ \hline 
EPSS & 

 Relative arclength convergence criterion for the detection of special solutions. 
 Most demos use \parf{ EPSS}=$10^{-4}$ or  \parf{ EPSS}=$10^{-5}$,
 which is the recommended value range.
 Generally, \parf{ EPSS} should be approximately 100 to 1000 times the value
 of \parf{ EPSL}, \parf{ EPSU}.
\\ \hline 
EPSU & 

 Relative convergence criterion for solution components in the Newton/Chord 
 method. Most demos use \parf{ EPSU}=$10^{-6}$ or \parf{ EPSU}=$10^{-7}$,
 which is the recommended value range.
\\ \hline 
IAD & 

This constant controls the mesh adaption~: 
\begin{itemize}
\item[-] \parf{ IAD=0}~:
  Fixed mesh. Normally, this choice should never be used, as it may result
  in spurious solutions. (Demo \parf{ ext}.)
\item[-] \parf{ IAD$>$0}~:  
  Adapt the mesh every \parf{ IAD} steps along the branch.
  Most demos use \parf{ IAD=3}, which is the strongly recommended value.
\end{itemize}
\\ \hline 
IADS & 

This constant controls the frequency of adaption of the 
pseudo-arclength stepsize.
\begin{itemize}
\item[-] \parf{ IADS=0}~: 
  Use fixed pseudo-arclength stepsize, i.e., the stepsize will
  be equal to the specified value of \parf{ DS} for every step.
  The computation of a branch will be discontinued as soon as
  the maximum number of iterations \parf{ ITNW} is reached.
  This choice is not recommended. 

(Demo \filef{ tim}.)
\item[-] \parf{ IADS$>$0}~:  
 Adapt the pseudo-arclength stepsize after every \parf{ IADS} steps.
  If the Newton/Chord iteration converges rapidly then 
  $\abs\parf{ DS}$ will be increased, but never beyond \parf{ DSMAX}.
  If a step fails then it will be retried with half
  the stepsize. This will be done repeatedly until the
  step is successful or until $\abs\parf{ DS}$ reaches \parf{ DSMIN}. 
  In the latter case nonconvergence will be signalled.
  The strongly recommended value is \parf{ IADS}=1, which is used in 
  almost all demos.
\end{itemize}
\\ \hline 
ICP & 

For each equation type and for each continuation calculation there is
a typical (``generic'') number of problem parameters that must be 
allowed to vary, in order for the calculations to be properly posed.
The constant \parf{ NICP} indicates how many free parameters have been specified,
while the array \parf{ ICP} actually designates these free parameters.
The parameter that appears first in the \parf{ ICP} list is called the 
``principal continuation parameter''.
Specific examples and special cases are described below.
\\ \hline 
IID & 

 This constant controls the amount of diagnostic output printed in \filef{ fort.9}~:
 the greater \parf{ IID} the more detailed the diagnostic output.
\begin{itemize}
\item[-] \parf{ IID=0}~:  
  Minimal diagnostic output. This setting is not recommended.
\item[-] \parf{ IID=2}~: 
  Regular diagnostic output. This is the recommended value of \parf{ IID}.
\item[-] \parf{ IID=3}~: 
  This setting gives additional diagnostic output for algebraic equations,
  namely the Jacobian and the residual vector at the starting point.
  This information, which is printed at the beginning of \filef{ fort.9},
  is useful for verifying whether the starting solution in \funcf{ stpnt} is indeed 
  a solution.
\item[-] \parf{ IID=4}~: 
  This setting gives additional diagnostic output for differential equations,
  namely the reduced system and the associated residual vector. 
  This information is printed for every step and for every Newton iteration,
  and should normally be suppressed.
  In particular it can be used to verify whether the starting solution
  is indeed a solution. For this purpose, the stepsize \parf{ DS} should
  be small, and one should look at the residuals printed in the \filef{ fort.9}
  output-file. (Note that the first residual vector printed in \filef{ fort.9} may
  be identically zero, as it may correspond to the computation of the starting
  direction. Look at the second residual vector in such case.)
  This residual vector has dimension 
  \parf{ NDIM}+\parf{ NBC}+\parf{ NINT}+1, which accounts for the \parf{ NDIM}
  differential equations, the \parf{ NBC} boundary conditions, the \parf{ NINT} user-defined
  integral constraints, and the pseudo-arclength equation.
  For proper interpretations of these data one may want to refer to the solution
  algorithm for solving the collocation system, as described in
  \citename{DoKeKe:91b} \citeyear{DoKeKe:91b}.
\item[-] \parf{ IID=5}~:
  This setting gives very extensive diagnostic output for differential equations,
  namely, debug output from the linear equation solver.
  This setting should not normally be used as it may result
  in a huge \filef{ fort.9} file. 
\end{itemize}
\\ \hline 
ILP & 

\begin{itemize}
\item[-] \parf{ ILP=0}~: 
  No detection of folds. This choice is recommended.
\item[-] \parf{ ILP=1}~: 
  Detection of folds. To be used if subsequent fold continuation is intended.
\end{itemize}2
\\ \hline 
IPLT & 

 This constant allows redefinition of the principal solution measure, which is
 printed as the second (real) column in the screen output and in the \filef{ fort.7}
 output-file~:
 
\begin{itemize}
\item[-]
  If \parf{ IPLT} = 0 then the $L_2$-norm is printed. Most demos use this setting.
  For algebraic problems, the standard definition of $L_2$-norm is used.
  For differential equations, the $L_2$-norm is defined as 
  $$ \sqrt{ \int_0^1 \sum_{k=1}^{NDIM} U_k(x)^2 ~ dx}~.$$
  Note that the interval of integration is $[0,1]$, the standard interval
 used by AUTO. For periodic solutions the independent variable is transformed
 to range from 0 to 1, before the norm is computed. The AUTO-constants THL(*) 
 and THU(*) (see Section~\ref{sec:NTHL} and Section~\ref{sec:NTHU})
 affect the definition of the $L_2$-norm.
\item[-]
  If 0 $<$ \parf{ IPLT} $\le$ \parf{ NDIM} then the maximum of the \parf{ IPLT}'th solution component 
  is printed.
\item[-]
  If $-$\parf{ NDIM} $\le$ \parf{ IPLT} $<$0 then the minimum of the \parf{ IPLT}'th solution component
  is printed.  (Demo \filef{ fsh}.)
\item[-]
  If \parf{ NDIM} $<$ \parf{ IPLT} $\le$ 2*\parf{ NDIM} then the integral 
  of the (\parf{ IPLT}$-$\parf{ NDIM})'th 
  solution component is printed. (Demos \filef{ exp}, \filef{ lor}.)
\item[-]
  If 2*\parf{ NDIM} $<$ \parf{ IPLT} $\le$ 3*\parf{ NDIM} 
  then the $L_2$-norm of the (\parf{ IPLT}$-$\parf{ NDIM})'th 
  solution component is printed. (Demo \filef{ frc}.)
\end{itemize}

Note that for algebraic problems the maximum and the minimum are identical.
Also, for ODEs the maximum and the minimum of a solution component are generally
much less accurate than the $L_2$-norm and component integrals.
Note also that the subroutine \funcf{ pvls} provides a second, more general way
of defining solution measures; see Section~\ref{sec:Parameter_over_specification}.
\\ \hline 
IPS & {\tiny 

This constant defines the problem type~:
\begin{itemize}
%=====================================================================
\item[-] \parf{ IPS=0}~: 
  An algebraic bifurcation problem.
  Hopf bifurcations will not be detected and stability
  properties will not be indicated in the \filef{ fort.7} output-file.
%=====================================================================
\item[-] \parf{ IPS=1}~: 
  Stationary solutions of ODEs with detection of Hopf bifurcations.
  The sign of PT, the point number, in \filef{ fort.7} is used 
  to indicate stability~: $-$ is stable , + is unstable.

 (Demo \filef{ ab}.)
%=====================================================================
\item[-] \parf{ IPS=$-$1}~:  
  Fixed points of the discrete dynamical system
  $u^{(k+1)}=f(u^{(k)},p ),$ with detection of Hopf bifurcations.
  The sign of PT in \filef{ fort.7} indicates stability~: 
  $-$ is stable , + is unstable.  
 (Demo \filef{ dd2}.)
%=====================================================================
\item[-] \parf{ IPS=$-$2}~: 
  Time integration using implicit Euler. 
  The \AUTO-constants \parf{ DS}, \parf{ DSMIN}, \parf{ DSMAX}, and \parf{ ITNW}, \parf{ NWTN} control 
  the stepsize.
  In fact, pseudo-arclength is used for ``continuation in time''. 
  Note that the time discretization is only first order accurate, 
  so that results should be carefully interpreted. 
  Indeed, this option has been included primarily for the detection 
  of stationary solutions, which can then be entered in the user-supplied
  subroutine \funcf{ stpnt}.  

 (Demo \filef{ ivp}.)
%=====================================================================
\item[-]  \parf{ IPS=2}~:
  Computation of periodic solutions. Starting data can be
  a Hopf bifurcation point (Run~2 of demo \filef{ ab}),
  a periodic orbit from a previous run (Run~4 of demo \filef{ pp2}),
  an analytically known periodic orbit (Run~1 of demo \filef{ frc}),
  or a numerically known periodic orbit (Demo \filef{ lor}).
  The sign of PT in \filef{ fort.7} is used to indicate
  stability~: $-$ is stable , + is unstable or unknown.
%=====================================================================
\item[-] \parf{ IPS=4}~: 
  A boundary value problem. Boundary conditions must be
  specified in the user-supplied subroutine \funcf{ bcnd}
  and integral constraints in \funcf{ icnd}. The \AUTO-constants
  \parf{ NBC} and \parf{ NINT} must be given correct values.
 (Demos \filef{ exp}, \filef{ int}, \filef{ kar}.)
%=====================================================================
\item[-] \parf{ IPS=5}~:
  Algebraic optimization problems. The objective function
  must be specified in the user-supplied subroutine \funcf{ fopt}. 
 (Demo \filef{ opt}.)
%=====================================================================
\item[-] \parf{ IPS=7}~:
  A boundary value problem with computation of Floquet multipliers. 
  This is a very special option; for most boundary value problems 
  one should use \parf{ IPS=4}.
  Boundary conditions must be
  specified in the user-supplied subroutine \funcf{ bcnd}
  and integral constraints in \funcf{ icnd}. The \AUTO-constants
  \parf{ NBC} and \parf{ NINT} must be given correct values.
%=====================================================================
\item[-] \parf{ IPS=9}~:
  This option is used in connection with the {\cal HomCont} algorithms
  described in 
  Chapters~\ref{ch:HomCont}-\ref{ch:HomCont_rev}
  for the  detection and continuation of homoclinic bifurcations.
  
 (Demos \filef{ san}, \filef{ mtn}, \filef{ kpr}, \filef{ cir}, \filef{ she},
  \filef{ rev}.)
%=====================================================================
\item[-] \parf{ IPS=11}~: 
  Spatially uniform solutions of a system of parabolic PDEs,
  with detection of traveling wave bifurcations.
  The user need only define the nonlinearity (in subroutine \funcf{ func}),
  initialize the wave speed in \parf{ PAR(10)}, initialize the diffusion 
  constants in \parf{ PAR(15,16,$\cdots$)}, and set a free equation parameter 
  in \parf{ ICP}(1).
  (Run~2 of demo \filef{ wav}.)
%=====================================================================
\item[-] \parf{ IPS=12}~: 
  Continuation of traveling wave solutions to a system of parabolic PDEs.
  Starting data can be a Hopf bifurcation point from a previous run 
  with \parf{ IPS}=11, or a traveling wave from a previous run with \parf{ IPS}=12.
  (Run~3  and Run~4 of demo \filef{ wav}.)
%=====================================================================
\item[-] \parf{ IPS=14}~:  
  Time evolution for a system of parabolic PDEs subject to periodic 
  boundary conditions. 
  Starting data may be solutions from a previous run with \parf{ IPS}=12 or 14. 
  Starting data can also be specified in \funcf{ stpnt}, in which case
  the wave length must be specified in \parf{ PAR(11)}, and the diffusion
  constants in \parf{ PAR(15,16,$\cdots$)}.
  \AUTO uses \parf{ PAR(14)} for the time variable.
  \parf{ DS}, \parf{ DSMIN}, and \parf{ DSMAX} govern the pseudo-arclength continuation 
  in the space-time variables.
  Note that the time discretization is only first order accurate, 
  so that results should be carefully interpreted. 
  Indeed, this option is mainly intended for the detection of stationary 
  waves.
  (Run~5 of demo \filef{ wav}.)
%=====================================================================
\item[-] \parf{ IPS=15}~:   
  Optimization of periodic solutions. The integrand of the
  objective functional must be specified in the user-supplied
  subroutine \funcf{ fopt}. Only \parf{ PAR(1-9)} should be used for
  problem parameters. \parf{ PAR(10)} is the value of the objective
  functional, \parf{ PAR(11)} the period, \parf{ PAR(12)} the norm of the
  adjoint variables, \parf{ PAR(14)} and \parf{ PAR(15)} are internal optimality
  variables. \parf{ PAR(21-29)} and \parf{ PAR(31)} are used to monitor the 
  optimality functionals associated with the problem parameters 
  and the period. 
  Computations can be started at a solution computed with \parf{ IPS}=2
  or \parf{ IPS}=15.
  For a detailed example see demo \filef{ ops}.
%=====================================================================
\item[-] \parf{ IPS=16}~:
  This option is similar to \parf{ IPS}=14, except that the user supplies the
  boundary conditions. Thus this option can be used for 
  time-integration of parabolic systems subject to 
  user-defined boundary conditions. For examples see the first runs
  of demos \filef{ pd1}, \filef{ pd2}, and \filef{ bru}. Note that
  the space-derivatives of the initial conditions must
  also be supplied in the user-supplied subroutine \funcf{ stpnt}. 
  The initial conditions must satisfy the boundary conditions.
  This option is mainly intended for the detecting stationary solutions.
%=====================================================================
 \item[-] \parf{ IPS=17}~: 
  This option can be used to continue stationary solutions
  of parabolic systems obtained from an evolution run with \parf{ IPS}=16.
  For examples see the second runs of demos \filef{ pd1} and \filef{ pd2}.
\end{itemize}
%=====================================================================
 }\\ \hline 
IRS & 

This constant sets the label of the solution where the computation
is to be restarted.
\begin{itemize}
\item[-] \parf{ IRS=0}~:  
  This setting is typically used in the first run of a new problem.
  In this case a starting solution must be defined in the user-supplied
  subroutine \filef{ stpnt}; see also Section~\ref{sec:Arguments_of_STPNT}.
  For representative examples of analytical starting solutions 
  see demos \filef{ ab} and \filef{ frc}.
  For starting from unlabeled numerical data see the {\it @fc} command
  (Section~\ref{sec:command_mode}) and demos \filef{ lor} and \filef{ pen}.
  
\item[-] \parf{ IRS$>$0}~: 
  Restart the computation at the previously computed solution with label \parf{ IRS}. 
  This solution is normally expected to be in the current data-file 
 \filef{ q.xxx}; see also the {\it @r} and {\it @R} commands in 
 Section~\ref{sec:command_mode}.
 Various \AUTO-constants can be modified when restarting.
\end{itemize}
\\ \hline 
ISP & 

This constant controls the detection of branch points,
period-doubling bifurcations, and torus bifurcations. 
\begin{itemize}
\item[-] \parf{ ISP=0}~:  
  This setting disables the detection of branch points, period-doubling 
  bifurcations, and torus bifurcations and the computation of 
  Floquet multipliers.
\item[-] \parf{ ISP=1}~:  
  Branch points are detected for algebraic equations, but not for
  periodic solutions and boundary value problems.
  Period-doubling bifurcations and torus bifurcations are not located either.
  However, Floquet multipliers are computed.
\item[-] \parf{ ISP=2}~: This setting enables the detection of all special 
 solutions.
 For periodic solutions and rotations, the choice \parf{ ISP}=2 should be used with
 care, due to potential inaccuracy in the computation of the
 linearized Poincar\'e map and possible rapid variation of the
 Floquet multipliers.
 The linearized Poincar\'e map always has a multiplier $z=1$.
 If this multiplier becomes inaccurate
 then the automatic detection of secondary periodic
 bifurcations will be discontinued and a
 warning message will be printed in \filef{ fort.9}.
 See also Section~\ref{sec:Bifurcations}.
\item[-] \parf{ ISP=3}~:  
  Branch points will be detected, but \AUTO will not monitor the 
  Floquet multipliers. Period-doubling and torus bifurcations will go undetected. 
  This option is useful for certain problems with non-generic Floquet behavior.
  The Floquet multipliers will be output to the diagnostic file.
\end{itemize}
\\ \hline 
ISW & 

 This constant controls branch switching at branch points for the case
 of differential equations.
 Note that branch switching is automatic for algebraic equations.
\begin{itemize}
\item[-] \parf{ ISW=1}~: This is the normal value of \parf{ ISW}.
\item[-] \parf{ ISW=$-$1}~:
  If \parf{ IRS} is the label of a branch point or a period-doubling
  bifurcation then branch switching will be done.
  For period doubling bifurcations it is recommended that \parf{ NTST} be increased.
  For examples see Run~2 and Run~3 of demo \filef{ lor}, where branch switching
  is done at period-doubling bifurcations, and Run~2 and Run~3 of demo \filef{ bvp},
  where branch switching is done at a transcritical branch point.
\item[-] \parf{ ISW=2}~:
  If \parf{ IRS} is the label of a fold, a Hopf bifurcation point, 
  or a period-doubling or torus bifurcation then a locus of such points will be
  computed. An additional free parameter must be specified for such 
  continuations; 
  see also Section~\ref{sec:Free_parameters}.
\end{itemize}
\\ \hline 
ITMX & 

 The maximum number of iterations allowed in the accurate
 location of special solutions, such as bifurcations, folds, 
 and user output points, by M\"uller's method with bracketing.
 The recommended value is \parf{ ITMX}=8, used in most demos.
\\ \hline 
ITNW & 

 The maximum number of combined Newton-Chord iterations.
 When this maximum is reached, the step will be retried with 
 half the stepsize.
 This is repeated until convergence, or until the minimum
 stepsize is reached. In the latter case the computation of
 the branch is discontinued and a message printed in \filef{ fort.9}.
 The recommended value is \parf{ ITNW}=5, but \parf{ ITNW}=7 may be used for 
 ``difficult'' problems, for example, 
 demos \filef{ spb}, \filef{ chu}, \filef{ plp}, etc.
\\ \hline 
JAC & 

 Used to indicate whether derivatives are supplied by the user
 or to be obtained by differencing~:
\begin{itemize}
\item[-] \parf{ JAC=0}~: 
  No derivatives are given by the user. (Most demos use \parf{ JAC}=0.)
\item[-] \parf{ JAC=1}~:  
  Derivatives with respect to state- and problem-parameters are given 
  in the user-supplied subroutines 
  \funcf{ func}, \funcf{ bcnd}, \funcf{ icnd} and \funcf{ fopt}, where 
  applicable.  This may be necessary for sensitive problems. 
  It is also recommended for computations in which \AUTO generates 
  an extended system, for example, when \parf{ ISW}=2.

  (Demos \filef{ int}, \filef{ dd2}, \filef{ obt}, \filef{ plp}, \filef{ ops}.)

  (For \parf{ ISW} see Section~\ref{sec:ISW}.) 
\end{itemize}
\\ \hline 
MXBF & 

 This constant, which is effective for algebraic problems only,
 sets the maximum number of bifurcations to be treated.
 Additional branch points will be noted, but the corresponding bifurcating
 branches will not be computed.
 If \parf{ MXBF} is positive then the bifurcating branches of the first \parf{ MXBF}
  branch points will be traced out in both directions.
 If \parf{ MXBF} is negative then the bifurcating branches of the first 
 $\abs{\parf{ MXBF}}$ branch points will be traced out in only one direction. 
\\ \hline 
NBC & 

 The number of boundary conditions as specified in the user-supplied
 subroutine \funcf{ bcnd}. 

 (Demos \filef{ exp}, \filef{ kar}.)
\\ \hline 
NCOL & 

 The number of Gauss collocation points per mesh interval,
 (2 $\le$ \parf{ NCOL} $\le$ 7).
 \parf{ NCOL} remains fixed during any given run, but can be changed
 when restarting at a previously computed solution.
 The choice \parf{ NCOL}=4, used in most demos, is recommended.
 If \parf{ NDIM} is ``large'' and the solutions ``very smooth'' then
 \parf{ NCOL}=2 may be appropriate.
\\ \hline 
NDIM & 

 Dimension of the system of equations as specified in the user-supplied
 subroutine \funcf{ func}.
\\ \hline 
NICP & 


%AUTODOC ICP START
%AUTODOC ICP SHORT
% ICP designates the free parameters.
%AUTODOC ICP FULL
% For each equation type and for each continuation calculation there is a
% typical (generic) number of problem parameters that must be allowed to
% vary, in order for the calculations to be properly posed. The constant NICP
% indicates how many free parameters have been specified, while the array ICP
% actually designates these free parameters. The parameter that appears first
% in the ICP list is called the principal continuation parameter.  Specific
% examples and special cases are described in the manual.
%AUTODOC ICP TEX

For each equation type and for each continuation calculation there is
a typical (``generic'') number of problem parameters that must be 
allowed to vary, in order for the calculations to be properly posed.
The constant \parf{ NICP} indicates how many free parameters have been specified,
while the array \parf{ ICP} actually designates these free parameters.
The parameter that appears first in the \parf{ ICP} list is called the 
``principal continuation parameter''.
Specific examples and special cases are described below.
%AUTODOC ICP END
\\ \hline 
NINT & 

 The number of integral conditions as specified in the user-supplied
 subroutine \funcf{ icnd}. 

(Demos \filef{ int}, \filef{ lin}, \filef{ obv}.)
\\ \hline 
NMX & 

The maximum number of steps to be taken along any branch.
\\ \hline 
NPR & 

 This constant can be used to regularly write \filef{ fort.8} plotting and restart 
 data.  
 IF \parf{ NPR}$>$0 then such output is written every \parf{ NPR} steps.
 IF \parf{ NPR}$=$0 or if \parf{ NPR}$\ge$\parf{ NMX} then no such output is written.
 Note that special solutions, such as branch points, folds, end points, etc., 
 are always written in \filef{ fort.8}.
 Furthermore, one can specify parameter values where plotting and restart 
 data is to be written; see Section~\ref{sec:NUZR}.
 For these reasons, and to limit the output volume, it is recommended that
 \parf{ NPR} output be kept to a minimum.
\\ \hline 
NTHL & 

By default, the pseudo-arclength stepsize includes all state variables
(or state functions) and all free parameters.
Under certain circumstances one may want to modify the weight accorded 
to individual parameters in the definition of stepsize.
For this purpose, \parf{ NTHL} defines the number of parameters whose weight 
is to be modified.
If \parf{ NTHL}=0 then all weights will have default value 1.0~.
If \parf{ NTHL}$>$0 then one must enter \parf{ NTHL} pairs,
             ~{\it Parameter Index} ~ {\it Weight}~,
with each pair on a separate line.

For example, for the computation of periodic solutions it is 
recommended that the period not be included in the pseudo-arclength 
continuation stepsize, in order to avoid period-induced limitations 
on the stepsize near orbits of infinite period. 
This exclusion can be accomplished by setting \parf{ NTHL=1}, with, 
on a separate line, the pair ~ 11 ~ 0.0 ~.
Most demos that compute periodic solutions use this option;
see for example demo \filef{ ab}.
\\ \hline 
NTHU & 

Under certain circumstances one may want to modify the weight accorded 
to individual state variables (or state functions) in the definition 
of stepsize.
For this purpose, \parf{ NTHU} defines the number of states whose weight 
is to be modified.
If \parf{ NTHU}=0 then all weights will have default value 1.0~.
If \parf{ NTHU}$>$0 then one must enter \parf{ NTHU} pairs,
             ~{\it State Index} ~ {\it Weight}~,
with each pair on a separate line.
At present none of the demos use this option.
\\ \hline 
NTST & 

 The number of mesh intervals used for discretization.
 \parf{ NTST} remains fixed during any particular run, but can be changed
 when restarting. 
 Recommended value of \parf{ NTST} : As small as possible to maintain convergence. 
 
 (Demos \filef{ exp}, \filef{ ab}, \filef{ spb}.)

 (For mesh adaption see \parf{IAD} in Section~\ref{sec:IAD}.)
\\ \hline 
NUZR & 

 This constant allows the setting of parameter values at which labeled plotting 
 and restart information is to be written in the \filef{ fort.8} output-file.
 Optionally, it also allows the computation to terminate at such a point.

\begin{itemize}
\item[-]
 Set \parf{ NUZR}=0 if no such output is needed. Many demos use this setting.
\item[-]
 If \parf{ NUZR}$>$0 then one must enter \parf{ NUZR} pairs,
            ~{\it Parameter-Index} ~ {\it Parameter-Value}~,
 with each pair on a separate line, to designate the parameters and the parameter
 values at which output is to be written.
 For examples see demos \filef{ exp}, \filef{ int}, and \filef{ fsh}.
\item[-]
 If such a parameter index is preceded by a minus sign then the computation will
 terminate at such a solution point.
 (Demos \filef{ pen} and \filef{ bru}.)
\end{itemize}

Note that \filef{ fort.8} output can also be written at selected values of 
overspecified parameters. For an example see demo \filef{ pvl}.
For details on overspecified parameters see 
Section~\ref{sec:Parameter_over_specification}.
\\ \hline 
NWTN & 

 After \parf{ NWTN} Newton iterations the Jacobian is frozen, i.e.,
 \AUTO uses full Newton for the first \parf{ NWTN} iterations
 and the Chord method for iterations \parf{ NWTN}+1 to \parf{ ITNW}.
 The choice \parf{ NWTN}=3 is strongly recommended and used in most demos.
 Note that this constant is only effective for ODEs, i.e., for solving
 the piecewise polynomial collocation equations.
 For algebraic systems \AUTO always uses full Newton.
\\ \hline 
RL0 & 

 The lower bound on the principal continuation parameter.
 (This is the parameter which appears first in the \parf{ ICP} list;
 see Section~\ref{sec:ICP}.). 
\\ \hline 
RL1 & 

 The upper bound on the principal continuation parameter. 
\\ \hline 
\end{longtable}
}
\end{document}
