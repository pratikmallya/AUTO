\subsection{commandAppend} \label{sec:clui_ref_commandAppend}\begin{minipage}{6in}\hrule\medskip\section*{Purpose}
Append data files.\section*{Description}

    Type commandAppend('xxx') to append the output-files fort.7, fort.8,
    fort.9, to existing data-files s.xxx, b.xxx, and d.xxx (if you are
    using the default filename templates).

    Type commandAppend('xxx','yyy') to append existing data-files s.xxx, b.xxx,
    and d.xxx to data-files s.yyy, b.yyy, and d.yyy (if you are using
    the default filename templates).
    \section*{Aliases}
ap append \medskip\hrule\end{minipage}\subsection{commandCat} \label{sec:clui_ref_commandCat}\begin{minipage}{6in}\hrule\medskip\section*{Purpose}
Print the contents of a file\section*{Description}

    Type 'commandCat xxx' to list the contents of the file 'xxx'.  This calls the
    Unix function 'cat' for reading the file.  
    \section*{Aliases}
cat \medskip\hrule\end{minipage}\subsection{commandCd} \label{sec:clui_ref_commandCd}\begin{minipage}{6in}\hrule\medskip\section*{Purpose}
Change directories.\section*{Description}
    
    Type 'commandCd xxx' to change to the directory 'xxx'.  This command
    understands both shell variables and home directory expansion.
    \section*{Aliases}
cd \medskip\hrule\end{minipage}\subsection{commandClean} \label{sec:clui_ref_commandClean}\begin{minipage}{6in}\hrule\medskip\section*{Purpose}
Clean the current directory.\section*{Description}

    Type commandClean() to clean the current directory.  This command will
    delete all files of the form fort.*, *.o, and *.exe.
    \section*{Aliases}
clean cl \medskip\hrule\end{minipage}\subsection{commandCopyAndLoadDemo} \label{sec:clui_ref_commandCopyAndLoadDemo}\begin{minipage}{6in}\hrule\medskip\section*{Purpose}
Copy a demo into the current directory and load it.\section*{Description}

    Type commandCopyAndLoadDemo('xxx') to copy all files from auto/2000/demos/xxx to the
    current user directory.  Here 'xxx' denotes a demo name; e.g.,
    'abc'.  Note that the 'dm' command also copies a Makefile to the
    current user directory. To avoid the overwriting of existing
    files, always run demos in a clean work directory.  NOTE: This
    command automatically performs the commandRunnerLoadName command
    as well.
    \section*{Aliases}
dm demo \medskip\hrule\end{minipage}\subsection{commandCopyDataFiles} \label{sec:clui_ref_commandCopyDataFiles}\begin{minipage}{6in}\hrule\medskip\section*{Purpose}
Copy data files.\section*{Description}

    Type commandCopyDataFiles('xxx','yyy') to copy the data-files c.xxx, d.xxx, b.xxx,
    and h.xxx to c.yyy, d.yyy, b.yyy, and h.yyy (if you are using the
    default filename templates).
    \section*{Aliases}
copy cp \medskip\hrule\end{minipage}\subsection{commandCopyDemo} \label{sec:clui_ref_commandCopyDemo}\begin{minipage}{6in}\hrule\medskip\section*{Purpose}
Copy a demo into the current directory.\section*{Description}

    Type commandCopyDemo('xxx') to copy all files from auto/2000/demos/xxx to the
    current user directory.  Here 'xxx' denotes a demo name; e.g.,
    'abc'.  Note that the 'dm' command also copies a Makefile to the
    current user directory. To avoid the overwriting of existing
    files, always run demos in a clean work directory.
    \section*{Aliases}
copydemo \medskip\hrule\end{minipage}\subsection{commandCopyFortFiles} \label{sec:clui_ref_commandCopyFortFiles}\begin{minipage}{6in}\hrule\medskip\section*{Purpose}
Save data files.\section*{Description}

    Type commandCopyFortFiles('xxx') to save the output-files fort.7, fort.8, fort.9,
    to b.xxx, s.xxx, d.xxx (if you are using the default filename
    templates).  Existing files with these names will be overwritten.
    \section*{Aliases}
sv save \medskip\hrule\end{minipage}\subsection{commandCreateGUI} \label{sec:clui_ref_commandCreateGUI}\begin{minipage}{6in}\hrule\medskip\section*{Purpose}
Show AUTOs graphical user interface.\section*{Description}

    Type commandCreateGUI() to start AUTOs graphical user interface.
    
    NOTE: This command is not implemented yet.
    \section*{Aliases}
gui \medskip\hrule\end{minipage}\subsection{commandDeleteDataFiles} \label{sec:clui_ref_commandDeleteDataFiles}\begin{minipage}{6in}\hrule\medskip\section*{Purpose}
Delete data files.\section*{Description}

    Type commandDeleteDataFiles('xxx') to delete the data-files d.xxx, b.xxx, and s.xxx
    (if you are using the default filename templates).
    \section*{Aliases}
delete dl \medskip\hrule\end{minipage}\subsection{commandDeleteFortFiles} \label{sec:clui_ref_commandDeleteFortFiles}\begin{minipage}{6in}\hrule\medskip\section*{Purpose}
Clear the current directory of fort files.\section*{Description}

    Type commandDeleteFortFiles() to clean the current directory.  This command will
    delete all files of the form fort.*.
    \section*{Aliases}
df deletefort \medskip\hrule\end{minipage}\subsection{commandDouble} \label{sec:clui_ref_commandDouble}\begin{minipage}{6in}\hrule\medskip\section*{Purpose}
Double a solution.\section*{Description}

    Type commandDouble() to double the solution in 'fort.7' and 'fort.8'.

    Type commandDouble('xxx') to double the solution in b.xxx and s.xxx (if you
    are using the default filename templates).
    \section*{Aliases}
double db \medskip\hrule\end{minipage}\subsection{commandInteractiveHelp} \label{sec:clui_ref_commandInteractiveHelp}\begin{minipage}{6in}\hrule\medskip\section*{Purpose}
Get help on the AUTO commands.\section*{Description}
    
    Type 'help' to list all commands with a online help.
    Type 'help xxx' to get help for command 'xxx'.
    \section*{Aliases}
man help \medskip\hrule\end{minipage}\subsection{commandLs} \label{sec:clui_ref_commandLs}\begin{minipage}{6in}\hrule\medskip\section*{Purpose}
List the current directory.\section*{Description}
    
    Type 'commandLs' to run the system 'ls' command in the current directory.  This
    command will accept whatever arguments are accepted by the Unix command
    'ls'.
    \section*{Aliases}
ls \medskip\hrule\end{minipage}\subsection{commandMoveFiles} \label{sec:clui_ref_commandMoveFiles}\begin{minipage}{6in}\hrule\medskip\section*{Purpose}
Move data-files to a new name.\section*{Description}

    Type commandMoveFiles('xxx','yyy') to move the data-files b.xxx, s.xxx, d.xxx,
    and c.xxx to b.yyy, s.yyy, d.yyy, and c.yyy (if you are using the
    default filename templates).  \section*{Aliases}
move mv \medskip\hrule\end{minipage}\subsection{commandParseConstantsFile} \label{sec:clui_ref_commandParseConstantsFile}\begin{minipage}{6in}\hrule\medskip\section*{Purpose}
Get the current continuation constants.\section*{Description}

    Type commandParseConstantsFile('xxx') to get a parsed version of the constants file
    c.xxx (if you are using the default filename templates).
    \section*{Aliases}
cn constantsget \medskip\hrule\end{minipage}\subsection{commandParseDiagramAndSolutionFile} \label{sec:clui_ref_commandParseDiagramAndSolutionFile}\begin{minipage}{6in}\hrule\medskip\section*{Purpose}
Parse both bifurcation diagram and solution.\section*{Description}

    Type commandParseDiagramAndSolutionFile('xxx') to get a parsed version of the diagram file b.xxx
    and solution file s.xxx (if you are using the default filename
    templates).
    \section*{Aliases}
bt diagramandsolutionget \medskip\hrule\end{minipage}\subsection{commandParseDiagramFile} \label{sec:clui_ref_commandParseDiagramFile}\begin{minipage}{6in}\hrule\medskip\section*{Purpose}
Parse a bifurcation diagram.\section*{Description}

    Type commandParseDiagramFile('xxx') to get a parsed version of the diagram file b.xxx
    (if you are using the default filename templates).
    \section*{Aliases}
dg diagramget \medskip\hrule\end{minipage}\subsection{commandParseSolutionFile} \label{sec:clui_ref_commandParseSolutionFile}\begin{minipage}{6in}\hrule\medskip\section*{Purpose}
Parse solution file:\section*{Description}

    Type commandParseSolutionFile('xxx') to get a parsed version of the solution file
    s.xxx (if you are using the default filename templates).
    \section*{Aliases}
sl solutionget \medskip\hrule\end{minipage}\subsection{commandPlotter} \label{sec:clui_ref_commandPlotter}\begin{minipage}{6in}\hrule\medskip\section*{Purpose}
2D plotting of data.\section*{Description}

    Type commandPlotter('xxx') to run the graphics program for the graphical
    inspection of the data-files b.xxx and s.xxx (if you are using the
    default filename templates).  The return value will be the handle
    for the graphics window.

    Type commandPlotter() to run the graphics program for the graphical
    inspection of the output-files 'fort.7' and 'fort.8'.  The return
    value will be the handle for the graphics window.
    \section*{Aliases}
p2 pl plot \medskip\hrule\end{minipage}\subsection{commandPlotter3D} \label{sec:clui_ref_commandPlotter3D}\begin{minipage}{6in}\hrule\medskip\section*{Purpose}
3D plotting of data.\section*{Description}

    Type commandPlotter3D('xxx') to run the graphics program for the graphical
    inspection of the data-files b.xxx and s.xxx (if you are using the
    default filename templates).  The return value will be the handle
    for the graphics window.

    Type commandPlotter3D() to run the graphics program for the graphical
    inspection of the output-files 'fort.7' and 'fort.8'.  The return
    value will be the handle for the graphics window.
    \section*{Aliases}
plot3 p3 \medskip\hrule\end{minipage}\subsection{commandQueryBranchPoint} \label{sec:clui_ref_commandQueryBranchPoint}\begin{minipage}{6in}\hrule\medskip\section*{Purpose}
Print the ``branch-point function''.\section*{Description}
    
    Type commandQueryBranchPoint() to list the value of the ``branch-point function'' 
    in the output-file fort.9. This function vanishes at a branch point.
    
    Type commandQueryBranchPoint('xxx') to list the value of the ``branch-point function''
    in the info file 'd.xxx'.
    \section*{Aliases}
br bp branchpoint \medskip\hrule\end{minipage}\subsection{commandQueryEigenvalue} \label{sec:clui_ref_commandQueryEigenvalue}\begin{minipage}{6in}\hrule\medskip\section*{Purpose}
Print eigenvalues of Jacobian (algebraic case).\section*{Description}

    Type commandQueryEigenvalue() to list the eigenvalues of the Jacobian 
    in fort.9. 
    (Algebraic problems.)

    Type commandQueryEigenvalue('xxx') to list the eigenvalues of the Jacobian 
    in the info file 'd.xxx'.
    \section*{Aliases}
eigenvalue ev eg \medskip\hrule\end{minipage}\subsection{commandQueryFloquet} \label{sec:clui_ref_commandQueryFloquet}\begin{minipage}{6in}\hrule\medskip\section*{Purpose}
Print the Floquet multipliers.\section*{Description}

    Type commandQueryFloquet() to list the Floquet multipliers
    in the output-file fort.9. 
    (Differential equations.)

    Type commandQueryFloquet('xxx') to list the Floquet multipliers 
    in the info file 'd.xxx'.
    \section*{Aliases}
fl floquet \medskip\hrule\end{minipage}\subsection{commandQueryHopf} \label{sec:clui_ref_commandQueryHopf}\begin{minipage}{6in}\hrule\medskip\section*{Purpose}
Print the value of the ``Hopf function''.\section*{Description}

    Type commandQueryHopf() to list the value of the ``Hopf function'' 
    in the output-file fort.9. This function
    vanishes at a Hopf bifurcation point.

    Type commandQueryHopf('xxx') to list the value of the ``Hopf function''
    in the info file 'd.xxx'.
    \section*{Aliases}
hb hp hopf lp \medskip\hrule\end{minipage}\subsection{commandQueryIterations} \label{sec:clui_ref_commandQueryIterations}\begin{minipage}{6in}\hrule\medskip\section*{Purpose}
Print the number of Newton interations.\section*{Description}

    Type commandQueryIterations() to list the number of Newton iterations per
    continuation step in fort.9. 

    Type commandQueryIterations('xxx') to list the number of Newton iterations per
    continuation step in the info file 'd.xxx'.
    \section*{Aliases}
iterations it \medskip\hrule\end{minipage}\subsection{commandQueryLimitpoint} \label{sec:clui_ref_commandQueryLimitpoint}\begin{minipage}{6in}\hrule\medskip\section*{Purpose}
Print the value of the ``limit point function''.\section*{Description}

    Type commandQueryLimitpoint() to list the value of the ``limit point function'' 
    in the output-file fort.9. This function
    vanishes at a limit point (fold).

    Type commandQueryLimitpoint('xxx') to list the value of the ``limit point function'' 
    in the info file 'd.xxx'.
    \section*{Aliases}
lm limitpoint \medskip\hrule\end{minipage}\subsection{commandQueryNote} \label{sec:clui_ref_commandQueryNote}\begin{minipage}{6in}\hrule\medskip\section*{Purpose}
Print notes in info file.\section*{Description}

    Type commandQueryNote() to show any notes 
    in the output-file fort.9.

    Type commandQueryNote('xxx') to show any notes 
    in the info file 'd.xxx'.
    \section*{Aliases}
nt note \medskip\hrule\end{minipage}\subsection{commandQuerySecondaryPeriod} \label{sec:clui_ref_commandQuerySecondaryPeriod}\begin{minipage}{6in}\hrule\medskip\section*{Purpose}
Print value of ``secondary-periodic bif. fcn''.\section*{Description}

    Type commandQuerySecondaryPeriod()  to list the value of the 
    ``secondary-periodic bifurcation function'' 
    in the output-file 'fort.9. This function
    vanishes at period-doubling and torus bifurcations.

    Type commandQuerySecondaryPeriod('xxx') to list the value of the
    ``secondary-periodic bifurcation function''
    in the info file 'd.xxx'.
    \section*{Aliases}
sc secondaryperiod sp \medskip\hrule\end{minipage}\subsection{commandQueryStepsize} \label{sec:clui_ref_commandQueryStepsize}\begin{minipage}{6in}\hrule\medskip\section*{Purpose}
Print continuation step sizes.\section*{Description}

    Type commandQueryStepsize() to list the continuation step size for each
    continuation step in  'fort.9. 

    Type commandQueryStepsize('xxx') to list the continuation step size for each
    continuation step in the info file 'd.xxx'.
    \section*{Aliases}
ss stepsize st \medskip\hrule\end{minipage}\subsection{commandRun} \label{sec:clui_ref_commandRun}\begin{minipage}{6in}\hrule\medskip\section*{Purpose}
Run AUTO.\section*{Description}

    Type commandRun([options]) to run AUTO with the given options.
    There are four possible options:
    \begin{verbatim}
    Long name   Short name    Description
    -------------------------------------------
    equation    e             The equations file
    constants   c             The AUTO constants file
    solution    s             The restart solution file
    homcont     h             The Homcont parameter file
    \end{verbatim}
    Options which are not explicitly set retain their previous value.
    For example one may type: commandRun(e='ab',c='ab.1') to use 'ab.c' as
    the equations file and c.ab.1 as the constants file (if you are
    using the default filename templates).

    Type commandRun('name') load all files with base 'name'.
    This does the same thing as running
    commandRun(e='name',c='name,s='name',h='name').
    \section*{Aliases}
r run rn \medskip\hrule\end{minipage}\subsection{commandRunnerConfigFort2} \label{sec:clui_ref_commandRunnerConfigFort2}\begin{minipage}{6in}\hrule\medskip\section*{Purpose}
Modify continuation constants.\section*{Description}

    Type commandRunnerConfigFort2('xxx',yyy) to change the constant 'xxx' to have
    value yyy.
    \section*{Aliases}
changeconstant cc ch \medskip\hrule\end{minipage}\subsection{commandRunnerLoadName} \label{sec:clui_ref_commandRunnerLoadName}\begin{minipage}{6in}\hrule\medskip\section*{Purpose}
Load files into the AUTO runner.\section*{Description}

    Type commandRunnerLoadName([options]) to modify AUTO runner.
    There are four possible options:
    \begin{verbatim}
    Long name   Short name    Description
    -------------------------------------------
    equation    e             The equations file
    constants   c             The AUTO constants file
    solution    s             The restart solution file
    homcont     h             The Homcont parameter file
    \end{verbatim}
    Options which are not explicitly set retain their previous value.
    For example one may type: commandRunnerLoadName(e='ab',c='ab.1') to use 'ab.c' as
    the equations file and c.ab.1 as the constants file (if you are
    using the default filename templates).

    Type commandRunnerLoadName('name') load all files with base 'name'.
    This does the same thing as running
    commandRunnerLoadName(e='name',c='name,s='name',h='name').
    \section*{Aliases}
ld load \medskip\hrule\end{minipage}\subsection{commandRunnerPrintFort2} \label{sec:clui_ref_commandRunnerPrintFort2}\begin{minipage}{6in}\hrule\medskip\section*{Purpose}
Print continuation parameters.\section*{Description}

    Type commandRunnerPrintFort2() to print all the parameters.
    Type commandRunnerPrintFort2('xxx') to return the parameter 'xxx'.
    \section*{Aliases}
pc pr printconstant \medskip\hrule\end{minipage}\subsection{commandShell} \label{sec:clui_ref_commandShell}\begin{minipage}{6in}\hrule\medskip\section*{Purpose}
Run a shell command.\section*{Description}
        
    Type 'shell xxx' to run the command 'xxx' in the Unix shell and display
    the results in the AUTO command line user interface.
    \section*{Aliases}
shell \medskip\hrule\end{minipage}\subsection{commandTriple} \label{sec:clui_ref_commandTriple}\begin{minipage}{6in}\hrule\medskip\section*{Purpose}
Triple a solution.\section*{Description}

    Type commandTriple() to triple the solution in 'fort.7' and 'fort.8'.

    Type commandTriple('xxx') to triple the solution in b.xxx and s.xxx (if you
    are using the default filename templates).
    \section*{Aliases}
tr triple \medskip\hrule\end{minipage}\subsection{commandUserData} \label{sec:clui_ref_commandUserData}\begin{minipage}{6in}\hrule\medskip\section*{Purpose}
Covert user-supplied data files.\section*{Description}

    Type commandUserData('xxx') to convert a user-supplied data file 'xxx.dat' to
    AUTO format. The converted file is called 's.dat'.  The original
    file is left unchanged.  AUTO automatically sets the period in
    PAR(10).  Other parameter values must be set in 'stpnt'. (When
    necessary, PAR(10) may also be redefined there.)  The
    constants-file file 'c.xxx' must be present, as the AUTO-constants
    'NTST' and 'NCOL' are used to define the new mesh.  For examples
    of using the 'userData' command see demos 'lor' and 'pen' (where
    it has the old name 'fc').
    \section*{Aliases}
us userdata \medskip\hrule\end{minipage}\subsection{commandWait} \label{sec:clui_ref_commandWait}\begin{minipage}{6in}\hrule\medskip\section*{Purpose}
Wait for the user to enter a key.\section*{Description}

    Type 'commandWait' to have the AUTO interface wait
    until the user hits any key (mainly used in scripts).
    \section*{Aliases}
wait \medskip\hrule\end{minipage}
