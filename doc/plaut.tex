%==============================================================================
%==============================================================================
\chapter{ The Graphics Program PLAUT.} \label{ch:PLAUT}
%==============================================================================
%==============================================================================
{\cal PLAUT} can be used to extract graphical
information from the \AUTO output-files \filef{ fort.7} and \filef{ fort.8},
or from the corresponding data-files \filef{ b.xxx} and \filef{ s.xxx}.
To invoke {\cal PLAUT}, use the the {\it @p} command defined in 
Section~\ref{sec:command_mode}.
The {\cal PLAUT} window (a Tektronix window) will appear, in which {\cal PLAUT}
commands can be entered.
FIXME:  This is not correct anymore
For examples of using {\cal PLAUT} see the tutorial demo \filef{ ab}, in particular,
Sections~\ref{sec:Tutorial_plotting} and \ref{sec:Tutorial_plotting_2p}.
See also demo \filef{ pp2} in Section~\ref{sec:Demos_pp2}.


\section{ Basic {\cal PLAUT}-Commands.} \label{sec:main_PLAUT_commands}
The principal {\cal PLAUT}-commands are 
\begin{itemize}
\item[\commandf{ bd0}]~:
  This command is useful for an initial overview of the bifurcation
  diagram as stored in \filef{ fort.7}.
  If you have not previously selected one of the default options 
  {\it d0, d1, d2, d3}, or {\it d4} described below then you will be asked
  whether you want solution labels, grid lines, titles, or labeled axes.

\item[\commandf{ bd}]~:
  This command is the same as the {\it bd0} command, except that you will be
  asked to enter the minimum and the maximum of the horizontal and 
  vertical axes.
  This is useful for blowing up portions of a previously displayed
  bifurcation diagram.

\item[\commandf{ ax}]~:
  With the {\it ax} command you can select any pair of columns of real
  numbers from \filef{ fort.7} as horizontal and vertical axis in the
  bifurcation diagram. (The default is columns 1 and 2).
  To determine what these columns represent, one can look at the
  screen ouput of the corresponding \AUTO run, or one can inspect the
  column headings in \filef{ fort.7}.
  
\item[\commandf{ 2d}]~:
  Upon entering the {\it 2d} command, the labels of all solutions stored 
  in \filef{ fort.8} will be listed and you can select one or more of these 
  for display. The number of solution components is also listed
  and you will be prompted to select two of these as horizontal and
  vertical axis in the display.
  Note that the first component is typically the independent 
  time or space variable scaled to the interval [0,1].

\item[\commandf{ sav}]~:
  To save the displayed plot in a file. You will be asked to enter
  a file name. Each plot must be stored in a separate new file.
  The plot is stored in compact {\cal PLOT10} format, which can be converted to 
  {\cal PostScript} format with the \AUTO-commands \filef{ @ps} and \filef{ @pr};
  see Section~\ref{sec:Printing_PLAUT_files}.

\item[\commandf{ cl}]~:  To clear the graphics window.

\item[\commandf{ lab}]~:
  To list the labels of all solutions stored in \filef{ fort.8}.
  Note that {\cal PLAUT} requires all labels to be distinct.
  In case of multiple labels you can use the \AUTO
  command {\it @lb} to relabel solutions in
  \filef{ fort.7} and \filef{ fort.8}.

\item[\commandf{ end}]~:  To end execution of {\cal PLAUT}.
\end{itemize}


\section{ Default Options.} \label{sec:PLAUT_default}
After entering the commands {\it bd0, bd}, or {\it 2d}, you will be asked whether you 
want solution labels, grid lines, titles, or axes labels.
For quick plotting it is convenient to bypass these selections.
This can be done by the default commands {\it d0, d1, d2, d3}, or {\it d4} below.
These can be entered as a single command 
or they can be entered as prefixes in the {\it bd0} and {\it bd} commands. 
Thus, for example, one can enter the command {\it d1bd0}.  

\begin{itemize}
\item[\commandf{ d0}]~:  Use solid curves, showing solution labels and symbols.  
\item[\commandf{ d1}]~:  Use solid curves, except use dashed curves for unstable
  solutions and for solutions of unknown stability.
  Show solution labels and symbols.
\item[\commandf{ d2}]~:  As {\it d1}, but with grid lines.  
\item[\commandf{ d3}]~:  As {\it d1}, except for periodic solutions use 
  solid circles if stable,
  and open circles if unstable or if the stability
  is unknown.
\item[\commandf{ d4}]~:  Use solid curves, without labels and symbols.  
\end{itemize}

If no default option {\it d0, d1, d2, d3}, or {\it d4} has been selected 
or if you want to override a default feature,
then the the following commands can be used.
These can be entered as individual commands or as prefixes.
For example, one can enter the command {\it sydpbd0}.

\begin{itemize}
\item[\commandf{ sy}]~:  Use symbols for special solution points, for example,
  open square = branch point,
  solid square = Hopf bifurcation.
\item[\commandf{ dp}]~:  ``Differential Plot'', i.e., show stability of the 
  solutions. Solid curves represent stable solutions.
  Dashed curves are used for unstable
  solutions and for solutions of unknown stability.
  For periodic solutions use solid/open circles
  to indicate stability/instability (or unknown
  stability).
\item[\commandf{ st}]~:  Set up titles and axes labels. 
\item[\commandf{ nu}]~:  Normal usage (reset special options). 
\end{itemize}


\section{ Other {\cal PLAUT}-Commands.} \label{sec:Other_PLAUT_commands}
The full {\cal PLAUT} program has several other capabilities, for example,

\begin{itemize}
\item[\commandf{ scr}]~:  To change the diagram size.
\item[\commandf{ rss}]~:  To change the size of special solution point symbols.
\end{itemize}


\section{ Printing {\cal PLAUT} Files.} \label{sec:Printing_PLAUT_files}
\begin{itemize}
\item[\commandf{ @ps}]~:
  Type {\it @ps fig.1} to convert a saved {\cal PLAUT} file \filef{ fig.1} 
  to {\cal PostScript} format
  in \filef{ fig.1.ps}.
\item[\commandf{ @pr}]~:
  Type {\it @pr fig.1} to convert a {\cal PLAUT} file \filef{ fig.1} to 
  {\cal PostScript} format and to
  print the resulting file \filef{ fig.1.ps}.
\end{itemize}

