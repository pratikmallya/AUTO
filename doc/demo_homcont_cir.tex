
%==============================================================================
%==============================================================================
\chapter{ {\cal HomCont} Demo : cir.} \label{ch:HomCont_cir}
%==============================================================================
%==============================================================================

%==============================================================================
%DEMO=cir======================================================================
%==============================================================================
\section{ Electronic Circuit of Freire {\it et al}.}
Consider the following model of a three-variable electronic circuit
\cite{FrRLuGaPo:93}
 \beq
\left \{ 
\begin{array}{rcl}
\dot{x} & = & \left [-(\beta+\nu) x + \beta y -a_3 x^3 
+b_3(y-x)^3\right ]/r, \\
\dot{y} & = & \beta x -(\beta+\gamma)y -z -b_3(y-x)^3, \\
\dot{z} & = & y.
\end{array}
\right.  
\label{5.fr1}
\eeq
These autonomous equations are also considered in the \AUTO demo \filef{ tor}.

First, we copy the demo into a new directory and compile
\begin{center}
\commandf{ @dm cir} \\
\end{center}
The system is contained in 
the equation-file \filef{ cir.c} and the initial run-time constants
are stored in \filef{ c.cir.1} and \filef{ h.cir.1}. We begin by starting from
the data from \filef{ cir.dat} for a saddle-focus homoclinic orbit 
at 
$\nu=-0.721309$, $\beta=0.6$, $\gamma=0$, $r=0.6$, $A_3=0.328578$ 
and $B_3=0.933578$, which was obtained by shooting over 
the time interval $2T=$\parf{ PAR(11)}$=36.13$.
We wish to follow the branch in the $(\beta,\nu)$-plane, but 
first we perform continuation in $(T,\nu)$ to obtain a better 
approximation to a homoclinic orbit.
\begin{center}
\commandf{ make first}
\end{center} 
yields the output
\begin{verbatim}
 BR  PT  TY LAB     PERIOD       L2-NORM    ...   PAR(1)     
  1  21  UZ   2  1.000000E+02  1.286637E-01 ... -7.213093E-01
  1  42  UZ   3  2.000000E+02  9.097899E-02 ... -7.213093E-01
  1  50  EP   4  2.400000E+02  8.305208E-02 ... -7.213093E-01
\end{verbatim}
Note that $\nu=$\parf{ PAR(1)} remains constant during the continuation
as the parameter values do not change, only the the length of
the interval over which the approximate homoclinic solution is computed.
Note from the eigenvalues, stored in \filef{ d.1} that this is a homoclinic
orbit to a saddle-focus with a one-dimensional unstable manifold.

We now restart at \parf{ LAB=3}, corresponding to a time interval $2T=200$,
and change the principal continuation parameters to be $(\nu,\beta)$.
The new constants defining the continuation are given in \filef{ c.cir.2}
and \filef{ h.cir.2}.
We also activate the test functions pertinent to codimension-two
singularities which may be encountered along a branch of saddle-focus
homoclinic orbits, viz.\ $\psi_2$, $\psi_4$, $\psi_5$, $\psi_9$ and $\psi_{10}$.
This must be specified in three ways: by choosing \parf{ NPSI=5} and appropriate
\parf{ IPSI(I)} in \filef{ h.cir.2}, by adding the corresponding parameter labels
to the list of continuation parameters \parf{ ICP(I)} in \filef{ c.cir.2}
(recall that these parameter indices are 20 more than the corresponding
$\psi$ indices), and finally adding USZR functions defining zeros of
these parameters in \filef{ c.cir.2}. Running 
\begin{center}  
\commandf{ make second}
\end{center} 
results in
\begin{verbatim}
BR  PT  TY LAB    PAR(1)     ...    PAR(2)     ...    PAR(25)       PAR(29)    
1   17  UZ   5 -7.256925E-01 ...  4.535645E-01 ... -1.765251E-05 -2.888436E-01
1   75  UZ   6 -1.014704E+00 ...  9.998966E-03 ...  1.664509E+00 -5.035997E-03
1   78  UZ   7 -1.026445E+00 ... -2.330391E-05 ...  1.710804E+00  1.165176E-05
1   81  UZ   8 -1.038012E+00 ... -1.000144E-02 ...  1.756690E+00  4.964621E-03  
1  100  EP   9 -1.164160E+00 ... -1.087732E-01 ...  2.230329E+00  5.042736E-02
\end{verbatim}
with results saved in \filef{ b.2, s.2, d.2}.
\begin{figure}[p]
\epsfysize 9.0cm
\centerline{\epsffile{include/cir1.ps}}
\caption{Solutions of the boundary value problem at labels 6 and 8, 
either side of the Shil'nikov-Hopf bifurcation}
\label{Fcircuit1}
\end{figure}
\begin{figure}[p]
\epsfysize 9.0cm
\centerline{\epsffile{include/cir2.ps}}
\caption{Phase portraits of three homoclinic orbits 
on the branch, showing the saddle-focus to saddle transition}
\label{Fcircuit2}
\end{figure}
Upon inspection of the output, note that label 5, where \parf{ PAR(25)}$\approx 0$, 
corresponds to a neutrally-divergent saddle-focus, $\psi_5=0$. 
Label 7, where \parf{ PAR(29)}$\approx 0$ corresponds to a local bifurcation, $\psi_9=0$, 
which we note from the eigenvalues stored in \filef{ d.2} corresponds to a \emp{
Shil'nikov-Hopf} bifurcation. Note that \parf{ PAR(2)} is also approximately zero
at label 7, which accords with the analytical observation that the origin of
(\ref{5.fr1}) undergoes a Hopf bifurcation when $\beta=0$.
Labels 6 and 8 are the user-defined output
points, the solutions at which are plotted in Fig.\ \ref{Fcircuit1}.
Note that solutions beyond label 7 (e.g., the plotted solution
at label 8) do not correspond to homoclinic orbits, but to 
\emp{ point-to-cycle} heteroclinic orbits (c.f.\ Section~2.2.1 of
\citeasnoun{ChKuSa:95}).

We now continue in the other direction along the branch. It turns out
that starting from the initial point in the other direction results in
missing a codim 2 point which is close to the starting point. Instead we
start from the first saved point from the previous computation
(label 5 in \filef{ s.2}):
\begin{center}
\it make third
\end{center}
The output
\begin{verbatim}
 BR  PT  TY LAB    PAR(1)     ...    PAR(2)        PAR(22)       PAR(24)    
  1   9  UZ  10 -7.204001E-01 ...  5.912315E-01 -1.725669E+00 -3.295862E-05
  1  18  UZ  11 -7.590583E-01 ...  7.428734E-01  3.432139E-05 -2.822988E-01
  1  26  UZ  12 -7.746686E-01 ...  7.746147E-01  5.833163E-01  1.637611E-07
  1  28  EP  13 -7.746628E-01 ...  7.746453E-01  5.908902E-01  1.426214E-04
\end{verbatim}
contains a neutral saddle-focus (a \emp{ Belyakov} transition) 
at \parf{ LAB=10} ($\psi_4=0$), a double real leading eigenvalue 
(saddle-focus to saddle transition) at \parf{ LAB =11} ($\psi_2=0$) 
and a neutral saddle at \parf{ LAB=12} ($\psi_4=0$). Data at several
points on the complete branch are plotted in Fig.\ \ref{Fcircuit2}.
If we had continued further (by increasing \parf{ NMX}), 
the computation would end at a no convergence error \parf{ TY=MX} owing 
to the homoclinic branch approaching a Bogdanov-Takens singularity 
at small amplitude. To compute further towards the BT point 
we would first need to continue to a higher value of \parf{ PAR(11)}.

\section{ Detailed \AUTO-Commands.}
\begin{table}[htbp]
\begin{center}
\begin{tabular}{| l | l |}
\hline
  \AUTO-COMMAND  & ACTION \\
\hline
%==============================================================================
  \commandf{ ! mkdir cir} & create an empty work directory \\ 
  \commandf{ cd cir} & change directory \\
  \commandf{ demo('cir')} & copy the demo files to the work directory \\
\hline
%==============================================================================
  \commandf{ us('cir')} & use the starting data in \filef{ cir.dat} to create \filef{ s.dat} \\ 
  \commandf{ run(c='cir.1',h='cir.1',s='dat')} &  increase the truncation interval; restart from \filef{ s.dat}\\ 
  \commandf{ sv('1')} & save output-files as \filef{ b.1, s.1, d.1} \\ 
\hline
%==============================================================================
  \commandf{ run(c='cir.2',h='cir.2',s='1')} &  continue saddle-focus homoclinic orbit; restart from \filef{ s.1} \\ 
  \commandf{ sv('2')} & save output-files as \filef{ b.2, s.2, d.2} \\ 
\hline
%=============================================================================
  \commandf{ run(c='cir.3',h='cir.3',s='2')} & generate adjoint variables  ; restart from \filef{ s.2} \\ 
  \commandf{ ap('2')} & append output-files as \filef{ b.2, s.2, d.2} \\ 
\hline
%==============================================================================
\end{tabular}
\caption{Detailed \AUTO-Commands for running demo \filef{ cir}.}
\label{tbl:demo_cir_1}
\end{center}
\end{table}


