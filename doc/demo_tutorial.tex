

%==============================================================================
%==============================================================================
\chapter{ \AUTO Demos : Tutorial.} \label{ch:Demos:_Tutorial}
%==============================================================================
%==============================================================================
\newpage
\section{ Introduction.} \label{sec:Tutorial_Introduction}
The directory \filef{ auto/2000/demos} has a large number of subdirectories,
for example \filef{ ab}, \filef{ pp2}, \filef{ exp}, etc.,
each containing all necessary files for certain illustrative calculations.
Each subdirectory, say \filef{ xxx}, corresponds to a particular equation
and contains one equations-file \filef{ xxx.c}
and one or more constants-files \filef{ c.xxx.i}, 
one for each successive run of the demo.
To see how the equations have been programmed, inspect the equations-file. 
To understand in detail how \AUTO is instructed to carry out a 
particular task, inspect the appropriate constants-file.
In this chapter we describe the tutorial demo \filef{ ab} in detail.
A brief description of other demos is given in later chapters.


\section{ ab : A Tutorial Demo.} \label{sec:Demos_ab}
%==============================================================================
%DEMO=ab=======================================================================
%==============================================================================
This demo illustrates the computation of 
stationary solutions,
Hopf bifurcations 
and 
periodic solutions,
and the computation loci of folds and Hopf bifurcation points.
The equations, that model an A $\to$ B  reaction, are those from
\citename{URP:74} \citeyear{URP:74}, namely
\begin{equation} \begin{array}{cl}
  u_1 ' &=  -u_1 + p_1 (1-u_1) e^{u_2}, \\
  u_2 ' &=  -u_2 +  p_1 p_2 ( 1-u_1) e^{u_2} - p_3 u_2.\\
\end{array} \end{equation}

\section{ Copying the Demo Files.}  \label{sec:Tutorial_copying}
The commands listed in Table~\ref{tbl:demo_ab_1}
will copy the demo files to your work directory.

\begin{table}[htbp]
\begin{center}
\begin{tabular}{| l | l |}
\hline
  {\cal Unix}-COMMAND  & ACTION \\
\hline
%==============================================================================
  \commandf{ auto}  & start the AUTO2000 Command Line User Interface\\ 
\hline
  \AUTO-COMMAND  & ACTION \\
\hline
  \commandf{ cd } & go to main directory (or other directory).\\
  \commandf{ ! mkdir ab}  & \parbox[t]{3in}{create an empty work directory.  
                            Note:  the '!' is used to signify a command 
                            which is sent to the shell.\vspace{0.2cm}}\\ 
  \commandf{ cd ab}  & change to the work directory.\\

  \commandf{ demo('ab')}  & copy the demo files to the work directory.\\
\hline
%==============================================================================
\end{tabular}
\caption{Copying the demo \filef{ ab} files.}
\label{tbl:demo_ab_1}
\end{center}
\end{table}

At this point you may want to see what files have been copied
to the work directory. 
In particular, you may want to edit the equations-file \filef{ ab.c}
to see how the equations have been entered (in subroutine \funcf{ func})
and how the starting solution has been set (in subroutine \funcf{ stpnt}).
Note that, initially, $p_1=0$ $p_2=14$, and $p_3=2$, for which
$u_1=u_2=0$ is a stationary solution.
 
\newpage
\section{ Executing all Runs Automatically.} \label{sec:Tutorial_all_runs}
To execute all prepared runs of demo \filef{ ab},
simply type one or both of the command given in Table~\ref{tbl:demo_ab_2}.

\begin{table}[htbp]
\begin{center}
\begin{tabular}{| l | l |}
\hline
  \AUTO-COMMAND  & ACTION \\
\hline
%==============================================================================
  \commandf{demofile('ab\_old.auto')}  & \parbox[t]{3in}{execute all runs of demo \filef{ ab} interactively using a new constants file for each run\vspace{0.2cm}}\\ 
  \commandf{demofile('ab\_new.auto')}  & \parbox[t]{3in}{execute all runs of demo \filef{ ab} interactively by modifying the constants file before each run\vspace{0.2cm}}\\ 
\hline
%==============================================================================
\end{tabular}
\caption{Executing all runs of demo \filef{ ab}.}
\label{tbl:demo_ab_2}
\end{center}
\end{table}

Each of the commands in Table~\ref{tbl:demo_ab_2} begins
a tutorial which will proceed one step each time
the user presses a key.  Each step consists of a
single \AUTO command preceded by instructions as
to what action the command performs.
The tutorial script \filef{ab\_old.auto} performs the
demo by reading in a sequence of \AUTO constants files
each of which corresponds to a step of the demo.
The tutorial script \filef{ab\_new.auto} performs the
demo by reading in a single \AUTO constants file
and then interactively modifying it to perform
each of the demo.  Both are valid and effective methods
for running \AUTO, with \filef{ab\_old.auto} being
similar to the way \AUTO was used before the advent of
the CLUI, and \filef{ab\_new.auto} using new functionality
provided by the CLUI.

Note that there are five separate runs.
In the first run, a branch of stationary solutions is traced out.
Along it, two folds (LP) and one Hopf bifurcation (HB) are located.
The free parameter is $p_1$. The other parameters remain fixed in this run.
Note also that only special, labeled solution points are printed on the screen.
More detailed results are saved 
in the data-files \filef{ b.ab}, \filef{ s.ab}, and \filef{ d.ab}.

The second run traces out the branch of periodic solutions that emanates
from the Hopf bifurcation. The free parameters are $p_1$ and the period.
The detailed results are appended to the existing data-files 
\filef{ b.ab}, \filef{ s.ab},
and \filef{ d.ab}.

In the third run, one of the folds detected in the first run is followed in
the two parameters $p_1$ and $p_3$, while $p_2$ remains fixed.
The fourth run continues this branch in opposite direction.
Similarly, in the fifth run, the Hopf bifurcation located in the first run 
is followed in the two parameters $p_1$ and $p_3$.
(In this example this is done in one direction only.)
The detailed results of these continuations are accumulated
in the data-files \filef{ b.2p}, \filef{ s.2p}, and \filef{ d.2p}.

The numerical results are given below
in somewhat abbreviated form.
Some differences in output are to be expected on different machines.
This does not mean that the results have different accuracy, but simply
that arithmetic differences have accumulated from step to step, possibly
leading to different step size decisions.

One could now use the \AUTO CLUI to graphically inspect the contents of the
data-files, but we shall do this later.
However, it may be useful to edit these files to view their contents.

Next, reset the work directory, by typing the command given
in Table~\ref{tbl:demo_ab_3}.

\newpage
\begin{table}[htbp]
\begin{center}
\begin{tabular}{| l | l |}
\hline
  \AUTO-COMMAND  & ACTION \\
\hline
%==============================================================================
  \commandf{ cl()}  & remove temporary files of demo \filef{ ab} \\ 
  \commandf{ dl('ab')}  & remove 'ab' data-files of demo \filef{ ab} \\ 
  \commandf{ dl('2p')}  & remove '2p' data-files of demo \filef{ ab} \\ 
\hline
%==============================================================================
\end{tabular}
\caption{Cleaning the demo \filef{ ab} work directory.}
\label{tbl:demo_ab_3}
\end{center}
\end{table}

\begin{center}
\vspace{-0.2in}
\begin{verbatim}
ab : first run : stationary solutions
 
BR    PT  TY LAB   PAR(1)       L2-NORM        U(1)         U(2)     
 1     1  EP   1  0.00000E+00  0.00000E+00  0.00000E+00  0.00000E+00
 1    33  LP   2  1.05739E-01  1.48439E+00  3.11023E-01  1.45144E+00
 1    70  LP   3  8.89318E-02  3.28824E+00  6.88982E-01  3.21525E+00
 1    90  HB   4  1.30899E-01  4.27186E+00  8.95080E-01  4.17704E+00
 1    92  EP   5  1.51241E-01  4.36974E+00  9.15589E-01  4.27275E+00
 Saved as *.ab
 
ab : second run : periodic solutions
 
BR    PT  TY LAB   PAR(1)       L2-NORM      MAX U(1)     MAX U(2)     PERIOD    
 4    30       6  1.19881E-01  3.98712E+00  9.91911E-01  7.02034E+00  2.721E+00
 4    60       7  1.15303E-01  3.14630E+00  9.99577E-01  9.95764E+00  6.147E+00
 4    90       8  1.05650E-01  2.21917E+00  9.99166E-01  9.36609E+00  1.399E+01
 4   120       9  1.05507E-01  1.69684E+00  9.99086E-01  9.29629E+00  9.956E+01
 4   150  EP  10  1.05507E-01  1.60388E+00  9.99789E-01  9.28146E+00  1.867E+03
 Appended to *.ab
 
ab : third run : a 2-parameter locus of folds
 
BR    PT  TY LAB   PAR(1)       L2-NORM        U(1)         U(2)       PAR(3)     
 2    27  LP  11  1.35335E-01  2.06012E+00  4.99653E-01  1.99861E+00  2.499E+00
 2   100  EP  12  1.09381E-08  2.13650E+01  9.53147E-01  2.13437E+01 -3.748E-01
 Saved as *.2p
 
ab : fourth run : the locus of folds in reverse direction
 
BR    PT  TY LAB   PAR(1)       L2-NORM        U(1)         U(2)       PAR(3)     
 2    35  EP  11 -1.31939E-03  9.96432E-01 -3.58651E-03  9.96426E-01 -1.050E+00
 Appended to *.2p
 
ab : fifth run : a 2-parameter locus of Hopf points
 
BR    PT  TY LAB   PAR(1)       L2-NORM        U(1)         U(2)       PAR(3)     
 4   100  EP  11  8.80940E-05  1.17440E+01  9.14609E-01  1.17083E+01  9.362E-02
 Appended to *.2p
\end{verbatim}
\end{center}

\newpage

\section{ Executing Selected Runs Automatically.} \label{sec:Tutorial_selected_runs}
As illustrated by the commands in Table~\ref{tbl:demo_ab_4}, 
one can also execute selected runs of demo \filef{ ab}.
In general, this cannot be done in arbitrary order, as any given
run may need restart data from a previous run.
Run~3 only requires the results of Run~1, so that the displayed 
command sequence is indeed appropriate.
The screen output of these runs will be identical to that of
the corresponding earlier runs, except for a change in solution 
labels in Run~3.

In real use there are two mains ways in which the \AUTO can
be used.  First, one can prepare a constants-file for each run.
In the illustrative runs below, the constants-files 
were carefully prepared in advance.
For example, the file \filef{ c.ab.1} contains the \AUTO-constants for Run~1,
\filef{ c.ab.3} contains the \AUTO-constants for Run~3, etc.

\begin{table}[htbp]
\begin{center}
\begin{tabular}{| l | l |}
\hline
  \AUTO-COMMAND  & ACTION \\
\hline
%==============================================================================
  \commandf{ ld("ab")}  & load the problem definition  \filef{ ab} \\ 
  \commandf{ run(c="ab.1")}  & execute the run which uses the constants in \filef{c.ab.1} \\ 
  \commandf{ sv("ab") }	&  save the results of the run into the files \filef{b.ab}, \filef{s.ab}, and \filef{d.ab} \\
  \commandf{ run(c="ab.3",s="ab")}  & execute the third run of demo \filef{ ab} \\ 
\hline
%==============================================================================
\end{tabular}
\caption{Selected runs of demo \filef{ ab}.}
\label{tbl:demo_ab_4}
\end{center}
\end{table}

On the other hand, one can use the CLUI to generate the constants
file at runtime.  In the example below, the constant file \filef{c.ab.1}
will be read in, and the CLUI will be used to make the appropriate
changes to perform the same calculation as in  
Table~\ref{tbl:demo_ab_4}.

\begin{table}[htbp]
\begin{center}
\begin{tabular}{| l | l |}
\hline
  \AUTO-COMMAND  & ACTION \\
\hline
%==============================================================================
  \commandf{ ld("ab")}  & load the problem definition  \filef{ ab} \\ 
  \commandf{ run(c="ab.1")}  & execute the run which uses the constants in \filef{c.ab.1} \\ 
  \commandf{ sv("ab") }	&  save the results of the run into the files \filef{b.ab}, \filef{s.ab}, and \filef{d.ab} \\
  \commandf{ cc("IRS",2) } &  start the new calculation from a solution with label 2 \\
  \commandf{ cc("ICP",[1,3]) } &  since we are following a locus of folds we require two free parameters \\
  \commandf{ cc("ISP",0) } & turn off detection of branch points \\
  \commandf{ cc("ISW",2) } & \parbox[t]{4in}{since we start at a fold the ISW parameter indicates we desire to compute a locus of such points \vspace{0.2cm}} \\
  \commandf{ cc("DSMAX",0.5) } & increase the maximum allowed step size \\
  \commandf{ run(s="ab")}  & execute the third run of demo \filef{ ab} \\ 
\hline
%==============================================================================
\end{tabular}
\caption{Selected runs of demo \filef{ ab}.}
\label{tbl:demo_ab_4a}
\end{center}
\end{table}

\section{ Using \AUTO-Commands.} \label{sec:Tutorial_AUTO_commands}
Next, with the commands in Table~\ref{tbl:demo_ab_5}, we execute the
first two runs of demo \filef{ ab} again, using commands similar
Table~\ref{tbl:demo_ab_5} that one would normally use in an actual
application.  We still use the demo constants-files that were prepared
in advance and assume you are in the directory into which
the \filef{ab} demo has already been copied

\begin{table}[htbp]
\begin{center}
\begin{tabular}{| l | l |}
\hline
  \AUTO-COMMAND  & ACTION \\
\hline
%==============================================================================
  \commandf{ cl()}  &  remove temporary files of any previous runs of the demo\\ 
  \commandf{ dl("ab")}  &  remove 'ab' data-files of any previous runs of the demo\\ 
  \commandf{ dl("2p")}  &  remove '2p' data-files of any previous runs of the demo\\ 
  \commandf{ ld("ab")}  &  make sure the problem definition is loaded\\ 
\hline
%==============================================================================
  \commandf{ run(c="ab.1")} & compute a stationary solution branch with folds and Hopf bifurcation \\  
  \commandf{ sv("ab")} & save output-files as \filef{ b.ab, s.ab, d.ab} \\ 
\hline
%==============================================================================
  \commandf{ run(c="ab.2",s="ab")} & compute a branch of periodic solutions from the Hopf point \\ 
  \commandf{ ap("ab")} & append the output-files to \filef{ b.ab, s.ab, d.ab} \\ 
\hline
%==============================================================================
\end{tabular}
\caption{Commands for Run~1 and Run~2 of demo \filef{ ab}.}
\label{tbl:demo_ab_4}
\end{center}
\end{table}
 
It is instructive to look at the constants-files
\filef{ c.ab.1} and \filef{ c.ab.2} used in the two runs above.
The significance of each \AUTO-constant set in these files
can be found in Chapter~\ref{ch:AUTO_constants}.
Note in particular the \AUTO-constants that were changed 
between the two runs; see Table~\ref{tbl:demo_ab_6}.
\begin{table}[htbp]
\begin{center}
\begin{tabular}{| l | r | r | l |}
\hline
  Constant &  Run~1  &  Run~2 & Reason for Change \\
\hline
%==============================================================================
  \parf{ IPS}  & 1  & 2  &  To compute periodic solutions in Run~2 \\  
\hline
  \parf{ IRS}  & 0  & 4  &  To specify the Hopf bifurcation restart label \\  
\hline
  \parf{ NICP}  & 1  & 2  &  The second run has two free parameters\\  
\hline
  \parf{ ICP}  & 1  &1,~10  &  To use and print \parf{ PAR(1)} and \parf{ PAR(10)} in Run~2\\  
\hline
  \parf{ NMX}  & 100 &150  &  To allow more continuation steps in Run~2 \\  
\hline
  \parf{ NPR}  & 100 & 30  &  To print output every 30 steps in Run~2 \\  
\hline
%==============================================================================
\end{tabular}
\caption{Differences in \AUTO-constants between \filef{ c.ab.1} and \filef{ c.ab.2}.}
\label{tbl:demo_ab_6}
\end{center}
\end{table}

Actually, for periodic solutions, \AUTO automatically adds \parf{ PAR(10)}
(the period) as second parameter.
However, for the period to be printed, one must specify the index 10
in the \parf{ ICP} list, as shown in Table~\ref{tbl:demo_ab_6}.

\section{ Plotting the Results with \AUTO.} \label{sec:Tutorial_plotting}
The bifurcation diagram computed in the runs above
is stored in the file \filef{ b.ab},
while each labeled solution is fully stored in \filef{ s.ab}.
To use \AUTO to graphically inspect these data-files,
type the \AUTO-command given in Table~\ref{tbl:demo_ab_7}.
The saved plots are shown in Figure~\ref{fig:ab_1}
and in Figure~\ref{fig:ab_2}.

Figure~\ref{fig:ab_1} shows the default view of the plotting tool,
which consists of a representation of the bifurcation diagram.  Step by step
instructions for creating Figure~\ref{fig:ab_2} are given below.

The plotting window consists of a menubar at the top, a plotting area,
and a control panel with four control widgets at the bottom.  The
first step in creating Figure~\ref{fig:ab_2} is to change the mode of
the plotting tool from ``bifurcation'' to ``solution''.  This is
accomplished by clicking on the widget marked ``Type'' on the bottom
control panel and setting it from ``bifurcation'' to ``solution''.  In
the plotting window will appear a plot of the first labeled solution in
\filef{s.ab}.  Unfortunately, this is an equilibrium solution, so only
a single point is plotted.  Since we wish to plot the periodic
solutions, we modify the widget marked ``Label'' by changing its value
from ``[1]'' to ``[6,7,10]'' (don't forget to hit the return key when
you are done modifying the value).  This signifies that instead of
plotting the solution with label 1 we want to plot the solutions with
labels 6, 7, and 10 simultaneously.  In the plotting window we now
have three curves, each of which is a plot of time versus the value of
the first state variable.  If we want a different plot, say the
values of the two state variables plotted against each other, we use
the two remaining widgets in the control panel, labeled ``X'' and
``Y''.  For example, if change the value of ``X'' from ``['t']'' to
``[0]'' and the value of ``Y'' from ``[0]'' to ``[1]'' we get a phase
plot of the period solutions (don't forget to hit the return key
when you are done modifying each value).  This plot is shown in
Figure~\ref{fig:ab_2}.  

The plotting tool can also be used to create Postscript files from
plots by selecting the ``File'' on the menubar and then selecting the
``Save Postscript...'' from the drop down menu.  This will bring up
a dialog into which the user can enter the filename of the postscript
file to save the plot in.  
Further information on the plotting tool can be found in
Section~\ref{clui:2d plotting}.

\begin{table}[htbp]
\begin{center}
\begin{tabular}{| l | l |}
\hline
  \AUTO-COMMAND  & ACTION \\
\hline
%==============================================================================
  \commandf{ plot("ab")} & run \AUTO to graph the contents of \filef{ b.ab} and \filef{ s.ab}; \\  
%==============================================================================
\hline
\end{tabular}
\caption{Command for plotting the files \filef{ b.ab} and \filef{ s.ab}.}
\label{tbl:demo_ab_7}
\end{center}
\end{table}

\begin{figure}[p]
\epsfysize 9.0cm
\centerline{\epsffile{include/ab1.ps}}
\caption{The bifurcation diagram of demo \filef{ ab}.}
\label{fig:ab_1}
\end{figure}
\begin{figure}[p]
\epsfysize 9.0cm
\centerline{\epsffile{include/ab2.ps}}
\caption{The phase plot of solutions 6, 7, and 10 in demo \filef{ ab}.}
\label{fig:ab_2}
\end{figure}


\section{ Following Folds and Hopf Bifurcations.} \label{sec:Tutorial_2_par}
The commands in Table~\ref{tbl:demo_ab_9} will execute the remaining
runs of demo \filef{ ab}.
Here, as in later demos, some of the \AUTO-constants that have been changed
between runs are indicated in the Table.
%==============================================================================
%==============================================================================
\begin{table}[htbp]
\begin{center}
\begin{tabular}{| l | l |}
\hline
   \AUTO-COMMAND  & ACTION \\
\hline
  \commandf{ run(c="ab.2",s="ab")} & \parbox[t]{4in}{ compute a locus of folds with changes (from \filef{ c.ab.2}) : IRS, NICP, ICP, ISW, DSMAX \vspace{0.2cm}}\\ 
  \commandf{ sv("2p")} & save output-files as \filef{ b.2p, s.2p, d.2p} \\ 
\hline
%==============================================================================
  \commandf{ run(c="ab.3",s="ab")} &  \parbox[t]{4in}{compute the  locus of folds in reverse direction with changes (from \filef{ c.ab.3}) : DS (sign)\vspace{0.2cm}}\\ 
  \commandf{ ap("2p")} &  append the output-files to \filef{ b.2p, s.2p, d.2p} \\ 
\hline
%==============================================================================
  \commandf{ run(c="ab.4",s="ab")} &  \parbox[t]{4in}{compute a locus of Hopf points with changes (from \filef{ c.ab.4}) : IRS\vspace{0.2cm}}\\ 
  \commandf{ ap("2p")} & append the output-files to \filef{ b.2p, s.2p, d.2p} \\ 
\hline
%==============================================================================
  \commandf{ run(c="ab.5",s="ab")} &  \parbox[t]{4in}{compute a locus of Hopf points in reverse direction with changes (from \filef{ c.ab.5}) : IRS\vspace{0.2cm}}\\ 
  \commandf{ ap("2p")} & append the output-files to \filef{ b.2p, s.2p, d.2p} \\ 
\hline
%==============================================================================
\end{tabular}
\caption{Commands for Runs~3, 4, and 5 of demo \filef{ ab}.}
\label{tbl:demo_ab_9}
\end{center}
\end{table}

\section{ Relabeling Solutions in the Data-Files.} \label{sec:Tutorial_relabeling}
Next we want to plot the two-parameter diagram computed in the last three runs.
However, the solution labels in these runs are not distinct.
This is due to the fact that in each of these three runs
the restart solution was read from \filef{ s.ab}, while the
computed solutions were stored in \filef{ s.2p}.
Consequently, these runs were unaware of each other's results, 
which led to non-unique labels.
For relabeling purpose, and more generally for file maintenance,
there is a utility program that can be invoked as indicated in 
Table~\ref{tbl:demo_ab_10}.
Its use is illustrated in Table~\ref{tbl:demo_ab_11}.
\begin{table}[htbp]
\begin{center}
\begin{tabular}{| l | l |}
\hline
  \AUTO-COMMAND  & ACTION \\
\hline
  \commandf{ rl("2p")} & run the relabeling  program on \filef{ b.2p} and \filef{ s.2p} \\ 
\hline
%==============================================================================
\end{tabular}
\caption{Command to run the relabeling program on \filef{ b.2p} and \filef{ s.2p}.}
\label{tbl:demo_ab_10}
\end{center}
\end{table}


\begin{table}[htbp]
\begin{center}
\begin{tabular}{| c | l |}
\hline
  RELABELING COMMAND  & ACTION \\
\hline
  l & list the labeled solutions in \filef{ s.2p} \\
  r & relabel the solutions  \\  
  l & list the new solution labeling  \\
  w & rewrite \filef{ b.2p} and \filef{ s.2p}  \\
\hline
%==============================================================================
\end{tabular}
\caption{Relabeling commands for the files \filef{ b.2p} and \filef{ s.2p}.}
\label{tbl:demo_ab_11}
\end{center}
\end{table}
\section{ Plotting the 2-Parameter Diagram.} \label{sec:Tutorial_plotting_2p}
To plot the files  \filef{ b.2p} and \filef{ s.2p},
enter the command listed in Table~\ref{tbl:demo_ab_12}.
The saved plot is shown in Figure~\ref{fig:ab_3}.

\begin{table}[htbp]
\begin{center}
\begin{tabular}{| l | l |}
\hline
  \AUTO-COMMAND  & ACTION \\
\hline
  \commandf{ plot("2p")} & run to graph the contents of \filef{ b.2p} and \filef{ s.2p}; \\ 
\hline
\end{tabular}
\caption{Command to plot the files \filef{ b.2p} and \filef{ s.2p}.}
\label{tbl:demo_ab_12}
\end{center}
\end{table}

\begin{figure}[t]
\epsfysize 9.0cm
\centerline{\epsffile{include/ab3.ps}}
\caption{Loci of folds and Hopf bifurcations for demo \filef{ ab}.}
\label{fig:ab_3}
\end{figure}


%%% Local Variables: ***
%%% tex-main-file: "auto.tex" ***








