%==============================================================================
%==============================================================================
\chapter{ Description of \AUTO-Constants.} \label{ch:AUTO_constants}
%==============================================================================
%==============================================================================
\section{ The \AUTO-Constants File.} \label{sec:The_AUTO_constants_file}
As described in Section~\ref{sec: User_supplied_files}, 
if the equations-file is \filef{ xxx.c} 
then the constants that define the computation 
are normally expected in the file  \filef{ c.xxx}.
The general format of this file is the same for all \AUTO runs.
For example, the file \filef{ c.ab} 
in directory \filef{ auto/2000/demos/ab} is listed below.
(The tutorial demo \filef{ ab} is described in detail in 
Chapter~\ref{ch:Demos:_Tutorial}.)  

\begin{verbatim}
2 1 0 1               NDIM,IPS,IRS,ILP
1   1                 NICP,(ICP(I),I=1,NICP)
50 4 3 1 1 0 0 0      NTST,NCOL,IAD,ISP,ISW,IPLT,NBC,NINT
100 0. 0.15 0. 100.   NMX,RL0,RL1,A0,A1
100 10 2 8 5 3 0      NPR,MXBF,IID,ITMX,ITNW,NWTN,JAC
1.e-6 1.e-6 0.0001    EPSL,EPSU,EPSS
0.01 0.005 0.05 1     DS,DSMIN,DSMAX,IADS
1                     NTHL,((I,THL(I)),I=1,NTHL)
11 0.
0                     NTHU,((I,THU(I)),I=1,NTHU)
0                     NUZR,((I,UZR(I)),I=1,NUZR)
\end{verbatim}

The significance of the \AUTO-constants, grouped by function, is 
described in the sections below. 
Representative demos that illustrate use of the \AUTO-constants
are also mentioned.

%=====================================================================
\section{ Problem Constants.} \label{sec:Problem_constants}
\subsection{\parf{ NDIM}}  \label{sec:NDIM}

%AUTODOC NDIM START
%AUTODOC NDIM SHORT
% Dimension of the system of equations as specified in the user-supplied
% subroutine func.
%AUTODOC NDIM FULL
% Dimension of the system of equations as specified in the user-supplied
% subroutine func.
%AUTODOC NDIM TEX

 Dimension of the system of equations as specified in the user-supplied
 subroutine \funcf{ func}.
%AUTODOC NDIM END

\subsection{\parf{ NBC}}  \label{sec:NBC}

%AUTODOC NBC START
%AUTODOC NBC SHORT
% The number of boundary conditions as specified in the user-supplied 
% subroutine bcnd.
%AUTODOC NBC FULL
% The number of boundary conditions as specified in the user-supplied 
% subroutine bcnd.
% (Demos exp, kar.)
%AUTODOC NBC TEX

 The number of boundary conditions as specified in the user-supplied
 subroutine \funcf{ bcnd}. 

 (Demos \filef{ exp}, \filef{ kar}.)
%AUTODOC NBC END

\subsection{\parf{ NINT}}  \label{sec:NINT}

%AUTODOC NINT START
%AUTODOC NINT SHORT
% The number of integral conditions as specified in the user-supplied sub
% routine icnd.
%AUTODOC NINT FULL
% The number of integral conditions as specified in the user-supplied sub
% routine icnd.
% (Demos int, lin, obv.)
%AUTODOC NINT TEX

 The number of integral conditions as specified in the user-supplied
 subroutine \funcf{ icnd}. 

(Demos \filef{ int}, \filef{ lin}, \filef{ obv}.)
%AUTODOC NINT END

\subsection{\parf{ JAC}}  \label{sec:JAC}

%AUTODOC JAC START
%AUTODOC JAC SHORT
% Used to indicate whether derivatives are supplied by the user or to be
% obtained by differencing.
%AUTODOC JAC FULL
% Used to indicate whether derivatives are supplied by the user or to be
% obtained by differencing :
% JAC=0 : No derivatives are given by the user. (Most demos use JAC=0.)
% JAC=1 : Derivatives with respect to state and problemparameters
%     are given in the user-supplied subroutines func, bcnd, icnd and fopt,
%     where applicable. This may be necessary for sensitive problems. It
%     is also recommended for computations in which AUTO generates an
%     extended system, for example, when ISW=2.
%     (Demos int, dd2, obt, plp, ops.)
%AUTODOC JAC TEX

 Used to indicate whether derivatives are supplied by the user
 or to be obtained by differencing~:
\begin{itemize}
\item[-] \parf{ JAC=0}~: 
  No derivatives are given by the user. (Most demos use \parf{ JAC}=0.)
\item[-] \parf{ JAC=1}~:  
  Derivatives with respect to state- and problem-parameters are given 
  in the user-supplied subroutines 
  \funcf{ func}, \funcf{ bcnd}, \funcf{ icnd} and \funcf{ fopt}, where 
  applicable.  This may be necessary for sensitive problems. 
  It is also recommended for computations in which \AUTO generates 
  an extended system, for example, when \parf{ ISW}=2.

  (Demos \filef{ int}, \filef{ dd2}, \filef{ obt}, \filef{ plp}, \filef{ ops}.)

  (For \parf{ ISW} see Section~\ref{sec:ISW}.) 
\end{itemize}
%AUTODOC JAC END
%=====================================================================
\section{ Discretization Constants.} \label{sec:Discretization_constants}
\subsection{\parf{ NTST}}  \label{sec:NTST}

%AUTODOC NTST START
%AUTODOC NTST SHORT
% The number of mesh intervals used for discretization. 
%AUTODOC NTST FULL
% The number of mesh intervals used for discretization. NTST remains fixed
% during any particular run, but can be changed when restarting. Recom
% mended value of NTST : As small as possible to maintain convergence.
% (Demos exp, ab, spb.)
%AUTODOC NTST TEX

 The number of mesh intervals used for discretization.
 \parf{ NTST} remains fixed during any particular run, but can be changed
 when restarting. 
 Recommended value of \parf{ NTST} : As small as possible to maintain convergence. 
 
 (Demos \filef{ exp}, \filef{ ab}, \filef{ spb}.)

 (For mesh adaption see \parf{IAD} in Section~\ref{sec:IAD}.)
%AUTODOC NTST END


\subsection{\parf{ NCOL}}  \label{sec:NCOL}

%AUTODOC NCOL START
%AUTODOC NCOL SHORT
% The number of Gauss collocation points per mesh interval, (2 < NCOL < 7).
%AUTODOC NCOL FULL
% The number of Gauss collocation points per mesh interval, (2 < NCOL < 7).
% NCOL remains fixed during any given run, but can be changed when
% restarting at a previously computed solution. The choice NCOL=4, used in
% most demos, is recommended. If NDIM is large and the solutions very
% smooth then NCOL=2 may be appropriate.
%AUTODOC NCOL TEX

 The number of Gauss collocation points per mesh interval,
 (2 $\le$ \parf{ NCOL} $\le$ 7).
 \parf{ NCOL} remains fixed during any given run, but can be changed
 when restarting at a previously computed solution.
 The choice \parf{ NCOL}=4, used in most demos, is recommended.
 If \parf{ NDIM} is ``large'' and the solutions ``very smooth'' then
 \parf{ NCOL}=2 may be appropriate.
%AUTODOC NCOL END

\subsection{\parf{ IAD}}  \label{sec:IAD}

%AUTODOC IAD START
%AUTODOC IAD SHORT
% This constant controls the mesh adaption.
%AUTODOC IAD FULL
% This constant controls the mesh adaption :
% IAD=0 : Fixed mesh. Normally, this choice should never be used, as it
%     may result in spurious solutions. (Demo ext.)
% IAD>0 : Adapt the mesh every IAD steps along the branch. Most demos
%     use IAD=3, which is the strongly recommended value.
%AUTODOC IAD TEX

This constant controls the mesh adaption~: 
\begin{itemize}
\item[-] \parf{ IAD=0}~:
  Fixed mesh. Normally, this choice should never be used, as it may result
  in spurious solutions. (Demo \parf{ ext}.)
\item[-] \parf{ IAD$>$0}~:  
  Adapt the mesh every \parf{ IAD} steps along the branch.
  Most demos use \parf{ IAD=3}, which is the strongly recommended value.
\end{itemize}
%AUTODOC IAD END

When computing  ``trivial'' solutions to a boundary value problem,
for example, when all solution components are constant, then the
mesh adaption may fail under certain circumstances, and overflow may
occur. In such case, try recomputing the solution branch with a fixed
mesh \parf{ (IAD=0)}. Be sure to set  \parf{ IAD} back to \parf{ IAD=3} 
for computing eventual non-trivial bifurcating solution branches.
%=====================================================================
\section{ Tolerances.} \label{sec:Tolerances}
\subsection{\parf{ EPSL}}  \label{sec:EPSL}

%AUTODOC EPSL START
%AUTODOC EPSL SHORT
% Relative convergence criterion for equations parameters in 
% the Newton/Chord method.
%AUTODOC EPSL FULL
% Relative convergence criterion for equation parameters in the Newton/Chord
% method. Most demos use EPSL=10e-6 or EPSL=10e-7, which is 
% the recommended value range.
%AUTODOC EPSL TEX

 Relative convergence criterion for equation parameters in the Newton/Chord 
 method. Most demos use \parf{ EPSL}=$10^{-6}$ or \parf{ EPSL}=$10^{-7}$,
 which is the recommended value range.
%AUTODOC EPSL END

\subsection{\parf{ EPSU}}  \label{sec:EPSU}

%AUTODOC EPSU START
%AUTODOC EPSU SHORT
% Relative convergence criterion for solution components in the 
% Newton/Chord method.
%AUTODOC EPSU FULL
% Relative convergence criterion for solution components in the Newton/Chord
% method. Most demos use EPSU=10e-6 or EPSU=10e-7 , which is the recom
% mended value range.
%AUTODOC EPSU TEX

 Relative convergence criterion for solution components in the Newton/Chord 
 method. Most demos use \parf{ EPSU}=$10^{-6}$ or \parf{ EPSU}=$10^{-7}$,
 which is the recommended value range.
%AUTODOC EPSU END

\subsection{\parf{ EPSS}}  \label{sec:EPSS}

%AUTODOC EPSS START
%AUTODOC EPSS SHORT
% Relative arclength convergence criterion for the detection of special 
% solutions. 
%AUTODOC EPSS FULL
% Relative arclength convergence criterion for the detection of special 
% solutions. Most demos use EPSS=10e-4 or EPSS=10e-5, which is the recommended
% value range. Generally, EPSS should be approximately 100 to 1000 times the
% value of EPSL, EPSU.
%AUTODOC EPSS TEX

 Relative arclength convergence criterion for the detection of special solutions. 
 Most demos use \parf{ EPSS}=$10^{-4}$ or  \parf{ EPSS}=$10^{-5}$,
 which is the recommended value range.
 Generally, \parf{ EPSS} should be approximately 100 to 1000 times the value
 of \parf{ EPSL}, \parf{ EPSU}.
%AUTODOC EPSS END
 
\subsection{\parf{ ITMX}}  \label{sec:ITMX}

%AUTODOC ITMX START
%AUTODOC ITMX SHORT
% The maximum number of iterations allowed in the accurate location of
% special solutions.
%AUTODOC ITMX FULL
% The maximum number of iterations allowed in the accurate location of
% special solutions, such as bifurcations, folds, and user output points, by
% Mullers method with bracketing. The recommended value is ITMX=8, used
% in most demos.
%AUTODOC ITMX TEX

 The maximum number of iterations allowed in the accurate
 location of special solutions, such as bifurcations, folds, 
 and user output points, by M\"uller's method with bracketing.
 The recommended value is \parf{ ITMX}=8, used in most demos.
%AUTODOC ITMX END

\subsection{\parf{ NWTN}}  \label{sec:NWTN}

%AUTODOC NWTN START
%AUTODOC NWTN SHORT
% After NWTN Newton iterations the Jacobian is frozen.
%AUTODOC NWTN FULL
% After NWTN Newton iterations the Jacobian is frozen, i.e., AUTO uses full
% Newton for the first NWTN iterations and the Chord method for iterations
% NWTN+1 to ITNW. The choice NWTN=3 is strongly recommended and used in
% most demos. Note that this constant is only effective for ODEs, i.e., for
% solving the piecewise polynomial collocation equations. 
% For algebraic systems AUTO always uses full Newton.
%AUTODOC NWTN TEX

 After \parf{ NWTN} Newton iterations the Jacobian is frozen, i.e.,
 \AUTO uses full Newton for the first \parf{ NWTN} iterations
 and the Chord method for iterations \parf{ NWTN}+1 to \parf{ ITNW}.
 The choice \parf{ NWTN}=3 is strongly recommended and used in most demos.
 Note that this constant is only effective for ODEs, i.e., for solving
 the piecewise polynomial collocation equations.
 For algebraic systems \AUTO always uses full Newton.
%AUTODOC NWTN END

\subsection{\parf{ ITNW}}  \label{sec:ITNW}

%AUTODOC ITNW START
%AUTODOC ITNW SHORT
% The maximum number of combined NewtonChord iterations. 
%AUTODOC ITNW FULL
% The maximum number of combined NewtonChord iterations. When this
% maximum is reached, the step will be retried with half the stepsize. This
% is repeated until convergence, or until the minimum stepsize is reached. In
% the latter case the computation of the branch is discontinued and a message
% printed in fort.9. The recommended value is ITNW=5, but ITNW=7 may be
% used for difficult problems, for example, demos spb, chu, plp, etc.
%AUTODOC ITNW TEX

 The maximum number of combined Newton-Chord iterations.
 When this maximum is reached, the step will be retried with 
 half the stepsize.
 This is repeated until convergence, or until the minimum
 stepsize is reached. In the latter case the computation of
 the branch is discontinued and a message printed in \filef{ fort.9}.
 The recommended value is \parf{ ITNW}=5, but \parf{ ITNW}=7 may be used for 
 ``difficult'' problems, for example, 
 demos \filef{ spb}, \filef{ chu}, \filef{ plp}, etc.
%AUTODOC ITNW END

%=====================================================================
\section{ Continuation Step Size.} \label{sec:step_size}
\subsection{\parf{ DS}}  \label{sec:DS}

%AUTODOC DS START
%AUTODOC DS SHORT
% The constant DS defines the pseudoarclength stepsize to be 
% used for the first attempted step along any branch.
%AUTODOC DS FULL
% AUTOuses pseudoarclength continuation for following solution branches.
% The pseudoarclength stepsize is the distance between the current solution
% and the next solution on a branch. By default, this distance includes all
% state variables (or state functions) and all free parameters. The constant
% DS defines the pseudoarclength stepsize to be used for the first attempted
% step along any branch. (Note that if IADS > 0 then DS will automatically be
% adapted for subsequent steps and for failed steps.) DS may be chosen positive
% or negative; changing its sign reverses the direction of computation. The re
% lation DSMIN < DS < DSMAX must be satisfied. The precise choice of DS is
% problemdependent; the demos use a value that was found appropriate after
% some experimentation.
%AUTODOC DS TEX

 \AUTO uses pseudo-arclength continuation for following solution branches.
 The pseudo-arclength stepsize is the distance between
 the current solution and the next solution on a branch.
 By default, this distance includes all state variables
 (or state functions) and all free parameters.
 The constant \parf{ DS} defines the pseudo-arclength stepsize to be used for the
 first attempted step along any branch. 
 (Note that if \parf{ IADS}$>$0 then \parf{ DS} will automatically be adapted
 for subsequent steps and for failed steps.)
 \parf{ DS} may be chosen positive or negative; changing its sign 
 reverses the direction of computation.
 The relation \parf{ DSMIN} $\le$ $\abs \parf{ DS}$ $\le$ \parf{ DSMAX} must be satisfied.
 The precise choice of \parf{ DS} is problem-dependent; the demos use a value 
 that was found appropriate after some experimentation.
%AUTODOC DS END
 

\subsection{\parf{ DSMIN}}  \label{sec:DSMIN}

%AUTODOC DSMIN START
%AUTODOC DSMIN SHORT
% This is minimum allowable absolute value of the pseudoarclength step
% size. 
%AUTODOC DSMIN FULL
% This is minimum allowable absolute value of the pseudoarclength step
% size. DSMIN must be positive. It is only effective if the pseudoarclength
% step is adaptive, i.e., if IADS>0. The choice of DSMIN is highly problem
% dependent; most demos use a value that was found appropriate after some
% experimentation.
%AUTODOC DSMIN TEX

 This is minimum allowable absolute value of the pseudo-arclength 
 stepsize. \parf{ DSMIN} must be positive.
 It is only effective if the pseudo-arclength step is adaptive,
 i.e., if \parf{ IADS}$>$0.
 The choice of \parf{ DSMIN} is highly problem-dependent;
 most demos use a value that was found appropriate after some
 experimentation.
 See also the discussion in Section~\ref{sec:Efficiency}.
%AUTODOC DSMIN END

\subsection{\parf{ DSMAX}}  \label{sec:DSMAX}

%AUTODOC DSMAX START
%AUTODOC DSMAX SHORT
% The maximum allowable absolute value of the pseudoarclength stepsize.
%AUTODOC DSMAX FULL
% The maximum allowable absolute value of the pseudoarclength stepsize.
% DSMAX must be positive. It is only effective if the pseudoarclength step is
% adaptive, i.e., if IADS>0. The choice of DSMAX is highly problemdependent;
% most demos use a value that was found appropriate after some experimentation.
%AUTODOC DSMAX TEX

 The maximum allowable absolute value of the pseudo-arclength stepsize.
 \parf{ DSMAX} must be positive.
 It is only effective if the pseudo-arclength step is adaptive,
 i.e., if \parf{ IADS}$>$0.
 The choice of \parf{ DSMAX} is highly problem-dependent; 
 most demos use a value that was found appropriate after some
 experimentation.
 See also the discussion in Section~\ref{sec:Efficiency}.
%AUTODOC DSMAX END

\subsection{\parf{ IADS}}  \label{sec:IADS}

%AUTODOC IADS START
%AUTODOC IADS SHORT
% This constant controls the frequency of adaption of the pseudoarclength
% stepsize.
%AUTODOC IADS FULL
% This constant controls the frequency of adaption of the pseudoarclength
% stepsize.
% IADS=0 : Use fixed pseudoarclength stepsize, i.e., the stepsize will be
%     equal to the specified value of DS for every step. The computation
%     of a branch will be discontinued as soon as the maximum number of
%     iterations ITNW is reached. This choice is not recommended.
%     (Demo tim.)
% IADS>0 : Adapt the pseudoarclength stepsize after every IADS steps.
%     If the Newton/Chord iteration converges rapidly then |DS| will be
%     increased, but never beyond DSMAX. If a step fails then it 
%     will be retried with half the stepsize. This will be done repeatedly 
%     until the step is successful or until |DS| reaches DSMIN. In the 
%     latter case nonconvergence will be signalled. The strongly 
%     recommended value is IADS=1, which is used in almost all demos.
%AUTODOC IADS TEX

This constant controls the frequency of adaption of the 
pseudo-arclength stepsize.
\begin{itemize}
\item[-] \parf{ IADS=0}~: 
  Use fixed pseudo-arclength stepsize, i.e., the stepsize will
  be equal to the specified value of \parf{ DS} for every step.
  The computation of a branch will be discontinued as soon as
  the maximum number of iterations \parf{ ITNW} is reached.
  This choice is not recommended. 

(Demo \filef{ tim}.)
\item[-] \parf{ IADS$>$0}~:  
 Adapt the pseudo-arclength stepsize after every \parf{ IADS} steps.
  If the Newton/Chord iteration converges rapidly then 
  $\abs\parf{ DS}$ will be increased, but never beyond \parf{ DSMAX}.
  If a step fails then it will be retried with half
  the stepsize. This will be done repeatedly until the
  step is successful or until $\abs\parf{ DS}$ reaches \parf{ DSMIN}. 
  In the latter case nonconvergence will be signalled.
  The strongly recommended value is \parf{ IADS}=1, which is used in 
  almost all demos.
\end{itemize}
%AUTODOC IADS END
  
\subsection{\parf{ NTHL}}  \label{sec:NTHL}

%AUTODOC NTHL START
%AUTODOC NTHL SHORT
% NTHL defines the number of parameters whose weight is to be modified.
%AUTODOC NTHL FULL
% By default, the pseudoarclength stepsize includes all state variables (or
% state functions) and all free parameters. Under certain circumstances one
% may want to modify the weight accorded to individual parameters in the
% definition of stepsize. For this purpose, NTHL defines the number of 
% parameters whose weight is to be modified. If NTHL=0 then all weights will
% have default value 1.0 . If NTHL>0 then one must enter NTHL pairs, Parameter
% Index Weight , with each pair on a separate line.
% For example, for the computation of periodic solutions it is recommended
% that the period not be included in the pseudoarclength continuation stepsize,
% in order to avoid periodinduced limitations on the stepsize near orbits of
% infinite period. This exclusion can be accomplished by setting NTHL=1, with,
% on a separate line, the pair 11 0.0 . Most demos that compute periodic
% solutions use this option; see for example demo ab.
%AUTODOC NTHL TEX

By default, the pseudo-arclength stepsize includes all state variables
(or state functions) and all free parameters.
Under certain circumstances one may want to modify the weight accorded 
to individual parameters in the definition of stepsize.
For this purpose, \parf{ NTHL} defines the number of parameters whose weight 
is to be modified.
If \parf{ NTHL}=0 then all weights will have default value 1.0~.
If \parf{ NTHL}$>$0 then one must enter \parf{ NTHL} pairs,
             ~{\it Parameter Index} ~ {\it Weight}~,
with each pair on a separate line.

For example, for the computation of periodic solutions it is 
recommended that the period not be included in the pseudo-arclength 
continuation stepsize, in order to avoid period-induced limitations 
on the stepsize near orbits of infinite period. 
This exclusion can be accomplished by setting \parf{ NTHL=1}, with, 
on a separate line, the pair ~ 11 ~ 0.0 ~.
Most demos that compute periodic solutions use this option;
see for example demo \filef{ ab}.
%AUTODOC NTHL END

\subsection{\parf{ NTHU}}  \label{sec:NTHU}

%AUTODOC NTHU START
%AUTODOC NTHU SHORT
% NTHU defines the number of states whose weight is to be modified.
%AUTODOC NTHU FULL
% Under certain circumstances one may want to modify the weight accorded
% to individual state variables (or state functions) in the definition 
% of stepsize.
% For this purpose, NTHU defines the number of states whose weight is to be
% modified. If NTHU=0 then all weights will have default value 1.0 . If NTHU>0
% then one must enter NTHU pairs, State Index Weight , with each pair on a
% separate line. At present none of the demos use this option.
%AUTODOC NTHU TEX

Under certain circumstances one may want to modify the weight accorded 
to individual state variables (or state functions) in the definition 
of stepsize.
For this purpose, \parf{ NTHU} defines the number of states whose weight 
is to be modified.
If \parf{ NTHU}=0 then all weights will have default value 1.0~.
If \parf{ NTHU}$>$0 then one must enter \parf{ NTHU} pairs,
             ~{\it State Index} ~ {\it Weight}~,
with each pair on a separate line.
At present none of the demos use this option.
%AUTODOC NTHU END
%=====================================================================
\section{ Diagram Limits.} \label{sec:Diagram_limits}

There are three ways to limit the computation of a branch~:

\begin{itemize}
\item[-]
By appropriate choice of the computational window 
defined by the constants \parf{ RL0}, \parf{ RL1}, \parf{ A0}, and \parf{ A1}.
One should always check that the starting solution lies within
this computational window, otherwise the computation will stop immediately
at the starting point.

\item[-]
By specifying the maximum number of steps, \parf{ NMX}.

\item[-]
By specifying a negative parameter index in the list associated 
with the constant \parf{ NUZR}; see Section~\ref{sec:NUZR}. 
\end{itemize}

\subsection{\parf{ NMX}} \label{sec:NMX}

%AUTODOC NMX START
%AUTODOC NMX SHORT
% The maximum number of steps to be taken along any branch.
%AUTODOC NMX FULL
% The maximum number of steps to be taken along any branch.
%AUTODOC NMX TEX

The maximum number of steps to be taken along any branch.
%AUTODOC NMX END

\subsection{\parf{ RL0}}  \label{sec:RL0}

%AUTODOC RL0 START
%AUTODOC RL0 SHORT
% The lower bound on the principal continuation parameter.
%AUTODOC RL0 FULL
% The lower bound on the principal continuation parameter. (This is the
% parameter which appears first in the ICP list;
%AUTODOC RL0 TEX

 The lower bound on the principal continuation parameter.
 (This is the parameter which appears first in the \parf{ ICP} list;
 see Section~\ref{sec:ICP}.). 
%AUTODOC RL0 END


\subsection{\parf{ RL1}}  \label{sec:RL1}

%AUTODOC RL1 START
%AUTODOC RL1 SHORT
% The upper bound on the principal continuation parameter.
%AUTODOC RL1 FULL
% The upper bound on the principal continuation parameter.
%AUTODOC RL1 TEX

 The upper bound on the principal continuation parameter. 
%AUTODOC RL1 END

\subsection{\parf{ A0}}  \label{sec:A0}

%AUTODOC A0 START
%AUTODOC A0 SHORT
% The lower bound on the principal solution measure.
%AUTODOC A0 FULL
% The lower bound on the principal solution measure. (By default, if
% IPLT=0, the principal solution measure is the L2 norm of the state vector or
% state vector function).
%AUTODOC A0 TEX

 The lower bound on the principal solution measure.
 (By default, if \parf{ IPLT}=0, the principal solution measure
 is the $L_2$-norm of the state vector or state vector function.
 See the \AUTO-constant \parf{ IPLT} in Section~\ref{sec:IPLT} 
 for choosing another principal solution measure.)
%AUTODOC A0 END

\subsection{\parf{ A1}}  \label{sec:A1}

%AUTODOC A1 START
%AUTODOC A1 SHORT
% The upper bound on the principal solution measure.
%AUTODOC A1 FULL
% The upper bound on the principal solution measure.
%AUTODOC A1 TEX

 The upper bound on the principal solution measure.
%AUTODOC A1 END

%=====================================================================
%=====================================================================
\section{ Free Parameters.} \label{sec:Free_parameters}


\subsection{\parf{ NICP, ICP}}  \label{sec:ICP}

%AUTODOC NICP START
%AUTODOC NICP SHORT
% The constant NICP indicates how many free parameters have been specified.
%AUTODOC NICP FULL
% For each equation type and for each continuation calculation there is a
% typical (generic) number of problem parameters that must be allowed to
% vary, in order for the calculations to be properly posed. The constant NICP
% indicates how many free parameters have been specified, while the array ICP
% actually designates these free parameters. The parameter that appears first
% in the ICP list is called the principal continuation parameter.  Specific
% examples and special cases are described in the manual.
%AUTODOC NICP TEX


%AUTODOC ICP START
%AUTODOC ICP SHORT
% ICP designates the free parameters.
%AUTODOC ICP FULL
% For each equation type and for each continuation calculation there is a
% typical (generic) number of problem parameters that must be allowed to
% vary, in order for the calculations to be properly posed. The constant NICP
% indicates how many free parameters have been specified, while the array ICP
% actually designates these free parameters. The parameter that appears first
% in the ICP list is called the principal continuation parameter.  Specific
% examples and special cases are described in the manual.
%AUTODOC ICP TEX

For each equation type and for each continuation calculation there is
a typical (``generic'') number of problem parameters that must be 
allowed to vary, in order for the calculations to be properly posed.
The constant \parf{ NICP} indicates how many free parameters have been specified,
while the array \parf{ ICP} actually designates these free parameters.
The parameter that appears first in the \parf{ ICP} list is called the 
``principal continuation parameter''.
Specific examples and special cases are described below.
%AUTODOC ICP END
%AUTODOC NICP END

%=====================================================================
\subsection{ Fixed points.}
The simplest case is the continuation of a solution branch to the system
$ f( u , p ) = 0$,  where $f(\cdot,\cdot), u \in \Rn$, cf. Equation~(\ref{1}).
Such a system arises in the continuation of ODE stationary solutions and 
in the continuation of fixed points of discrete dynamical systems.
There is only one free parameter here, so \parf{ NICP}=1.

As a concrete example, consider Run~1 of demo \filef{ ab},
where \parf{ NICP=1}, with \parf{ ICP(1)=1}. 
Thus, in this run \parf{ PAR(1)} is designated as the free parameter.

%=====================================================================
\subsection{ Periodic solutions and rotations.}
The continuation of periodic solutions and rotations generically requires 
two parameters, namely, one problem parameter and the period.
Thus, in this case  \parf{ NICP}=2.
For example, in Run~2 of demo \parf{ ab} we have \parf{ NICP}=2,
with \parf{ ICP}(1)=1 and \parf{ ICP}(2)=10.
Thus, in this run, the free parameters are \parf{ PAR(1)} and \parf{ PAR(10)}.
(Note that \AUTO reserves \parf{ PAR(10)} for the period, or
 \parf{ PAR(11)} if you use a Fortran equations-file.)

Actually, for periodic solutions, one can set \parf{ NICP}=1 and only specify 
the index of the free problem parameter, as \AUTO will automatically 
add the parameter with the period (10 or 11).
However, in this case the period will not appear in the screen output 
and in the \filef{ fort.7} output-file. 

For fixed period orbits one must set \parf{ NICP}=2 and specify two free problem 
parameters.
For example, in Run~7 of demo \filef{ pp2}, we have \parf{ NICP}=2, with 
\parf{ PAR(1)} and \parf{ PAR(2)}
specified as free problem parameters.
The period is fixed in this run.
If the period is large then such a continuation provides a simple and 
effective method for computing a locus of homoclinic orbits.
%=====================================================================
\subsection{ Folds and Hopf bifurcations.}
The continuation of folds for algebraic problems and the continuation of
Hopf bifurcations requires two free problem parameters, i.e.,  \parf{ NICP}=2.
For example, to continue a fold in Run~3 of demo \filef{ ab}, we have \parf{ NICP}=2, 
with \parf{ PAR(1)} and \parf{ PAR(3)} specified as free parameters.
Note that one must set \parf{ ISW}=2 for computing such loci of special solutions.
Also note that in the continuation of folds the principal continuation parameter
must be the one with respect to which the fold was located.

%=====================================================================
\subsection{ Folds and period-doublings.}
The continuation of folds, for periodic orbits and rotations,
and the continuation of period-doubling bifurcations require two free 
problem parameters plus the free period. Thus, one would normally set \parf{ NICP}=3.
For example, in Run~6 of demo \filef{ pen}, where a locus of period-doubling
bifurcations is computed for rotations, we have \parf{ NICP}=3, 
with \parf{ PAR(2)}, \parf{ PAR(3)}, and the period specified as free parameters. 
Note that one must set \parf{ ISW}=2 for computing such loci of special solutions.
Also note that in the continuation of folds the principal continuation parameter
must be the one with respect to which the fold was located.

Actually, one may set \parf{ NICP}=2, and only specify the problem parameters,
as \AUTO will automatically add the period.
For example, in Run~3 of demo \filef{ plp}, where a locus of folds is computed 
for periodic orbits, we have \parf{ NICP}=2, with \parf{ PAR(4)} and \parf{ PAR(1)} specified
as free parameters. 
However, in this case the period will not appear in the screen output 
and in the \filef{ fort.7} output-file. 

To continue a locus of folds or period-doublings with fixed period, simply
set \parf{ NICP}=3 and specify three problem parameters, not including
the period.

%=====================================================================
\subsection{ Boundary value problems.}
The simplest case is that of boundary value problems where 
\parf{ NDIM}=\parf{ NBC} 
and where \parf{ NINT}=0.
Then, generically, one free problem parameter is required for computing 
a solution branch.
For example, in demo \filef{ exp}, we have \parf{ NDIM}=\parf{ NBC}=2, \parf{ NINT}=0. 
Thus \parf{ NICP}=1.
Indeed, in this demo one free parameter is designated,
namely \parf{ PAR(1)}.

More generally, for boundary value problems with integral constraints,
the generic number of free parameters is \parf{ NBC} + \parf{ NINT}$-$\parf{ NDIM} +1.
For example, in demo \filef{ lin}, we have \parf{ NDIM}=2, \parf{ NBC}=2, and \parf{ NINT}=1.
Thus \parf{ NICP}=2. 
Indeed, in this demo two free parameters are designated,
namely \parf{ PAR(1)} and \parf{ PAR(3)}.

%=====================================================================
\subsection{ Boundary value folds.}
To continue a locus of folds for a general boundary value problem
with integral constraints, set \parf{ NICP}=\parf{ NBC}+\parf{ NINT}$-$\parf{ NDIM}+2, 
and specify this number of parameter indices to designate the free parameters.

%=====================================================================
\subsection{ Optimization problems.}
In algebraic optimization problems one must set \parf{ ICP}(1)=9,
as \AUTO uses \parf{ PAR(9)} as principal continuation parameter
to monitor the value of the objective function.
Furthermore, one must designate one free equation parameter in \parf{ ICP}(2). 
Thus, \parf{ NICP}=2 in the first run.

Folds with respect to \parf{ PAR(9)} correspond to extrema of the objective function.
In a second run one can restart at such a fold, with an additional
free equation parameter specified in \parf{ ICP}(3).
Thus, \parf{ NICP}=3 in the second run.

The above procedure can be repeated.
For example, folds from the second run can be continued in a third run
with three equation parameters specified in addition to \parf{ PAR(9)}.
Thus, \parf{ NICP}=4 in the third run.

For a simple example see demo \filef{ opt}, where a four-parameter extremum
is located.
Note that \parf{ NICP}=5 in each of the four constants-files of this demo, 
with the indices of \parf{ PAR(9)} and \parf{ PAR(1)-PAR(4)} specified in \parf{ ICP}.
Thus, in the first three runs, there are overspecified parameters.
However, \AUTO will always use the correct number of parameters.
Although the overspecified parameters will be printed, their values will
remain fixed. 

%=====================================================================
\subsection{ Internal free parameters.}
The actual continuation scheme in \AUTO may use additional free
parameters that are automatically added.
The simplest example is the computation of periodic solutions and rotations,
where \AUTO automatically adds the period, if not specified.
The computation of loci of folds, Hopf bifurcations, and period-doublings
also requires additional internal continuation parameters.
These will be automatically added, and their indices will be greater
than 10.


%=====================================================================
\subsection{ Parameter overspecification.} \label{sec:Parameter_over_specification}
The number of specified parameter indices is allowed to be be greater 
than the generic number.
In such case there will be ``overspecified'' parameters, whose values
will appear in the screen and \filef{ fort.7} output, but which are not
part of the continuation process.
A simple example is provided by demo \filef{ opt}, where the first three runs
have overspecified parameters whose values, although constant, are printed.

There is, however, a more useful application of parameter overspecification.
In the user-supplied subroutine \filef{ pvls} one can define solution measures
and assign these to otherwise unused parameters.
Such parameters can then be overspecified, in order to print them
on the screen and in the \filef{ fort.7} output.
It is important to note that such overspecified parameters must appear
at the end of the \parf{ ICP} list, as they cannot be used as true continuation
parameters.

For an example of using parameter overspecification for printing user-defined
solution measures, see demo \filef{ pvl}.
This is a boundary value problem (Bratu's equation) which has
only one true continuation parameter, namely \parf{ PAR(1)}.
Three solution measures are defined in the subroutine \filef{ pvls}, namely,
the $L_2$-norm of the first solution component,
the minimum of the second component, and
the left boundary value of the second component.
These solution measures are assigned to \parf{ PAR(2), PAR(3)}, and \parf{ PAR(4)}, respectively.
In the constants-file \filef{ c.pvl} we have \parf{ NICP}=4, with \parf{ PAR(1)-PAR(4)}
specified as parameters.
Thus, in this example, \parf{ PAR(2)-PAR(4)} are overspecified.
Note that \parf{ PAR(1)} must appear first in the \parf{ ICP} list;
the other parameters cannot be used as true continuation parameters.
%=====================================================================
%=====================================================================
\section{ Computation Constants.} \label{sec:Computation_constants}
\subsection{\parf{ ILP}}  \label{sec:ILP}

%AUTODOC ILP START
%AUTODOC ILP SHORT
% This constant controls the detection of fold.
%AUTODOC ILP FULL
% ILP=0 : No detection of folds. This choice is recommended.
% ILP=1 : Detection of folds. To be used if subsequent fold continuation
%   is intended.
%AUTODOC ILP TEX

\begin{itemize}
\item[-] \parf{ ILP=0}~: 
  No detection of folds. This choice is recommended.
\item[-] \parf{ ILP=1}~: 
  Detection of folds. To be used if subsequent fold continuation is intended.
\end{itemize}2
%AUTODOC ILP END
 
\subsection{\parf{ ISP}}  \label{sec:ISP}

%AUTODOC ISP START
%AUTODOC ISP SHORT
% This constant controls the detection of branch points, perioddoubling
% bifurcations, and torus bifurcations.
%AUTODOC ISP FULL
% This constant controls the detection of branch points, perioddoubling
% bifurcations, and torus bifurcations.
% ISP=0 : This setting disables the detection of branch points, period
%     doubling bifurcations, and torus bifurcations and the computation of
%     Floquet multipliers.
% ISP=1 : Branch points are detected for algebraic equations, but not
%     for periodic solutions and boundary value problems. Perioddoubling
%     bifurcations and torus bifurcations are not located either. However,
%     Floquet multipliers are computed.
% ISP=2 : This setting enables the detection of all special solutions. For
%     periodic solutions and rotations, the choice ISP=2 should be used with
%     care, due to potential inaccuracy in the computation of the linearized
%     Poincare map and possible rapid variation of the Floquet multipliers.
%     The linearized Poincare map always has a multiplier z = 1. If this mul
%     tiplier becomes inaccurate then the automatic detection of secondary
%     periodic bifurcations will be discontinued and a warning message will
%     be printed in fort.9. See also Section .
% ISP=3 : Branch points will be detected, but AUTO will not monitor
%     the Floquet multipliers. Perioddoubling and torus bifurcations will go
%     undetected. This option is useful for certain problems with nongeneric
%     Floquet behavior.
%AUTODOC ISP TEX

This constant controls the detection of branch points,
period-doubling bifurcations, and torus bifurcations. 
\begin{itemize}
\item[-] \parf{ ISP=0}~:  
  This setting disables the detection of branch points, period-doubling 
  bifurcations, and torus bifurcations and the computation of 
  Floquet multipliers.
\item[-] \parf{ ISP=1}~:  
  Branch points are detected for algebraic equations, but not for
  periodic solutions and boundary value problems.
  Period-doubling bifurcations and torus bifurcations are not located either.
  However, Floquet multipliers are computed.
\item[-] \parf{ ISP=2}~: This setting enables the detection of all special 
 solutions.
 For periodic solutions and rotations, the choice \parf{ ISP}=2 should be used with
 care, due to potential inaccuracy in the computation of the
 linearized Poincar\'e map and possible rapid variation of the
 Floquet multipliers.
 The linearized Poincar\'e map always has a multiplier $z=1$.
 If this multiplier becomes inaccurate
 then the automatic detection of secondary periodic
 bifurcations will be discontinued and a
 warning message will be printed in \filef{ fort.9}.
 See also Section~\ref{sec:Bifurcations}.
\item[-] \parf{ ISP=3}~:  
  Branch points will be detected, but \AUTO will not monitor the 
  Floquet multipliers. Period-doubling and torus bifurcations will go undetected. 
  This option is useful for certain problems with non-generic Floquet behavior.
  The Floquet multipliers will be output to the diagnostic file.
\end{itemize}
%AUTODOC ISP END

\subsection{\parf{ ISW}}  \label{sec:ISW}

%AUTODOC ISW START
%AUTODOC ISW SHORT
% This constant controls branch switching at branch points for the case of
% differential equations.
%AUTODOC ISW FULL
% This constant controls branch switching at branch points for the case of
% differential equations. Note that branch switching is automatic for algebraic
% equations.
% ISW=1  : This is the normal value of ISW.
% ISW=-1 : If IRS is the label of a branch point or a perioddoubling
%     bifurcation then branch switching will be done. For period doubling
%     bifurcations it is recommended that NTST be increased. For examples
%     see Run 2 and Run 3 of demo lor, where branch switching is done at
%     perioddoubling bifurcations, and Run 2 and Run 3 of demo bvp, where
%     branch switching is done at a transcritical branch point.
% ISW=2  : If IRS is the label of a fold, a Hopf bifurcation point, or a
%     perioddoubling or torus bifurcation then a locus of such points will
%     be computed. An additional free parameter must be specified for such
%     continuations.
%AUTODOC ISW TEX

 This constant controls branch switching at branch points for the case
 of differential equations.
 Note that branch switching is automatic for algebraic equations.
\begin{itemize}
\item[-] \parf{ ISW=1}~: This is the normal value of \parf{ ISW}.
\item[-] \parf{ ISW=$-$1}~:
  If \parf{ IRS} is the label of a branch point or a period-doubling
  bifurcation then branch switching will be done.
  For period doubling bifurcations it is recommended that \parf{ NTST} be increased.
  For examples see Run~2 and Run~3 of demo \filef{ lor}, where branch switching
  is done at period-doubling bifurcations, and Run~2 and Run~3 of demo \filef{ bvp},
  where branch switching is done at a transcritical branch point.
\item[-] \parf{ ISW=2}~:
  If \parf{ IRS} is the label of a fold, a Hopf bifurcation point, 
  or a period-doubling or torus bifurcation then a locus of such points will be
  computed. An additional free parameter must be specified for such 
  continuations; 
  see also Section~\ref{sec:Free_parameters}.
\end{itemize}
%AUTODOC ISW END

\subsection{\parf{ MXBF}}  \label{sec:MXBF}

%AUTODOC MXBF START
%AUTODOC MXBF SHORT
% This constant, which is effective for algebraic problems only, sets the
% maximum number of bifurcations to be treated.
%AUTODOC MXBF FULL
% This constant, which is effective for algebraic problems only, sets the
% maximum number of bifurcations to be treated. Additional branch points will
% be noted, but the corresponding bifurcating branches will not be computed.
% If MXBF is positive then the bifurcating branches of the first MXBF branch
% points will be traced out in both directions. If MXBF is negative then the
% bifurcating branches of the first |MXBF| branch points will be traced out in
% only one direction.
%AUTODOC MXBF TEX

 This constant, which is effective for algebraic problems only,
 sets the maximum number of bifurcations to be treated.
 Additional branch points will be noted, but the corresponding bifurcating
 branches will not be computed.
 If \parf{ MXBF} is positive then the bifurcating branches of the first \parf{ MXBF}
  branch points will be traced out in both directions.
 If \parf{ MXBF} is negative then the bifurcating branches of the first 
 $\abs{\parf{ MXBF}}$ branch points will be traced out in only one direction. 
%AUTODOC MXBF END

\subsection{\parf{ IRS}}  \label{sec:IRS}

%AUTODOC IRS START
%AUTODOC IRS SHORT
% This constant sets the label of the solution where the computation is to
% be restarted.
%AUTODOC IRS FULL
% This constant sets the label of the solution where the computation is to
% be restarted.
% IRS=0 : This setting is typically used in the first run of a new problem.
%     In this case a starting solution must be defined in the user-supplied
%     subroutine stpnt; see also Section . For representative examples of
%     analytical starting solutions see demos ab and frc. For starting from
%     unlabeled numerical data see the @fc command (Section ) and demos
%     lor and pen.
% IRS>0 : Restart the computation at the previously computed solution
%     with label IRS. This solution is normally expected to be in the cur
%     rent datafile q.xxx; see also the @r and @R commands in Section .
%     Various AUTOconstants can be modified when restarting.
%AUTODOC IRS TEX

This constant sets the label of the solution where the computation
is to be restarted.
\begin{itemize}
\item[-] \parf{ IRS=0}~:  
  This setting is typically used in the first run of a new problem.
  In this case a starting solution must be defined in the user-supplied
  subroutine \filef{ stpnt}; see also Section~\ref{sec:Arguments_of_STPNT}.
  For representative examples of analytical starting solutions 
  see demos \filef{ ab} and \filef{ frc}.
  For starting from unlabeled numerical data see the {\it @fc} command
  (Section~\ref{sec:command_mode}) and demos \filef{ lor} and \filef{ pen}.
  
\item[-] \parf{ IRS$>$0}~: 
  Restart the computation at the previously computed solution with label \parf{ IRS}. 
  This solution is normally expected to be in the current data-file 
 \filef{ q.xxx}; see also the {\it @r} and {\it @R} commands in 
 Section~\ref{sec:command_mode}.
 Various \AUTO-constants can be modified when restarting.
\end{itemize}
%AUTODOC IRS END

\subsection{\parf{ IPS}}  \label{sec:IPS}

%AUTODOC IPS START
%AUTODOC IPS SHORT
% This constant defines the problem type.
%AUTODOC IPS FULL
% This constant defines the problem type :
% IPS=0 : An algebraic bifurcation problem. Hopf bifurcations will not
%     be detected and stability properties will not be indicated in the fort.7
%     outputfile.
% IPS=1 : Stationary solutions of ODEs with detection of Hopf bifurca
%     tions. The sign of PT, the point number, in fort.7 is used to indicate
%     stability :  - is stable , + is unstable.
%     (Demo ab.)
% IPS=1 : Fixed points of the discrete dynamical system u^(k+1) = f(u^k,p),
%     with detection of Hopf bifurcations. The sign of PT in
%     fort.7 indicates stability :  - is stable , + is unstable. (Demo dd2.)
%     IPS=2 : Time integration using implicit Euler. The AUTOconstants
%     DS, DSMIN, DSMAX, and ITNW, NWTN control the stepsize. In fact, pseudo
%     arclength is used for continuation in time. Note that the time dis
%     cretization is only first order accurate, so that results should be care
%     fully interpreted. Indeed, this option has been included primarily for
%     the detection of stationary solutions, which can then be entered in the
%     user-supplied subroutine stpnt.
%     (Demo ivp.)
% IPS=2 : Computation of periodic solutions. Starting data can be a Hopf
%     bifurcation point (Run 2 of demo ab), a periodic orbit from a previous
%     run (Run 4 of demo pp2), an analytically known periodic orbit (Run 1
%     of demo frc), or a numerically known periodic orbit (Demo lor). The
%     sign of PT in fort.7 is used to indicate stability : - is stable , + is
%     unstable or unknown.
% IPS=4 : A boundary value problem. Boundary conditions must be
%     specified in the user-supplied subroutine bcnd and integral constraints
%     in icnd. The AUTOconstants NBC and NINT must be given correct
%     values. (Demos exp, int, kar.)
% IPS=5 : Algebraic optimization problems. The objective function must
%     be specified in the user-supplied subroutine fopt. (Demo opt.)
% IPS=7 : A boundary value problem with computation of Floquet multi
%     pliers. This is a very special option; for most boundary value problems
%     one should use IPS=4. Boundary conditions must be specified in the
%     user-supplied subroutine bcnd and integral constraints in icnd. The
%     AUTOconstants NBC and NINT must be given correct values.
% IPS=9 : This option is used in connection with the HomCont algo
%     rithms described in Chapters  for the detection and continuation
%     of homoclinic bifurcations.
%     (Demos san, mtn, kpr, cir, she, rev.)
% IPS=11 : Spatially uniform solutions of a system of parabolic PDEs,
%     with detection of traveling wave bifurcations. The user need only
%     define the nonlinearity (in subroutine func), initialize the wave 
%     speed in PAR(10), initialize the diffusion constants in PAR(15,16,...),
%     and set a free equation parameter in ICP(1). (Run 2 of demo wav.)
% IPS=12 : Continuation of traveling wave solutions to a system of parabolic
%     PDEs. Starting data can be a Hopf bifurcation point from a previous
%     run with IPS=11, or a traveling wave from a previous run with IPS=12.
%     (Run 3 and Run 4 of demo wav.)
% IPS=14 : Time evolution for a system of parabolic PDEs subject to
%     periodic boundary conditions. Starting data may be solutions from a
%     previous run with IPS=12 or 14. Starting data can also be specified in
%     stpnt, in which case the wave length must be specified in PAR(11), and
%     the diffusion constants in PAR(15,16,...). AUTO uses PAR(14) for the
%     time variable. DS, DSMIN, and DSMAX govern the pseudoarclength con
%     tinuation in the spacetime variables. Note that the time discretization
%     is only first order accurate, so that results should be carefully inter
%     preted. Indeed, this option is mainly intended for the detection of
%     stationary waves. (Run 5 of demo wav.)
% IPS=15 : Optimization of periodic solutions. The integrand of the
%     objective functional must be specified in the user-supplied subroutine
%     fopt. Only PAR(19) should be used for problem parameters. PAR(10)
%     is the value of the objective functional, PAR(11) the period, PAR(12)
%     the norm of the adjoint variables, PAR(14) and PAR(15) are internal
%     optimality variables. PAR(2129) and PAR(31) are used to monitor
%     the optimality functionals associated with the problem parameters and
%     the period. Computations can be started at a solution computed with
%     IPS=2 or IPS=15. For a detailed example see demo ops.
%     IPS=16 : This option is similar to IPS=14, except that the user sup
%     plies the boundary conditions. Thus this option can be used for time
%     integration of parabolic systems subject to userdefined boundary con
%     ditions. For examples see the first runs of demos pd1, pd2, and bru.
%     Note that the spacederivatives of the initial conditions must also be
%     supplied in the user-supplied subroutine stpnt. The initial conditions
%     must satisfy the boundary conditions. This option is mainly intended
%     for the detecting stationary solutions.
% IPS=17 : This option can be used to continue stationary solutions of
%     parabolic systems obtained from an evolution run with IPS=16. For
%     examples see the second runs of demos pd1 and pd2.
%AUTODOC IPS TEX

This constant defines the problem type~:
\begin{itemize}
%=====================================================================
\item[-] \parf{ IPS=0}~: 
  An algebraic bifurcation problem.
  Hopf bifurcations will not be detected and stability
  properties will not be indicated in the \filef{ fort.7} output-file.
%=====================================================================
\item[-] \parf{ IPS=1}~: 
  Stationary solutions of ODEs with detection of Hopf bifurcations.
  The sign of PT, the point number, in \filef{ fort.7} is used 
  to indicate stability~: $-$ is stable , + is unstable.

 (Demo \filef{ ab}.)
%=====================================================================
\item[-] \parf{ IPS=$-$1}~:  
  Fixed points of the discrete dynamical system
  $u^{(k+1)}=f(u^{(k)},p ),$ with detection of Hopf bifurcations.
  The sign of PT in \filef{ fort.7} indicates stability~: 
  $-$ is stable , + is unstable.  
 (Demo \filef{ dd2}.)
%=====================================================================
\item[-] \parf{ IPS=$-$2}~: 
  Time integration using implicit Euler. 
  The \AUTO-constants \parf{ DS}, \parf{ DSMIN}, \parf{ DSMAX}, and \parf{ ITNW}, \parf{ NWTN} control 
  the stepsize.
  In fact, pseudo-arclength is used for ``continuation in time''. 
  Note that the time discretization is only first order accurate, 
  so that results should be carefully interpreted. 
  Indeed, this option has been included primarily for the detection 
  of stationary solutions, which can then be entered in the user-supplied
  subroutine \funcf{ stpnt}.  

 (Demo \filef{ ivp}.)
%=====================================================================
\item[-]  \parf{ IPS=2}~:
  Computation of periodic solutions. Starting data can be
  a Hopf bifurcation point (Run~2 of demo \filef{ ab}),
  a periodic orbit from a previous run (Run~4 of demo \filef{ pp2}),
  an analytically known periodic orbit (Run~1 of demo \filef{ frc}),
  or a numerically known periodic orbit (Demo \filef{ lor}).
  The sign of PT in \filef{ fort.7} is used to indicate
  stability~: $-$ is stable , + is unstable or unknown.
%=====================================================================
\item[-] \parf{ IPS=4}~: 
  A boundary value problem. Boundary conditions must be
  specified in the user-supplied subroutine \funcf{ bcnd}
  and integral constraints in \funcf{ icnd}. The \AUTO-constants
  \parf{ NBC} and \parf{ NINT} must be given correct values.
 (Demos \filef{ exp}, \filef{ int}, \filef{ kar}.)
%=====================================================================
\item[-] \parf{ IPS=5}~:
  Algebraic optimization problems. The objective function
  must be specified in the user-supplied subroutine \funcf{ fopt}. 
 (Demo \filef{ opt}.)
%=====================================================================
\item[-] \parf{ IPS=7}~:
  A boundary value problem with computation of Floquet multipliers. 
  This is a very special option; for most boundary value problems 
  one should use \parf{ IPS=4}.
  Boundary conditions must be
  specified in the user-supplied subroutine \funcf{ bcnd}
  and integral constraints in \funcf{ icnd}. The \AUTO-constants
  \parf{ NBC} and \parf{ NINT} must be given correct values.
%=====================================================================
\item[-] \parf{ IPS=9}~:
  This option is used in connection with the {\cal HomCont} algorithms
  described in 
  Chapters~\ref{ch:HomCont}-\ref{ch:HomCont_rev}
  for the  detection and continuation of homoclinic bifurcations.
  
 (Demos \filef{ san}, \filef{ mtn}, \filef{ kpr}, \filef{ cir}, \filef{ she},
  \filef{ rev}.)
%=====================================================================
\item[-] \parf{ IPS=11}~: 
  Spatially uniform solutions of a system of parabolic PDEs,
  with detection of traveling wave bifurcations.
  The user need only define the nonlinearity (in subroutine \funcf{ func}),
  initialize the wave speed in \parf{ PAR(10)}, initialize the diffusion 
  constants in \parf{ PAR(15,16,$\cdots$)}, and set a free equation parameter 
  in \parf{ ICP}(1).
  (Run~2 of demo \filef{ wav}.)
%=====================================================================
\item[-] \parf{ IPS=12}~: 
  Continuation of traveling wave solutions to a system of parabolic PDEs.
  Starting data can be a Hopf bifurcation point from a previous run 
  with \parf{ IPS}=11, or a traveling wave from a previous run with \parf{ IPS}=12.
  (Run~3  and Run~4 of demo \filef{ wav}.)
%=====================================================================
\item[-] \parf{ IPS=14}~:  
  Time evolution for a system of parabolic PDEs subject to periodic 
  boundary conditions. 
  Starting data may be solutions from a previous run with \parf{ IPS}=12 or 14. 
  Starting data can also be specified in \funcf{ stpnt}, in which case
  the wave length must be specified in \parf{ PAR(10)}, and the diffusion
  constants in \parf{ PAR(14,15,$\cdots$)}.
  \AUTO uses \parf{ PAR(13)} for the time variable.
  \parf{ DS}, \parf{ DSMIN}, and \parf{ DSMAX} govern the pseudo-arclength continuation 
  in the space-time variables.
  Note that the time discretization is only first order accurate, 
  so that results should be carefully interpreted. 
  Indeed, this option is mainly intended for the detection of stationary 
  waves.
  (Run~5 of demo \filef{ wav}.)
%=====================================================================
\item[-] \parf{ IPS=15}~:   
  Optimization of periodic solutions. The integrand of the
  objective functional must be specified in the user-supplied
  subroutine \funcf{ fopt}. Only \parf{ PAR(1-9)} should be used for
  problem parameters. \parf{ PAR(9)} is the value of the objective
  functional, \parf{ PAR(10)} the period, \parf{ PAR(11)} the norm of the
  adjoint variables, \parf{ PAR(13)} and \parf{ PAR(14)} are internal optimality
  variables. \parf{ PAR(20--28)} and \parf{ PAR(30)} are used to monitor the 
  optimality functionals associated with the problem parameters 
  and the period. 
  Computations can be started at a solution computed with \parf{ IPS}=2
  or \parf{ IPS}=15.
  For a detailed example see demo \filef{ ops}.
%=====================================================================
\item[-] \parf{ IPS=16}~:
  This option is similar to \parf{ IPS}=14, except that the user supplies the
  boundary conditions. Thus this option can be used for 
  time-integration of parabolic systems subject to 
  user-defined boundary conditions. For examples see the first runs
  of demos \filef{ pd1}, \filef{ pd2}, and \filef{ bru}. Note that
  the space-derivatives of the initial conditions must
  also be supplied in the user-supplied subroutine \funcf{ stpnt}. 
  The initial conditions must satisfy the boundary conditions.
  This option is mainly intended for the detecting stationary solutions.
%=====================================================================
 \item[-] \parf{ IPS=17}~: 
  This option can be used to continue stationary solutions
  of parabolic systems obtained from an evolution run with \parf{ IPS}=16.
  For examples see the second runs of demos \filef{ pd1} and \filef{ pd2}.
\end{itemize}
%=====================================================================
%AUTODOC IPS END


\section{ Output Control.} \label{sec:Output_control}
\subsection{\parf{ NPR}}  \label{sec:NPR}

%AUTODOC NPR START
%AUTODOC NPR SHORT
% This constant can be used to regularly write fort.8 plotting and restart
% data. 
%AUTODOC NPR FULL
% This constant can be used to regularly write fort.8 plotting and restart
% data. IF NPR>0 then such output is written every NPR steps. IF NPR=0
% or if NPRNMX then no such output is written. Note that special solutions,
% such as branch points, folds, end points, etc., are always written in fort.8.
% Furthermore, one can specify parameter values where plotting and restart
% data is to be written; see Section . For these reasons, and to limit the
% output volume, it is recommended that NPR output be kept to a minimum.
%AUTODOC NPR TEX

 This constant can be used to regularly write \filef{ fort.8} plotting and restart 
 data.  
 IF \parf{ NPR}$>$0 then such output is written every \parf{ NPR} steps.
 IF \parf{ NPR}$=$0 or if \parf{ NPR}$\ge$\parf{ NMX} then no such output is written.
 Note that special solutions, such as branch points, folds, end points, etc., 
 are always written in \filef{ fort.8}.
 Furthermore, one can specify parameter values where plotting and restart 
 data is to be written; see Section~\ref{sec:NUZR}.
 For these reasons, and to limit the output volume, it is recommended that
 \parf{ NPR} output be kept to a minimum.
%AUTODOC NPR END

\subsection{\parf{ IID}} \label{sec:IID} 

%AUTODOC IID START
%AUTODOC IID SHORT
% This constant controls the amount of diagnostic output printed in fort.9.
%AUTODOC IID FULL
% This constant controls the amount of diagnostic output printed in fort.9 :
% the greater IID the more detailed the diagnostic output.
% IID=0 : Minimal diagnostic output. This setting is not recommended.
% IID=2 : Regular diagnostic output. This is the recommended value of IID.
% IID=3 : This setting gives additional diagnostic output for algebraic
%     equations, namely the Jacobian and the residual vector at the starting
%     point. This information, which is printed at the beginning of fort.9,
%     is useful for verifying whether the starting solution in stpnt is indeed
%     a solution.
% IID=4 : This setting gives additional diagnostic output for differen
%     tial equations, namely the reduced system and the associated resid
%     ual vector. This information is printed for every step and for every
%     Newton iteration, and should normally be suppressed. In particular it
%     can be used to verify whether the starting solution is indeed a solu
%     tion. For this purpose, the stepsize DS should be small, and one should
%     look at the residuals printed in the fort.9 outputfile. (Note that the
%     first residual vector printed in fort.9 may be identically zero, as it
%     may correspond to the computation of the starting direction. Look at
%     the second residual vector in such case.) This residual vector has di
%     mension NDIM+NBC+NINT+1, which accounts for the NDIM differential
%     equations, the NBC boundary conditions, the NINT userdefined integral
%     constraints, and the pseudoarclength equation. For proper interpreta
%     tions of these data one may want to refer to the solution algorithm for
%     solving the collocation system, as described in  ().
%     IID=5 : This setting gives very extensive diagnostic output for differ
%     ential equations, namely, debug output from the linear equation solver.
%     This setting should not normally be used as it may result in a huge
%     fort.9 file.
%AUTODOC IID TEX

 This constant controls the amount of diagnostic output printed in \filef{ fort.9}~:
 the greater \parf{ IID} the more detailed the diagnostic output.
\begin{itemize}
\item[-] \parf{ IID=0}~:  
  Minimal diagnostic output. This setting is not recommended.
\item[-] \parf{ IID=2}~: 
  Regular diagnostic output. This is the recommended value of \parf{ IID}.
\item[-] \parf{ IID=3}~: 
  This setting gives additional diagnostic output for algebraic equations,
  namely the Jacobian and the residual vector at the starting point.
  This information, which is printed at the beginning of \filef{ fort.9},
  is useful for verifying whether the starting solution in \funcf{ stpnt} is indeed 
  a solution.
\item[-] \parf{ IID=4}~: 
  This setting gives additional diagnostic output for differential equations,
  namely the reduced system and the associated residual vector. 
  This information is printed for every step and for every Newton iteration,
  and should normally be suppressed.
  In particular it can be used to verify whether the starting solution
  is indeed a solution. For this purpose, the stepsize \parf{ DS} should
  be small, and one should look at the residuals printed in the \filef{ fort.9}
  output-file. (Note that the first residual vector printed in \filef{ fort.9} may
  be identically zero, as it may correspond to the computation of the starting
  direction. Look at the second residual vector in such case.)
  This residual vector has dimension 
  \parf{ NDIM}+\parf{ NBC}+\parf{ NINT}+1, which accounts for the \parf{ NDIM}
  differential equations, the \parf{ NBC} boundary conditions, the \parf{ NINT} user-defined
  integral constraints, and the pseudo-arclength equation.
  For proper interpretations of these data one may want to refer to the solution
  algorithm for solving the collocation system, as described in
  \citename{DoKeKe:91b} \citeyear{DoKeKe:91b}.
\item[-] \parf{ IID=5}~:
  This setting gives very extensive diagnostic output for differential equations,
  namely, debug output from the linear equation solver.
  This setting should not normally be used as it may result
  in a huge \filef{ fort.9} file. 
\end{itemize}
%AUTODOC IID END

\subsection{\parf{ IPLT}}  \label{sec:IPLT}

%AUTODOC IPLT START
%AUTODOC IPLT SHORT
% This constant allows redefinition of the principal solution measure.
%AUTODOC IPLT FULL
% This constant allows redefinition of the principal solution measure, which
% is printed as the second (real) column in the screen output and in the fort.7
% outputfile :
% If IPLT = 0 then the L2 norm is printed. Most demos use this setting.
% For algebraic problems, the standard definition of L2 norm is used. For
% differential equations, the L2 norm is defined as in the manual.
% Note that the interval of integration is [0,1], the standard interval used
% by AUTO. For periodic solutions the independent variable is trans
% formed to range from 0 to 1, before the norm is computed. The AUTO
% constants THL(*) and THU(*) affect
% the definition of the L2 norm.
% If 0 < IPLT < NDIM then the maximum of the IPLTth solution component 
% is printed.
% If -NDIM < IPLT< 0 then the minimum of the IPLTth solution component 
% is printed. (Demo fsh.)
% If NDIM < IPLT < 2*NDIM then the integral of the (IPLTNDIM)th
% solution component is printed. (Demos exp, lor.)
% If 2*NDIM < IPLT < 3*NDIM then the L2 norm of the (IPLTNDIM)th
% solution component is printed. (Demo frc.)
% Note that for algebraic problems the maximum and the minimum are
% identical. Also, for ODEs the maximum and the minimum of a solution
% component are generally much less accurate than the L2 norm and 
% component integrals. Note also that the subroutine pvls provides 
% a second, more general way of defining solution measures.
%AUTODOC IPLT TEX

 This constant allows redefinition of the principal solution measure, which is
 printed as the second (real) column in the screen output and in the \filef{ fort.7}
 output-file~:
 
\begin{itemize}
\item[-]
  If \parf{ IPLT} = 0 then the $L_2$-norm is printed. Most demos use this setting.
  For algebraic problems, the standard definition of $L_2$-norm is used.
  For differential equations, the $L_2$-norm is defined as 
  $$ \sqrt{ \int_0^1 \sum_{k=1}^{NDIM} U_k(x)^2 ~ dx}~.$$
  Note that the interval of integration is $[0,1]$, the standard interval
 used by AUTO. For periodic solutions the independent variable is transformed
 to range from 0 to 1, before the norm is computed. The AUTO-constants THL(*) 
 and THU(*) (see Section~\ref{sec:NTHL} and Section~\ref{sec:NTHU})
 affect the definition of the $L_2$-norm.
\item[-]
  If 0 $<$ \parf{ IPLT} $\le$ \parf{ NDIM} then the maximum of the \parf{ IPLT}'th solution component 
  is printed.
\item[-]
  If $-$\parf{ NDIM} $\le$ \parf{ IPLT} $<$0 then the minimum of the \parf{ IPLT}'th solution component
  is printed.  (Demo \filef{ fsh}.)
\item[-]
  If \parf{ NDIM} $<$ \parf{ IPLT} $\le$ 2*\parf{ NDIM} then the integral 
  of the (\parf{ IPLT}$-$\parf{ NDIM})'th 
  solution component is printed. (Demos \filef{ exp}, \filef{ lor}.)
\item[-]
  If 2*\parf{ NDIM} $<$ \parf{ IPLT} $\le$ 3*\parf{ NDIM} 
  then the $L_2$-norm of the (\parf{ IPLT}$-$\parf{ NDIM})'th 
  solution component is printed. (Demo \filef{ frc}.)
\end{itemize}

Note that for algebraic problems the maximum and the minimum are identical.
Also, for ODEs the maximum and the minimum of a solution component are generally
much less accurate than the $L_2$-norm and component integrals.
Note also that the subroutine \funcf{ pvls} provides a second, more general way
of defining solution measures; see Section~\ref{sec:Parameter_over_specification}.
%AUTODOC IPLT END


\subsection{\parf{ NUZR}}  \label{sec:NUZR} 

%AUTODOC NUZR START
%AUTODOC NUZR SHORT
%
%AUTODOC NUZR FULL
% This constant allows the setting of parameter values at which labeled
% plotting and restart information is to be written in the fort.8 outputfile.
% Optionally, it also allows the computation to terminate at such a point.
% Set NUZR=0 if no such output is needed. Many demos use this setting.
% If NUZR>0 then one must enter NUZR pairs:
% 
%   ParameterIndex ParameterValue
% 
% with each pair on a separate line, to designate the parameters
% and the parameter values at which output is to be written. 
% For examples see demos exp, int, and fsh.
% If such a parameter index is preceded by a minus sign then the computation 
% will terminate at such a solution point. (Demos pen and bru.)
% Note that fort.8 output can also be written at selected values of over
% specified parameters. For an example see demo pvl. For details on 
% overspecified parameters see the manual.
%AUTODOC NUZR TEX

 This constant allows the setting of parameter values at which labeled plotting 
 and restart information is to be written in the \filef{ fort.8} output-file.
 Optionally, it also allows the computation to terminate at such a point.

\begin{itemize}
\item[-]
 Set \parf{ NUZR}=0 if no such output is needed. Many demos use this setting.
\item[-]
 If \parf{ NUZR}$>$0 then one must enter \parf{ NUZR} pairs,
            ~{\it Parameter-Index} ~ {\it Parameter-Value}~,
 with each pair on a separate line, to designate the parameters and the parameter
 values at which output is to be written.
 For examples see demos \filef{ exp}, \filef{ int}, and \filef{ fsh}.
\item[-]
 If such a parameter index is preceded by a minus sign then the computation will
 terminate at such a solution point.
 (Demos \filef{ pen} and \filef{ bru}.)
\end{itemize}

Note that \filef{ fort.8} output can also be written at selected values of 
overspecified parameters. For an example see demo \filef{ pvl}.
For details on overspecified parameters see 
Section~\ref{sec:Parameter_over_specification}.
%AUTODOC NUZR END
%=====================================================================
