%==============================================================================
%==============================================================================
\chapter{ Graphical User Interface.} \label{ch:GUI}
%==============================================================================
%==============================================================================
\section{ General Overview.} \label{sec:GUI_Overview}
%%%FIXME
\emp{ Please note:}  as of \date{\today} the GUI is being updated, so the
documentation is this chapter is not being actively maintained.
The old GUI is provided with this release of \AUTO, but it is unsupported
and may not be included in future releases.

The \AUTOold graphical user interface (GUI) is a tool
for creating and editing equations-files and constants-files;
see Section~\ref{sec: User_supplied_files}
 for a description of these files.
The GUI can also be used to run \AUTO and to manipulate and plot
output-files and data-files; 
see Section~\ref{sec:command_mode} for corresponding commands.
To use the GUI for a new equation, change to an empty work directory.
For an existing equations-file, change to its directory.
(\emp{ Do not activate the GUI in the directory \filef{ auto/2000} 
or in any of its subdirectories.})
Then type 

\centerline { @\commandf{ auto}, }

or its abbreviation @\commandf{ a}.
Here we assume that the \AUTO aliases have been activated; 
see Section~\ref{sec:Installation}.
The GUI includes a window for editing the equations-file,
and four groups of buttons, namely,
the \commandf{ Menu Bar} at the top of the GUI,
the \commandf{ Define Constants}-buttons at the center-left,
the \commandf{ Load Constants}-buttons at the lower left,
and the \commandf{ Stop- and Exit}-buttons.

{\bf Note :}~
Most GUI buttons are activated by point-and-click action with 
the \commandf{ left} mouse button. 
If a beep sound results then the \commandf{ right} mouse button must be used. 

\subsection{ The Menu bar.}
It contains the main buttons for running \AUTO
and for manipulating the equations-file, the constants-file,
the output-files, and the data-files.
In a typical application, these buttons are used from left to right.
First the \commandf{ Equations} are defined and, if necessary, \commandf{ Edited},
before being \commandf{ Written}.
Then the \AUTO-constants are \commandf{ Defined}.
This is followed by the actual \commandf{ Run} of \AUTO.
The resulting output-files can be \commandf{ Saved} as data-files,
or they can be \commandf{ Appended} to existing data-files.
Data-files can be \commandf{ Plotted} with the graphics program {\cal PLAUT},
and various file operations can be done with the \commandf{ Files}-button.
Auxiliary functions are provided by the \commandf{ Demos-}, \commandf{ Misc-},
and \commandf{ Help}-buttons.
The Menu Bar buttons are described in more detail 
in Section~\ref{sec:GUI_Menu_bar}.


\subsection{ The Define-Constants-buttons.}
These have the same function as
the \commandf{ Define}-button on the  Menu Bar, namely to set and change
\AUTO-constants.
However, 
for the \commandf{ Define}-button all constants appear in one panel, 
while 
for the Define Constants-buttons they
are grouped by function, 
as in Chapter~\ref{ch:AUTO_constants}, namely
\commandf{ Problem} definition constants,
\commandf{ Discretization} constants,
convergence \commandf{ Tolerances},
continuation \commandf{ Step Size},
diagram \commandf{ Limits},
designation of free \commandf{ Parameters},
constants defining the \commandf{ Computation},
and
constants that specify \commandf{ Output} options.


\subsection{ The Load-Constants-buttons.}
The \commandf{ Previous}-button can be used to load an existing \AUTO-constants file.
Such a file is also loaded, if it exists,
by the \commandf{ Equations}-button on the \commandf{ Menu Bar}.
The \commandf{ Default}-button can be used
to load  default values of all \AUTO-constants. 
Custom editing is normally necessary.


\subsection{ The Stop- and Exit-buttons.}
The \commandf{ Stop}-button can be used to abort execution of an \AUTO-run.
This should be done only in exceptional circumstances.
Output-files, if any, will normally be incomplete and should be deleted.
Use the \commandf{ Exit}-button to end a session.


\section{ The Menu Bar.} \label{sec:GUI_Menu_bar}
\subsection{ Equations-button.}
This pull-down menu contains the items
\commandf{ Old}, to load an existing equations-file,
\commandf{ New}, to load a model equations-file,
and
\commandf{ Demo}, to load a selected demo equations-file.
Equations-file names are of the form \filef{ xxx.c}.
The corresponding constants-file \filef{ c.xxx} is also loaded if it exists.
The equation name \filef{ xxx} remains active until redefined.

\subsection{ Edit-button.}
This pull-down menu contains the items
\commandf{ Cut} and \commandf{ Copy}, 
to be performed on text in the GUI window
highlighted by click-and-drag action of the mouse,
and the item \commandf{ Paste}, which places editor buffer text at the
location of the cursor.



\subsection{ Write-button.}
This pull-down menu contains the item
\commandf{ Write},
to write the loaded files \filef{ xxx.c} and \filef{ c.xxx},
by the active equation name,
and the item
\commandf{ Write As}
to write these files by a selected new name, which then becomes the active name.


\subsection{ Define-button.}
Clicking this button will display the full \AUTO-constants panel.
Most of its text fields can be edited,
but some have restricted input values that can be selected with
the right mouse button.
Some text fields will display a subpanel for entering data.
To actually apply changes made in the panel, click the
\commandf{ OK-} or \commandf{ Apply}-button at the bottom of the panel.



\subsection{ Run-button.}
Clicking this button will write the constants-file \filef{ c.xxx} and run \AUTO.
If the equations-file has been edited then it should first be rewritten 
with the \commandf{ Write}-button. 


\subsection{ Save-button.}
This pull-down menu contains the item
\commandf{ Save},
to save the output-files \filef{ fort.7}, \filef{ fort.8}, \filef{ fort.9},
as \filef{ b.xxx}, \filef{ s.xxx}, \filef{ d.xxx}, respectively.
Here \filef{ xxx} is the active equation name.
It also contains the item
\commandf{ Save As}, 
to save the output-files under another name. 
Existing data-files with the selected name, if any, will be overwritten.


\subsection{ Append-button.}
This pull-down menu contains the item
\commandf{ Append},
to append the output-files \filef{ fort.7}, \filef{ fort.8}, \filef{ fort.9},
to existing data-files \filef{ b.xxx}, \filef{ s.xxx}, \filef{ d.xxx}, respectively.
Here \filef{ xxx} is the active equation name.
It also contains the item
\commandf{ Append To}, 
to append the output-files to other existing data-files.

\subsection{ Plot-button.}
This pull-down menu contains the items
\commandf{ Plot},
to run the plotting program {\cal PLAUT} for the data-files 
\filef{ b.xxx} and \filef{ s.xxx},
where \filef{ xxx} is the active equation name,
and the item
\commandf{ Name}, 
to run {\cal PLAUT} with other data-files.


\subsection{ Files-button.}
This pull-down menu contains 
the item 
\commandf{ Restart}, to redefine the restart file.
Normally, when restarting from a previously computed solution,
the restart data is expected in the file \filef{ s.xxx},
where \filef{ xxx} is the active equation name.
Use the \commandf{ Restart}-button to read the restart data from another data-file
in the immediately following run.  
The pull-down menu also contains the following items~:
\begin{itemize}
\item[-]\commandf{ Copy},~ to copy  
  \filef{ b.xxx}, \filef{ s.xxx}, \filef{ d.xxx}, \filef{ c.xxx},
  to
  \filef{ b.yyy}, \filef{ s.yyy}, \filef{ d.yyy}, \filef{ c.yyy}, resp.;

\item[-]\commandf{ Append},~ to append data-files
  \filef{ b.xxx}, \filef{ s.xxx}, \filef{ d.xxx},
  to
  \filef{ b.yyy}, \filef{ s.yyy}, \filef{ d.yyy}, resp.;

\item[-]\commandf{ Move},~ to move 
  \filef{ b.xxx}, \filef{ s.xxx}, \filef{ d.xxx}, \filef{ c.xxx},
  to
  \filef{ b.yyy}, \filef{ s.yyy}, \filef{ d.yyy}, \filef{ c.yyy}, resp.;

\item[-]\commandf{ Delete},~ to delete data-files
  \filef{ b.xxx}, \filef{ s.xxx}, \filef{ d.xxx};  

\item[-]\commandf{ Clean}, to delete all files of the form 
  \filef{ fort.*}, \filef{ *.o}, and \filef{ *.exe}.  
\end{itemize}


\subsection{ Demos-button.}
This pulldown menu contains the items
\commandf{ Select},
to view and run a selected \AUTO demo in the demo directory,
and
\commandf{ Reset},
to restore the demo directory to its original state.
Note that demo files can be copied to the user work directory
with the \commandf{ Equations/Demo}-button.


\subsection{ Misc.-button.}
This pulldown menu contains the items
\commandf{ Tek Window}
and
\commandf{ VT102 Window},
for opening windows;
\commandf{ Emacs}
and
\commandf{ Xedit},
for editing files,
and
\commandf{ Print}, for printing the active equations-file \filef{ xxx.c}.


\subsection{ Help-button.}
This pulldown menu contains the items
\commandf{ \AUTO-constants}, for help on \AUTO-constants,
and
\commandf{ User Manual}, for viewing the user manual; i.e., this document.


\section{ Using the GUI.} \label{sec:Using_the_GUI}
\AUTO-commands are described in Section~\ref{sec:command_mode} and
illustrated in the demos.
In Table~\ref{tbl:CM_GUI} we list the main \AUTO-commands 
together with the corresponding GUI button.

\begin{table}[htbp]
\begin{center}
\begin{tabular}{| l | l |}
\hline
\commandf{ @r }  & \commandf{ Run} \\  
\hline
\commandf{ @sv }  & \commandf{ Save}  \\ 
\hline
\commandf{ @ap }  & \commandf{ Append} \\ 
\hline
\commandf{ @p }  & \commandf{ Plot}  \\ 
\hline
\commandf{ @cp }  & \commandf{ Files/Copy}  \\ 
\hline
\commandf{ @mv }  & \commandf{ Files/Move}  \\ 
\hline
\commandf{ @cl }  & \commandf{ Files/Clean} \\ 
\hline
\commandf{ @dl }  & \commandf{ Files/Delete} \\  
\hline
\commandf{ @dm }  & \commandf{ Equations/Demo} \\  
\hline
\end{tabular}
\caption{Command Mode - GUI correspondences.}
\label{tbl:CM_GUI}
\end{center}
\end{table}


The \AUTO-command \commandf{ @r xxx yyy} is given in the GUI as follows~:
click \commandf{ Files/Restart} and enter \filef{ yyy} as data.
Then click \commandf{ Run}.
As noted in Section~\ref{sec:command_mode}, 
this will run \AUTO with the current equations-file
\filef{ xxx.c} and the current constants-file \filef{ c.xxx}, 
while expecting restart data in \filef{ s.yyy}.
The \AUTO-command \commandf{ @ap xxx yyy} is given in the GUI by
clicking \commandf{ Files/Append}.

\section{ Customizing the GUI.} \label{sec:Customizing_the_GUI}
\subsection{ Print-button.}
The \commandf{ Misc/Print}-button on the Menu Bar can be customized 
by editing the file \filef{ GuiConsts.h} in directory \filef{ auto/2000/include}.

\subsection{ GUI colors.}
GUI colors can be customized by creating an X resource file.
Two model files can be found in directory \filef{ auto/2000/gui}, namely,
\filef{ Xdefaults.1} and \filef{ Xdefaults.2}.
To become effective, edit one of these, if desired,
and copy it to \filef{ .Xdefaults} in your home directory.
Color names can often be found in the system file \filef{ /usr/lib/X11/rgb.txt}.

\subsection{ On-line help.}
The file \filef{ auto/2000/include/GuiGlobal.h}
contains on-line help on \AUTO-constants and demos.
The text can be updated, subject to a modifiable maximum length.
On SGI machines this is 10240 bytes,
which can be increased, for example, to 20480 bytes, 
by replacing the line
\commandf{ CC = cc -Wf, -XNl10240 -O}
in \filef{ auto/2000/gui/Makefile} by
\commandf{ CC = cc -Wf, -XNl20480 -O}
On other machines, the maximum message length is the system defined maximum
string literal length.

