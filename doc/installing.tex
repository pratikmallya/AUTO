%==============================================================================
%==============================================================================
\chapter{Installing \AUTO.} \label{ch:Installing_AUTO}
%==============================================================================
%==============================================================================
\section{ Typographical Conventions }

This manual uses the following conventions.

\begin{tabular}{|l|l|}
\hline 
\commandf{command} & This font is used for commands which you can type in. \\ \hline
\parf{PAR}         & This font is used for AUTO parameters. \\ \hline
\filef{filename}   & This font is used for file and directory names. \\ \hline
\envf{variable}    & This font is used for environment variable. \\ \hline
\webf{site}        & This font is used for world wide web and ftp sites. \\ \hline
\funcf{function}   & This font is used for function names. \\ \hline
\end{tabular}


\section{ Installation.} \label{sec:Installation}
The \AUTO files 
are available via HTTP from \\
\webf{http://sourceforge.net/project/showfiles.php?group\_id=21781}.

\medskip
\begin{tabular}{|l|l|}
\hline
bzipped Postscript manual            &   \webf{auto2000-\autoversion.ps.bz2}\\ \hline
gzipped Postscript manual            &   \webf{auto2000-\autoversion.ps.gz}\\ \hline
compressed Postscript manual         &   \webf{auto2000-\autoversion.ps.Z}\\ \hline
tarred and gzipped source code        &   \webf{auto2000-\autoversion.tar.gz}\\ \hline
tarred and bzipped source code        &   \webf{auto2000-\autoversion.tar.bz2}\\ \hline
tarred and compressed source code    &   \webf{auto2000-\autoversion.tar.Z}\\ \hline
zipped source code                   &   \webf{auto2000-\autoversion.zip}\\ \hline
\end{tabular}
\medskip

Below it is assumed that you are using the Unix shell \filef{ tcsh} or
\filef{ bash}
and that the file \filef{ auto2000-\autoversion.tar.Z} is in your main directory.

While in your main directory, enter the commands
\commandf{ uncompress auto2000-\autoversion.tar.Z},
followed by
\commandf{ tar xvfo auto2000-\autoversion.tar}.
This will result in a directory \filef{ auto}, 
with one subdirectory, \filef{ auto/2000}. 
Type \commandf{ cd auto/2000}  
to change directory to \filef{ auto/2000}.
Then type 
\commandf{ ./configure}~,
to check your system for required compilers and libraries.
Once the \filef{ configure} script has finished you 
may then type \commandf{ make} to compile \AUTO
and its ancillary software.
The \filef{ configure} script is designed to detect the details
of your system which \AUTO requires to compile successfully.
If either the \filef{ configure} script or the \filef{ make} command
should fail, you may assist the \filef{ configure} script by giving
it various command line options.  Please type \commandf{ ./configure --help}
for more details.
Upon compilation, you may type 
\commandf{ make clean}  
to remove unnecessary files.

There is a new CLUI under development
which includes some of the capabilities of the old GUI and will
eventually be the recommend way to run AUTO.  More information on the
CLUI may be found in Chapter~\ref{ch:CLUI}.  The new CLUI does not
require any additional options to be passed to the \commandf{configure}
script.

To run the new Command Line User Interface (CLUI) and the old command
language you need to set your environment variables correctly.
Assuming \AUTO is installed in your home directory, the following
commands set your environment variables so that you will be able to
run the \AUTO commands, and may be placed into your \filef{.login},
\filef{.profile}, or \filef{.cshrc} file, as appropriate.  If you are
using a \commandf{sh} compatible shell, such as \commandf{sh},
\commandf{bash}, \commandf{ksh}, or \commandf{ash} enter the command
\commandf{ source~ {\rm \$}HOME/auto/2000/cmds/auto.env.sh}.  On the
other hand, if you are using a \commandf{csh} compatible shell, such
as \commandf{csh} or \commandf{tcsh}, enter the command \commandf{
source~ {\rm \$}HOME/auto/2000/cmds/auto.env.csh}.

There is an old and unsupported Graphical User Interface (GUI)
which requires the {\cal X-Window} system and {\cal Motif}, and it is not
compiled by default.  Note that \AUTO can be very effectively run in
Command Mode, i.e., the GUI is not strictly necessary.  To compile
\AUTO with the old GUI, type \commandf{ ./configure --enable-gui} and then
\filef{ make} in directory \filef{ auto/2000}.  

The PostScript conversion command \filef{ @ps} will be enabled
if the \filef{ configure} script detects the appropriate software,
but
you may have to enter the correct printer name in \filef{ auto/2000/cmds/@pr}.

To generate the on-line manual, type \commandf{ make} in \filef{ auto/2000/doc}.

To prepare \AUTO for transfer to another machine, type \commandf{ make superclean}
in directory \filef{ auto/2000} before creating the \filef{ tar}-file. 
This will remove all executable, object, and other non-essential files, and
thereby reduce the size of the package.

\AUTO can be tested by typing \commandf{ make $>$ TEST {\rm \&}} in 
directory \filef{ auto/2000/test}.
This will execute a selection of demos from \filef{ auto/2000/demos} and write a
summary of the computations in the file \filef{ TEST}.
The contents of \filef{ TEST} can then be compared to other test result files
in directory \filef{ auto/2000/test}. Note that minor differences are
to be expected due to architecture and compiler differences.

Some {\cal EISPACK} routines used by \AUTO for computing eigenvalues and
Floquet multipliers are included in the package
(\citename{EISPACK:76} \citeyear{EISPACK:76}).

\subsection{Installation on Mac OS X}
\AUTO runs on Mac OS X using the above instructions provided that
you have the development tools
installed. You do not need to start an X server to run AUTO. To be able
to plot, \AUTO uses \filef{ pythonw} instead of \filef{ python}. This
should happen automatically.

\subsection{Installation on Windows}
\AUTO runs on Windows as above using the UNIX-like environment Cygwin
(see \webf{http://www.cygwin.com}). You should have at least the
default installation, \filef{  gcc}, \filef{ g77 } or \filef{ gfortran
}, \filef{ make }, and \filef{ python } installed.
An X server is not necessary.

A more light-weight alternative is the combination of MinGW and MSYS
(see \webf{http://www.mingw.org}), combined with a native Win32
version of Python, obtained at \webf{http://www.python.org}. Note that
the Python CLUI does not work in the default MSYS shell environment
(the rxvt window), but you have to start MSYS using, for instance,
\filef{ c:\textbackslash msys\textbackslash 1.0\textbackslash
bin\textbackslash sh $--$login $-$i } in a \filef{ cmd.exe }
window. Then \filef{ auto } can be started as usual, provided that
\filef{ python.exe } can be found using the environment variable
\filef{ \$PATH. }

\section{ Restrictions on Problem Size.} \label{sec:Restrictions}
There are size restrictions in the file \filef{ auto/2000/src/auto\_c.h}
on the following \AUTO-constants~:
the effective number of equation parameters \parf{ NPAR},
and the number of stored branch points \parf{NBIF} for algebraic problems.
See Chapter~\ref{ch:AUTO_constants} 
for the significance these constants. 
Their maxima are denoted by the  corresponding constant followed by an X.
For example, \parf{ NPARX} in \filef{ auto\_c.h} denotes the maximum value 
of \parf{ NPAR}.
If the maxima of \parf{NBIF} is exceeded in an \AUTO-run then a message 
will be printed.
On the other hand, the maximum value of \parf{ NPAR}, 
if exceeded, may lead to unreported errors.
Upon installation \parf{ NPARX}=50; it should never be decreased below that value;
see also Section~\ref{sec:Restrictions_on_PAR}.
Size restrictions can be changed by editing \filef{ auto\_c.h}.
This must be followed by recompilation by typing \commandf{ make} 
in directory \filef{ auto/2000/src}.

Note that in certain cases the \filef{ effective dimension} may be greater
than the user dimension.
For example, for the continuation of folds,
the effective dimension is 2\parf{ NDIM}+1 for algebraic equations,
and 2\parf{ NDIM} for ordinary differential equations, respectively.
Similarly, for the continuation of Hopf bifurcations,
the effective dimension is 3\parf{ NDIM}+2.
 
 
\section{Compatibility with Older Versions.} \label{sec:Compatibility}
There are two changes compared to early versions of \AUTOolder:
The user-supplied equations-files must contain the subroutine \funcf{pvls}.
For an example of use of \funcf{pvls} see the demo \filef{ pvl} 
in Section~\ref{sec:Demos_pvl}.
There is also a small change in the \filef{ q.xxx/s.xxx} data-file.
If necessary, older \AUTOolder files can be converted using the 
\filef{ @94to97} command; see Section~\ref{sec:command_mode}.
Data files from \AUTOold are fully compatible
with \AUTOc, but as \AUTOc is written in {\cal C}
user defined function files from \AUTOold, which
are generally in {\cal Fortran}, are ideally rewritten.
This is not strictly necessary, however, as AUTO2000 has wrappers that
can call {\cal Fortran} code.

\section{Parallel Version.} \label{sec:Parallel}
\AUTOc contains code which allows
it to run in on various types of parallel computers.  Namely,
it can use either the Pthreads library for running on
shared-memory multi-processors, or the MPI message passing library.
When the \filef{ configure} script is run it will try to
find the above two libraries, and if it is successful
it will include their functionality into \AUTOc.
To force the \filef{ configure} script not to use either
of the above libraries, one may type \commandf{ ./configure --without-mpi}
or \commandf{ ./configure --without-pthreads}, and then type \commandf{ make}.
One may even preclude both by typing 
\commandf{ ./configure --without-mpi --without-pthreads} and then
typing \commandf{ make}.  On the other hand, unless there is some
particular difficulty, we recommend that that the 
\filef{ configure} script be used without arguments, since the
parallel version of \AUTOc may easily be controlled,
and even run in a serial mode,  
through the use of command line options at run time.
The command line options are listed in Table~\ref{tbl:Command_line_args}.

\begin{table}[htbp]
\begin{center}
\begin{tabular}{| l | l |}
\hline
-v&    Give verbose output.\\
\hline
-m&    Use the Message Passing Interface library for parallelization.\\
\hline
-t&    \begin{minipage}{4in}
       Use the Pthreads library for parallelization.
       This option takes one of three arguments.
       \begin{description}
       \item[conpar] parallelizes the condensation of parameters routine.
       \item[setubv] parallelizes the Jacobian setup routine.
       \item[both]   parallelizes both routines.
       \end{description}
       In general the recommended option is 'both'.
       \end{minipage}\\
\hline
-\#&    \parbox{4in}{The number of processing units to use 
                     (currently only used with the -t option).}\\
\hline
\end{tabular}
\caption{Command line options.}
\label{tbl:Command_line_args}
\end{center}
\end{table}
 
For example, to run the \AUTOc executable \filef{ auto.exe}
in serial mode you just type \filef{ auto.exe}.  To run the
same command in parallel using the Pthreads library
on 4 processors you type \filef{ auto.exe -t both -\# 4}.
If you were to try and run the above command on a machine
which did not have the Pthreads library, the command
would exit with an error and inform you that the Pthreads
library is not available.

Running the MPI version is somewhat more complex because of the fact
that MPI normally uses some external program for starting the
computational processes.  The exact name and command line options of
this external program depends on your MPI installation.  A common name
for this MPI external program is \filef{ mpirun}, and a common command
line option which defines the number of computational processes is \filef{
-np}.  Accordingly, if you wanted to run the MPI version of \AUTOc 
on four processors, with the above external program, you would
type \filef{ mpirun -np 4 auto.exe -m}.  Please see your local MPI
documentation for more detail.  As with the Pthreads library, if you
were to try and run the above command on a machine which did not have
MPI, the command would exit with an error and inform you that MPI is
not available.

The commands in the \filef{ auto/2000/cmds} directory and described
in Chapter~\ref{ch:How_to_run_AUTO} may be used with the
parallel version as well, by setting the \envf{ AUTO\_COMMAND\_PREFIX}
and \envf{ AUTO\_COMMAND\_ARGS} environment variables.  For example, to
the run \AUTOc in parallel using the Pthreads library on 4
processors just type \commandf{ setenv AUTO\_COMMAND\_ARGS ``-t both -\#
4''} and then use the commands in \filef{ auto/2000/cmds} normally.  To
run \AUTOold in parallel using the MPI library on 4
processors just type \commandf{ setenv AUTO\_COMMAND\_ARGS ``-m''} and 
\commandf{setenv AUTO\_COMMAND\_PREFIX ``mpirun -np 4''}, and then use the
commands in \filef{ auto/2000/cmds} normally.  The previous examples
assumed you are using the \commandf{csh} shell or the \commandf{tcsh} shell, for
other shells you should modify the commands appropriately.








