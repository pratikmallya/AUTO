
%==============================================================================
%==============================================================================
\chapter{ {\cal HomCont} Demo : mtn.} \label{ch:HomCont_mtn}
%==============================================================================
%==============================================================================

%==============================================================================
%DEMO=mtn======================================================================
%==============================================================================
\section{ A Predator-Prey Model with Immigration.}
Consider the following system of two equations \cite{Sc:95}
\begin{equation} \label{sn.1} \begin{array}{rcl}
\dot{X} & = & RX\left(1-{\ds \frac{X}{K}}\right) - 
{\ds \frac{A_1XY}{B_1+X}} + D_0K \\
\dot{Y} & = & E_1 {\ds \frac{A_1XY}{B_1+X}} - D_1Y - 
{\ds\frac{A_2ZY^2}{B_2^2+Y^2}}.
\end{array} \end{equation}
\begin{figure}[b]
\epsfysize 10.0cm
\centerline{\epsffile{include/mtn1.ps}}
\caption{Parametric portrait of the predator-prey system }
\label{SNF.1}
\end{figure}
The values of all parameters except $(K,Z)$ are set as follows~:
$$
R=0.5,\ A_1=0.4,\ B_1=0.6,\ D_0=0.01,\ E_1=0.6,\ A_2=1.0,\ B_2=0.5,\ D_1=0.15.
$$
\par
\noindent
The parametric portrait of the system (\ref{sn.1}) on the
$(Z,K)$-plane is presented in Figure \ref{SNF.1}. It contains fold
($t_{1,2}$) and Hopf ($H$) bifurcation curves, as well as a homoclinic
bifurcation curve $P$. The fold curves meet at a cusp singular point
$C$, while the Hopf and the homoclinic curves originate at a
Bogdanov-Takens point $BT$. Only the homoclinic curve $P$ will be 
considered here, the other bifurcation curves can be computed using
{\tt AUTO} or,
for example, {\cal locbif} \cite{KhKuLeNi:93}.

\section{Continuation of Central Saddle-Node Homoclinics.}
Local bifurcation analysis shows that at $K=6.0,\ Z=0.06729762\ldots$,
the system has a saddle-node equilibrium 
$$
(X^0,Y^0) = (5.738626\ldots,0.5108401\ldots),
$$
with one zero and one negative eigenvalue. Direct simulations reveal a
homoclinic 
orbit to this saddle-node, departing and returning along its central
direction (i.e., tangent to the null-vector).
\par
Starting from this solution, stored in the file {\tt mtn.dat}, we
continue the saddle-node central homoclinic orbit 
with respect to the parameters $K$ and $Z$ by copying the
demo and running it
\begin{center}
{\it @dm mtn}\\
{\it make first}
\end{center}
The file {\tt mtn.c} contains approximate
parameter values
$$
K={\tt PAR(1)}=6.0,\ Z={\tt PAR(2)}=0.06729762,
$$
as well as the coordinates of the saddle-node
$$
X^0={\tt PAR(12)}=5.738626,\ Y^0={\tt PAR(13)}=0.5108401,
$$
and the length of the truncated time-interval
$$
T_0={\tt PAR(11)} = 1046.178 \: .
$$
Since a homoclinic orbit to a saddle-node is being followed, we have also
made the choices
$$
{\tt IEQUIB = 2 \quad NUNSTAB =0 \quad NSTAB = 1   }
$$
in {\tt h.mtn.1}. The two test-functions, $\psi_{15}$ and $\psi_{16}$, 
to detect non-central saddle-node homoclinic
orbits are also activated, which must be specified in three ways. 
Firstly, in {\tt h.mtn.1}, {\tt NPSI} is
set to 2 and the active test functions {\tt IPSI(I),I=1,2}
are chosen as 15 and 16. This sets up the monitoring of these
test functions. Secondly, in {\tt c.mtn.1} user-defined functions
({\tt NUZR=2}) are set up to look for zeros of the parameters
corresponding to these test functions. Recall that the
parameters to be zeroed are always the test functions plus 20.
Finally, these parameters are included in the list of continuation
parameters ({\tt NICP,(ICP(I),I=1 NICP)}).

Among the output there is a line 
\begin{verbatim}
  BR    PT  TY LAB    PAR(1)    ...     PAR(2)        PAR(35)       PAR(36)    
   1    27  UZ   5  6.10437E+00 ...   6.932475E-02 -6.782898E-07  8.203437E-02
\end{verbatim}
indicating that a zero of the test function {\tt IPSI(1)=15} 
This means that at
$$
D_1=(K^1,Z^1)=(6.6104\ldots, 0.069325\ldots)
$$
the homoclinic orbit to the saddle-node becomes {\it non-central}, namely,
it returns to the equilibrium along the stable eigenvector, forming a
non-smooth loop. The output is saved in {\tt b.1}, {\tt s.1} and {\tt d.1}. 
Repeating computations in the opposite direction along the curve, 
{\tt IRS=1, DS=-0.01} in {\tt c.mtn.2}, 
\begin{center}
{\it make second}
\end{center}
one obtains 
\begin{verbatim}
  BR    PT  TY LAB    PAR(1)     ...    PAR(2)        PAR(35)       PAR(36)  
   1    34  UZ   9  5.180323E+00 ...  6.385506E-02  3.349720E-09  9.361957E-02
\end{verbatim}
which means another non-central saddle-node homoclinic bifurcation occurs
at
$$
D_2=(K^2,Z^2)=(5.1803\ldots,0.063855\ldots).
$$
Note that these data were obtained using a smaller value of {\tt NTST} than
the original computation (compare {\tt c.mtn.1} with {\tt c.mtn.2}). The
high original value of {\tt NTST} was only necessary for the first few steps
because the original solution is specified on a uniform mesh. 

\section{Switching between Saddle-Node and Saddle Homoclinic Orbits.}
Now we can switch to continuation of saddle homoclinic orbits at the
located codim 2 points $D_1$ and $D_2$. 
\begin{center}
{\it make third}
\end{center}
starts from $D_1$. Note that now 
\begin{center}
{\tt 
NUNSTAB = 1 \quad IEQUIB = 1}  
\end{center}
has been specified in {\tt h.mtn.3}. Also, test functions $\psi_9$
and $\psi_{10}$ have been activated in order
to monitor for non-hyperbolic equilibria along the homoclinic locus. 
We get the following output
\begin{verbatim}
  BR    PT  TY LAB    PAR(1)     ...    PAR(2)        PAR(29)       PAR(30)    
   1    10      11  7.114523E+00 ...  7.081751E-02 -4.649861E-01  3.183429E-03
   1    20      12  9.176810E+00 ...  7.678731E-02 -4.684912E-01  1.609294E-02
   1    30      13  1.210834E+01 ...  8.543468E-02 -4.718871E-01  3.069638E-02
   1    40  EP  14  1.503788E+01 ...  9.428036E-02 -4.743794E-01  4.144558E-02
\end{verbatim}
The fact that {\tt PAR(29)} and {\tt PAR(30)} do not change sign indicates 
that there are no further non-hyperbolic equilibria
along this branch. Note that restarting in the opposite direction with {\tt IRS=11,
DS=-0.02} 
\begin{center}
{\it make fourth}
\end{center}
will detect the same codim 2 point $D_1$ but now as a zero
of the test-function $\psi_{10}$
\begin{verbatim}
  BR    PT  TY LAB    PAR(1)     ...    PAR(2)        PAR(29)       PAR(30)
  1    10  UZ  15  6.610459E+00  ...  6.932482E-02 -4.636603E-01  1.725013E-09    
\end{verbatim}
Note that the values of {\tt PAR(1)} and {\tt PAR(2)} differ from that at label {\tt 4} 
only in the sixth significant figure. 

Actually, the program runs further and
eventually computes the point $D_2$ and the whole lower branch of $P$
emanating from it, however, the solutions between $D_1$ and $D_2$
should be considered as spurious\footnote{\label{ft1} The program actually
computes the saddle-saddle heteroclinic orbit bifurcating from the
non-central saddle-node homoclinic at the point $D_1$, see
\citeasnoun[Fig. 2]{ChKuSa:95}, and continues it to the one emanating from
$D_2$.}, therefore we do not save these data.
The reliable way to compute the lower branch of $P$ is to restart computation
of saddle homoclinic orbits in the other direction from the point $D_2$
\begin{center}
{\it make fifth}
\end{center}
This gives the lower branch of $P$ approaching the BT point
(see Figure \ref{SNF.1})
\begin{verbatim}
  BR    PT  TY LAB    PAR(1)     ...   PAR(2)        PAR(29)       PAR(30)    
   1    10      15  4.966429E+00 ... 6.298418E-02 -4.382426E-01  4.946824E-03
   1    20      16  4.925379E+00 ... 7.961214E-02 -3.399102E-01  3.288447E-02
   1    30      17  7.092267E+00 ... 1.587114E-01 -1.692842E-01  3.876291E-02
   1    40  EP  18  1.101819E+01 ... 2.809825E-01 -3.482651E-02  2.104384E-02
\end{verbatim}
The data are appended to the stored results in {\tt b.1}, {\tt s.1} and
{\tt d.1}. One could now display all data using the \AUTO
command {\it @p 1} to reproduce the curve $P$ shown in Figure
\ref{SNF.1}.
\par
\begin{figure}[p]
\epsfysize 10.0cm
\centerline{\epsffile{include/mtn2.ps}}
\caption{Approximation by a large-period cycle}
\label{SNF.2}
\end{figure}
\begin{figure}[p]
\epsfysize 9.0cm
\centerline{\epsffile{include/mtn3.ps}}
\caption{Projection onto the ($K,D_0$)-plane of the 
three-parameter curve of non-central  saddle-node homoclinic orbit}
\label{SNF.3}
\end{figure}
%
It is worthwhile to compare the homoclinic curves computed above with
a curve $T_0=const$ along which the system has a limit cycle of constant large
period $T_0=1046.178$, which can easily be computed using \AUTO or
{\cal locbif}. Such a curve is plotted in Figure \ref{SNF.2}. 
It obviously approximates well the saddle homoclinic loci of $P$, but 
demonstrates much bigger
deviation from the saddle-node homoclinic segment $D_1D_2$. This happens
because the period of the limit cycle grows to infinity while approaching both
types of homoclinic orbit, but with {\it different asymptotics}: 
as $-\ln\|\alpha-\alpha^*\|$, in the saddle homoclinic case, and 
as $\|\alpha-\alpha^*\|^{-1}$ in the saddle-node case. 

\section{Three-Parameter Continuation.}
Finally, we can follow the curve of non-central saddle-node homoclinic
orbits in three parameters. The extra continuation parameter is
$D_0$={\tt PAR(3)}.  To achieve this we restart at label {\tt 4},
corresponding to the codim 2 point $D_1$. We return to continuation of
saddle-node homoclinics, {\tt NUNSTAB=0},{\tt IEQUIB=2}, but append the
defining equation $\psi_{15}=0$ to the continuation problem
(via {\tt NFIXED=1}, {\tt IFIXED(1)=15}). The new
continuation problem is specified in {\tt c.mtn.6} and {\tt h.mtn.6}.
\begin{center}
{\it make sixth}
\end{center}
Notice that we set {\tt ILP=1} and choose {\tt PAR(3)} as the first 
continuation parameter so that \AUTO can detect limit points 
with respect to this parameter. We also make a user-defined function
({\tt NUZR=1})
to detect intersections with the plane $D_0=0.01$.
We get among other output
\begin{verbatim}
  BR    PT  TY LAB    PAR(3)        L2-NORM    ...    PAR(1)        PAR(2)
   1    22  LP  19  1.081212E-02  5.325894E+00 ...  5.673631E+00  6.608184E-02
   1    31  UZ  20  1.000000E-02  4.819681E+00 ...  5.180317E+00  6.385503E-02
\end{verbatim}
the first line of which represents the $D_0$ value at which 
the homoclinic curve $P$ has a tangency with the branch $t_2$ 
of fold bifurcations. Beyond this value of $D_0$,
$P$ consists entirely of saddle homoclinic orbits. The data at label {\tt 20} 
reproduce the coordinates of the point $D_2$. The results of this
computation and a similar one starting from $D_1$ in the opposite direction
(with {\tt DS=-0.01}) are displayed in Figure \ref{SNF.3}.
%

\newpage
\section{ Detailed \AUTO-Commands.}
\begin{table}[htbp]
\begin{center}
\begin{tabular}{| l | l |}
\hline
  \AUTO-COMMAND  & ACTION \\
\hline
%==============================================================================
  {\it ! mkdir mtn} & create an empty work directory \\ 
  {\it cd mtn} & change directory \\
  {\it demo('mtn')} & copy the demo files to the work directory \\
\hline
%==============================================================================
  {\it us('mtn')} & use the starting data in {\tt mtn.dat} to create {\tt s.dat} \\ 
  {\it run(c='mtn.1',h='mtn.1',s='dat')} &  continue saddle-node homoclinic orbit\\
  {\it sv('1') } & save output-files as {\tt b.1, s.1, d.1} \\ 
\hline
%==============================================================================
  {\it run(c='mtn.2',h='mtn.2',s='1')} & continue in opposite direction; restart from {\tt s.1} \\ 
  {\it ap('1') } & append output-files to {\tt b.1, s.1, d.1} \\ 
\hline
%=============================================================================
  {\it run(c='mtn.3',h='mtn.3',s='1')} & switch to saddle homoclinic orbit  ; restart from {\tt s.1} \\ 
  {\it ap('1') } & append output-files to {\tt b.1, s.1, d.1} \\ 
\hline
%==============================================================================
  {\it run(c='mtn.4',h='mtn.4',s='1')} & continue in reverse direction; restart from {\tt s.1} \\ 
  {\it sv('4') } & save output-files as {\tt b.4, s.4, d.4} \\ 
\hline
%=============================================================================
  {\it run(c='mtn.5',h='mtn.5',s='1')} & other saddle homoclinic orbit branch; restart from {\tt s.1} \\
  {\it ap('1') } & append output-files to {\tt b., s.1, d.1} \\ 
\hline
%==============================================================================
  {\it run(c='mtn.6',h='mtn.6',s='1')} & 3-parameter non-central saddle-node homoclinic. \\ 
  {\it sv('6') } & save output-files as {\tt b.6, s.6, d.6} \\ 
\hline
%==============================================================================
\end{tabular}
\caption{Detailed \AUTO-Commands for running demo {\tt mtn}.}
\label{tbl:demo_mtn_1}
\end{center}
\end{table}


