%==============================================================================
%==============================================================================
\chapter{ {\cal HomCont}.} \label{ch:HomCont}
%==============================================================================
%==============================================================================
\section{ Introduction.} \label{sec:HomCont_Intro}
{\cal HomCont} is a collection of subroutines for the continuation 
of homoclinic solutions to ODEs in two or more parameters.
The accurate detection and multi-parameter continuation of certain
codimension-two singularities is allowed for, including all known
cases that involve a unique homoclinic orbit at the singular point.
Homoclinic connections to hyperbolic and non-hyperbolic equilibria are 
allowed as are certain heteroclinic orbits. 
Homoclinic orbits in reversible systems can also be computed.
The theory behind the methods used is
explained in \citeasnoun{ChKu:94}, \citeasnoun{BaCh:94},
\citename{Sa:95} \citeyear{Sa:95,Sa:95a}, \citeasnoun{ChKuSa:95} and
references therein.  The final cited paper contains a concise
description of the present version. 

The current implementation of {\cal HomCont} must be considered as experimental,
and updates are anticipated.
The {\cal HomCont} subroutines are in the file \filef{ auto/2000/src/autlib5.c}. 
Expert users wishing to modify the routines may look there.
Note also that at present, {\cal HomCont} can be run only in 
\AUTO Command Mode and not with the GUI. 


\section{{\cal HomCont} Files and Subroutines.} \label{sec:HomCont_files}

In order to run {\cal HomCont} one must prepare an equations file \filef{ xxx.c}, 
where \filef{ xxx} is the name of the example, 
and two constants-files \filef{ c.xxx} and \filef{ h.xxx}.
The first two of these files are in the standard \AUTO format, 
whereas the \filef{ h.xxx} file
contains constants that are specific to homoclinic continuation.
The choice \parf{ IPS}=9 in \filef{ c.xxx} specifies the problem as
being homoclinic continuation, in which case \filef{ h.xxx} is required.

The equation-file \filef{ kpr.c} serves as a sample for new equation
files. It contains the C subroutines 
\funcf{ func}, \funcf{ stpnt}, \funcf{ pvls}, \funcf{ bcnd}, \funcf{ icnd} 
and \funcf{ fopt}. The final three are
dummy subroutines which are never needed for homoclinic continuation.
Note a minor difference in \funcf{ stpnt} and \funcf{ pvls} with other 
\AUTO equation-files, in that the common block 
\parf{ /BLHOM/} is required.

The constants-file \filef{ c.xxx} is identical in format to other
\AUTO constants-files. Note that the values of the constants
\parf{ NBC} and \parf{ NINT} are irrelevant, as these are set
automatically by the choice \parf{ IPS}=9. Also, the choice \parf{ JAC=1}
is strongly recommended, because the Jacobian is used extensively for
calculating the linearization at the equilibria and hence for
evaluating boundary conditions and certain test functions. However,
note that \parf{ JAC=1} does not necessarily mean that {\cal auto} will
use the analytically specified Jacobian for continuation.

\section{ {\cal HomCont}-Constants.} \label{sec:HomCont_Constants}
An example for the additional file \filef{ h.xxx} is listed below:
\begin{verbatim}
          1 2 1 1 1    NUNSTAB,NSTAB,IEQUIB,ITWIST,ISTART
          0            NREV,(/,I,IREV(I)),I=1,NREV)
          1            NFIXED,(/,I,IFIXED(I)),I=1,NFIXED)
            13
          1            NPSI,(/,I,IPSI(I)),I=1,NPSI)
            9 10 13
\end{verbatim}
The constants specified in \filef{ h.xxx} have the following meaning. 

\subsection{\parf{ NUNSTAB}}  \label{sec:NUNSTAB}

Number of unstable eigenvalues of the left-hand equilibrium (the equilibrium 
approached by the orbit as $t \to -\infty$).


\subsection{\parf{ NSTAB}}  \label{sec:NSTAB}
Number of stable eigenvalues of the right-hand equilibrium (the equilibrium
approached by the orbit as $t \to +\infty$).

\subsection{\parf{ IEQUIB}}  \label{sec:IEQUIB}
\begin{itemize}
\item[-] \parf{ IEQUIB=0}~: 
Homoclinic orbits to hyperbolic equilibria;  
the equilibrium is specified explicitly in \funcf{ pvls} and stored in
\parf{ par[10+I]}, \parf{ I=1,NDIM}.
\item[-] \parf{ IEQUIB=1}~: 
Homoclinic orbits to hyperbolic equilibria;  
the equilibrium is solved for during continuation. Initial values for
the equilibrium are stored in \parf{ par[10+I]}, \parf{ I=1,NDIM} in \funcf{ stpnt}.
\item[-] \parf{ IEQUIB=2}~: 
Homoclinic orbits to a saddle-node; initial values for
the equilibrium are stored in \parf{ par[10+I]}, \parf{ I=1,NDIM} in \funcf{ stpnt}.
\item[-] \parf{ IEQUIB=-1}~: 
Heteroclinic orbits to hyperbolic equilibria;
the equilibria are specified explicitly in \funcf{ pvls} and stored in
\parf{ par[10+I]},  
\parf{ I=1,NDIM} (left-hand equilibrium) and \parf{ par[10+I]}, 
\parf{ I=NDIM+1,2*NDIM} (right-hand equilibrium). 
\item[-] \parf{ IEQUIB=-2}~: 
Heteroclinic orbits to hyperbolic equilibria;
the equilibria are solved for during continuation. Initial values are
specified in \funcf{ stpnt} and stored in \parf{ par[10+I]}, \parf{ I=1,NDIM} (left-hand equilibrium), 
\parf{ par[10+I]}, \parf{ I=NDIM+1,2*NDIM} (right-hand equilibrium).
\end{itemize}

\subsection{\parf{ ITWIST}}  \label{sec:ITWIST}
\begin{itemize}
\item[-] \parf{ ITWIST=0}~: 
the orientation of the homoclinic orbit is not computed.
\item[-] \parf{ ITWIST=1}~: 
the orientation of the homoclinic orbit is computed. For this purpose, the
adjoint variational equation is solved for the unique bounded
solution. If \parf{ IRS = 0}, an initial solution to the adjoint equation
must be specified as well. However, if \parf{ IRS>0} and \parf{ ITWIST} 
has just been increased from zero, then \AUTO will
automatically generate the initial solution to the adjoint. 
In this case, a dummy Newton-step should be performed, see 
Section~\ref{sec:Starting_strategies} for more details.
\end{itemize}

\subsection{\parf{ ISTART}}  \label{sec:ISTART}
\begin{itemize}
\item[-] \parf{ ISTART=1}~:  
This option is obsolete in the current version. 
It may be used as a flag that a solution is to be
restarted from a previously computed point or
from numerical data converted into \AUTO format using \commandf{us}.
In this case \parf{ IRS>0}.
%If \parf{ IRS=0} then starting data are read from the file 
%\filef{ fort.4} (copied from \filef{xxx.dat} in the examples).
%These data must be $t,U$ in multi-column format at each 
%point with $t$ in the interval $[0,1]$. If \parf{ IRS}$\neq$0$,
%this value of \parf{ ISTART} should be used to read starting data from
%a previously-obtained output point from \AUTO.
\item[-] \parf{ ISTART=2}~: 
If \parf{ IRS=0}, an explicit solution must be specified in the
subroutine \funcf{ stpnt} in the usual format. 
\item[-] \parf{ ISTART=3}~: 
The ``homotopy'' approach is used for starting, see
Section~\ref{sec:Starting_strategies}
for more details. Note that this is not available with the choice 
\parf{ IEQUIB=2}.
\item[-] \parf{ ISTART=4}~:
A phase-shift is performed for homoclinic orbits to let the
equilibrium (either fixed or non-fixed, depending on IEQUIB)
correspond to $t=0$ and $t=1$. This is necessary if a periodic
orbit that is close to a homoclinic orbit is continued into
a homoclinic orbit.
\item[-] \parf{ ISTART=-N, $N=1,2,3,\ldots$}~:
Homoclinic branch switching: this description is for reference only
and we refer to Chapter~\ref{ch:HomCont_hbs} to see how this can be used
in actual practice and to \citeasnoun{OlChKr:02} for
theory and background.

The orbit is split into $N+1$ parts and
AUTO sees it as an $(N+1)\times$\parf{NDIM}-dimensional object.
The first part $u_0$ goes from the equilibrium to the point $x_0$ that is 
furthest from the equilibrium. 
Then follow $N-1$ shifted copies of the orbit, which travel
from the point $x_0$ back to the point $x_0$. The last part $U_N$
goes from the point $x_0$ back to the equilibrium. 
The derivatives $\dot{x}_0$ with respect to time
of the point that is furthest from the equilibrium are stored at the 
parameters \parf{par[NPARX-NDIM...NPARX-1]}.

If \parf{ITWIST=1}, and this was also the case in the preceding run,
then a copy of the adjoint vector $\Psi$ at $x_0$ is stored at the parameters
\parf{par[NPARX-NDIM*2...NPARX-NDIM-1]} and Lin's method can be used
to do homoclinic branch switching. To be more precise, the individual parts
$u_i$ and $u_{i+1}$ are at distances $\varepsilon_i$ away from each
other, along the Lin vector $Psi$, at the left- and right-hand end
points. These gaps $\varepsilon_i$ are at parameters
\parf{par[19+2*i]}. Moreover, each part (except $u_{N+1}$) ends at
at a Poincar\'e section which goes through $x_0$ and is perpendicular
to $\dot{x}_0$.

The times $T_i$ that each part $u_i$ takes are stored as follows: 
$T_0=$\parf{par[9]}, $T_N=$\parf{par[10]} and $T_i=$\parf{par[18+2*i]}
for $i=1\ldots N-1$. Through a continuation in problem parameters,
gaps $\varepsilon_i$, and times $T_i$ 
it is possible to switch from a $1$-homoclinic to
an $N$-homoclinic orbit.

If \parf{ITWIST=0}, the adjoint vector is not computed and Lin's
method is not used. Instead, AUTO produces a gap
$\varepsilon$=\parf{par[21]} at the right-hand end point $p$ of
$u_{N+1}$, measuring the distance between the stable manifold of the
equilibrium and $p$. This technique can also be used to find 2-homoclinic
orbits, by varying in $\varepsilon$ and $T_1$, similar to the method
described before, but only if the unstable manifold in
one-dimensional. Because this method is more limited than the method
using Lin vectors, we do not recommend it for normal usage.

To switch back to a normal homoclinic orbit, set \parf{ISTART} back to
a positive value such as 1. Now HomCont has lost all the information
about the adjoint, so if \parf{ITWIST} is set to 0, HomCont
does a normal continuation without the adjoint, and
if \parf{ITWIST} is set to 1, one needs to do a Newton dummy step
first to recalculate the adhoint.
\end{itemize}

\subsection{\parf{ NREV, IREV}}  \label{sec:IREV}
If \parf{ NREV=1} then it is assumed that
the system is reversible under the transformation 
$t \to -t$ and $U(i) \to -U(i)$ for all $i$ with 
\parf{ IREV(i)>0}. Then only half the homoclinic solution is
solved for with right-hand boundary conditions specifying
that the solution is symmetric under the reversibility
(see \citeasnoun{ChSp:93}). The number of free parameters
is then reduced by one. Otherwise \parf{ IREV=0}.

\subsection{\parf{ NFIXED, IFIXED}}  \label{sec:IFIXED}
Number and labels of test functions that are held fixed. 
E.g., with \parf{ NFIXED=1} one can compute a locus in
one extra parameter of a singularity defined by 
test function \parf{ PSI(IFIXED(1))=0}.

\subsection{\parf{ NPSI, IPSI}}  \label{sec:IPSI}
Number and labels of activated test functions for detecting homoclinic
bifurcations, see Section~\ref{sec:HomCont_Test_functions} 
for a list. If a test function is activated then the
corresponding parameter (\parf{ IPSI(I)+19}) 
must be added to the list of continuation parameters \parf{ NICP,(ICP(I),I=1 NICP)}
and zero of this parameter added to the list of user-defined
output points \parf{ NUZR,} \parf{ (/,I,par[I]),I=1, NUZR} in \filef{ c.xxx}.

\section{ Restrictions on {\cal HomCont} Constants.}
Note that certain combinations of these constants are not allowed
in the present implementation. In particular,
\begin{itemize}
\item[-] 
The computation of orientation \parf{ ITWIST=1} is not
implemented for \parf{ IEQUIB<0} (heteroclinic orbits), 
\parf{ IEQUIB=2} (saddle-node homoclinics),
\parf{ IREV=1} (reversible systems), \parf{ ISTART=3} (homotopy
method for starting), or if the equilibrium contains complex
eigenvalues in its linearization.  
\item[-] The homotopy method \parf{ ISTART=3} is not fully implemented
for heteroclinic connections \parf{ IEQUIB<0}, saddle-node homoclinic
orbits \parf{ IEQUIB=2} or reversible systems \parf{ IREV=1}.
\item[-] Certain test functions are not valid for certain forms
of continuation 
(see Section~\ref{sec:HomCont_Test_functions} below); 
for example
\parf{ PSI(13)} and \parf{ PSI(14)} only make sense if 
\parf{ ITWIST=1} and \parf{ PSI(15)} and \parf{ PSI(16)} only apply
to \parf{ IEQUIB=2}.
\end{itemize}

\section{ Restrictions on the Use of \parf{ par}.}
The parameters \parf{ par[0]} -- \parf{ par[8]} can be used freely by
the user. The other parameters are used as follows~:

\begin{itemize}

\item[-] \parf{ par[10]}~: 
The value of \parf{ par[10]} equals the length of the time interval over
which a homoclinic solution is computed. Also referred to as ``period''.
This must be specified in \funcf{ stpnt}.

\item[-] \parf{ par[9]}~: 
If \parf{ ITWIST=1} then \parf{ par[9]} is used internally as a
dummy parameter so that the adjoint equation is well-posed.

\item[-] \parf{ par[11]-par[19]}~:
These are used for specifying the 
equilibria and (if \parf{ ISTART=3}) the artificial parameters of
the homotopy method (see Section~\ref{sec:Starting_strategies} below).

\item[-] \parf{ par[20]-par[35]}~: 
These parameters are used for storing the test functions 
(see Section~\ref{sec:HomCont_Test_functions}).
\end{itemize}

The output is in an identical format to \AUTO except that
additional information at each computed point is written 
in \filef{ fort.9}. This information comprises the eigenvalues of
the (left-hand) equilibrium, the values of each activated test
function and, if \parf{ ITWIST=1}, 
whether the saddle homoclinic loop is orientable
or not.
Note that the statement about orientability is only meaningful if the
leading eigenvalues are not complex and the homoclinic solution is not
in a flip configuration, that is, none of the test functions 
$\psi_i$ for $i=11,12,13,14$ is zero (or close to zero), 
see Section~\ref{sec:HomCont_Test_functions}.
 Finally, the values of the \parf{ NPSI} activated test functions are written. 

\section{ Test Functions.} \label{sec:HomCont_Test_functions}
Codimension-two homoclinic orbits are detected along branches of codim
1 homoclinics by locating zeroes of certain test functions
$\psi_i$. The test functions that are ``switched on'' during any
continuation are given by the choice of the labels $i$, and are
specified by the parameters \parf{ NPSI,(/,I,IPSI(I)),I=1,NPSI)} in \filef{
h.xxx}.  Here \parf{ NPSI} gives the number of activated test functions
and \parf{ IPSI(1),$\ldots$,IPSI(NPSI)} give the labels of
the test functions (numbers between 1 and 16). A zero of
each labeled test function defines a certain codimension-two 
homoclinic singularity, specified as follows.
The notation used for eigenvalues is the same as that in
\citeasnoun{ChKu:94} or \citeasnoun{ChKuSa:95}. 

\begin{itemize}
\item[-] $ i=1$: 
Resonant eigenvalues (neutral saddle); $\mu_1=-\lambda_1$.
\item[-] $ i=2$: 
Double real leading stable eigenvalues (saddle to saddle-focus
transition); $\mu_1=\mu_2$. 
\item[-] $ i=3$: 
Double real leading unstable eigenvalues (saddle to saddle-focus
transition);\\ 
$\lambda_1=\lambda_2$. 
\item[-] $ i=4$: 
Neutral saddle, saddle-focus or bi-focus (includes $ i=1$);
$\mbox{Re}(\mu_1)  =  - \mbox{Re}(\lambda_1)$. 
\item[-] $ i=5$: 
Neutrally-divergent saddle-focus (stable eigenvalues complex);\\
$\mbox{Re}(\lambda_1) = - \mbox{Re}(\mu_1) - \mbox{Re}(\mu_2)$.
\item[-] $ i=6$: 
Neutrally-divergent saddle-focus (unstable eigenvalues complex);\\
$\mbox{Re}(\mu_1) = - \mbox{Re}(\lambda_1) - \mbox{Re}(\lambda_2)$. 
\item[-] $ i=7$: 
Three leading eigenvalues (stable);
$\mbox{Re}(\lambda_1) = - \mbox{Re}(\mu_1) - \mbox{Re}(\mu_2)$. 
\item[-] $ i=8$: 
Three leading eigenvalues (unstable);
$\mbox{Re}(\mu_1) = - \mbox{Re}(\lambda_1) - \mbox{Re}(\lambda_2)$.
\item[-] $ i=9$: 
Local bifurcation (zero eigenvalue or Hopf): 
number of stable eigenvalues decreases; $\mbox{Re}(\mu_1)=0$.
\item[-] $ i=10$: 
Local bifurcation (zero eigenvalue or Hopf): 
number of unstable eigenvalues decreases; $\mbox{Re}(\lambda_1)=0$.
\item[-] $ i=11$: 
Orbit flip with respect to leading stable direction 
(e.g., 1D unstable manifold).
\item[-] $ i=12$: 
Orbit flip with respect to leading unstable direction, 
(e.g., 1D stable manifold).
\item[-] $ i=13$: 
Inclination flip with respect to stable manifold
(e.g., 1D unstable manifold).
\item[-] $ i=14$: 
Inclination flip with respect to unstable manifold
(e.g., 1D stable manifold).
\item[-] $ i=15$: 
Non-central homoclinic to saddle-node (in stable manifold).
\item[-] $ i=16$: 
Non-central homoclinic to saddle-node (in unstable manifold).
\end{itemize}

Expert users may wish to add their own test functions by editing 
the function \parf{ PSIHO} in \filef{ autlib5.c}.

{\it It is important to remember that, in order to specify activated 
test functions, it is required to also 
add the corresponding label $+20$ to the list of continuation
parameters and a zero of this parameter to the list of user-defined
output points. Having done this, the corresponding parameters
are output to the screen and zeros are accurately located.} 

\section{ Starting Strategies.} \label{sec:Starting_strategies}
There are four possible starting procedures for continuation. 

\begin{itemize}

\item[{\bf(i)}]
Data can be read from a previously-obtained output point from \AUTO
 (e.g., from continuation of a periodic orbit up to large period;
note that if the end-point of the data stored is not close to the
equilibrium, a phase shift must be performed by setting
\parf{ISTART=4}). These data can be read from fort.8 (saved to \filef{
s.xxx}) by making \parf{ IRS} correspond to the label of the data
point in question.

\item[{\bf(ii)}]
Data from numerical integration (e.g.,\ computation of a stable
periodic orbit, or an approximate homoclinic obtained by shooting)  
can be read in from a data file using the general \AUTO 
utility \commandf{ us} (see earlier in the manual). 
The  numerical data should be stored in
a file  \filef{ xxx.dat}, in multi-column format according to the read statement
\begin{verbatim}
       READ(...,*) T(J),(U(I,J),I=1,NDIM)
\end{verbatim}
where $T$ runs in the interval $[0,1]$.
After running \commandf{ us} the restart data is stored in
the format of a previously computed solution in \filef{ s.dat}.
When starting from this solution \parf{ IRS} should be set to 1 and 
the value of \parf{ ISTART} is irrelevant.

\item[{\bf(iii)}]
By setting \parf{ ISTART=2},  
an explicit homoclinic solution can be specified in the routine \funcf{ stpnt} in the usual \AUTO format, that is 
$U=...(T)$ where $T$ is scaled to lie in the
interval$[0,1]$. 

\item[{\bf(iv)}]
The choice \parf{ ISTART=3}, allows for
a homotopy method to be used to approach a homoclinic orbit
starting from a small approximation to a solution to the 
linear problem in the unstable manifold \cite{DoFrMo:93}. For
details of implementation, the reader is referred to 
Section~5.1.2.\ of \citeasnoun{ChKu:94}, under the simplification
that we do not solve for the adjoint $u(t)$ here. The basic idea
is to start with a small solution in the unstable manifold, and perform
continuation in \parf{ par[10]=}$2T$ and dummy initial-condition 
parameters $\xi_i$ in order to satisfy the correct right-hand boundary
conditions, which are defined by zeros of other dummy parameters
$\omega_i$. More precisely, the left-hand end point is placed in the
tangent space to the unstable manifold of the saddle and is characterized by
\parf{ NUNSTAB} coordinates $\xi_i$ satisfying the condition
$$
\xi_1^2 + \xi_2^2 + \ldots +\xi_\parf{ NUNSTAB}^2  = \eps_0^2,
$$
where $\eps_0$ is a user-defined small number.
At the right-hand end point, \parf{ NUNSTUB} values $\omega_i$ 
measure the deviation of this point from the tangent
space to the stable manifold of the saddle. 
\par
Suppose that \parf{ IEQUIB=0,1} and set \parf{ IP=11+IEQUIB*NDIM}. Then
\par
\medskip
\begin{center}
\begin{tabular}{ll}
\parf{ par[IP]} & :$\ \ \eps_0$\\
\parf{ par[IP+i]} &  :$\ \ \xi_i$, \parf{ i=1,2,...,NUNSTAB}\\
\parf{ par[IP+NUNSTAB+i]} & :$\ \ \omega_i$, \parf{ i=1,2,...,NUNSTAB}
\end{tabular}
\end{center}
\par
\medskip
{\it Note that to avoid interference with the test functions 
(i.e. \parf{ par[20]-par[35]}), one must have \parf{ IP+2*NUNSTAB < 21}.} 
\par
If an $\omega_i$ is vanished, it can be frozen while another dummy or system parameter is allowed to
vary in order to make consequently all $\omega_i=0$. The resulting final solution
gives the initial homoclinic orbit provided the right-hand end point
is sufficiently close to the saddle. 
See Chapter~\ref{ch:HomCont_kpr} for an example, 
however, we recommend the homotopy method only for ``expert users''.
\end{itemize}

To compute the orientation of a homoclinic orbit (in order to detect
inclination-flip bifurcations) it is necessary to compute, in tandem,
a solution to the modified adjoint variational equation, by setting
\parf{ ITWIST=1}. In order to obtain starting data for such a
computation when restarting from a point where just the homoclinic
is computed, upon increasing \parf{ ITWIST} to 1, \AUTO generates
trivial data for the adjoint. Because the adjoint equations are
linear, only a single step of Newton's method is required to
enable these trivial data to converge to the correct unique bounded
solution. This can be achieved by making a single continuation step in a
trivial parameter (i.e. a parameter that does not appear
in the problem). 

Decreasing \parf{ ITWIST} to 0 automatically deletes the data for the adjoint
from the continuation problem.


\section{ Notes on Running {\cal HomCont} Demos.} \label{sec:HomCont_Tutorial_examples}
{\cal HomCont} demos are given in the following chapters.
To copy all files of a demo \filef{ xxx} (for example, \filef{ san}),
move to a clean directory and type {\it demo('xxx')}.
Simply typing {\it make} or {\it make all} will then automatically
execute all runs of the demo.
%To automatically run a demo in ``step-by-step'' mode,
%type  {\it make first}, {\it make second}, etc.,
%to run each separate computation of the demo. 
At each step, the user is encouraged to plot the data
saved by using the command {\it plot} (e.g., {\it plot('1')} plots the data
saved in \filef{ b.1} and \filef{ s.1}).

Of course, in a real application, the runs will not have been prepared
in advance, and \AUTO-commands must be used.
Such commands can be found in a table at the end of each chapter.
A sequence of detailed \AUTO-commands will be given in these tables
as illustrated in Table~\ref{tbl:HomCont_demos_1} 
and Table~\ref{tbl:HomCont_demos_2} for two representative runs of 
{\cal HomCont} demo \filef{ san}.

The user is encouraged to copy the format of one of these demos
when constructing new examples.

The output of the {\cal HomCont} demos reproduced in  the following chapters
is somewhat machine dependent, as already noted 
in Section~\ref{sec:Tutorial_all_runs}.
In exceptional circumstances, \AUTO may reach its maximum number of
steps \parf{ NMX} before a certain output point, or the label of
an output point may change. In such case the user may have
to make appropriate changes in the \AUTO constants-files.


\begin{table}[htbp]
\begin{center}
\begin{tabular}{| l | l |}
\hline
  COMMAND  & ACTION \\
\hline
%==============================================================================
  {\it ld('san')}                & load the problem defition    \\ 
  {\it run(c='san.1',h='san.1')} & get the HomCont constants-file and run \AUTO/{\cal HomCont}\\
  {\it sv('6')}                  & save output-files as \filef{ b.6, s.6, d.6}  \\ 
\hline
%==============================================================================
\end{tabular}
\caption{ An example of \AUTO-Commands.}
\label{tbl:HomCont_demos_1}
\end{center}
\end{table}


\begin{table}[htbp]
\begin{center}
\begin{tabular}{| l | l |}
\hline
  COMMAND  & ACTION \\
\hline
%==============================================================================
  {\it run(c='san.9',h='san.9',s='6')} & \parbox[t]{3in}{get the HomCont constants-file and run \AUTO/{\cal HomCont}; restart solution read from \filef{ s.6}\vspace{0.2cm}}\\
  {\it ap('6')}                        & append output-files to \filef{ b.6, s.6, d.6}  \\ 
\hline
%==============================================================================
\end{tabular}
\caption{ Another example of \AUTO-Commands.}
\label{tbl:HomCont_demos_2}
\end{center}
\end{table}

