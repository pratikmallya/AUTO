
%==============================================================================
%==============================================================================
\chapter{ \AUTO Demos : Connecting orbits.} \label{ch:Demos_Heteroclinics}
%==============================================================================
%==============================================================================

\newpage
%==============================================================================
%DEMO=fsh======================================================================
%==============================================================================
\section{ fsh : A Saddle-Node Connection.} \label{sec:Demos_fsh}
This demo illustrates the computation of travelling wave front solutions
to the Fisher equation,
\begin{equation} \begin{array}{cl}
  & w_t = w_{xx} + f(w),
  \qquad -\infty < x < \infty,
  \quad  t > 0,  \\
  & f(w) \equiv w(1-w) .  \\
\end{array} \end{equation}
We look for solutions of the form $w(x,t)=u(x+ct)$, where
$c$ is the wave speed.
This gives the first order system
\begin{equation} \begin{array}{cl}
  &  u_1'(z)  = u_2(z),  \\
  &  u_2'(z)  = c u_2(z) - f\bigl(u_1(z)\bigr).  \\
\end{array} \end{equation}
Its fixed point $(0,0)$ has two positive eigenvalues when $c>2$.
The other fixed point, $(1,0)$, is a saddle point.
A branch of orbits connecting the two fixed points
requires one free parameter; see 
\citename{FrDo:91} \citeyear{FrDo:91}.
Here we take this parameter to be the wave speed $c$.

In the first run a starting connecting orbit is computed 
by continuation in the period $T$.
This procedure can be used generally for time integration of an ODE with \AUTO.
Starting data in \funcf{stpnt} correspond to a point on the approximate stable manifold
of $(1,0)$, with $T$ small.
In this demo the ``free'' end point of the orbit necessary approaches the
unstable fixed point $(0,0)$.
A computed orbit with sufficiently large $T$ is then chosen as restart orbit
in the second run, where, typically, one replaces $T$ by $c$ as continuation
parameter.
However, in the second run below, we also add a phase condition, 
and both $c$ and $T$ remain free.



\begin{table}[htbp]
\begin{center}
\begin{tabular}{| l | l |}
\hline
  \AUTO-COMMAND  & ACTION \\
\hline
%==============================================================================
  \commandf{ ! mkdir fsh} & create an empty work directory \\ 
  \commandf{ cd fsh} & change directory \\
  \commandf{ demo('fsh')} & copy the demo files to the work directory \\
\hline
%==============================================================================
  \commandf{ run(c='fsh.1')} & continuation in the period $T$, with $c$ fixed; no phase condition \\ 
  \commandf{ sv('0')} & save output-files as \filef{ b.0, s.0, d.0} \\ 
\hline
%==============================================================================
  \commandf{ run(c='fsh.2',s='0')} & \parbox[t]{3in}{continuation in $c$ and $T$, with active phase condition. Constants changed : \parf{ IRS, ICP, NINT, DS} \vspace{0.2cm}} \\ 
  \commandf{ sv('fsh')} & save output-files as \filef{ b.fsh, s.fsh, d.fsh} \\ 
\hline
%==============================================================================
\end{tabular}
\caption{Commands for running demo \filef{ fsh}.}
\label{tbl:demo_fsh}
\end{center}
\end{table}

\newpage
%==============================================================================
%DEMO=nag======================================================================
%==============================================================================
\section{ nag : A Saddle-Saddle Connection.} \label{sec:Demos_nag}
This demo illustrates the computation of traveling wave front solutions
to Nagumo's equation,
\begin{equation} \begin{array}{cl}
  & w_t = w_{xx} + f(w,a),
  \qquad -\infty < x < \infty,
  \quad  t > 0,  \\
  & f(w,a) \equiv w(1-w)(w-a), \qquad 0<a<1.  \\
\end{array} \end{equation}
We look for solutions of the form $w(x,t)=u(x+ct)$, where
$c$ is the wave speed.
This gives the first order system
\begin{equation} \begin{array}{cl}
  &  u_1'(z)  = u_2(z),  \\
  &  u_2'(z)  = c u_2(z) - f\bigl(u_1(z),a\bigr),  \\
\end{array} \end{equation}
where $z=x+ct$, and $' = d/dz$.
If $a=1/2$ and $c=0$ then there are two analytically known
heteroclinic connections, one of which is given by
$$ u_1(z) = {
  {e^{{1 \over 2} \sqrt{2} z}}
  \over
  {1 + e^{{1 \over 2} \sqrt{2} z}}  },
  \qquad  u_2(z) = u_1'(z),  \qquad  -\infty < z < \infty.
  $$
The second heteroclinic connection is obtained by reflecting the
phase plane representation of the first with respect to the
$u_1$-axis.
In fact, the two connections together constitute a heteroclinic cycle.
One of the exact solutions is used below as starting orbit.
To start from the second exact solution, change SIGN=-1 in the  
subroutine \funcf{stpnt} in \filef{ nag.c} and repeat the computations below;
see also
\citename{FrDo:91} \citeyear{FrDo:91}.

\begin{table}[htbp]
\begin{center}
\begin{tabular}{| l | l |}
\hline
  \AUTO-COMMAND  & ACTION \\
\hline
%==============================================================================
  \commandf{ ! mkdir nag} & create an empty work directory \\ 
  \commandf{ cd nag} & change directory \\
  \commandf{ demo('nag')} & copy the demo files to the work directory \\
\hline
%==============================================================================
  \commandf{ run(c='nag.1')} & compute part of first branch of heteroclinic orbits \\ 
  \commandf{ sv('nag')} & save output-files as \filef{ b.nag, s.nag, d.nag} \\ 
\hline
%==============================================================================
  \commandf{ run(c='nag.2',s='nag')} & \parbox[t]{3in}{compute first branch in opposite direction.  Constants changed : \parf{ DS}  \vspace{0.2cm}}\\ 
  \commandf{ ap('nag')} & append output-files to \filef{ b.nag, s.nag, d.nag} \\ 
\hline
%==============================================================================
\end{tabular}
\caption{Commands for running demo \filef{ nag}.}
\label{tbl:demo_nag}
\end{center}
\end{table}

\newpage
%==============================================================================
%DEMO=stw======================================================================
%==============================================================================
\section{ stw : Continuation of Sharp Traveling Waves.} \label{sec:Demos_stw}
This demo illustrates the computation of sharp traveling wave front solutions
to nonlinear diffusion problems of the form
$$ w_t = A(w) w_{xx} + B(w) w_x^{2} + C(w),  $$
with
$A(w) = a_1 w + a_2 w^{2}$,
$B(w) = b_0 + b_1 w + b_2 w^{2}$,
and
$C(w) = c_0 + c_1 w + c_2 w^{2}$.
Such equations can have \commandf{ sharp traveling wave fronts} as solutions, i.e., solutions of the form
$w(x,t)=u(x+ct)$ for which there is a $z_0$ such that
$u(z)=0$ for $z \ge z_0$,
$u(z) \not= 0$ for $z < z_0$, and
$u(z) \rightarrow constant$ as $z \rightarrow -\infty$.
These solutions are actually generalized solutions, since they need
not be differentiable at $z_0$.

Specifically, in this demo a homotopy path will be computed 
from an analytically known exact sharp traveling wave solution of
$$ w_t = 2w w_{xx} + 2 w_x^{2} + w(1-w),  \leqno(1) $$
to a corresponding sharp traveling wave of
$$ w_t = (2w+w^{2}) w_{xx} + w w_x^{2} + w(1-w). \leqno(2) $$
This problem is also considered in
\citename{DoKeKe:91b} \citeyear{DoKeKe:91b}.
For these two special cases the functions $A,B,C$ are defined
by the coefficients in Table~\ref{tbl:demo_stw_1}.

\begin{table}[htbp]
\begin{center}
\begin{tabular}{| l | l | l | l | l | l | l | l | l |}
\hline
         & $a_1$ & $a_2$ & $b_0$ &$b_1$ &$b_2$ &$c_0$ &$c_1$ &$c_2$ \\
\hline
Case (1) &  2    &   0   &   2   &  0   &  0   &  0   &  1   &  -1 \\
\hline
Case (2) &  2    &   1   &   0   &  1   &  0   &  0   &  1   &  -1 \\
\hline
\end{tabular}
\caption{Problem coefficients in demo \filef{ stw}.}
\label{tbl:demo_stw_1}
\end{center}
\end{table}

With $w(x,t)=u(x+ct)$, $z=x+ct$, one obtains the reduced system
\begin{equation} \begin{array}{cl}
  & u_1'(z) = u_2,  \\
  & u_2'(z) = \bigl[c u_2 - B(u_1) u_2^{2} - C(u_1) \bigr]/A(u_1). \\
\end{array} \end{equation}
To remove the singularity when $u_1=0$, we apply a
nonlinear transformation of the independent variable 
(see \citename{Ar:80} \citeyear{Ar:80}), viz.,
${d / d \tilde z} = A(u_1) {d / dz}$,
which changes the above equation into
\begin{equation} \begin{array}{cl}
  & u_1'(\tilde z) = A(u_1) u_2,  \\
  & u_2'(\tilde z) = c u_2 - B(u_1) u_2^{2} - C(u_1). \\
\end{array} \end{equation}
Sharp traveling waves then correspond to heteroclinic connections
in this transformed system.

\newpage 
Finally, we  map $ [0,T] \rightarrow [0,1] $
by the transformation $\xi = \tilde z / T$.
With this scaling of the independent variable, the reduced system
becomes
\begin{equation} \begin{array}{cl}
  & u_1'(\xi) = T A(u_1) u_2,  \\
  & u_2'(\xi) = T \bigl[ c u_2 - B(u_1) u_2^{2} - C(u_1)\bigr]. \\
\end{array} \end{equation}
For Case~1 this equation has a known exact solution, namely,
$$ u(\xi) = { 1 \over 1 + exp(T\xi) }, \qquad
  v(\xi) = { -{1 \over 2}  \over 1 + exp(-T\xi) }. $$
This solution has wave speed $c=1$.
In the limit as $T \rightarrow \infty$ its phase plane trajectory
connects the stationary points $(1,0)$ and $(0,-{1 \over 2})$.
 
The sharp traveling wave in Case~2
can now be obtained using the following homotopy.
Let
$(a_1,a_2,b_0,b_1,b_2) =
  (1-\lambda) (2,0,2,0,0) + \lambda (2,1,0,1,0)$.
Then as $\lambda$ varies continuously from 0 to 1, the parameters
$(a_1,a_2,b_0,b_1,b_2)$
vary continously from the values for Case~1
  to the values for Case~2.


\begin{table}[htbp]
\begin{center}
\begin{tabular}{| l | l |}
\hline
  \AUTO-COMMAND  & ACTION \\
\hline
%==============================================================================
  \commandf{ ! mkdir stw} & create an empty work directory \\ 
  \commandf{ cd stw} & change directory \\
  \commandf{ demo('stw')} & copy the demo files to the work directory \\
\hline
%==============================================================================
  \commandf{ run(c='stw.1')} & continuation of the sharp traveling wave \\ 
  \commandf{ sv('stw')} & save output-files as \filef{ b.stw, s.stw, d.stw} \\ 
\hline
%==============================================================================
\end{tabular}
\caption{Commands for running demo \filef{ stw}.}
\label{tbl:demo_stw_2}
\end{center}
\end{table}
