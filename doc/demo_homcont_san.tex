

%==============================================================================
%==============================================================================
\chapter{ {\cal HomCont} Demo : san.} \label{ch:HomCont_san}
%==============================================================================
%==============================================================================

%==============================================================================
%DEMO=san======================================================================
%==============================================================================
\section{ Sandstede's Model.}
\newcommand{\ti}{\tilde}
Consider the system \cite{Sa:95b}
\begin{equation} \label{bs1} \begin{array}{rcl}
\dot{x} & = & a \, x + b \, y - a \, x^2 + (\ti \mu - \alpha \, z) \, x
\, (2-3x) \\
\dot{y} & = & b \, x + a \, y - \frac{3}{2} \, b \, x^2 - 
\frac{3}{2} \, a \, x \, y - (\ti \mu - \alpha \, z) \, 2 \, y \\
\dot{z} & = & c \, z + \mu \, x + \gamma\, x\, y + \alpha \, 
\beta \, (x^2 \, (1-x) - y^2)

\end{array} \end{equation}
as given in the file \filef{ san.c}.
Choosing the constants appearing
in (\ref{bs1}) appropriately allows for computing inclination and
orbit flips as well as non-orientable resonant bifurcations, see
\cite{Sa:95b} for details and proofs. The starting point for all
calculations is $a=0$, $b=1$ where there exists an explicit solution
given by  
$$ 
(x(t),y(t),z(t)) = 
\left( 1 - \left(\frac{1-e^t}{1+e^t}\right)^2 , 4 \, e^t \,
\frac{1-e^t}{(1+e^t)^3} , 0 \right). 
$$
This solution is specified in the routine \funcf{ stpnt}.

\section{Inclination Flip.}
We start by copying the demo to the current work directory 
and running the first step
\begin{center}
\commandf{ @dm san }\\
\commandf{ make first }
\end{center}
This computation starts from the analytic solution above with 
$a=0$, $b=1$, $c=-2$, $\alpha=0$, $\beta=1$ and 
$\gamma = \mu=\ti \mu =0$. The homoclinic solution is followed in the
parameters $(a,\ti \mu)$ \parf{ =(PAR(1), PAR(8))} up to $a=0.25$. 
The output is summarised on the screen as
\begin{verbatim}
   BR  PT  TY LAB    PAR(1)        L2-NORM           PAR(8)     
    1   1  EP   1  0.000000E+00  4.000000E-01 ...  0.000000E+00
    1   5  UZ   2  2.500000E-01  4.030545E-01 ... -3.620329E-11
    1  10  EP   3  7.384434E-01  4.339575E-01 ... -9.038826E-09
\end{verbatim}
and saved in more detail as \filef{ b.1}, \filef{ s.1} and \filef{ d.1}.

Next we want to add a solution to the adjoint equation to the
solution obtained at $a=0.25$. This is
achieved by making the change \parf{ ITWIST = 1} saved in \filef{ h.san.2},
and \parf{ IRS = 2},  \parf{ NMX = 2} and \parf{ ICP(1) = 9} saved in 
\filef{ c.san.2}. We also disable any
user-defined functions \parf{ NUZR=0}. The computation so-defined 
is a single step in a trivial parameter \parf{ PAR(9)} (namely a parameter
that does not appear in the problem). The effect is to perform a Newton
step to enable \AUTO to converge to a solution of the adjoint equation.
\begin{center}
\commandf{ make second}
\end{center} 
The output is stored in \filef{ b.2}, \filef{ s.2}  and \filef{ d.2}.

We can now continue the homoclinic plus
adjoint in $(\alpha,\ti \mu)$ \parf{ =(PAR(4), PAR(8))} by
changing the constants (stored in \filef{ c.san.3}) to read
\parf{ IRS = 4}, \parf{ NMX = 50} and \parf{ ICP(1) = 4}.
We also add \parf{ PAR(10)} to the list of continuation parameters
\parf{ NICP,(ICP(I),I=1 NICP)}. Here \parf{ PAR(10)} is a dummy parameter used in
order to make the continuation of the adjoint well posed. Theoretically,
it should be zero if the computation of the adjoint is successful
\cite{Sa:95b}.
The test functions for detecting resonant bifurcations 
(\parf{ ISPI(1)=1}) and inclination flips (\parf{ ISPI(1)=13}) are
also activated. Recall that this should be specified in
three ways. First we add \parf{ PAR(21)} and \parf{ PAR(33)}
to the list of continuation parameters in \filef{ c.san.3}, second we set up user defined
output at zeros of these parameters in the same file, and finally we set \parf{ NPSI=2}
\parf{ (IPSI(1),IPSI(2))=1,13} in \filef{ h.san.3}. We also add to \filef{ c.san.3} another user zero
for detecting when \parf{ PAR(4)=1.0}.
Running 
\begin{center}
\commandf{ make third }\\
\end{center}
reads starting data from \filef{ s.2} and outputs to the screen
\begin{verbatim}
 BR  PT  TY LAB    PAR(4)     ...    PAR(8)        PAR(10)    ...    PAR(33)    
  1  20       5  7.847219E-01 ... -3.001440E-11 -4.268884E-09 ... -1.441124E+01
  1  27  UZ   6  1.000000E+00 ... -3.844872E-11 -4.460769E-09 ... -5.701675E+00
  1  35  UZ   7  1.230857E+00 ... -5.833977E-11 -4.530541E-09 ...  9.434843E-06
  1  40       8  1.383969E+00 ... -8.133899E-11 -4.671817E-09 ...  1.348810E+00
  1  50  EP   9  1.695209E+00 ... -1.386324E-10 -5.098460E-09 ...  5.311065E-01
\end{verbatim}
Full output is stored in \filef{ b.3}, \filef{ s.3} and \filef{ d.3}. 
\begin{figure}[b]
\epsfysize 9.0cm
\centerline{\epsffile{include/san1.ps}}
\caption{Second versus third component of the solution to the adjoint
equation at labels 5, 7 and 9}
\label{Ftest1}
\end{figure}
Note that the artificial parameter $\epsilon=$\parf{ PAR(10)} is zero within
the allowed tolerance. At label \parf{ 7}, a zero of test function $\psi_{13}$ has
been detected which corresponds to an inclination flip with respect to
the stable manifold. That the orientation of the homoclinic loop
changes as the branch passes through this point can be read from
the information in \filef{ d.3}.
However in \filef{ d.3}, the line 
\begin{verbatim} 
ORIENTABLE (    0.2982090775D-03)
\end{verbatim}
at \parf{ PT=35} would seems to contradict the 
detection of the inclination flip at this point. Nonetheless, the
important fact is the zero of the test function; and note that 
the value of the variable indicating the orientation is 
small compared to its value at the other regular points. 
Data for the adjoint equation at \parf{ LAB= 5, 7} and \parf{ 9} at
and on either side of the inclination flip are presented in 
Fig.\ \ref{Ftest1}. The switching of the solution between components
of the leading unstable left eigenvector is apparent.
Finally, we remark that the Newton step in the dummy 
parameter \parf{ PAR(20)} performed above is crucial
to obtain convergence. Indeed, if instead we try to continue the
homoclinic orbit and the solution of the adjoint equation directly by
setting
\begin{verbatim}
  ITWIST = 1   IRS = 2   NMX = 50   ICP(1) = 4   NPUSZR = 0
\end{verbatim}
(as saved in \filef{ c.san.4}) and running
\begin{center}
\commandf{ make fourth}
\end{center}
we obtain a no convergence error.

\section{Non-orientable Resonant Eigenvalues.}
Inspecting the output saved in the third run,
we observe the existence of a non-orientable homoclinic orbit at label 
\parf{ 7} corresponding to \parf{ N=40}. We restart at this label, with
the first continuation parameter being once again $a=$\parf{ PAR(1)}, 
by changing constants and storing them in \filef{ c.san.5} according to 
\begin{verbatim}
   IRS = 7     DS = -0.05D0    NMX = 20    ICP(1) = 1
\end{verbatim}
Running, 
 \begin{center}
\commandf{ make fifth}\\
\end{center}
the output at label \parf{ 10}
\begin{verbatim}
  BR    PT  TY LAB    PAR(1)           PAR(8)        PAR(10)       PAR(21)       
  1     8  UZ  10 -1.304570E-07  ... 3.874816E-12 -1.468457E-09 -2.609139E-07 
\end{verbatim}
indicates that \AUTO has detected a zero of
\parf{ PAR(21)}, implying that a non-orientable resonant bifurcation occurred at that
point.

\section{Orbit Flip.}
In this section we compute an orbit flip. To this end we restart
from the original explicit solution, without computing the orientation. We 
begin by separately performing continuation in $(\alpha,\ti \mu)$, 
$(\beta,\ti \mu)$, $(a,\ti \mu)$, $(b,\ti \mu)$ and $(\mu, \ti \mu)$
in order to reach the parameter values 
$(a,b,\alpha,\beta, \mu)=(0.5,3,1,0,0.25)$.
The sequence of continuations up to the desired parameter values 
are run via
\begin{center}
\it make sixth\\
make seventh\\
make eighth\\
make ninth\\ 
make tenth\\
\end{center}
with appropriate continuation parameters and user output values
set in the corresponding files \filef{ c.san.xx}. 
All the output is saved to \filef{ s.6}.

The final saved point \parf{ LAB=10} contains a homoclinic solution at
the desired parameter values. From here we perform continuation in
the negative direction of $(\mu,\ti \mu)=$ (\parf{ PAR(7),PAR(8)}) with
the test function $\psi_{11}$ for orbit flips with respect to the
stable manifold activated.
\begin{center}
\it make eleventh
\end{center}
The output detects an inclination flip (by a zero of \parf{ PAR(31)}) 
at \parf{ PAR(7)=0} 
\begin{verbatim}
  BR    PT  TY LAB    PAR(7)      ...    PAR(8)        PAR(31)    
  1     5  UZ  12  2.394737E-07   ...  6.434492E-08 -4.133994E-06
\end{verbatim}
at which parameter value the homoclinic orbit is contained in the $(x,y)$-plane
(see Fig.\ \ref{Ftest2}).

\begin{figure}[t]
\epsfysize 9.0cm
\centerline{\epsffile{include/san2.ps}}
\caption{Orbits on either side of the orbit flip bifurcation. The critical
orbit is contained in the $(x,y)$-plane}
\label{Ftest2}
\end{figure}

Finally, we demonstrate that the orbit flip can be continued as 
three parameters (\parf{ PAR(6), PAR(7), PAR(8)}) are varied. 
\begin{center}
\it make twelfth
\end{center}
\begin{verbatim}
 BR    PT  TY LAB    PAR(7)       ...    PAR(8)        PAR(6)     
   1     5      14 -5.374538E-19  ... -1.831991E-10 -3.250000E-01
   1    10      15 -6.145911E-19  ... -2.628607E-10 -8.250001E-01
   1    15      16 -4.947133E-19  ... -2.361151E-10 -1.325000E+00
   1    20  EP  17 -5.792940E-19  ... -3.075527E-10 -1.825000E+00
\end{verbatim}
The orbit flip continues to be defined by a planar homoclinic orbit
at \parf{ PAR(7)=PAR(8)=0}.

\newpage
\section{ Detailed \AUTO-Commands.}

\begin{table}[htbp]
\begin{center}
\begin{tabular}{| l | l |}
\hline
  \AUTO-COMMAND  & ACTION \\
\hline
%==============================================================================
  \commandf{ ! mkdir san} & create an empty work directory \\ 
  \commandf{ cd san} & change directory \\
  \commandf{ demo('san')} & copy the demo files to the work directory \\
\hline
%==============================================================================
  \commandf{ run(c='san.1',h='san.1') } &  continuation in \parf{ PAR(1)} \\ 
  \commandf{ sv('1') } & save output-files as \filef{ b.1, s.1, d.1} \\ 
\hline
%==============================================================================
  \commandf{ run(c='san.2',h='san.2',s='1') } & generate adjoint variables; restart from \filef{ s.1} \\ 
  \commandf{ sv('2') } & save output-files as \filef{ b.2, s.2, d.2} \\ 
\hline
%=============================================================================
  \commandf{ run(c='san.3',h='san.3',s='2') } & continue homoclinic orbit and adjoint; restart from \filef{ s.2} \\ 
  \commandf{ sv('3') } & save output-files as \filef{ b.3, s.3, d.3} \\ 
\hline
%==============================================================================
  \commandf{ run(c='san.4',h='san.4',s='1') } & no convergence without dummy step; restart from \filef{ s.1} \\ 
  \commandf{ sv('4') } &  save output-files as \filef{ b.4, s.4, d.4} \\ 
\hline
%=============================================================================
  \commandf{ run(c='san.5',h='san.5',s='3') } & continue non-orientable orbit; restart from \filef{ s.3} \\
  \commandf{ sv('5') } & save output-files as \filef{ b.5, s.5, d.5} \\ 
\hline
%==============================================================================
\end{tabular}
\caption{Detailed \AUTO-Commands for running demo \filef{ san}.}
\label{tbl:demo_san_1}
\end{center}
\end{table}


\begin{table}[htbp]
\begin{center}
\begin{tabular}{| l | l |}
\hline
  \AUTO-COMMAND  & ACTION \\
\hline
%==============================================================================
  \commandf{ run(c='san.6',h='san.6',s='san') } & restart and homotopy to \parf{ PAR(4)}=1.0 \\ 
  \commandf{ sv('6') } & save output-files as \filef{ b.6, s.6, d.6} \\ 
\hline
%==============================================================================
  \commandf{ run(c='san.7',h='san.7',s='6') } & homotopy to \parf{ PAR(5)}=0.0; restart from \filef{ s.6} \\ 
  \commandf{ ap('6') } & append output-files to \filef{ b.6, s.6, d.6} \\ 
\hline
%==============================================================================
  \commandf{ run(c='san.8',h='san.8',s='6') } & homotopy to \parf{ PAR(1)}=0.5; restart from \filef{ s.6} \\ 
  \commandf{ ap('6') } & append output-files to \filef{ b.6, s.6, d.6} \\ 
\hline
  \commandf{ run(c='san.9',h='san.9',s='6') } & homotopy to \parf{ PAR(2)}=3.0; restart from \filef{ s.6} \\ 
  \commandf{ ap('6') } & append output-files to \filef{ b.6, s.6, d.6} \\ 
\hline
%==============================================================================
  \commandf{ run(c='san.10',h='san.10',s='6') } & homotopy to \parf{ PAR(7)}=0.25; restart from \filef{ s.6} \\ 
  \commandf{ ap('6') } & append output-files to \filef{ b.6, s.6, d.6} \\ 
\hline
%==============================================================================
  \commandf{ run(c='san.11',h='san.11',s='6') } & continue in \parf{ PAR(7)} to detect orbit flip; restart from \filef{ s.6} \\ 
  \commandf{ sv('11') } & save output-files as \filef{ b.11, s.11, d.11} \\ 
\hline
%==============================================================================
  \commandf{ run(c='san.12',h='san.12',s='11') } & three-parameter continuation of orbit flip; restart from \filef{ s.11} \\ 
  \commandf{ sv('12') } & save output-files as \filef{ b.12, s.12, d.12} \\ 
\hline
%==============================================================================
\end{tabular}
\caption{Detailed \AUTO-Commands for running demo \filef{san}.}
\label{tbl:demo_san_2}
\end{center}
\end{table}

