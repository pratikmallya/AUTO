
%==============================================================================
%==============================================================================
\chapter{ \AUTO Demos : Periodic solutions.} \label{ch:Demos_Periodic}
%==============================================================================
%==============================================================================

%==============================================================================
%DEMO=lrz======================================================================
%==============================================================================
\newpage
\section{ lrz : The Lorenz Equations.} \label{sec:Demos_lrz}
This demo computes two symmetric homoclinic orbits in the Lorenz equations
\begin{equation} \begin{array}{cl}
  u_1' &=  p_3 (u_2 - u_1), \\
  u_2' &=  p_1 u_1 - u_2 - u_1 u_3,  \\
  u_3' &=  u_1 u_2 - p_2 u_3. \\ \end{array} \end{equation}
Here $p_1$ is the free parameter, and $p_2=8/3$, $p_3=10$.
The two homoclinic orbits correspond to the final, large period orbits 
on the two periodic solution branches.

\begin{table}[htbp]
\begin{center}
\begin{tabular}{| l | l |}
\hline
  \AUTO-COMMAND  & ACTION \\
\hline
%==============================================================================

  \commandf{ ! mkdir lrz} & create an empty work directory \\ 
  \commandf{ cd lrz} & change directory \\
  \commandf{ demo('lrz')} & copy the demo files to the work directory \\
\hline
%==============================================================================
  \commandf{ ld('lrz')} & load the problem definition \\ 
  \commandf{ run(c='lrz.1')} & compute stationary solutions \\ 
  \commandf{ sv('lrz')} & save output-files as \filef{ b.lrz, s.lrz, d.lrz} \\ 
\hline
%==============================================================================
  \commandf{ run(c='lrz.2',s='lrz')} & \parbox[t]{3in}{ compute periodic solutions; the final orbit is near-homoclinic.  Constants changed : {\tt IPS, IRS, NICP, ICP, NMX, NPR, DS} \vspace{0.2cm}} \\ 
  \commandf{ ap('lrz')} & append the output-files to \filef{ b.lrz, s.lrz, d.lrz} \\ 
\hline
%==============================================================================
  \commandf{ run(c='lrz.3',s='lrz')} & \parbox[t]{3in}{ compute the symmetric periodic solution branch.  Constants changed : {\tt IRS} \vspace{0.2cm}} \\ 
  \commandf{ ap('lrz')} & append the output-files to \filef{ b.lrz, s.lrz, d.lrz} \\ 
\hline
\end{tabular}
\caption{Commands for running demo \filef{ lrz}.}
\label{tbl:demo_lrz}
\end{center}
\end{table}

\newpage
%==============================================================================
%DEMO=abc======================================================================
%==============================================================================
\section{ abc : The A $\to$ B $\to$ C Reaction.} \label{sec:Demos_abc}
This demo illustrates the computation of 
stationary solutions,
Hopf bifurcations 
and
periodic solutions
in the A $\to$ B $\to$ C reaction 
(\citename{DoHe:83} \citeyear{DoHe:83}).
\begin{equation} \begin{array}{cl}
  u_1 ' &=  -u_1 + p_1 (1-u_1) e^{u_3}, \\
  u_2 ' &=  -u_2 +  p_1 e^{u_3} ( 1-u_1 - p_5 u_2 ),\\
  u_3 ' &=  -u_3 - p_3 u_3 + p_1 p_4 e^{u_3}  
  ( 1-u_1 + p_2 p_5 u_2 ),\\ \end{array} \end{equation}
with $p_2=1$, $p_3=1.55$, $p_4=8$, and $p_5=0.04$. 
The free parameter is $p_1$.


\begin{table}[htbp]
\begin{center}
\begin{tabular}{| l | l |}
\hline
  \AUTO-COMMAND  & ACTION \\ 
\hline
%==============================================================================
  \commandf{ ! mkdir abc} & create an empty work directory \\ 
  \commandf{ cd abc} & change directory \\
  \commandf{ demo('abc')} & copy the demo files to the work directory \\
\hline
%==============================================================================
  \commandf{ ld('abc')} & load the problem definition \\ 
  \commandf{ run(c='abc.1')} & compute the stationary solution branch with Hopf bifurcations \\ 
  \commandf{ sv('abc')} & save output-files as \filef{ b.abc, s.abc, d.abc} \\ 
\hline
%==============================================================================
  \commandf{ run(c='abc.2',s='abc')} & \parbox[t]{3in}{ compute a branch of periodic solutions from the first Hopf point.  Constants changed : {\tt IRS, IPS, NICP, ICP} \vspace{0.2cm}} \\ 
  \commandf{ ap('abc')} & append the output-files to \filef{ b.abc, s.abc, d.abc} \\ 
\hline
%==============================================================================
  \commandf{ run(c='abc.3',s='abc')} & \parbox[t]{3in}{  compute a branch of periodic solutions from the second Hopf point.  Constants changed : {\tt IRS, NMX} \vspace{0.2cm}} \\ 
  \commandf{ ap('abc')} & append the output-files to \filef{ b.abc, s.abc, d.abc} \\ 
\hline
\end{tabular}
\caption{Commands for running demo \filef{ abc}.}
\label{tbl:demo_abc}
\end{center}
\end{table}

\newpage
%==============================================================================
%DEMO=pp2======================================================================
%==============================================================================
\section{ pp2 : A 2D Predator-Prey Model.} \label{sec:Demos_pp2}
This demo illustrates a variety of calculations.
The equations, which model a predator-prey system with harvesting, are
\begin{equation} \begin{array}{cl}
  u_1 ' &= p_2 u_1 (1 - u_1 ) - u_1 u_2 - p_1 (1-e^{-p_3 u_1}) ,\\
  u_2 ' &= -u_2  + p_4 u_1 u_2  .\end{array} \end{equation}
Here $p_1$ is the principal continuation parameter,
$p_3=5$, $p_4=3$, and, initially,  $p_2=3$.
For two-parameter computations $p_2$ is also free.
%The use of {\cal PLAUT} is also illustrated. The saved plots are shown
%in Figure~\ref{fig:pp2_1} and  Figure~\ref{fig:pp2_2}.
\begin{table}[htbp]
\begin{center}
\begin{tabular}{| l | l |}
\hline
  \AUTO-COMMAND  & ACTION \\
\hline
%==============================================================================
  \commandf{ ! mkdir pp2} & create an empty work directory \\ 
  \commandf{ cd pp2} & change directory \\ 
  \commandf{ demo('pp2')} & copy the demo files to the work directory \\ 
\hline
%============================================================================== 
  \commandf{ ld('pp2')} & load the problem definition \\ 
  \commandf{ run(c='pp2.1')} & 1st run; stationary solutions \\ 
  \commandf{ sv('pp2')} & save output-files as \filef{ b.pp2, s.pp2, d.pp2} \\ 
\hline
%============================================================================== 
  \commandf{ run(c='pp2.2',s='pp2')} & \parbox[t]{3in}{ 2nd run; restart at a labeled solution.   Constants changed : {\tt IRS, RL1} \vspace{0.2cm}} \\ 
  \commandf{ ap('pp2')} & append output-files to \filef{ b.pp2, s.pp2, d.pp2} \\ 
\hline
%==============================================================================
  \commandf{ run(c='pp2.3',s='pp2')} & \parbox[t]{3in}{ 3rd run; periodic solutions.  Constants changed : {\tt IRS, IPS, ILP} \vspace{0.2cm}} \\ 
  \commandf{ ap('pp2')} & append output-files to \filef{ b.pp2, s.pp2, d.pp2} \\ 
\hline
%==============================================================================
  \commandf{ run(c='pp2.4',s='pp2')} & \parbox[t]{3in}{ 4th run; restart at a labeled periodic solution.  Constants changed : {\tt IRS, NTST } \vspace{0.2cm}} \\ 
  \commandf{ ap('pp2')} & append output-files to \filef{ b.pp2, s.pp2, d.pp2} \\ 
\hline
%==============================================================================
  \commandf{ run(c='pp2.5',s='pp2')} & \parbox[t]{3in}{ 5th run; continuation of folds.  Constants changed : {\tt IRS, IPS, ISW, ICP } \vspace{0.2cm}} \\ 
  \commandf{ sv('lp')} & save output-files as \filef{ b.lp, s.lp, d.lp} \\ 
\hline
%==============================================================================
  \commandf{ run(c='pp2.6',s='pp2')} & \parbox[t]{3in}{ 6th run; continuation of Hopf bifurcations.  Constants changed : {\tt IRS} \vspace{0.2cm}} \\ 
  \commandf{ sv('hb')} & save output-files as \filef{ b.hb, s.hb, d.hb} \\ 
\hline
%============================================================================== 
  \commandf{ run(c='pp2.7',s='pp2')} & \parbox[t]{3in}{ 7th run; continuation of homoclinic orbits.  Constants changed : {\tt IRS, IPS, ISP} \vspace{0.2cm}} \\ 
  \commandf{ sv('hom')} & save output-files as \filef{ b.hom, s.hom, d.hom} \\
\hline
%==============================================================================
\end{tabular}
\caption{Commands for running demo \filef{ pp2}.}
\label{tbl:demo_pp2_1}
\end{center}
\end{table}


% \begin{table}[htbp]
% \begin{center}
% \begin{tabular}{| l | l |}
% \hline
%   \AUTO-COMMAND  & ACTION \\
% \hline
%   \commandf{ @p pp2} & run {\cal PLAUT} to graph the contents of \filef{ b.pp2} and \filef{ s.pp2}; \\ 
% \hline
%   {\cal PLAUT}-COMMAND  & ACTION \\
% \hline
%   \commandf{ d1}  & set convenient defaults\\ 
%   \commandf{ bd0}  & plot the default bifurcation diagram; $L_2$-norm versus $p_1$ \\ 
%   \commandf{ cl}  & clear the screen  \\
% \hline
%   \commandf{ ax}  & select axes \\ 
%   \commandf{ 1 3}  & select real columns 1 and 3 in \filef{ b.pp2} \\ 
%   \commandf{ bd0}  & plot the bifurcation diagram; $max~u_1$ versus $p_1$ \\
%   \commandf{ cl}  & clear the screen  \\
% \hline
%   \commandf{ d3}  & choose other default settings \\ 
%   \commandf{ bd0}  & bifurcation diagram \\ 
% \hline
%   \commandf{ bd}  & get blow-up of current bifurcation diagram \\ 
%   \commandf{ 0~ 1 ~-0.25~ 1} & enter diagram limits  \\
%   \commandf{ sav}  & save plot (see Figure~\ref{fig:pp2_1})\\
%   \commandf{ fig.1}  & upon prompt, enter a new file name, e.g., \filef{ fig.1} \\
%   \commandf{ cl}  & clear the screen  \\
% \hline
%   \commandf{ 2d}  & enter 2D mode, for plotting labeled solutions\\ 
%   \commandf{ 12 13 14}  & select labeled orbits 12, 13, and 14 in \filef{ s.pp2}\\ 
%   \commandf{ d}  & default orbit display; $u_1$ versus time\\
% \hline
%   \commandf{ 1 3}  & select columns 1 and 3 in \filef{ s.pp2} \\
%   \commandf{ d}  & display the orbits; $u_2$ versus time\\
% \hline
%   \commandf{ 2 3}  & select columns 2 and 3 in \filef{ s.pp2} \\
%   \commandf{ d}  & phase plane display; $u_2$ versus $u_1$\\
%   \commandf{ sav}  & save plot  (see Figure~\ref{fig:pp2_2})\\
%   \commandf{ fig.2}  & upon prompt, enter a new file name \\
%   \commandf{ ex}  & exit from 2D mode  \\
% \hline
%   \commandf{ end}  & exit from {\cal PLAUT} \\
% \hline
% %==============================================================================
% \end{tabular}
% \caption{Plotting commands for demo \filef{ pp2}.}
% \label{tbl:demo_pp2_2}
% \end{center}
% \end{table}

% \begin{figure}[p]
% \epsfysize 9.0cm
% \centerline{\epsffile{include/pp21.ps}}
% \caption{The bifurcation diagram of demo \filef{ pp2}.}
% \label{fig:pp2_1}
% \end{figure}

% \begin{figure}[p]
% \epsfysize 9.0cm
% \centerline{\epsffile{include/pp22.ps}}
% \caption{The phase plot of solutions 12, 13, and 14 in demo \filef{ pp2}.}
% \label{fig:pp2_2}
% \end{figure}


\newpage
%==============================================================================
%DEMO=lor======================================================================
%==============================================================================
\section{ lor : Starting an Orbit from Numerical Data.} \label{sec:Demos_lor}
This demo illustrates how to start the computation of a branch of
periodic solutions from numerical data obtained, for example, from an
initial value solver.
As an illustrative application we consider the Lorenz equations
\begin{equation} \begin{array}{cl}
  u_1' &=  p_3 (u_2 - u_1), \\
  u_2' &=  p_1 u_1 - u_2 - u_1 u_3,  \\
  u_3' &=  u_1 u_2 - p_2 u_3. \\\end{array} \end{equation}
Numerical simulations with a simple initial value solver show the
existence of a stable periodic orbit when $p_1=280$, $p_2=8/3$, $p_3=10$.
Numerical data representing one complete periodic oscillation are
contained in the file \filef{ lor.dat}. 
Each row in \filef{ lor.dat} contains four real numbers, namely,
the time variable $t$, $u_1$, $u_2$ and $u_3$.
The correponding parameter values are defined in the user-supplied subroutine
\filef{ stpnt}.
The \AUTO-command \commandf{us('lor')} then converts the data in \filef{ lor.dat}
to a labeled
\AUTO solution (with label~1) in a new file \filef{ s.dat}.
The mesh will be suitably adapted to the solution, using the number of
mesh intervals \parf{ NTST} and the number of collocation point per mesh
interval \parf{ NCOL} specified in the constants-file \filef{ c.lor}.
(Note that the file \filef{ s.dat} should be used for restart only.
Do not append new output-files to \filef{ s.dat}, as the command \commandf{us('lor')}
only creates \filef{ s.dat}, with no corresponding \filef{ b.dat}.)

\begin{table}[htbp]
\begin{center}
\begin{tabular}{| l | l |}
\hline
  \AUTO-COMMAND  & ACTION \\
\hline
%==============================================================================
  \commandf{ ! mkdir lor} & create an empty work directory \\ 
  \commandf{ cd lor} & change directory \\
  \commandf{ demo('lor')} & copy the demo files to the work directory \\
\hline
%==============================================================================
  \commandf{ ld('lor')} & load the problem definition \\ 
  \commandf{ us('lor')} & convert \filef{ lor.dat} to \AUTO format in \filef{ s.dat} \\ 
\hline
%==============================================================================
  \commandf{ run(c='lor.1',s='dat')} & compute a solution branch, restart from \filef{ s.dat} \\ 
  \commandf{ sv('lor')} & save output-files as \filef{ b.lor, s.lor, d.lor} \\ 
\hline
%==============================================================================
  \commandf{ run(c='lor.2',s='lor')} & \parbox[t]{3in}{ switch branches at a period-doubling detected in the first run.  Constants changed : {\tt IRS, ISW, NTST} \vspace{0.2cm}} \\ 
  \commandf{ ap('lor')} & append the output-files to \filef{ b.lor, s.lor, d.lor} \\ 
\hline
%==============================================================================
\end{tabular}
\caption{Commands for running demo \filef{ lor}.}
\label{tbl:demo_lor}
\end{center}
\end{table}

\newpage
%==============================================================================
%DEMO=frc======================================================================
%==============================================================================
\section{ frc : A Periodically Forced System.} \label{sec:Demos_frc}
This demo illustrates the computation of periodic solutions
to a periodically forced system.
In \AUTO this can be done by adding a nonlinear oscillator with
the desired periodic forcing as one of the solution components.
An example of such an oscillator is
\begin{equation} \begin{array}{cl}
 x'&=x + \beta y - x (x^{2} + y^{2}),  \\
 y'&=-\beta x + y - y (x^{2} + y^{2}), \\\end{array} \end{equation}
which has the asymptotically stable solution $x=sin (\beta t)$,
$y=cos (\beta t)$.
We couple this oscillator to the Fitzhugh-Nagumo equations~:
\begin{equation} \begin{array}{cl}
 v'&=\bigl( F(v) - w \bigr) / \eps,  \\
 w'&=v - dw - \bigl( b + r \sin(\beta t) \bigr) ,
\end{array} \end{equation}
by replacing $\sin(\beta t)$ by $x$.
Above, $F(v) = v (v-a) (1-v)$ and $a,b,\eps$ and $d$ are fixed.
The first run is a homotopy from $r=0$, where a solution is known analytically,
to $r=0.2$.
Part of the solution branch with $r=0.2$ and varying $\beta$ 
is computed in the second run.
For detailed results see 
\citename{AlDoOt:90} \citeyear{AlDoOt:90}.


\begin{table}[htbp]
\begin{center}
\begin{tabular}{| l | l |}
\hline
  \AUTO-COMMAND  & ACTION \\
\hline
%==============================================================================
  \commandf{ ! mkdir frc} & create an empty work directory \\ 
  \commandf{ cd frc} & change directory \\
  \commandf{ demo('frc')} & copy the demo files to the work directory \\
\hline
%============================================================================== 
  \commandf{ ld('frc')} & load the problem definition \\ 
  \commandf{ run(c='frc.1')} & homotopy to $r=0.2$ \\ 
  \commandf{ sv('0')} & save output-files as \filef{ b.0, s.0, d.0} \\ 
\hline
%==============================================================================
  \commandf{ run(c='frc.2',s='0')} & \parbox[t]{3in}{ compute solution branch; restart from \filef{ s.0}.  Constants changed : {\tt IRS, ICP(1), NTST, NMX, DS, DSMAX} \vspace{0.2cm}} \\ 
  \commandf{ sv('frc')} & save output-files as \filef{ b.frc, s.frc, d.frc} \\ 
\hline
%==============================================================================
\end{tabular}
\caption{Commands for running demo \filef{ frc}.}
\label{tbl:demo_frc}
\end{center}
\end{table}

\newpage
%==============================================================================
%DEMO=ppp======================================================================
%==============================================================================
\section{ ppp :  Continuation of Hopf Bifurcations.} \label{sec:Demos_ppp}
This demo illustrates the continuation of Hopf bifurcations in a 3-dimensional 
predator prey model (\citename{Do:84} \citeyear{Do:84}).
This curve contain branch points, where one locus of Hopf points
bifurcates from another locus of Hopf points.
The equations are
\begin{equation} \begin{array}{cl}
  u_1 ' &= u_1(1-u_1) - p_4 u_1 u_2  ,  \\
  u_2 ' &= -p_2 u_2 + p_4 u_1 u_2 - p_5 u_2 u_3
  -p_1(1-e^{-p_6 u_2}) \\
  u_3 ' &= -p_3 u_3  + p_5 u_2 u_3  .  \\  
\end{array} \end{equation}
Here $p_2=1/4$,  $p_3=1/2$,  $p_4=3$,  $p_5=3$,  $p_6=5$,
and $p_1$ is the free parameter.
In the continuation of Hopf points the parameter $p_4$
is also free.

\begin{table}[htbp]
\begin{center}
\begin{tabular}{| l | l |}
\hline
  \AUTO-COMMAND  & ACTION \\
\hline
%==============================================================================
  \commandf{ ! mkdir ppp} & create an empty work directory \\ 
  \commandf{ cd ppp} & change directory \\
  \commandf{ demo('ppp')} & copy the demo files to the work directory \\
\hline
%==============================================================================
  \commandf{ ld('ppp')} & load the problem definition \\ 
  \commandf{ run(c='ppp.1')} & compute stationary solutions; detect Hopf bifurcations \\ 
  \commandf{ sv('ppp')} & save output-files as \filef{ b.ppp, s.ppp, d.ppp} \\ 
\hline
%==============================================================================
  \commandf{ run(c='ppp.2',s='ppp')} & \parbox[t]{3in}{ compute a branch of periodic solutions.  Constants changed : {\tt IPS, IRS, ICP} \vspace{0.2cm}} \\ 
  \commandf{ ap('ppp')} & append the output-files to \filef{ b.ppp, s.ppp, d.ppp} \\ 
\hline
%==============================================================================

  \commandf{ run(c='ppp.3',s='ppp')} & compute Hopf bifurcation curves \\ 
  \commandf{ sv('hb')} & save the output-files as \filef{ b.hb, s.hb, d.hb} \\ 
\hline
%==============================================================================
\end{tabular}
\caption{Commands for running demo \filef{ ppp}.}
\label{tbl:demo_ppp_1}
\end{center}
\end{table}


\newpage
%==============================================================================
%DEMO=plp======================================================================
%==============================================================================
\section{ plp : Fold Continuation for Periodic Solutions.} \label{sec:Demos_plp}
This demo, which corresponds to computations in 
\citename{DoKeKe:91a} \citeyear{DoKeKe:91a}, shows how one can
continue a fold on a branch of periodic solution in two parameters.
The calculation of a locus of Hopf bifurcations is also included.
The equations, that model a one-compartment activator-inhibitor system 
(\citename{JPK:80} \citeyear{JPK:80}),
are given by
\begin{equation} \begin{array}{cl}
 s' &= (s_{0} - s) - \rho R (s,a), \\
 a' &=\alpha (a_{0} - a) - \rho R (s,a), \\
\end{array} \end{equation}
where
$$ R(s,a)={s a \over 1+s+ \kappa s^{2} },
\qquad \kappa  > 0. $$
The free parameter is $\rho$.
In the fold continuation $s_0$ is also free.

\begin{table}[htbp]
\begin{center}
\begin{tabular}{| l | l |}
\hline
  \AUTO-COMMAND  & ACTION \\
\hline
%==============================================================================
  \commandf{ ! mkdir plp} & create an empty work directory \\ 
  \commandf{ cd plp} & change directory \\
  \commandf{ demo('plp')} & copy the demo files to the work directory \\
\hline
%==============================================================================
  \commandf{ ld('plp')} & load the problem definition \\ 
  \commandf{ run(c='plp.1')} & 1st run; compute a stationary solution branch and locate HBs \\ 
  \commandf{ sv('plp')} & save output-files as \filef{ b.plp, s.plp, d.plp} \\ 
\hline
%==============================================================================
  \commandf{ run(c='plp.2',s='plp')} & \parbox[t]{3in}{ compute a branch of periodic solutions and locate a fold.  Constants changed : \parf{ IPS, IRS, NMX} \vspace{0.2cm}} \\ 
  \commandf{ ap('plp')} & append output-files to \filef{ b.plp, s.plp, d.plp} \\ 
\hline
%==============================================================================
  \commandf{ run(c='plp.3',s='plp')} & \parbox[t]{3in}{ Compute a locus of Hopf bifurcation points.  Constants changed : \parf{ IPS, ICP, ISW, NMX, RL1} \vspace{0.2cm}} \\ 
  \commandf{ sv('2p')} & save output-files as \filef{ b.2p, s.2p, d.2p} \\ 
\hline
%==============================================================================
  \commandf{ run(c='plp.4',s='plp')} & \parbox[t]{3in}{ generate starting data for the fold continuation.  Constants changed : \parf{ IPS, IRS, ICP, NMX} \vspace{0.2cm}} \\ 
  \commandf{ sv('tmp')} & save output-files as \filef{ b.tmp, s.tmp, d.tmp} \\ 
\hline
%==============================================================================
  \commandf{ run(c='plp.5',s='tmp')} & \parbox[t]{3in}{  fold continuation; restart data from \filef{ s.tmp}.  Constants changed : \parf{ IRS, NUZR} \vspace{0.2cm}} \\ 
  \commandf{ ap('2p')} & append output-files to \filef{ b.2p, s.2p, d.2p} \\ 
\hline
%==============================================================================
  \commandf{ run(c='plp.6',s='2p')} & \parbox[t]{3in}{ compute an isola of periodic solutions; restart data from \filef{ s.2p}.  Constants changed : \parf{ IRS, ISW, NMX, NUZR } \vspace{0.2cm}} \\ 
  \commandf{ sv('iso')} & save output-files as \filef{ b.iso, s.iso, d.iso} \\ 
\hline
%==============================================================================
\end{tabular}
\caption{Commands for running demo \filef{ plp}.}
\label{tbl:demo_plp}
\end{center}
\end{table}

\newpage
%==============================================================================
%DEMO=pp3======================================================================
%==============================================================================
\section{ pp3 : Period-Doubling Continuation.} \label{sec:Demos_pp3}
This demo illustrates the computation of stationary solutions,
Hopf bifurcations, and periodic solutions, branch switching at a period-doubling
bifurcation, and the computation of a locus of period-doubling bifurcations.
The equations model a 3D predator-prey system with harvesting 
(\citename{Do:84} \citeyear{Do:84}).
\begin{equation} \begin{array}{cl}
  u_1 ' &= u_1(1-u_1) - p_4 u_1 u_2  ,  \\
  u_2 ' &= -p_2 u_2 + p_4 u_1 u_2 - p_5 u_2 u_3
  -p_1(1-e^{-p_6 u_2}) \\
  u_3 ' &= -p_3 u_3  + p_5 u_2 u_3  .  \\\end{array} \end{equation}
The free parameter is $p_1$, except in the period-doubling 
continuation, where both $p_1$ and $p_2$ are free.

\begin{table}[htbp]
\begin{center}
\begin{tabular}{| l | l |}
\hline
  \AUTO-COMMAND  & ACTION \\
\hline
%==============================================================================
  \commandf{ ! mkdir pp3} & create an empty work directory \\ 
  \commandf{ cd pp3} & change directory \\
  \commandf{ demo('pp3')} & copy the demo files to the work directory \\
\hline
%==============================================================================
  \commandf{ ld('pp3')} & load the problem definition \\ 
  \commandf{ run(c='pp3.1')} & 1st run; stationary solutions \\ 
  \commandf{ sv('pp3')} & save output-files as \filef{ b.pp3, s.pp3, d.pp3} \\ 
\hline
%==============================================================================
  \commandf{ run(c='pp3.2',s='pp3')} & \parbox[t]{3in}{ compute a branch of periodic solutions.  Constants changed : \parf{ IRS, IPS, NMX} \vspace{0.2cm}} \\ 
  \commandf{ ap('pp3} & append output-files to \filef{ b.pp3, s.pp3, d.pp3} \\ 
\hline
%==============================================================================
  \commandf{ run(c='pp3.3',s='pp3')} & \parbox[t]{3in}{ compute the branch bifurcating at the period-doubling.   Constants changed : \parf{ IRS, ISW, NTST} \vspace{0.2cm}} \\ 
  \commandf{ ap('pp3')} & append output-files to \filef{ b.pp3, s.pp3, d.pp3} \\ 
\hline
%==============================================================================
  \commandf{ run(c='pp3.4',s='pp3')} & \parbox[t]{3in}{ generate starting data for the period-doubling continuation. Constants changed : \parf{ ISW} \vspace{0.2cm}} \\ 
  \commandf{ sv('tmp')} & save output-files as \filef{ b.tmp, s.tmp, d.tmp} \\ 
\hline
%==============================================================================
  \commandf{ run(c='pp3.5',s='tmp')} & \parbox[t]{3in}{  period-doubling continuation; restart from \filef{ s.tmp}.  Constants changed : \parf{ IRS} \vspace{0.2cm}} \\ 
  \commandf{ sv('2p')} & save output-files as \filef{ b.2p, s.2p, d.2p} \\ 
\hline
%==============================================================================
\end{tabular}
\caption{Commands for running demo \filef{ pp3}.}
\label{tbl:demo_pp3}
\end{center}
\end{table}

\newpage
%==============================================================================
%DEMO=tor======================================================================
%==============================================================================
\section{ tor : Detection of Torus Bifurcations.} \label{sec:Demos_tor}
This demo uses a model in 
\citename{FrRLuGaPo:93} \citeyear{FrRLuGaPo:93}
 to illustrate the detection of a torus bifurcation. 
It also illustrates branch switching at a secondary periodic bifurcation
with double Floquet multiplier at $z=1$.
The computational results also include folds, homoclinic orbits,
and period-doubling bifurcations.
Their continuation is not illustrated here;
see instead the demos \filef{ plp}, \filef{ pp2}, and \filef{ pp3}, respectively.  
The equations are
\begin{equation} \begin{array}{cl}
  x'(t) & = \bigr[ -(\beta+\nu)x + \beta y - a_3 x^3 + b_3 (y-x)^3 \bigr] / r,\\
  y'(t) &= \beta x - (\beta + \gamma) y - z - b_3 (y-x)^3, \\
  z'(t) &= y,\end{array} \end{equation}
where $\gamma=-0.6$, $r=0.6$, $a_3=0.328578$, and $b_3=0.933578$.
Initially $\nu=-0.9$ and $\beta=0.5$.

\begin{table}[htbp]
\begin{center}
\begin{tabular}{| l | l |}
\hline
  \AUTO-COMMAND  & ACTION \\
\hline
%==============================================================================
  \commandf{ ! mkdir tor} & create an empty work directory \\ 
  \commandf{ cd tor} & change directory \\
  \commandf{ demo('tor')} & copy the demo files to the work directory \\
\hline
%==============================================================================
  \commandf{ ld('tor')} & load the problem definition \\ 
  \commandf{ run(c='tor.1')} & 1st run; compute a stationary solution branch with Hopf bifurcation \\ 
  \commandf{ sv('1')} & save output-files as \filef{ b.1, s.1, d.1} \\ 
\hline
%==============================================================================
  \commandf{ run(c='tor.2',s='1')} & \parbox[t]{3in}{ compute a branch of periodic solutions; restart from \filef{ s.1}.   Constants changed : \parf{ IPS, IRS } \vspace{0.2cm}} \\ 
  \commandf{ ap('1')} & append output-files to \filef{ b.1, s.1, d.1} \\ 
\hline
%==============================================================================
  \commandf{ run(c='tor.3',s='1')} & \parbox[t]{3in}{ compute a bifurcating branch of periodic solutions; restart from \filef{ s.1}.  Constants changed : \parf{ IRS, ISW, NMX} \vspace{0.2cm}} \\ 
  \commandf{ ap('1')} & append output-files to \filef{ b.1, s.1, d.1} \\ 
\hline
%==============================================================================
\end{tabular}
\caption{Commands for running demo \filef{ tor}.}
\label{tbl:demo_tor}
\end{center}
\end{table}

\newpage
%==============================================================================
%DEMO=pen======================================================================
%==============================================================================
\section{ pen : Rotations of Coupled Pendula.} \label{sec:Demos_pen}
This demo illustrates the computation of rotations, i.e., solutions that
are periodic, modulo a phase gain of an even multiple of $\pi$.
\AUTO checks the starting data for components with such a phase gain
and, if present, it will automatically adjust the computations accordingly.
The model equations, a system of two coupled  pendula, 
(\citename{DoArOt:91} \citeyear{DoArOt:91}),
are given by
\begin{equation} \begin{array}{cl}
 & \phi_1'' + \eps \phi_1' + \sin \phi_1 
  = I + \gamma(\phi_2-\phi_1), \\
 & \phi_2'' + \eps \phi_2' + \sin \phi_2 
  = I + \gamma(\phi_1-\phi_2) ,\\
\end{array} \end{equation}
or, in equivalent first order form,
\begin{equation} \begin{array}{cl}
 & \phi_1'  =  \psi_1, \\
 & \phi_2'  =  \psi_2, \\
 & \psi_1'  = - \eps \psi_1 - \sin \phi_1 + I + \gamma(\phi_2-\phi_1), \\
 & \psi_2'  = - \eps \psi_2 - \sin \phi_2 + I + \gamma(\phi_1-\phi_2).\\
\end{array} \end{equation}
Throughout $\gamma=0.175$. Initially, $\eps=0.1$ and $I=0.4$.

Numerical data representing one complete rotation are
contained in the file \filef{ pen.dat}. 
Each row in \filef{ pen.dat} contains five real numbers, namely,
the time variable $t$, $\phi_1$, $\phi_2$, $\psi_1$ and $\psi_2$.
The correponding parameter values are defined in the user-supplied subroutine
\funcf{ stpnt}.

Actually, in this example, a scaled time variable $t$ is given in \filef{ pen.dat}. 
For this reason the period (\parf{ PAR(11)}) is also set in \funcf{ stpnt}.
Normally \AUTO would automatically set the period according to
the data in \filef{ pen.dat}.

The \AUTO-command \commandf{us('pen')} converts the data in \filef{ pen.dat}
to a labeled
\AUTO solution (with label~1) in a new file \filef{ s.dat}.
The mesh will be suitably adapted to the solution, using the number of
mesh intervals \parf{ NTST} and the number of collocation point per mesh
interval \parf{ NCOL} specified in the constants-file \filef{ c.pen}.
(Note that the file \filef{ s.dat} should be used for restart only.
Do not append new output-files to \filef{ s.dat}, as the command \commandf{us('pen')}
only creates \filef{ s.dat}, with no corresponding \filef{ b.dat}.)

The first run, with $I$ as free problem parameter,
starts from the converted solution with label~1 in \filef{ pen.dat}.
A period-doubling bifurcation is located, and the period-doubled branch
is computed in the second run.
Two branch points are located, and the bifurcating
branches are traced out in the third and fourth run, respectively.
The fifth run generates starting data for the subsequent computation of
a locus of period-doubling bifurcations.
The actual computation is done in the sixth run, with $\eps$ and $I$
as free problem parameters.

\begin{table}[htbp]
\begin{center}
\begin{tabular}{| l | l |}
\hline
  \AUTO-COMMAND  & ACTION \\
\hline
%==============================================================================
  \commandf{ ! mkdir pen} & create an empty work directory \\ 
  \commandf{ cd pen} & change directory \\
  \commandf{ demo('pen')} & copy the demo files to the work directory \\
\hline
%==============================================================================
  \commandf{ ld('pen')} & load the problem definition \\ 
  \commandf{ us('pen')} & convert \filef{ pen.dat} to \AUTO format in \filef{ s.dat} \\ 
\hline
%==============================================================================
  \commandf{ run(c='pen.1',s='dat')} & locate a period doubling bifurcation; restart from \filef{ s.dat} \\ 
  \commandf{ sv('pen')} & save output-files as \filef{ b.pen, s.pen, d.pen} \\ 
\hline
%==============================================================================
  \commandf{ run(c='pen.2',s='pen')} & \parbox[t]{3in}{ a branch of  period-doubled (and out-of-phase) rotations.   Constants changed : \parf{ IPS, NTST, ISW, NMX} \vspace{0.2cm}} \\ 
  \commandf{ ap('pen')} & append output-files tp \filef{ b.pen, s.pen, d.pen} \\ 
\hline
%============================================================================== 
  \commandf{ run(c='pen.3',s='pen')} & \parbox[t]{3in}{  a secondary bifurcating branch (without bifurcation detection).  Constants changed : \parf{ IRS, ISP} \vspace{0.2cm}} \\ 
  \commandf{ ap('pen')} & append output-files to \filef{ b.pen, s.pen, d.pen} \\ 
\hline
%==============================================================================
  \commandf{ run(c='pen.4',s='pen')} & \parbox[t]{3in}{  another secondary bifurcating branch (without bifurcation detection).  Constants changed : \parf{ IRS} \vspace{0.2cm}} \\ 
  \commandf{ ap('pen')} & append output-files to \filef{ b.pen, s.pen, d.pen} \\ 
\hline
%==============================================================================
  \commandf{ run(c='pen.5',s='pen')} & \parbox[t]{3in}{  generate starting data for period doubling continuation.  Constants changed : \parf{ IRS, ICP, ICP, ISW, NMX} \vspace{0.2cm}} \\ 
  \commandf{ sv('t')} & save output-files as \filef{ b.t, s.t, d.t} \\ 
\hline
%==============================================================================
  \commandf{ run(c='pen.6',s='t')} & \parbox[t]{3in}{  compute a locus of period doubling bifurcations; restart from \filef{ s.t}.  Constants changed : \parf{ IRS} \vspace{0.2cm}} \\ 
  \commandf{ sv('pd')} & save output-files as \filef{ b.pd, s.pd, d.pd} \\ 
%\hline
%==============================================================================
%%  \commandf{ @pn pen} & run an animation program to view the solutions in \filef{ s.pen} \\ 
%%  & (on SGI machines only; see also the file \filef{ auto/2000/pendula/README}).
%  \\ 
\hline
%==============================================================================
\end{tabular}
\caption{Commands for running demo \filef{ pen}.}
\label{tbl:demo_pen}
\end{center}
\end{table}

\newpage
%==============================================================================
%DEMO=chu======================================================================
%==============================================================================
\section{ chu :  A Non-Smooth System (Chua's Circuit).} \label{sec:Demos_chu}
Chua's circuit 
is one of the simplest electronic devices to exhibit complex behavior. 
For related calculations see
\citename{KhRoCh:93} \citeyear{KhRoCh:93}.
The equations modeling the circuit are
\begin{equation} \begin{array}{cl}
 u_1' &=  \alpha \bigl[~ u_2 - h(u_1) ~\bigr]~,\\ 
 u_2' &=  u_1 - u_2 + u_3~, \\  
 u_3' &=  - \beta~ u_2~,  
\end{array} \end{equation}
where
$$ h(x) = a_1 x + {1 \over 2}~ (a_0 - a_1) ~
  \bigl\{ \abs{x+1} -  \abs{x-1} \bigr\}~,$$
and where we take
$\beta = 14.3$, $a_0 = - 1/7$, $a_1 = 2/7$.

Note that $h(x)$ is not a smooth function, and hence the solution 
to the equations  may have non-smooth derivatives.
However, for the orthogonal collocation method to attain its optimal accuracy,
it is necessary that the solution be sufficiently smooth.
Moreover, the adaptive mesh selection strategy will fail
if the solution or one of its lower order derivatives has discontinuities.
For these reasons  we use the smooth approximation
$$ \abs{x} ~\approx~ {2 x \over \pi } ~ {\rm arctan}(Kx),$$
which get better as $K$ increases.
In the numerical calculations below we use $K = 10$.
The free parameter is $\alpha$.


\begin{table}[htbp]
\begin{center}
\begin{tabular}{| l | l |}
\hline
  \AUTO-COMMAND  & ACTION \\
\hline
%==============================================================================
  \commandf{ ! mkdir chu} & create an empty work directory \\ 
  \commandf{ cd chu} & change directory \\
  \commandf{ demo('chu')} & copy the demo files to the work directory \\
\hline
%==============================================================================
  \commandf{ ld('chu')} & load the problem definition \\ 
  \commandf{ run(c='chu.1')} & 1st run; stationary solutions \\ 
  \commandf{ sv('chu')} & save output-files as \filef{ b.chu, s.chu, d.chu} \\ 
\hline
%==============================================================================
  \commandf{ run(c='chu.2',s='chu')} & \parbox[t]{3in}{ 2nd run; periodic solutions, with detection of period-doubling.  constants changed : \parf{ IPS, IRS, ICP, ICP} \vspace{0.2cm}} \\ 
  \commandf{ ap('chu')} & append the output-files to \filef{ b.chu, s.chu, d.chu} \\ 
\hline
%==============================================================================
\end{tabular}
\caption{Commands for running demo \filef{ chu}.}
\label{tbl:demo_chu}
\end{center}
\end{table}

\newpage
%==============================================================================
%DEMO=phs======================================================================
%==============================================================================
\section{ phs : Effect of the Phase Condition.} \label{sec:Demos_phs}
This demo illustrates the effect of the phase condition 
on the computation of periodic solutions.
We consider the differential equation
\begin{equation} \begin{array}{cl}
 u_1'&= \lambda u_1 - u_2,  \\
 u_2'&= u_1 (1-u_1) .  \\
\end{array} \end{equation}
This equation has a Hopf bifurcation from the trivial solution at $\lambda=0$. 
The bifurcating branch of periodic solutions
is vertical and along it the period increases monotonically.
The branch terminates in a homoclinic orbit containing the
saddle point $(u_1,u_2)=(1,0)$.
Graphical inspection of the computed periodic orbits,
for example $u_1$ versus the scaled time variable $t$,
shows how the phase condition has the effect of keeping the ``peak'' 
in the solution in the same location.

\begin{table}[htbp]
\begin{center}
\begin{tabular}{| l | l |}
\hline
  \AUTO-COMMAND  & ACTION \\
\hline
%==============================================================================
  \commandf{ ! mkdir phs} & create an empty work directory \\ 
  \commandf{ cd phs} & change directory \\
  \commandf{ demo('phs')} & copy the demo files to the work directory \\
\hline
%==============================================================================
  \commandf{ ld('phs')} & load the problem definition \\ 
  \commandf{ run(c='phs.1')} & detect Hopf bifurcation \\ 
  \commandf{ sv('phs')} & save output-files as \filef{ b.phs, s.phs, d.phs} \\ 
\hline
%==============================================================================
  \commandf{ run(c='phs.2',s='phs')} & \parbox[t]{3in}{ compute periodic solutions. Constants changed : \parf{ IRS, IPS, NPR} \vspace{0.2cm}} \\ 
  \commandf{ ap('phs')} & append output-files to \filef{ b.phs, s.phs, d.phs} \\ 
\hline
%==============================================================================
\end{tabular}
\caption{Commands for running demo \filef{ phs}.}
\label{tbl:demo_phs}
\end{center}
\end{table}

\newpage
%==============================================================================
%DEMO=ivp======================================================================
%==============================================================================
\section{ ivp :  Time Integration with Euler's Method.} \label{sec:Demos_ivp}
This demo uses Euler's method to locate a stationary solution of the
following predator-prey system with harvesting~:

\begin{equation} \begin{array}{cl}
  u_1 ' &= p_2 u_1 (1 - u_1 ) - u_1 u_2 - p_1 (1-e^{-p_3 u_1}) ,\\
  u_2 ' &= -u_2  + p_4 u_1 u_2  ,\\\end{array} \end{equation}
where all problem parameters have a fixed value.
The equations are the same as those in demo \filef{ pp2}.
The continuation parameter is the independent time variable, namely \parf{ PAR(14)}.

Note that Euler time integration is only first order accurate, so that
the time step must be sufficiently small to ensure correct results.
Indeed, this option has been added only as a convenience, and should 
generally be used only to locate stationary states.
Note that the \AUTO-constants \parf{ DS}, \parf{ DSMIN}, and \parf{ DSMAX}
control the step size
in the space consisting of time, here \parf{ PAR(14)}, and the state vector,
here $(u_1,u_2)$.

\begin{table}[htbp]
\begin{center}
\begin{tabular}{| l | l |}
\hline
  \AUTO-COMMAND  & ACTION \\
\hline
%==============================================================================
  \commandf{ ! mkdir ivp} & create an empty work directory \\ 
  \commandf{ cd ivp} & change directory \\
  \commandf{ demo('ivp')} & copy the demo files to the work directory \\
\hline
%==============================================================================
  \commandf{ ld('ivp')} & load the problem definition \\ 
  \commandf{ run(c='ivp.1')} & time integration \\ 
  \commandf{ sv('ivp')} & save output-files as \filef{ b.ivp, s.ivp, d.ivp} \\ 
\hline
%==============================================================================
\end{tabular}
\caption{Commands for running demo \filef{ ivp}.}
\label{tbl:demo_ivp}
\end{center}
\end{table}

