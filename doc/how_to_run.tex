
%==============================================================================
%==============================================================================
\chapter{ How to Run \AUTO.}  \label{ch:How_to_run_AUTO}
%==============================================================================
%==============================================================================
\section{ User-Supplied Files.} \label{sec: User_supplied_files}
The user must prepare the two files described below.
This can be done with the GUI described in Chapter~\ref{ch:CLUI}, 
or independently.

\subsection{ The equations-file \filef{ xxx.c}} 
A source file \filef{ xxx.c} containing the C subroutines
\funcf{ func}, \funcf{ stpnt}, \funcf{ bcnd}, \funcf{ icnd}, \funcf{ fopt}, and \funcf{ pvls}.
Here \filef{ xxx} stands for a user-selected name. 
If any of these subroutines is irrelevant 
to the problem then its body need not be completed.
Examples are in \filef{ auto/2000/demos}, where, e.g.,
the file \filef{ ab/ab.c} defines a two-dimensional dynamical system,
and the file \filef{ exp/exp.c} defines a boundary value problem.
The simplest way to create a new equations-file is to copy 
an appropriate demo file.
%For a fully documented equations-file see \filef{ auto/2000/gui/aut.f}.
In GUI mode, this file can be directly loaded with the GUI-button 
\commandf{ Equations/New}; see Section~\ref{sec:GUI_Menu_bar}.

Alternatively you can use a Fortran equations-file \filef{ xxx.f }
(the same one used in \AUTOold).
However, arrays in Fortran are 1-based as opposed to
of the 0 based C-arrays. In particular, this affects the parameter
array, where the period-parameter \parf{ PAR(11)} in Fortran
corresponds to \parf{ par[10]} in C, and you need to add 1 to all
fixed parameter indices mentioned in Chapters~\ref{ch:AUTO_constants}
and ~\ref{ch:HomCont}

\subsection{ The constants-file \filef{ c.xxx}} 
\AUTO-constants for \filef{ xxx.c} are normally expected 
in a corresponding file \filef{ c.xxx}.
Specific examples include \filef{ ab/c.ab}
and \filef{ exp/c.exp} in \filef{ auto/2000/demos}.
See Chapter~\ref{ch:AUTO_constants}
for the significance of each constant.

\newpage
\section{ User-Supplied Subroutines.} \label{sec: User_supplied_subroutines}
The purpose of each of the user-supplied subroutines in
the file \filef{ xxx.c} is described below.
  
\begin{itemize}
\item[-] \funcf{ func}~:~
  defines the function $f(u,p)$ in (\ref{1}), (\ref{2}), or (\ref{3}).
\item[-] \funcf{ stpnt}~:~
  This subroutine is called only if \parf{ IRS}=0 
(see Section~\ref{sec:IRS} for \parf{ IRS}),
  which typically is the case for the first run.
  It defines a starting solution $(u,p)$ of (\ref{1}) or (\ref{2}).
  The starting solution should not be a branch point.\\ 
  (Demos \filef{ ab}, \filef{ exp}, \filef{ frc}, \filef{ lor}.)  
\item[-] \funcf{ bcnd}~:~ 
  A subroutine \funcf{ bcnd} that defines the boundary conditions. \\
  (Demo \filef{ exp}, \filef{ kar}.)
\item[-] \funcf{ icnd}~:~ 
  A subroutine \funcf{ icnd} that defines the integral conditions. \\ 
  (Demos \filef{ int}, \filef{ lin}.)  
\item[-] \funcf{ fopt}~:~ 
  A subroutine \funcf{ fopt} that defines the objective functional. \\ 
  (Demos \filef{ opt}, \filef{ ops}.)  
\item[-] \funcf{ pvls}~:~
  A subroutine \funcf{ pvls} for defining ``solution measures''. \\
  (Demo \filef{ pvl}.)
\end{itemize}
 

\section{ Arguments of \funcf{ stpnt}.} \label{sec:Arguments_of_STPNT}
Note that the arguments of \funcf{ stpnt} depend on the solution type~:
\begin{itemize}
\item[-]
  When starting from a fixed point or an analytically or numerically known space-dependent 
  solution, \funcf{ stpnt} must have four arguments, namely, 
(\parf{NDIM,U,PAR,T}).
  Here T is the independent space variable which
  takes values in the interval $[0,1]$.  T is ignored in 
  the case of fixed points.\\ 
  (Demos \filef{exp} and \filef{ab}.)
\item[-]
  Similarly, when starting from an analytically known 
  time-periodic solution or rotation, the arguments of \funcf{ stpnt} are
  (\parf{NDIM,U,PAR,T}), where T denotes the independent time variable which
  takes values in the interval $[0,1]$.
  In this case one must also specify the period in \parf{PAR(11)}. \\  
  (Demos \filef{frc}, \filef{lor}, \filef{pen}.)
\item[-]
  When using the \commandf{@fc} command (Section~\ref{sec:command_mode})
  for conversion of numerical data,
  \funcf{ stpnt} must have four arguments, namely, (\parf{NDIM,U,PAR,T}).
  In this case only the parameter values need to be defined in \funcf{ stpnt}.
  (Demos \filef{lor} and \filef{pen}.)
\end{itemize}
 
\section{ User-Supplied Derivatives.} \label{sec:derivatives}
If \AUTO-constant \parf{ JAC} equals 0 
then derivatives need not be specified in 
\funcf{ func}, \funcf{ bcnd}, \funcf{ icnd}, and \funcf{ fopt}; see Section~\ref{sec:JAC}.
If \parf{ JAC=1} then derivatives must be given.
This may be necessary for sensitive 
problems, and is recommended for computations in which \AUTO 
generates an extended system.
Examples of user-supplied derivatives can be found in
demos  \filef{ dd2}, \filef{ int}, \filef{ plp}, \filef{ opt}, and \filef{ ops}.

\section{ Output Files.} \label{sec:Output_files}
\AUTO writes four output-files~:
\begin{itemize}
\item[-] \filef{ fort.6}~:~
  A summary of the computation is written in \filef{ fort.6}, which usually
  corresponds to the window in which \AUTO is run. 
  Only special, labeled solution points are noted, namely those listed
  in Table~\ref{tbl:Solution_Types}.
  The letter codes in the Table are used in the screen output.
  The numerical codes are used internally and in
  the \filef{ fort.7} and \filef{ fort.8} output-files described below.

\begin{table}[htbp]
\begin{center}
\begin{tabular}{| l | r | l |}
\hline
 BP & (1)  & Branch point (algebraic systems) \\
\hline
 LP & (2)  & Fold (algebraic systems) \\
\hline
 HB & (3)  & Hopf bifurcation \\
\hline
  & (4)  & User-specified regular output point \\
\hline
 UZ & (-4)  & Output at user-specified parameter value \\
\hline
 LP & (5)  & Fold (differential equations) \\
\hline
 BP & (6)  & Branch point (differential equations) \\
\hline
 PD & (7)  & Period doubling bifurcation \\
\hline
 TR & (8)  & Torus bifurcation \\
\hline
 EP & (9)  & End point of branch; normal termination \\
\hline
 MX & (-9)  & Abnormal termination; no convergence \\
\hline
\end{tabular}
\caption{Solution Types.}
\label{tbl:Solution_Types}
\end{center}
\end{table}
 

\item[-] \filef{ fort.7}~:~ 
  The \filef{ fort.7} output-file contains the bifurcation diagram.
  Its format is the same as the \filef{ fort.6} (screen) output, 
  but the \filef{ fort.7} output is more extensive, as every solution point has 
  an output line printed.
\item[-] \filef{ fort.8}~:~ 
  The \filef{ fort.8} output-file contains complete graphics and restart data
  for selected, labeled solutions. 
  The information per solution is generally much more extensive than
  that in \filef{ fort.7}. 
  The \filef{ fort.8} output should normally be kept to a minimum.
\item[-] \filef{ fort.9}~:~
  Diagnostic messages, convergence history, eigenvalues, and 
  Floquet multipliers are written in \filef{ fort.9}.
  It is strongly recommended that this output be habitually inspected.
  The amount of diagnostic data can be controlled via the \AUTO-constant \parf{ IID};
  see Section~\ref{sec:IID}.
\end{itemize}

The user has some control over the \filef{ fort.6} (screen) and \filef{ fort.7} output 
via the \AUTO-constant \parf{ IPLT} (Section~\ref{sec:IPLT}).
Furthermore, the subroutine \funcf{ pvls} can be used to define ``solution measures''
which can then be printed by ``parameter overspecification'';
see Section~\ref{sec:Parameter_over_specification}.
For an example see demo \filef{ pvl}.

The \AUTO-commands \commandf{ @sv}, \commandf{ @ap}, and \commandf{ @df} can be used 
to manipulate  the output-files \filef{ fort.7}, \filef{ fort.8},
and \filef{ fort.9}.
Furthermore, the \AUTO-command \commandf{ @lb} can be used to delete and
relabel solutions simultaneously in \filef{ fort.7} and \filef{ fort.8}.
For details see Section~\ref{sec:command_mode}.

The graphics program {\cal PLAUT} can be used to graphically inspect 
the data in \filef{ fort.7} and \filef{ fort.8}; see Chapter~\ref{ch:PLAUT}.
 
