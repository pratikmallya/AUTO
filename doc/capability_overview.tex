%==============================================================================
%==============================================================================
\chapter{ Overview of Capabilities.} \label{ch:Overview}
%==============================================================================
%==============================================================================
\section{ Summary.} \label{sec:Summary}
\AUTO can do a limited bifurcation analysis of algebraic systems
\begin{equation} \label{1} 
  f( u , p ) = 0 ,  \qquad  f(\cdot,\cdot) , u \in \Rn,
\end{equation}
and of systems of ordinary differential equation (ODEs) of the form
\begin{equation} \label{2} 
 u'(t) = f\bigl( u(t) , p \bigr) , 
  \qquad  f(\cdot,\cdot) , u(\cdot) \in \Rn,
\end{equation}
Here $p$ denotes one or more free parameters.

It can also do certain stationary solution and wave calculations for the 
partial differential equation (PDE)
\begin{equation} \label{3} 
  u_t = D u_{xx} + f( u , p ), 
  \qquad  f(\cdot,\cdot) , u(\cdot) \in \Rn,
\end{equation}
where $D$ denotes a diagonal matrix of diffusion constants.
The basic algorithms used in the package,
as well as related algorithms, can be found in 
\citename{HBK:77} \citeyear{HBK:77},
\citename{HBK:86} \citeyear{HBK:86},
\citename{DoKeKe:91a} \citeyear{DoKeKe:91a},
\citename{DoKeKe:91b} \citeyear{DoKeKe:91b}.

Below, the basic capabilities of \AUTO are specified in more detail.
Some representative demos are also indicated.
 
\section{ Algebraic Systems.} \label{sec:algebraic_systems}
Specifically, for (\ref{1}) the program can~:~
 
\begin{itemize}
\item[-]
  Compute solution branches.\\  (Demo \filef{ ab}; Run~1.) 
\item[-]
  Locate branch points and automatically compute
  bifurcating branches. \\ (Demo \filef{ pp2}; Run~1.)
\item[-]
  Locate Hopf bifurcation points and continue these in two
  parameters. \\ (Demo \filef{ ab}; Runs~1 and 5.)
\item[-]
  Locate folds (limit points) and continue these 
  in two parameters. \\(Demo \filef{ ab}; Runs~1,3,4.)
\item[-]
  Do each of the above for fixed points
  of the discrete dynamical system 
  $u^{(k+1)}= f( u^{(k)}, p )$ \\ (Demo \filef{ dd2}.)
\item[-]
  Find extrema of an objective function along solution branches
  and successively continue such extrema in more parameters.
  \\ (Demo \filef{ opt}.)
\end{itemize}


\section{ Ordinary Differential Equations.} \label{sec:ODEs}
For the ODE (\ref{2}) the program can~:~
 
\begin{itemize}
\item[-]
  Compute branches of stable and unstable periodic
  solutions and
  compute the Floquet multipliers, that determine stability, along
  these branches.
  Starting data for the computation of periodic orbits are
  generated automatically at Hopf bifurcation points. \\
  (Demo \filef{ ab}; Run~2.)
\item[-]
  Locate folds, branch points, period doubling bifurcations,
  and bifurcations to tori, along branches of periodic solutions. 
  Branch switching is possible at branch points and at period 
  doubling bifurcations.  \\
  (Demos \filef{ tor}, \filef{ lor}.)
\item[-]  Continue folds and period-doubling bifurcations, 
  in two parameters. \\ (Demos \filef{ plp}, \filef{ pp3}.)
  The continuation of orbits of fixed period is also
  possible. This is the simplest way to compute curves of
  homoclinic orbits, if the period is sufficiently large.
  \\ (Demo \filef{ pp2}.)
\item[-]  Do each of the above for {\it rotations}, i.e., when some of the
  solution components are periodic modulo a phase gain of a
  multiple of $2 \pi$. \\
  (Demo \filef{ pen}.)
\item[-]  Follow curves of homoclinic orbits and detect and continue
  various codimension-2 bifurcations, using the {\cal HomCont} algorithms of 
  \citename{ChKu:94} \citeyear{ChKu:94},
  \citename{ChKuSa:95} \citeyear{ChKuSa:95}.\\
  (Demos  \filef{ san}, \filef{ mnt}, \filef{ kpr}, \filef{ cir},
  \filef{ she}, \filef{ rev}.)
\item[-]  Locate extrema of an integral objective functional along a branch 
  of periodic solutions and successively continue such extrema 
  in more parameters. \\
  (Demo \filef{ ops}.)
\item[-]
  Compute curves of solutions to (\ref{2}) on $[0,1]$, subject to general
  nonlinear boundary and integral conditions.
  The boundary conditions need not be separated, i.e., they may
  involve both $u(0)$ and $u(1)$ simultaneously.
  The side conditions may also depend on parameters.
  The number of boundary conditions plus the number of integral
  conditions need not equal the dimension of the ODE, 
  provided there is a corresponding number of additional
  parameter variables. \\
  (Demos \filef{ exp}, \filef{ int}.)
\item[-]
  Determine folds and branch points along
  solution branches to the above boundary value problem.
  Branch switching is possible at branch points.
  Curves of folds can be computed in two parameters.\\
  (Demos \filef{ bvp}, \filef{ int}.)
\end{itemize}
 


\section{ Parabolic PDEs.} \label{sec:Parabolic_PDEs}
For (\ref{3}) the program can~:~
 
\begin{itemize}
\item[-]
  Trace out branches of spatially homogeneous solutions.
  This amounts to a bifurcation analysis of the algebraic
  system (\ref{1}). However, \AUTO uses a related system instead,
  in order to enable the detection of bifurcations to wave train
  solutions of given wave speed. More precisely, bifurcations
  to wave trains are detected as Hopf bifurcations along fixed
  point branches of the related ODE
  \begin{equation} \label{4} \begin{array}{cl}
  & u'(z) = v(z) ,\\
  & v'(z) =-D^{-1}  \bigl[ c~v(z) + f\bigl( u(z) , p \bigr) \bigr], \\
  \end{array} \end{equation}
  where $z = x - ct$ , with the wave speed $c$ specified by the user.\\
  (Demo \filef{ wav}; Run~2.) 
\item[-]
  Trace out branches of periodic wave solutions to (\ref{3}) that emanate
  from a Hopf bifurcation point of Equation~\ref{4}.
  The wave speed $c$ is  fixed along such a branch, but
  the wave length $L$, i.e., the period of periodic solutions 
  to (\ref{4}),
  will normally vary. If the wave length $L$ becomes large,
  i.e., if a homoclinic orbit of Equation~\ref{4} is approached,
  then the wave tends to a solitary wave solution of (\ref{3}). \\
  (Demo \filef{ wav}; Run~3.) 
\item[-]
  Trace out branches of waves of fixed wave length $L$ in two parameters. 
  The wave speed $c$ may be chosen as one of these parameters.
  If $L$ is large then such a continuation gives a branch
  of approximate solitary wave solutions to (\ref{3}).\\
  (Demo \filef{ wav}; Run~4.) 
\item[-]
  Do time evolution calculations for (\ref{3}), given periodic
  initial data on the interval $[0,L]$.
  The initial data must be specified on $[0,1]$ and
  $L$ must be set separately because of internal scaling.
  The initial data may be given analytically or
  obtained from a previous computation of wave trains, solitary
  waves, or from a previous evolution calculation.
  Conversely, if an evolution calculation results in a
  stationary wave then this wave can be used as starting data
  for a wave continuation calculation.\\
  (Demo \filef{ wav}; Run~5.)
\item[-]
  Do time evolution calculations for (\ref{3}) subject to user-specified
  boundary conditions.
  As above, the initial data must be specified on $[0,1]$ and the space
  interval length $L$ must be specified separately.
  Time evolution computations of (\ref{3}) are adaptive in space and
  in time. Discretization in time is not very accurate~: only
  implicit Euler. Indeed, time integration of (\ref{3}) has only been
  included as a convenience and it is not very efficient.
  (Demos \filef{ pd1}, \filef{ pd2}.)
\item[-]
  Compute curves of stationary solutions to (\ref{3}) subject to user-specified
  boundary conditions.
  The initial data may be given analytically, obtained from a previous 
  stationary solution computation, or from a time evolution calculation.\\
  (Demos \filef{ pd1}, \filef{ pd2}.)
\end{itemize}
 
In connection with periodic waves,
note that (\ref{4}) is just a special case of (\ref{2}) and
that its fixed point analysis is a special case of (\ref{1}).
One advantage of the built-in capacity of \AUTO to deal with
problem (\ref{3}) is that the user need only specify $f$, $D$, and $c$.
Another advantage is the compatibility of output data for
restart purposes. This allows switching back and forth between
evolution calculations and wave computations.

\section{ Discretization.} \label{sec:Discretization}
  \AUTO discretizes ODE boundary value problems
  (which includes periodic solutions) by the method of orthogonal 
  collocation using piecewise polynomials with 2-7 collocation points 
  per mesh interval (\citename{dBSw:73} \citeyear{dBSw:73}).
  The mesh automatically adapts to the solution to equidistribute
  the local discretization error (\citename{RuCr:78} \citeyear{RuCr:78}).
  The number of mesh intervals and the number of collocation points
  remain constant during any given run, although they may be changed 
  at restart points.
  The implementation is \AUTO-specific. In particular, the choice of
  local polynomial basis
  and the algorithm for solving the linearized collocation systems
  were specifically designed for use in numerical bifurcation analysis.
  
 