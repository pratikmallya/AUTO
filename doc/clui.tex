%==============================================================================
%==============================================================================
\chapter{ Command Line User Interface.} \label{ch:CLUI}
%==============================================================================
%==============================================================================
%\floatstyle{boxed}
%\restylefloat{figure}

\section{ Typographical Conventions }
This chapter uses the following conventions.
All code examples will be in in the following font.

{\small \begin{center} \begin{boxedverbatim}
AUTO> copydemo("ab")
Copying demo ab ... done 
\end{boxedverbatim} 
\end{center}
}

To distinguish commands which are typed to the Unix
shell from those which are typed to the \AUTOc
command line user interface (CLUI) we will use the
following two prompts.

\begin{tabular}{|l|l|}
\hline 
\verb!>! & Commands which follow this prompt are for the Unix shell. \\ \hline
\verb!AUTO>!   & Commands which follow this prompt are for the \AUTOc CLUI. \\ \hline
\end{tabular}

\section{ General Overview.} \label{sec:CLUI_Overview}
The \AUTOc command line user interface (CLUI) is similar
to the command language described in Section~\ref{sec:command_mode}
in that it facilitates the interactive creating and editing of 
equations-files and constants-files.
It differs from the other command language in that it is based 
on the object-oriented scripting language \python~ (see \citename{Lut:96} \citeyear{Lut:96})
and provides extensive programming capabilities.
This chapter will provide documentation for the \AUTOc CLUI commands,
but is not intended as a tutorial for the \python~ language.
We will attempt to make this chapter self contained by describing
all \python~ constructs that we use in the examples, but
for more extensive documentation on the \python~ language,
including tutorials and pointers to further documentation,
please see \citename{Lut:96} \citeyear{Lut:96} or the
web page \webf{http://www.python.org} which contains
an excellent tutorial at 
\webf{http://www.python.org/doc/current/tut/tut.html}.

To use the CLUI for a new equation, change to an empty directory.
For an existing equations-file, change to its directory.
(\emp{Do not activate the CLUI in the directory \filef{auto/2000} 
or in any of its subdirectories.})
Then type 

\centerline {\commandf{auto}.}

If your command search path has been correctly set (see
Section~\ref{sec:Installation}), this command will start the \AUTOc CLUI
interactive interpretor and provide you with the \AUTOc CLUI prompt.

\begin{figure}[htbp]
{\small 
\begin{center} \begin{boxedverbatim}
> auto
Initializing
Python 1.5.2 (1, Feb  1 2000, 16:32:16)  [GCC egcs-2.91.66 19990314/Linux 
Copyright 1991-1995 Stichting Mathematisch Centrum, Amsterdam
(AUTOInteractiveConsole)
AUTO> 
\end{boxedverbatim}
\end{center}
}
\caption[Starting the \AUTOc CLUI.]
{Typing \commandf{auto} at the Unix shell prompt starts the
\AUTOc CLUI.  }
\label{exa:clui_starting}
\end{figure}

In addition to the examples in the following sections there are
several example scripts which can be found in
\filef{auto/2000/demos/python} and are listed in
Table~\ref{tbl:demo_scripts}.  These scripts are fully annotated and
provide good examples of how \AUTOc CLUI scripts are written.  The
scripts in \filef{auto/2000/demos/python/n-body} are espcially lucid
examples and preform various related parts of a calculation involving
the gravitional N-body problem.  
Scripts which end in the
suffix \filef{.auto} are called ``basic'' scripts and can
be run by typing \commandf{auto scriptname.auto}.
The scripts show in Section~\ref{sec:clui_first_example}
and Section~\ref{sec:clui_complex_example} are examples
of basic scripts.
Scripts which end in the
suffix \filef{.xauto} are called ``expert'' scripts and can
be run by typing \commandf{autox scriptname.xauto}.
More information on expert scripts can be 
found in Section~\ref{sec:clui_extending}.
See the \filef{README} file in that
directory for more information.

\begin{table}[htbp]
\begin{center}
\begin{tabular}{| l | l |}
\hline
Script & Description \\
\hline
demo1.auto & \begin{minipage}{3in}\smallskip The demo script from Section~\ref{sec:clui_first_example}.\smallskip\end{minipage} \\
\hline
demo2.auto & \begin{minipage}{3in}\smallskip The demo script from Section~\ref{sec:clui_complex_example}.\smallskip\end{minipage} \\
\hline
userScript.xauto & \begin{minipage}{3in}\smallskip The expert demo script from Figure~\ref{exa:clui_complex_function}.\smallskip\end{minipage} \\
\hline
userScript.py & \begin{minipage}{3in}\smallskip The loadable expert demo script from Figure~\ref{exa:clui_complex_interactive}.\smallskip\end{minipage} \\
\hline
fullTest.auto & \begin{minipage}{3in}\smallskip A script which uses the entire \AUTOc command set, except for the plotting commands.\smallskip\end{minipage} \\
\hline
plotter.auto & \begin{minipage}{3in}\smallskip A demonstration of some of the plotting capabilities of \AUTOc. \smallskip\end{minipage}\\
\hline
fullTest.auto & \begin{minipage}{3in}\smallskip A script which implements the tutorial from Section~\ref{sec:Demos_ab}. \smallskip\end{minipage}\\

\hline n-body/compute\_lagrange\_points\_family.auto 
& \begin{minipage}{3in}\smallskip A basic script which computes and plots all of the
``Lagrange points'' as a function of the ratio of the masses of
the two planets.\smallskip\end{minipage}\\

\hline n-body/compute\_lagrange\_points\_0.5.auto
& \begin{minipage}{3in}\smallskip A basic script which computes all of the ``Lagrange
points'' for the case where the masses of the two planets are
equal, and saves the data.
\smallskip\end{minipage}\\

\hline n-body/compute\_periodic\_family.xauto
& \begin{minipage}{3in}\smallskip An expert script which starts at a ``Lagrange
point'' computed by compute\_lagrange\_points\_0.5.auto
and continues in the ratio of the masses until
a specified mass ratio is reached.  It then computes
a family of periodic orbits for each pair of
purely complex eigenvalues.
\smallskip\end{minipage}\\

\hline n-body/to\_matlab.xauto
& \begin{minipage}{3in}\smallskip A script which takes a set of \AUTOc data files and creates
a set of files formatted for importing into Matlab
for either plotting or further calculations.
\smallskip\end{minipage}\\
\hline
\end{tabular}
\caption[Available demo scripts.]
{The various demonstration scripts for the \AUTOc CLUI.}
\label{tbl:demo_scripts}
\end{center}
\end{table}

\section{ First Example } \label{sec:clui_first_example}

We begin with a simple example of the \AUTOc CLUI.  In this example we
copy the \filef{ab} demo from the \AUTOc installation directory and
run it.  For more information on the \filef{ab} demo see
Section~\ref{sec:Demos_ab}.
The commands listed in Table~\ref{tbl:example_clui_1}
will copy the demo files to your work directory and run
the first part of the demo.
The results of running these commands are shown in Figure~\ref{exa:clui_first_example}.


\begin{table}[htbp]
\begin{center}
\begin{tabular}{| l | l |}
\hline
  {\cal Unix}-COMMAND  & ACTION \\
\hline
%==============================================================================
  \commandf{ auto}  & start the \AUTOc CLUI\\ 
\hline
  \AUTOc CLUI COMMAND  & ACTION \\
\hline
  \commandf{ copydemo('ab')}  & copy the demo files to the work directory\\
  \commandf{ load(equation='ab')}  & load the filename \filef{ab.c} into memory\\
  \commandf{ load(constants='ab.1')}  & load the contents of the file \filef{r.ab.1} into memory\\
  \commandf{ run()}  & run \AUTOc with the current set of files\\
\hline
%==============================================================================
\end{tabular}
\caption[Running the demo \filef{ ab} files.]
{Running the demo \filef{ ab} files.}
\label{tbl:example_clui_1}
\end{center}
\end{table}

\begin{figure}[htbp]
{\small \begin{center} \begin{boxedverbatim}
> auto
Initializing
Python 1.5.2 (#1, Feb  1 2000, 16:32:16)  [GCC egcs-2.91.66 19990314/Linux 
(egcs- on linux-i386
Copyright 1991-1995 Stichting Mathematisch Centrum, Amsterdam
(AUTOInteractiveConsole)
AUTO> copydemo('ab')
Copying demo ab ... done
AUTO> load(equation='ab')
Runner configured
AUTO> load(constants='ab.1')
Runner configured
AUTO> run()
gcc -O -DPTHREADS -O -I/home/amavisitors/redrod/src/auto/2000/include -c ab.c
gcc -O ab.o -o ab.exe /home/amavisitors/redrod/src/auto/2000/lib/*.o  
-lpthread -L/home/amavisitors/redrod/src/auto/2000/lib -lauto_f2c -lm
Starting ab ...
 
   1     1  EP   1  0.000000E+00  0.000000E+00  0.000000E+00  0.000000E+00
   1    33  LP   2  1.057390E-01  1.484391E+00  3.110230E-01  1.451441E+00
   1    70  LP   3  8.893185E-02  3.288241E+00  6.889822E-01  3.215250E+00
   1    90  HB   4  1.308998E-01  4.271867E+00  8.950803E-01  4.177042E+00
   1    92  EP   5  1.512417E-01  4.369748E+00  9.155894E-01  4.272750E+00

 Total Time    9.502E-02
ab ... done
AUTO> 
\end{boxedverbatim}
\end{center} 
}
\caption[The first example of the \AUTOc CLUI.]
{Typing \commandf{auto} at the Unix shell prompt starts the
\AUTOc CLUI.  The rest of the commands are interpreted by
the \AUTOc CLUI.}
\label{exa:clui_first_example}
\end{figure}


Let us examine more closely what action each of the commands 
performs.  First, \commandf{copydemo('ab')} 
(Section~\ref{sec:clui_ref_commandCopyDemo} in the reference) copies the files in 
\filef{ \$AUTO\_DIR/demo/ab} into the work directory.  

Next, \commandf{load(equation='ab')} 
(Section~\ref{sec:clui_ref_commandRunnerLoadName} in the reference)
informs the \AUTOc CLUI that the name of
the user defined function file is \filef{ab.c}.  The command
\commandf{load} is one of the most commonly used commands in the
\AUTOc CLUI, since it reads and parses the user files which are
manipulated by other commands.  The \AUTOc CLUI stores this setting
until it is changed by a command, such as another \commandf{load}
command.  The idea of storing information is one of the ideas that
sets the CLUI apart from the command language described in
Section~\ref{sec:command_mode}.

Next, \commandf{load(constants='ab.1')} parses the \AUTO
constants file \filef{c.ab.1} and reads it into memory.  Note that
\emp{changes to the file \filef{c.ab.1} after it has been loaded in
will not be used by \AUTOc unless it is loaded in again after the
changes are made}.  

Finally, \commandf{run()} 
(Section~\ref{sec:clui_ref_commandRun} in the reference)
uses the user defined functions loaded
by the \commandf{load(equation='ab')} command, and the \AUTO constants
loaded by the \commandf{load(constants='ab.1')} to run \AUTOc.

Figure~\ref{exa:clui_first_example} showed two of 
the file types that the 
\commandf{load} command can read into memory, namely 
the user defined function file and the \AUTO constants
file (Section~\ref{sec: User_supplied_files}).  
There are two other files types that can be read
in using the \commandf{load} command, and they are
the restart solution file (Section~\ref{sec:Output_files})
and the {\cal HomCont} parameter
file (Section~\ref{sec:HomCont_files}).

Note that the name given to the load
command is not the same as the filename which is read
in, for example \commandf{load(constants='ab.1')} reads in 
the file \filef{c.ab.1}.  This difference is 
a result of the automatic transformation of the
filenames by the 
\AUTOc CLUI into the
standard names used by \AUTOc.  
The standard filename
transformations are show in Table~\ref{tbl:clui_filename_translation}. 

\begin{table}[htbp]
\begin{center}
\begin{tabular}{| l | l | l | l |}
\hline
Long name & Short name & Name entered & Transformed file name \\
\hline
equation  & e          & foo          &foo.c \\
\hline
constants  & c          & foo          & c.foo \\
\hline
solution  & s          & foo          & s.foo \\
\hline
bifurcationDiagram  & b          & foo          &b.foo \\
\hline
diagnostics  & d          & foo          &d.foo \\
\hline
homcont  & h          & foo          &h.foo \\
\hline
\end{tabular}
\caption[Standard \AUTOc CLUI filename translations.]
{This table shows the standard \AUTOc CLUI filename
translations.  In \filef{load} and \filef{run}
commands either the long name or the short name may be
used for loading the appropriate files.}
\label{tbl:clui_filename_translation}
\end{center}
\end{table}

Since the \commandf{load} command is so common, there are
various shorthand versions of it.  First, there are short versions
of the various arguments as shown in Table~\ref{tbl:clui_filename_translation}.
For example, the command \commandf{load(constants='ab.1')} can 
be shortened to \commandf{load(c='ab.1')}.
Next, several different
files may be loaded at once using the same \commandf{load} command.
For example, the two commands in Figure~\ref{exa:clui_two_command}
have the same effect as the single command in 
Figure~\ref{exa:clui_one_command}.

\begin{figure}[htbp]
{\small \begin{center} \begin{boxedverbatim}
AUTO> load(e='ab')
Runner configured
AUTO> load(c='ab.1)
Runner configured
\end{boxedverbatim}
\end{center} 
}
\caption[Loading two files individually]
{Loading two files individually.}
\label{exa:clui_two_command}
\end{figure}

\begin{figure}[htbp]
{\small \begin{center} \begin{boxedverbatim}
AUTO> load(e='ab',c='ab.1')
Runner configured
\end{boxedverbatim}
\end{center} 
}
\caption[Loading two files at the same time]
{Loading two files at the same time.}
\label{exa:clui_one_command}
\end{figure}

Also, since it is common that several files will be loaded that
have the same base name \commandf{load('ab')} performs the same 
action as \commandf{load(e='ab',c='ab',s='ab',h='ab')}.  Note, 
for the command \commandf{load('ab')} it is only required that 
\filef{ab.c} and \filef{c.ab} exist;  \filef{s.ab} and
\filef{h.ab} are optional, and if they do not exist,
no error message will be given.  


\section{ Scripting }

Section~\ref{sec:clui_first_example} showed commands
being interactively entered at the \AUTOc CLUI
prompt, but since the \AUTOc CLUI is based 
on \python~ one has the ability to write
scripts for performing sequences of commands
automatically.  A \python~ script is very similar
to the interactive mode shown in Section~\ref{sec:clui_first_example}
except that the commands are placed in a file and
read all at once.  For example, if the
commands from Figure~\ref{exa:clui_first_example} where placed 
into the file \filef{demo1.auto}, in the format shown in 
Figure~\ref{exa:clui_first_script}, then the commands
could be run all at once by typing \commandf{auto demo1.auto}.
See Figure~\ref{exa:clui_run_first_script} for the
full output.

\begin{figure}[htb]
{\small \begin{center} \begin{boxedverbatim}
copydemo('ab')
load(equation='ab')
load(constants='ab.1')
run()
\end{boxedverbatim}
\end{center} 
}
\caption[First example of a \AUTOc CLUI script.]
{The commands from Figure~\ref{exa:clui_first_example}
and they would appear in a \AUTOc CLUI script file.
The source for this script can be found in \filef{\$AUTO\_DIR/demos/python/demo1.auto}.
}
\label{exa:clui_first_script}
\end{figure}

\begin{figure}[htb]
{\small \begin{center} \begin{boxedverbatim}
> cat demo1.auto
copydemo('ab')
load(equation='ab')
load(constants='ab.1')
run()

> auto demo1.auto 
Initializing
Copying demo ab ... done
Runner configured
Runner configured
gcc -O -DPTHREADS -O -I/home/amavisitors/redrod/src/auto/2000/include -c ab.c
gcc -O ab.o -o ab.exe /home/amavisitors/redrod/src/auto/2000/lib/*.o  -lpthread 
-L/home/amavisitors/redrod/src/auto/2000/lib -lauto_f2c -lm
Starting ab ...
 
   1     1  EP   1  0.000000E+00  0.000000E+00  0.000000E+00  0.000000E+00
   1    33  LP   2  1.057390E-01  1.484391E+00  3.110230E-01  1.451441E+00
   1    70  LP   3  8.893185E-02  3.288241E+00  6.889822E-01  3.215250E+00
   1    90  HB   4  1.308998E-01  4.271867E+00  8.950803E-01  4.177042E+00
   1    92  EP   5  1.512417E-01  4.369748E+00  9.155894E-01  4.272750E+00

 Total Time    8.740E-02
ab ... done
> 
\end{boxedverbatim}
\end{center} 
}
\caption[Figure of running a simple \AUTOc CLUI script.]
{This Figure starts by listing the contents
of the \filef{demo1.auto} file using the Unix
\commandf{cat} command.  The file is then run through
the \AUTOc CLUI by typing {auto demo1.auto} 
and the output is shown.}
\label{exa:clui_run_first_script}
\end{figure}


\section{ Second Example } \label{sec:clui_complex_example}

In Section~\ref{sec:clui_first_example} we showed a very simple
\AUTOc CLUI script, in this Section we will describe a more
complex example, which introduces several new \AUTOc CLUI
commands as well as some basic \python~ constructs for conditionals
and looping.  We will not provide an exhaustive reference for
the \python~ language, but only 
the very basics.  For more extensive documentation we refer the
reader to \citename{Lut:96} \citeyear{Lut:96} or the
web page \webf{http://www.python.org}.
In this section we will describe each line of the script
in detail, and the full text of the script is in
Figure~\ref{exa:clui_complex_script}.

\begin{figure}[htbp]
{\small \begin{center} \begin{boxedverbatim}
copydemo('bvp')

ld('bvp')
run()
sv('bvp')
ld(s='bvp')
data = sl('bvp')
ch("NTST",50)
for solution in data:
    if solution["Type name"] == "BP":
        ch("IRS", solution["Label"])
        ch("ISW", -1)
        # Compute forward
        run()
        ap('bvp')
        # Compute back
        ch("DS",-pr("DS"))
        run()
        ap('bvp')

plot('bvp')
wait()
\end{boxedverbatim}
\end{center} 
}
\caption[A complex example of a \AUTOc CLUI script.]
{This Figure shows a more complex \AUTOc CLUI script.
The source for this script can be found in \filef{\$AUTO\_DIR/demos/python/demo2.auto}.
}
\label{exa:clui_complex_script}
\end{figure}

The script begins with a section, extracted into 
Figure~\ref{exa:clui_complex_first}, which performs a task 
identical to that shown in Figure~\ref{exa:clui_first_example}
except that the shorthand discussed in 
Section~\ref{sec:clui_first_example} is used for the 
\commandf{ld} command.

\begin{figure}[htbp]
{\small \begin{center} \begin{boxedverbatim}
copydemo('bvp')

ld('bvp')
run()
\end{boxedverbatim}
\end{center} 
}
\caption[The first part of the  complex \AUTOc CLUI script.]
{The first part of the  complex \AUTOc CLUI script.}
\label{exa:clui_complex_first}
\end{figure}

The next section of the script, extracted into 
Figure~\ref{exa:clui_complex_second}, introduces
three new \AUTOc CLUI commands.  First, \commandf{sv('bvp')}
(Section~\ref{sec:clui_ref_commandCopyDataFiles} in the reference)
saves the results of the \AUTOc run into files using
the base name \filef{bvp} and the filename extensions in
Table~\ref{tbl:clui_filename_translation}.  For example,
in this case the bifurcation diagram file \filef{fort.7}
will be saved as \filef{b.bvp}, 
the solution file \filef{fort.8}
will be saved as \filef{s.bvp}, and  
the diagnostics file \filef{fort.9}
will be saved as \filef{d.bvp}.  Next, \commandf{ld(s='bvp')}
loads the solution file \filef{s.bvp} into memory
so that it can be used by \AUTOc for further calculations.

Up to this point all of the commands presented have had
analogs in the command language discussed in 
Section~\ref{sec:command_mode}, and the \AUTOc CLUI has
been designed in this way to make it easy for users to
migrate from the old command language to the
\AUTOc CLUI.  The next command, namely \commandf{data = sl('bvp')}
(Section~\ref{sec:clui_ref_commandParseSolutionFile} in the reference)
is the first command which has no analog in the 
old command language.
The command \commandf{sl('bvp')} parses the file \filef{s.bvp}
and returns a python object which encapsulates the
information contained in the file and presents it to the
user in an easy to use format.  Accordingly, the
command \commandf{data = sl('bvp')} stores this easy to
use representation of the object in the \python~
variable \commandf{data}.
Note, variables in \python~ are different from those in languages
such as {\cal C} in that their type does not have to be
declared before they are created.
Finally, \commandf{ch("NTST",50)}  
(Section~\ref{sec:clui_ref_commandRunnerConfigFort2} in the reference)
changes the \parf{NTST} value to \parf{50} (see Section~\ref{sec:NDIM}).
To be precise, the command \commandf{ch("NTST",50)} only modifies
the ``in memory'' version of the \AUTOc constants created
by the \commandf{ld('bvp')} command.
The original file \filef{c.bvp} is \emp{not} modified.

\begin{figure}[htbp]
{\small \begin{center} \begin{boxedverbatim}
sv('bvp')
ld(s='bvp')
data = sl('bvp')
ch("NTST",50)
\end{boxedverbatim}
\end{center} 
}
\caption[The second part of the  complex \AUTOc CLUI script.]
{The second part of the  complex \AUTOc CLUI script.}
\label{exa:clui_complex_second}
\end{figure}

The next section of the script, extracted into 
Figure~\ref{exa:clui_complex_third}, shows as example
of looping and conditionals in an \AUTOc CLUI script.
The first line \commandf{for solution in data:} is
the \python~ syntax for loops.  The \commandf{data} variable
was defined in Figure~\ref{exa:clui_complex_second}
to be the parsed version of an \AUTOc \filef{fort.8}
file, and accordingly contains a list of the solutions
from the \filef{fort.8} file.  The command 
\commandf{for solution in data:} is used to loop
over all solutions in the data variable by 
setting the variable \commandf{solution} to
be one of the solutions in \commandf{data}
and then calling the rest of the code in the
block.  

\python~ differs from most other computer languages in that
blocks of code are not defined by some delimiter, such
as \commandf{\{\}} in {\cal C}, but by indentation.  
In Figure~\ref{exa:clui_complex_script} the commands
\commandf{plot('bvp')} and \commandf{wait()} are not
part of the loop, because they are indented 
differently.  This can be confusing first time users of \python~,
but it has the advantage that the code
is forced to have a consistent indentation style.

The next command in the script, \commandf{if solution["Type name"] == "BP":}
is a \python~ conditional.  
It examines the contents of the
variable \commandf{solution} (which is one of the entries
in the array of solutions \commandf{data}) 
and checks to see if the condition
\commandf{solution["Type name"] == "BP"} holds.  
For parsed \filef{fort.8}
files \commandf{Type name} \commandf{BP} corresponds to a bifurcation
point.  Accordingly, the function of this loop and conditional
is to examine every solution in the \filef{fort.8} file and
run the following commands if the solution is a bifurcation
point.

The next line
is \commandf{ch("IRS", solution["Label"])} which changes
the ``in memory'' version of the \AUTOc constants file to 
set \parf{IRS} (see Section~\ref{sec:IRS}) equal to the label
of the bifurcation point. We then use \commandf{ch("ISW", -1)} to change
the \AUTOc constant \parf{ISW} to \commandf{-1}, which
indicates a branch switch (see Section~\ref{sec:ISW}).

We then use a \commandf{run()} command to perform the
calculation of the bifurcating branch and then append the
data to the \filef{s.bvp}, \filef{b.bvp}, and
\filef{d.bvp} files with the \commandf{ap('bvp')}
command (Section~\ref{sec:clui_ref_commandAppend} in the reference).
In addition, as can be seen in Figure~\ref{exa:clui_complex_third},
the \commandf{\#} character is the \python~ comment character.
When the \python~ interpretor encounters a \commandf{\#} character
it ignores everything from that character to the end of the line.

Finally, we us \commandf{ch("DS",-pr("DS"))} to change the
\AUTOc initial step size from positive to negative, which allows
us to compute the bifurcating branch in the other direction
(see Section~\ref{sec:DS}).  Running the \AUTOc calculation
with the \commandf{run()} command and appending the
data the appropriate files with the \commandf{ap('bvp')} command
completes the body of the loop.

\begin{figure}[htbp]
{\small \begin{center} \begin{boxedverbatim}
for solution in data:
    if solution["Type name"] == "BP":
        ch("IRS", solution["Label"])
        ch("ISW", -1)
        # Compute forward
        run()
        ap('bvp')
        # Compute back
        ch("DS",-pr("DS"))
        run()
        ap('bvp')
\end{boxedverbatim}
\end{center} 
}
\caption[The second part of the  complex \AUTOc CLUI script.]
{The second part of the  complex \AUTOc CLUI script.}
\label{exa:clui_complex_third}
\end{figure}

Now that the section of script shown in 
Figure~\ref{exa:clui_complex_third} has finished computing the
bifurcation diagram, the command \commandf{plot('bvp')}
brings up a plotting window 
(Section~\ref{sec:clui_ref_commandPlotter} in the reference),
and the command \commandf{wait()} causes the \AUTOc CLUI
to wait for input.  You may now exit the \AUTOc CLUI
by pressing any key in the window in which you started
the \AUTOc CLUI.

\section{ Extending the \AUTOc CLUI } \label{sec:clui_extending}

The code in Figure~\ref{exa:clui_complex_script}
performed a very useful and common procedure, it started an \AUTOc
calculation and performed additional continuations
at every point which \AUTOc detected as a bifurcation.
Unfortunately, the script as written can only be used
for the \filef{bvp} demo.  In this section we will 
generalize the script in Figure~\ref{exa:clui_complex_script}
for use with any demo, and demonstrate how it
can be imported back into the interactive
mode to create a new command
for the \AUTOc CLUI.  Several examples of such
``expert'' scripts can be found in \filef{auto/2000/demos/python/n-body}.

Just as loops and conditionals can be used in \python~,
one can also define functions.  For example,
Figure~\ref{exa:clui_complex_function} is a
functional version of script from 
Figure~\ref{exa:clui_complex_script}.
The changes are actually quite minor.  
The first line, \commandf{from AUTOclui import *},
includes the definitions of the \AUTOc CLUI commands,
and must be included in all \AUTOc CLUI scripts
which define functions.
The next line, 
\commandf{def myRun(demo):},
begins the function definition, and 
creates a function named \commandf{myRun} which
takes one argument \commandf{demo}.  The
rest of the script is the same except that it
has been indented to indicate that it is
part of the function definition, and all occurrences
of string \commandf{'bvp'} have been replaced
with the variable \commandf{demo}.
Finally we have added a line \commandf{myRun('bvp')}
which actually calls the function we have 
created and runs the same computation as
the original script.

\begin{figure}[htbp]
{\small \begin{center} \begin{boxedverbatim}
from AUTOclui import *

def myRun(demo):

    copydemo(demo)

    ld(demo)
    run()
    sv(demo)
    ld(s=demo)
    data = sl(demo)
    ch("NTST",50)
    for solution in data:
        if solution["Type name"] == "BP":
            ch("IRS", solution["Label"])
            ch("ISW", -1)
            # Compute forward
            run()
            ap(demo)
            # Compute back
            ch("DS",-pr("DS"))
            run()
            ap(demo)

    plot(demo)
    wait()

myRun('bvp')

\end{boxedverbatim}
\end{center} 
}
\caption[A complex \AUTOc CLUI script as a function.]
{This Figure shows a complex \AUTOc CLUI script
written as a function.
The source for this script can be found in \filef{\$AUTO\_DIR/demos/python/userScript.xauto}.
}
\label{exa:clui_complex_function}
\end{figure}

While the script in Figure~\ref{exa:clui_complex_function} is 
only slightly different then the one showed in 
Figure~\ref{exa:clui_complex_script} it is much more powerful.
Not only can it be used as a script for running any demo
by modifying the last line, it can be read back into
the interactive mode of the \AUTOc CLUI and
used to create a new command, 
as in Figure~\ref{exa:clui_complex_interactive}.
First, we create a file called \filef{userScript.py}
which contains the script from 
Figure~\ref{exa:clui_complex_function}, with
one minor modification.  We want the function only
to run when we use it interactively, not when
the file \filef{userScript.py} is read in, so we
remove the last line where the function is called.
We start the \AUTOc CLUI with the Unix command
\commandf{auto}, and once the \AUTOc CLUI is running
we use the command \commandf{from userScript import *},
to import the file \filef{userScript.py} into the
\AUTOc CLUI.  The \commandf{import} command makes
all functions in that file available for
our use (in this case \commandf{myRun} is the only 
one).  It is important to note that 
\commandf{from userScript import *} does \emp{not}
use the \filef{.py} extension on the file name. 
After importing our new function, we may use it
just like any other function in the \AUTOc
CLUI, for example by typing \commandf{myRun('bvp')}.

\begin{figure}[htbp]
{\small \begin{center} \begin{boxedverbatim}
> cp \$AUTO\_DIR/python/demo/userScript.py .
> ls
userScript.py
> cat userScript.py
# This is an example script for the AUTO2000 command line user
# interface.  See the "Command Line User Interface" chapter in the
# manual for more details.
from AUTOclui import *

def myRun(demo):

    copydemo(demo)

    ld(demo)
    run()
    sv(demo)
    ld(s=demo)
    data = sl(demo)
    ch("NTST",50)
    for solution in data:
        if solution["Type name"] == "BP":
            ch("IRS", solution["Label"])
            ch("ISW", -1)
            # Compute forward
            run()
            ap(demo)
            # Compute back
            ch("DS",-pr("DS"))
            run()
            ap(demo)

    plot(demo)
    wait()

> auto
Initializing
Python 1.5.2 (#1, Feb  1 2000, 16:32:16)  [GCC egcs-2.91.66 19990314/Linux 
(egcs- on linux-i386
Copyright 1991-1995 Stichting Mathematisch Centrum, Amsterdam
(AUTOInteractiveConsole)
AUTO> from userScript import *
AUTO> myRun('bvp')
...
\end{boxedverbatim}
\end{center} 
}
\caption[An example of using a user created function interactively.]
{This Figure shows the functional version of the
 \AUTOc CLUI from Figure~\ref{exa:clui_complex_function} being 
used as an extension to the \AUTOc CLUI.  The source
code for this script can be found in 
\filef{ \$AUTO\_DIR/python/demo/userScript.py }}
\label{exa:clui_complex_interactive}
\end{figure}

\section{Bifurcation Diagram Files}
Using the \commandf{commandParseDiagramFile} 
command (Section~\ref{sec:clui_ref_commandParseDiagramFile} in the reference)
the user can parse and read into memory an \AUTOc bifurcation
diagram file.   For example, the command  \commandf{commandParseDiagramFile('ab')}
would parse the file \filef{b.ab} (if you are using the standard
filename translations from Table~\ref{tbl:clui_filename_translation})
and return an object which encapsulates
the bifurcation diagram in an easy to use form.

The object returned by the \commandf{commandParseDiagramFile} 
is a list of all of the solutions in the appropriate
bifurcation diagram file, and each solution is a Python
dictionary with entries for each piece of
data for the solution.  For example, the sequence of commands
in Figure~\ref{exa:clui parse diagram}, prints out the
label of the first solution in a bifurcation diagram.
The queriable parts of the object are listed in
Table~\ref{tbl:clui parse diagram}.

The individual elements of the array may be accessed 
in two ways, either by index of the solution using the
\commandf{[]} syntax or by label number using the
\commandf{()} syntax.  For example, assume that the parsed object is contained
in a variable \commandf{data}.  
The first solution may be accessed 
using the command \commandf{data[0]}, while the solution with
label \commandf{57} may be accessed using the command \commandf{data(57)}.

This class has two methods that are particularily useful for creating
data which can be used in other programs.  First, there is a method
called \commandf{toArray} which takes a bifurcation diagram and
returns a standard Python array.  Second, there is a method called
\commandf{writeRawFilename} which will create a standard ASCII file
which contains the bifurcation diagram.  For example, we again assume
that the parsed object is contained in a variable \commandf{data}.  If
one wanted to have the bifurcation diagram returned as a Python array
one would type \commandf{data.toArray()}.  Similarily, if one wanted
to write out the bifurcation diagram to the file \commandf{outputfile}
one would type \commandf{data.writeRawFilename('outputfile')}.

\begin{figure}[htbp]
{\small \begin{center} \begin{boxedverbatim}
AUTO> data=dg('ab')
Parsed file: b.ab
AUTO> print data[0]
{'LAB': 6, 'TY name': 'EP', 'data': [0.0, 0.0, 0.0, 0.0], 'section': 12, 
'BR': 2, 'PT': 1, 'TY number': 9}
AUTO> print data[0]['LAB']
6
AUTO> 
\end{boxedverbatim}
\end{center} 
}
\caption[An example of parsing a bifurcation diagram.]
{This figure shows an example of parsing a bifurcation diagram.
The first command, \commandf{data=dg('ab')}, reads in the bifurcation
diagram and puts it into the variable \commandf{data}.
The second command, \commandf{print data[0]} prints out all of
the data about the first solution in the list.
The third command, \commandf{print data[0]['LAB']}, prints
out the label of the first point.
}
\label{exa:clui parse diagram}
\end{figure}

\begin{table}[htbp]
\begin{center}
\begin{tabular}{| l | l |}
\hline
Query string & Meaning \\
\hline
TY name &  The short name for the solution type (see Table~\ref{tbl:clui type translation}). \\
\hline
TY number &  The number of the solution type (see Table~\ref{tbl:clui type translation}). \\
\hline
BR  &  The branch number. \\
\hline
PT  &  The point number. \\
\hline
LAB  &  The solution label, if any. \\
\hline
section  &  A unique identifier for each branch in a file with multiple branches. \\
\hline
data  &  An array which contains the \AUTOc output. \\
\hline
\end{tabular}
\caption[Contents of a bifurcation diagram object.]
{This table shows the strings that can be used to
query a bifurcation diagram object and their
meanings.}
\label{tbl:clui parse diagram}
\end{center}
\end{table}

\begin{table}[htbp]
\begin{center}
\begin{tabular}{| l | l | l |}
\hline
Type & Short Name & Number \\
\hline
No Label & No Label &  \\
\hline
Branch point (algebraic problem) & BP & 1 \\
\hline
Fold (algebraic problem) & LP & 2 \\
\hline
Hopf bifurcation (algebraic problem) & HB & 3 \\
\hline
Regular point (every NPR steps) & RG & 4 \\
\hline
User requested point & UZ & -4 \\
\hline
Fold (ODE) & LP & 5 \\
\hline
Bifurcation point (ODE) & BP & 6 \\
\hline
Period doubling bifurcation (ODE) & PD & 7 \\
\hline
Bifurcation to invarient torus (ODE) & TR & 8 \\
\hline
Normal begin or end & EP & 9 \\
\hline
Abnormal termination & MX & -9 \\
\hline
\end{tabular}
\caption[Type translations.]
{This table shows the the various types of points
that can be in solution and bifurcation diagram
files, with their short names and numbers.}
\label{tbl:clui type translation}
\end{center}
\end{table}

\section{Solution Files}
Using the \commandf{commandParseSolutionFile} 
command (Section~\ref{sec:clui_ref_commandParseSolutionFile} in the reference)
the user can parse and read into memory an \AUTOc bifurcation
solution file.   For example, the command  \commandf{commandParseSolutionFile('ab')}
would parse the file \filef{b.ab} (if you are using the standard
filename translations from Table~\ref{tbl:clui_filename_translation})
and return an object which encapsulates
the bifurcation solution in a easy to use form.

The object returned by the \commandf{commandParseSolutionFile} 
is a list of all of the solutions in the appropriate
bifurcation solution file, and each solution is a Python
dictionary with entries for each  piece of
data for the solution.  For example, the sequence of commands
in Figure~\ref{exa:clui parse solution}, prints out the
label of the first solution in a bifurcation solution.
The queriable parts of the object are listed in
Table~\ref{tbl:clui parse solution}.

\begin{figure}[htbp]
{\small \begin{center} \begin{boxedverbatim}
AUTO> data=sl()
Parsed file: fort.8
AUTO> print data[0]
'Branch number': 2
'ISW': 1
'Label': 6
'NCOL': 0
'NTST': 0
'Parameters': [0.0, 14.0, 2.0, 0.0, 0.0, 0.0]
'Point number': 1
'Type name': 'EP'
'Type number': 9
'p': [0.0, 14.0, 2.0, 0.0, 0.0, 0.0]
'parameters': [0.0, 14.0, 2.0, 0.0, 0.0, 0.0]
AUTO> print data[0]['Label']
6
AUTO> data[0]["data"][0]
{'t': 0.0, 'u': [0.0, 0.0]}
\end{boxedverbatim}
\end{center} 
}
\caption[An example of parsing a bifurcation solution.]
{This figure shows an example of parsing a bifurcation solution.
The first command, \commandf{data=dg('ab')}, reads in the bifurcation
solution and puts it into the variable \commandf{data}.
The second command, \commandf{print data[0]} prints out all of
the data about the first solution in the list.
The third command, \commandf{print data[0]['Label']}, prints
out the label of the first point.  The last command prints
the value of the solution at the first point of the
first solution.
}
\label{exa:clui parse solution}
\end{figure}

\begin{table}[htbp]
\begin{center}
\begin{tabular}{| l | l |}
\hline
Query string & Meaning \\
\hline
data  & \begin{minipage}{4in} \smallskip An array which contains the \AUTOc output. \smallskip \end{minipage} \\
\hline
Branch number & \begin{minipage}{4in} \smallskip The number of the branch to which the solution belongs. \smallskip \end{minipage} \\ 
\hline
ISW & \begin{minipage}{4in} \smallskip  The ISW value used to start the calcluation.  See Section~\ref{sec:ISW}. \smallskip \end{minipage} \\ 
\hline
Label & \begin{minipage}{4in} \smallskip The label of the solution.   \smallskip \end{minipage} \\ 
\hline
NCOL & \begin{minipage}{4in} \smallskip The number of collocation points used to compute the solution.  See Section~\ref{sec:NCOL}. \smallskip \end{minipage} \\ 
\hline
NTST & \begin{minipage}{4in} \smallskip The number of mesh intervals used to compute the solution.  See Section~\ref{sec:NTST}. \smallskip \end{minipage} \\ 
\hline
Parameters & \begin{minipage}{4in} \smallskip The value of all of the parameters for the solution. \smallskip \end{minipage} \\ 
\hline
Point number & \begin{minipage}{4in} \smallskip The number of the point in the given branch.  \smallskip \end{minipage} \\  
\hline
Type name & \begin{minipage}{4in} \smallskip A short string which describes the type of the solution (see Table~\ref{tbl:clui type translation}). \smallskip \end{minipage} \\  
\hline
Type number & \begin{minipage}{4in} \smallskip A number which describes the type of the solution (see Table~\ref{tbl:clui type translation}).  \smallskip \end{minipage} \\  
\hline
p & \begin{minipage}{4in} \smallskip The value of all of the parameters for the solution. (This is an alias for 'Parameter'). \smallskip \end{minipage} \\  
\hline
parameters & \begin{minipage}{4in} \smallskip The value of all of the parameters for the solution. (This is an alias for 'Parameter').\smallskip \end{minipage} \\  
\hline
\end{tabular}
\caption[Contents of a bifurcation solution object.]
{This table shows the strings that can be used to
query a bifurcation solution object and their
meanings.}
\label{tbl:clui parse solution}
\end{center}
\end{table}

The individual elements of the array may be accessed 
in two ways, either by the index of the solution using the
\commandf{[]} syntax or by label number using the
\commandf{()} syntax.  For example, sssume that the parsed object is contained
in a variable \commandf{data}.  
The first solution may be accessed 
using the command \commandf{data[0]}, while the solution with
label \commandf{57} may be accessed using the command \commandf{data(57)}.

This class has two methods that are particularily useful
for creating data which can be used in other programs.
First, there is a method called \commandf{toArray} which
takes a solution and returns a standard Python array.
Second, there is a method called \commandf{writeRawFilename}
which will create a standard ASCII file which contains
the solution.  The first element of each row will be
the 't' value and the following elements will be the
values of the components at that 't' value.
For example, we again assume that the parsed object is contained
in a variable \commandf{data}.  If one wanted to have the
solution with label \commandf{57} returned as a Python
array one would type \commandf{data(57).toArray()}.
Similarily, if one wanted to write out the solution
to the file \commandf{outputfile} one would type
\commandf{data(57).writeRawFilename('outputfile')}.

\section{ The \filef{.autorc} File }

Much of the default behavior of the \AUTOc CLUI
can be controlled by the \filef{.autorc} file.
The \filef{.autorc} file can exist in
either the main \AUTOc directory, the users
home directory, or the current directory.  For any
option which is defined in more then one file, 
the \filef{.autorc} file
in the current directory (if it exists) takes precedence, 
followed by the \filef{.autorc} file
in the users home directory (if it exists), and then the
\filef{.autorc} file in the main \AUTOc directory.  Hence, 
options may be defined on either a per directory, per
user, or global basis.  

The first section of the \filef{.autorc} file
begins with the line \commandf{[AUTO\_command\_aliases]}
and this section defines short names, or aliases,
for the \AUTOc CLUI commands.  
Each line thereafter is a definition of
a command, similiar to 
\commandf{branchPoint     =commandQueryBranchPoint}.
The right hand side of the assignment
is the internal \AUTOc CLUI name for the command,
while the left hand side is the desired alias.  
Aliases and
internal names may be used interchangably, but the
intention is that the aliases will be more commonly
used.  A default set of aliases is provided, and
these aliases will be used in the examples in the
rest of this Chapter.  The default aliases
are listed in the reference in Section~\ref{sec:clui reference}.

%FIXME:  Fix the documentation here
{\em NOTE:  Defaults for the plotting tool may be included in
the .autorc file as well.  The documentation for this
is under developement, but the file \filef{\$AUTO\_DIR/.autorc}
contains examples of how these options may be set.}

\section{Two Dimensional Plotting Tool}\label{clui:2d plotting}

The two dimensional plotting tool can be run by using the command
\commandf{plot()} to plot the files \filef{fort.7} and \filef{fort.8}
after a calculation has been run, or using the command
\commandf{plot('foo')} to plote the data in the
files \filef{s.foo} and \filef{b.foo}.  

The menu bar provides two buttons.  The \commandf{File}
button brings up a menu which allows the user to save
the current plot as a Postscript file or
to quit the plotter.
The \commandf{Options} button allows the plotter
configuration options to be modified.
The available options are decribed in
Table~\ref{tbl:clui 2d plotter specific options}.  In addition, the options can 
be set from within the CLUI.  For example,
the set of commands in Figure~\ref{exa:plotter_example} shows how to
create a plotter and change its background color to black.  The
demo script \filef{auto/2000/demo/python/plotter.py} contains several
examples of changing options in plotters.

Pressing the right mouse button in the plotting window brings
up a menu of buttons which control several aspects
of the plotting window.  The top two toggle buttons
control what function the left button performs.  
The \commandf{print value} button causes the
left button to print out the numerical value underneath
the pointer when it is clicked.
When \commandf{zoom} button is checked the left
mouse button may be held down to create
a box in the plot.  When the left button
is released the plot will zoom to the selected
portion of the diagram.
The \commandf{unzoom} button returns the
diagram to the default zoom. 
The \commandf{Postscript} button allows the user
to save the plot as a Postscript file.
The \commandf{Configure...} button brings up
the dialog for setting configuration options.

\begin{figure}[htbp]
{\small \begin{center} \begin{boxedverbatim} 
AUTO> plot=pl()
Created plotter
AUTO> plot.config(bg="black")
AUTO> 
\end{boxedverbatim}
\end{center} 
}
\caption[Configuring a two-dimensional plotter.]
{This example shows how a plotter is created, and
how the background color may be changed to black.
All other configuration options are set similarily.
Note, the above commands assume that the files
fort.7 and fort.8 exist in the current directory.
}
\label{exa:plotter_example}
\end{figure}

\begin{longtable}{| l | l |}
\hline
Query string & Meaning \\
\hline
background  &  The background color of the plot. \\
\hline
bifurcation\_column\_defaults  & A set of bifurcation columns the user is likely to use. \\
\hline
bifurcation\_diagram  &  A parsed bifurcation diagram file to plot. \\
\hline
bifurcation\_diagram\_filename  & The filename of the bifurcation diagram to plot. \\
\hline
bifurcation\_symbol  &  The symbol to use for bifurcation points. \\ 
\hline
bifurcation\_x  & The column to plot along the X-axis for bifurcation diagrams. \\
\hline
bifurcation\_y  & The column to plot along the Y-axis for bifurcation diagrams. \\
\hline
color\_list  &  A list of colors to use for multiple plots. \\
\hline
decorations  & Turn on or off the axis, tick marks, etc. \\
\hline
error\_symbol  &    The symbol to use for error points. \\ 
\hline
foreground  &  The background color of the plot. \\
\hline
grid  &  Turn on or off the grid. \\
\hline
hopf\_symbol  &    The symbol to use for Hopf bifurcation points. \\ 
\hline
index  & An array of indicies to plot.\\
\hline
label  & An array of labels to plot.\\
\hline
label\_defaults  & A set of labels that the user is likely to use. \\
\hline
limit\_point\_symbol  &    The symbol to use for limit points. \\ 
\hline
mark\_t  &  The t value to marker with a small ball. \\      
\hline
maxx  & The upper bound for the x-axis of the plot. \\
\hline
maxy  & The upper bound for the y-axis of the plot. \\
\hline
minx  &  The lower bound for the x-axis of the plot. \\
\hline
miny  & The lower bound for the y-axis of the plot. \\
\hline
period\_doubling\_symbol  &   The symbol to use for period doubling bifurcation points. \\ 
\hline
runner  &  The runner object from which to get data. \\       
\hline
special\_point\_colors  &    An array of colors used to mark special points. \\ 
\hline
special\_point\_radius  &    The radius of the spheres used to mark special points. \\ 
\hline
solution  &  A parsed solution file to plot. \\
\hline
solution\_column\_defaults  & A set of solution columns the user is likely to use.\\
\hline
solution\_filename  & The filename of the solution to plot. \\
\hline
solution\_x  &  The column to plot along the X-axis for solutions. \\
\hline
solution\_y  & The column to plot along the Y-axis for solutions. \\
\hline
symbol\_font  &  The font to use for marker symbols. \\
\hline
symbol\_color  & The color to use for the marker symbols. \\
\hline
tick\_label\_template  & A string which defines the format of the tick labels. \\
\hline
tick\_length  &  The length of the tick marks. \\
\hline
torus\_symbol  &    The symbol to use for torus bifurcation points. \\ 
\hline
type  & The type of the plot, either ``solution'' or ``bifurcation''. \\  
\hline
user\_point\_symbol  &   The symbol to use for user defined output points. \\ 
\hline
xlabel  & The label for the x-axis. \\
\hline
xmargin  & The margin between the graph and the right and left edges. \\
\hline
xticks  & The number of ticks on the x-axis. \\
\hline
ylabel  & The label for the y-axis. \\
\hline
ymargin  & The margin between the graph and the top and bottom edges. \\
\hline
yticks  & The number of ticks on the y-axis. \\
\hline
\caption[The options for the \AUTOc CLUI two dimensional
plotting window.]
{This table shows the options that
can be set for the \AUTOc CLUI two dimensional
plotting window and their meanings.}
\label{tbl:clui 2d plotter specific options}
\end{longtable}

\section{Three Dimensional Plotting Tool}
%FIXME:  Fix the documentation here
{\em NOTE:  the documentation in this section is under developement.}

The \AUTOc three dimensional plotting tool can use DataViewer or
OpenInventor for rendering three dimensional representations of
bifurcation diagrams and solutions and is under active development.
Neither DataViewer nor OpenInventor are provided with \AUTOc and must
be downloaded seperately.  If you are interested in the three
dimensional plotting tool please contact \webf{redrod@acm.org}.

\floatstyle{plain}
\restylefloat{figure}

\clearpage

\pagebreak
\section{ Quick Reference } \label{sec:clui quick reference}

In this section we have created a table of all of the \AUTOc CLUI
commands, their abbreviations, and a one line description of what
function they perform.  Each command may be entered using 
its full name or any of its aliases.

\begin{longtable}{|l|p{1.5in}|p{2in}|}
\hline 
Command & Aliases & Description\\ \hline 
commandAppend & ap append  & Append data files.\\ \hline 
commandCat & cat  & Print the contents of a file\\ \hline 
commandCd & cd  & Change directories.\\ \hline 
commandClean & clean cl  & Clean the current directory.\\ \hline 
commandCopyAndLoadDemo & dm demo  & Copy a demo into the current directory and load it.\\ \hline 
commandCopyDataFiles & copy cp  & Copy data files.\\ \hline 
commandCopyDemo & copydemo  & Copy a demo into the current directory.\\ \hline 
commandCopyFortFiles & sv save  & Save data files.\\ \hline 
commandCreateGUI & gui  & Show AUTOs graphical user interface.\\ \hline 
commandDeleteDataFiles & delete dl  & Delete data files.\\ \hline 
commandDeleteFortFiles & df deletefort  & Clear the current directory of fort files.\\ \hline 
commandDouble & double db  & Double a solution.\\ \hline 
commandInteractiveHelp & man help  & Get help on the AUTO commands.\\ \hline 
commandLs & ls  & List the current directory.\\ \hline 
commandMoveFiles & move mv  & Move data-files to a new name.\\ \hline 
commandParseConstantsFile & cn constantsget  & Get the current continuation constants.\\ \hline 
commandParseDiagramAndSolutionFile & bt diagramandsolutionget  & Parse both bifurcation diagram and solution.\\ \hline 
commandParseDiagramFile & dg diagramget  & Parse a bifurcation diagram.\\ \hline 
commandParseSolutionFile & sl solutionget  & Parse solution file:\\ \hline 
commandPlotter & p2 pl plot  & 2D plotting of data.\\ \hline 
commandPlotter3D & plot3 p3  & 3D plotting of data.\\ \hline 
commandQueryBranchPoint & br bp branchpoint  & Print the ``branch-point function''.\\ \hline 
commandQueryEigenvalue & eigenvalue ev eg  & Print eigenvalues of Jacobian (algebraic case).\\ \hline 
commandQueryFloquet & fl floquet  & Print the Floquet multipliers.\\ \hline 
commandQueryHopf & hb hp hopf lp  & Print the value of the ``Hopf function''.\\ \hline 
commandQueryIterations & iterations it  & Print the number of Newton interations.\\ \hline 
commandQueryLimitpoint & lm limitpoint  & Print the value of the ``limit point function''.\\ \hline 
commandQueryNote & nt note  & Print notes in info file.\\ \hline 
commandQuerySecondaryPeriod & sc secondaryperiod sp  & Print value of ``secondary-periodic bif. fcn''.\\ \hline 
commandQueryStepsize & ss stepsize st  & Print continuation step sizes.\\ \hline 
commandRun & r run rn  & Run AUTO.\\ \hline 
commandRunnerConfigFort2 & changeconstant cc ch  & Modify continuation constants.\\ \hline 
commandRunnerLoadName & ld load  & Load files into the AUTO runner.\\ \hline 
commandRunnerPrintFort2 & pc pr printconstant  & Print continuation parameters.\\ \hline 
commandShell & shell  & Run a shell command.\\ \hline 
commandTriple & tr triple  & Triple a solution.\\ \hline 
commandUserData & us userdata  & Covert user-supplied data files.\\ \hline 
commandWait & wait  & Wait for the user to enter a key.\\ \hline 
\end{longtable}

\pagebreak
\section{ Reference }  \label{sec:clui reference}
\subsection{commandAppend} \label{sec:clui_ref_commandAppend}\begin{minipage}{6in}\hrule\medskip\section*{Purpose}
Append data files.\section*{Description}

    Type commandAppend('xxx') to append the output-files fort.7, fort.8,
    fort.9, to existing data-files s.xxx, b.xxx, and d.xxx (if you are
    using the default filename templates).

    Type commandAppend('xxx','yyy') to append existing data-files s.xxx, b.xxx,
    and d.xxx to data-files s.yyy, b.yyy, and d.yyy (if you are using
    the default filename templates).
    \section*{Aliases}
ap append \medskip\hrule\end{minipage}\subsection{commandCat} \label{sec:clui_ref_commandCat}\begin{minipage}{6in}\hrule\medskip\section*{Purpose}
Print the contents of a file\section*{Description}

    Type 'commandCat xxx' to list the contents of the file 'xxx'.  This calls the
    Unix function 'cat' for reading the file.  
    \section*{Aliases}
cat \medskip\hrule\end{minipage}\subsection{commandCd} \label{sec:clui_ref_commandCd}\begin{minipage}{6in}\hrule\medskip\section*{Purpose}
Change directories.\section*{Description}
    
    Type 'commandCd xxx' to change to the directory 'xxx'.  This command
    understands both shell variables and home directory expansion.
    \section*{Aliases}
cd \medskip\hrule\end{minipage}\subsection{commandClean} \label{sec:clui_ref_commandClean}\begin{minipage}{6in}\hrule\medskip\section*{Purpose}
Clean the current directory.\section*{Description}

    Type commandClean() to clean the current directory.  This command will
    delete all files of the form fort.*, *.o, and *.exe.
    \section*{Aliases}
clean cl \medskip\hrule\end{minipage}\subsection{commandCopyAndLoadDemo} \label{sec:clui_ref_commandCopyAndLoadDemo}\begin{minipage}{6in}\hrule\medskip\section*{Purpose}
Copy a demo into the current directory and load it.\section*{Description}

    Type commandCopyAndLoadDemo('xxx') to copy all files from auto/2000/demos/xxx to the
    current user directory.  Here 'xxx' denotes a demo name; e.g.,
    'abc'.  Note that the 'dm' command also copies a Makefile to the
    current user directory. To avoid the overwriting of existing
    files, always run demos in a clean work directory.  NOTE: This
    command automatically performs the commandRunnerLoadName command
    as well.
    \section*{Aliases}
dm demo \medskip\hrule\end{minipage}\subsection{commandCopyDataFiles} \label{sec:clui_ref_commandCopyDataFiles}\begin{minipage}{6in}\hrule\medskip\section*{Purpose}
Copy data files.\section*{Description}

    Type commandCopyDataFiles('xxx','yyy') to copy the data-files c.xxx, d.xxx, b.xxx,
    and h.xxx to c.yyy, d.yyy, b.yyy, and h.yyy (if you are using the
    default filename templates).
    \section*{Aliases}
copy cp \medskip\hrule\end{minipage}\subsection{commandCopyDemo} \label{sec:clui_ref_commandCopyDemo}\begin{minipage}{6in}\hrule\medskip\section*{Purpose}
Copy a demo into the current directory.\section*{Description}

    Type commandCopyDemo('xxx') to copy all files from auto/2000/demos/xxx to the
    current user directory.  Here 'xxx' denotes a demo name; e.g.,
    'abc'.  Note that the 'dm' command also copies a Makefile to the
    current user directory. To avoid the overwriting of existing
    files, always run demos in a clean work directory.
    \section*{Aliases}
copydemo \medskip\hrule\end{minipage}\subsection{commandCopyFortFiles} \label{sec:clui_ref_commandCopyFortFiles}\begin{minipage}{6in}\hrule\medskip\section*{Purpose}
Save data files.\section*{Description}

    Type commandCopyFortFiles('xxx') to save the output-files fort.7, fort.8, fort.9,
    to b.xxx, s.xxx, d.xxx (if you are using the default filename
    templates).  Existing files with these names will be overwritten.
    \section*{Aliases}
sv save \medskip\hrule\end{minipage}\subsection{commandCreateGUI} \label{sec:clui_ref_commandCreateGUI}\begin{minipage}{6in}\hrule\medskip\section*{Purpose}
Show AUTOs graphical user interface.\section*{Description}

    Type commandCreateGUI() to start AUTOs graphical user interface.
    
    NOTE: This command is not implemented yet.
    \section*{Aliases}
gui \medskip\hrule\end{minipage}\subsection{commandDeleteDataFiles} \label{sec:clui_ref_commandDeleteDataFiles}\begin{minipage}{6in}\hrule\medskip\section*{Purpose}
Delete data files.\section*{Description}

    Type commandDeleteDataFiles('xxx') to delete the data-files d.xxx, b.xxx, and s.xxx
    (if you are using the default filename templates).
    \section*{Aliases}
delete dl \medskip\hrule\end{minipage}\subsection{commandDeleteFortFiles} \label{sec:clui_ref_commandDeleteFortFiles}\begin{minipage}{6in}\hrule\medskip\section*{Purpose}
Clear the current directory of fort files.\section*{Description}

    Type commandDeleteFortFiles() to clean the current directory.  This command will
    delete all files of the form fort.*.
    \section*{Aliases}
df deletefort \medskip\hrule\end{minipage}\subsection{commandDouble} \label{sec:clui_ref_commandDouble}\begin{minipage}{6in}\hrule\medskip\section*{Purpose}
Double a solution.\section*{Description}

    Type commandDouble() to double the solution in 'fort.7' and 'fort.8'.

    Type commandDouble('xxx') to double the solution in b.xxx and s.xxx (if you
    are using the default filename templates).
    \section*{Aliases}
double db \medskip\hrule\end{minipage}\subsection{commandInteractiveHelp} \label{sec:clui_ref_commandInteractiveHelp}\begin{minipage}{6in}\hrule\medskip\section*{Purpose}
Get help on the AUTO commands.\section*{Description}
    
    Type 'help' to list all commands with a online help.
    Type 'help xxx' to get help for command 'xxx'.
    \section*{Aliases}
man help \medskip\hrule\end{minipage}\subsection{commandLs} \label{sec:clui_ref_commandLs}\begin{minipage}{6in}\hrule\medskip\section*{Purpose}
List the current directory.\section*{Description}
    
    Type 'commandLs' to run the system 'ls' command in the current directory.  This
    command will accept whatever arguments are accepted by the Unix command
    'ls'.
    \section*{Aliases}
ls \medskip\hrule\end{minipage}\subsection{commandMoveFiles} \label{sec:clui_ref_commandMoveFiles}\begin{minipage}{6in}\hrule\medskip\section*{Purpose}
Move data-files to a new name.\section*{Description}

    Type commandMoveFiles('xxx','yyy') to move the data-files b.xxx, s.xxx, d.xxx,
    and c.xxx to b.yyy, s.yyy, d.yyy, and c.yyy (if you are using the
    default filename templates).  \section*{Aliases}
move mv \medskip\hrule\end{minipage}\subsection{commandParseConstantsFile} \label{sec:clui_ref_commandParseConstantsFile}\begin{minipage}{6in}\hrule\medskip\section*{Purpose}
Get the current continuation constants.\section*{Description}

    Type commandParseConstantsFile('xxx') to get a parsed version of the constants file
    c.xxx (if you are using the default filename templates).
    \section*{Aliases}
cn constantsget \medskip\hrule\end{minipage}\subsection{commandParseDiagramAndSolutionFile} \label{sec:clui_ref_commandParseDiagramAndSolutionFile}\begin{minipage}{6in}\hrule\medskip\section*{Purpose}
Parse both bifurcation diagram and solution.\section*{Description}

    Type commandParseDiagramAndSolutionFile('xxx') to get a parsed version of the diagram file b.xxx
    and solution file s.xxx (if you are using the default filename
    templates).
    \section*{Aliases}
bt diagramandsolutionget \medskip\hrule\end{minipage}\subsection{commandParseDiagramFile} \label{sec:clui_ref_commandParseDiagramFile}\begin{minipage}{6in}\hrule\medskip\section*{Purpose}
Parse a bifurcation diagram.\section*{Description}

    Type commandParseDiagramFile('xxx') to get a parsed version of the diagram file b.xxx
    (if you are using the default filename templates).
    \section*{Aliases}
dg diagramget \medskip\hrule\end{minipage}\subsection{commandParseSolutionFile} \label{sec:clui_ref_commandParseSolutionFile}\begin{minipage}{6in}\hrule\medskip\section*{Purpose}
Parse solution file:\section*{Description}

    Type commandParseSolutionFile('xxx') to get a parsed version of the solution file
    s.xxx (if you are using the default filename templates).
    \section*{Aliases}
sl solutionget \medskip\hrule\end{minipage}\subsection{commandPlotter} \label{sec:clui_ref_commandPlotter}\begin{minipage}{6in}\hrule\medskip\section*{Purpose}
2D plotting of data.\section*{Description}

    Type commandPlotter('xxx') to run the graphics program for the graphical
    inspection of the data-files b.xxx and s.xxx (if you are using the
    default filename templates).  The return value will be the handle
    for the graphics window.

    Type commandPlotter() to run the graphics program for the graphical
    inspection of the output-files 'fort.7' and 'fort.8'.  The return
    value will be the handle for the graphics window.
    \section*{Aliases}
p2 pl plot \medskip\hrule\end{minipage}\subsection{commandPlotter3D} \label{sec:clui_ref_commandPlotter3D}\begin{minipage}{6in}\hrule\medskip\section*{Purpose}
3D plotting of data.\section*{Description}

    Type commandPlotter3D('xxx') to run the graphics program for the graphical
    inspection of the data-files b.xxx and s.xxx (if you are using the
    default filename templates).  The return value will be the handle
    for the graphics window.

    Type commandPlotter3D() to run the graphics program for the graphical
    inspection of the output-files 'fort.7' and 'fort.8'.  The return
    value will be the handle for the graphics window.
    \section*{Aliases}
plot3 p3 \medskip\hrule\end{minipage}\subsection{commandQueryBranchPoint} \label{sec:clui_ref_commandQueryBranchPoint}\begin{minipage}{6in}\hrule\medskip\section*{Purpose}
Print the ``branch-point function''.\section*{Description}
    
    Type commandQueryBranchPoint() to list the value of the ``branch-point function'' 
    in the output-file fort.9. This function vanishes at a branch point.
    
    Type commandQueryBranchPoint('xxx') to list the value of the ``branch-point function''
    in the info file 'd.xxx'.
    \section*{Aliases}
br bp branchpoint \medskip\hrule\end{minipage}\subsection{commandQueryEigenvalue} \label{sec:clui_ref_commandQueryEigenvalue}\begin{minipage}{6in}\hrule\medskip\section*{Purpose}
Print eigenvalues of Jacobian (algebraic case).\section*{Description}

    Type commandQueryEigenvalue() to list the eigenvalues of the Jacobian 
    in fort.9. 
    (Algebraic problems.)

    Type commandQueryEigenvalue('xxx') to list the eigenvalues of the Jacobian 
    in the info file 'd.xxx'.
    \section*{Aliases}
eigenvalue ev eg \medskip\hrule\end{minipage}\subsection{commandQueryFloquet} \label{sec:clui_ref_commandQueryFloquet}\begin{minipage}{6in}\hrule\medskip\section*{Purpose}
Print the Floquet multipliers.\section*{Description}

    Type commandQueryFloquet() to list the Floquet multipliers
    in the output-file fort.9. 
    (Differential equations.)

    Type commandQueryFloquet('xxx') to list the Floquet multipliers 
    in the info file 'd.xxx'.
    \section*{Aliases}
fl floquet \medskip\hrule\end{minipage}\subsection{commandQueryHopf} \label{sec:clui_ref_commandQueryHopf}\begin{minipage}{6in}\hrule\medskip\section*{Purpose}
Print the value of the ``Hopf function''.\section*{Description}

    Type commandQueryHopf() to list the value of the ``Hopf function'' 
    in the output-file fort.9. This function
    vanishes at a Hopf bifurcation point.

    Type commandQueryHopf('xxx') to list the value of the ``Hopf function''
    in the info file 'd.xxx'.
    \section*{Aliases}
hb hp hopf lp \medskip\hrule\end{minipage}\subsection{commandQueryIterations} \label{sec:clui_ref_commandQueryIterations}\begin{minipage}{6in}\hrule\medskip\section*{Purpose}
Print the number of Newton interations.\section*{Description}

    Type commandQueryIterations() to list the number of Newton iterations per
    continuation step in fort.9. 

    Type commandQueryIterations('xxx') to list the number of Newton iterations per
    continuation step in the info file 'd.xxx'.
    \section*{Aliases}
iterations it \medskip\hrule\end{minipage}\subsection{commandQueryLimitpoint} \label{sec:clui_ref_commandQueryLimitpoint}\begin{minipage}{6in}\hrule\medskip\section*{Purpose}
Print the value of the ``limit point function''.\section*{Description}

    Type commandQueryLimitpoint() to list the value of the ``limit point function'' 
    in the output-file fort.9. This function
    vanishes at a limit point (fold).

    Type commandQueryLimitpoint('xxx') to list the value of the ``limit point function'' 
    in the info file 'd.xxx'.
    \section*{Aliases}
lm limitpoint \medskip\hrule\end{minipage}\subsection{commandQueryNote} \label{sec:clui_ref_commandQueryNote}\begin{minipage}{6in}\hrule\medskip\section*{Purpose}
Print notes in info file.\section*{Description}

    Type commandQueryNote() to show any notes 
    in the output-file fort.9.

    Type commandQueryNote('xxx') to show any notes 
    in the info file 'd.xxx'.
    \section*{Aliases}
nt note \medskip\hrule\end{minipage}\subsection{commandQuerySecondaryPeriod} \label{sec:clui_ref_commandQuerySecondaryPeriod}\begin{minipage}{6in}\hrule\medskip\section*{Purpose}
Print value of ``secondary-periodic bif. fcn''.\section*{Description}

    Type commandQuerySecondaryPeriod()  to list the value of the 
    ``secondary-periodic bifurcation function'' 
    in the output-file 'fort.9. This function
    vanishes at period-doubling and torus bifurcations.

    Type commandQuerySecondaryPeriod('xxx') to list the value of the
    ``secondary-periodic bifurcation function''
    in the info file 'd.xxx'.
    \section*{Aliases}
sc secondaryperiod sp \medskip\hrule\end{minipage}\subsection{commandQueryStepsize} \label{sec:clui_ref_commandQueryStepsize}\begin{minipage}{6in}\hrule\medskip\section*{Purpose}
Print continuation step sizes.\section*{Description}

    Type commandQueryStepsize() to list the continuation step size for each
    continuation step in  'fort.9. 

    Type commandQueryStepsize('xxx') to list the continuation step size for each
    continuation step in the info file 'd.xxx'.
    \section*{Aliases}
ss stepsize st \medskip\hrule\end{minipage}\subsection{commandRun} \label{sec:clui_ref_commandRun}\begin{minipage}{6in}\hrule\medskip\section*{Purpose}
Run AUTO.\section*{Description}

    Type commandRun([options]) to run AUTO with the given options.
    There are four possible options:
    \begin{verbatim}
    Long name   Short name    Description
    -------------------------------------------
    equation    e             The equations file
    constants   c             The AUTO constants file
    solution    s             The restart solution file
    homcont     h             The Homcont parameter file
    \end{verbatim}
    Options which are not explicitly set retain their previous value.
    For example one may type: commandRun(e='ab',c='ab.1') to use 'ab.c' as
    the equations file and c.ab.1 as the constants file (if you are
    using the default filename templates).

    Type commandRun('name') load all files with base 'name'.
    This does the same thing as running
    commandRun(e='name',c='name,s='name',h='name').
    \section*{Aliases}
r run rn \medskip\hrule\end{minipage}\subsection{commandRunnerConfigFort2} \label{sec:clui_ref_commandRunnerConfigFort2}\begin{minipage}{6in}\hrule\medskip\section*{Purpose}
Modify continuation constants.\section*{Description}

    Type commandRunnerConfigFort2('xxx',yyy) to change the constant 'xxx' to have
    value yyy.
    \section*{Aliases}
changeconstant cc ch \medskip\hrule\end{minipage}\subsection{commandRunnerLoadName} \label{sec:clui_ref_commandRunnerLoadName}\begin{minipage}{6in}\hrule\medskip\section*{Purpose}
Load files into the AUTO runner.\section*{Description}

    Type commandRunnerLoadName([options]) to modify AUTO runner.
    There are four possible options:
    \begin{verbatim}
    Long name   Short name    Description
    -------------------------------------------
    equation    e             The equations file
    constants   c             The AUTO constants file
    solution    s             The restart solution file
    homcont     h             The Homcont parameter file
    \end{verbatim}
    Options which are not explicitly set retain their previous value.
    For example one may type: commandRunnerLoadName(e='ab',c='ab.1') to use 'ab.c' as
    the equations file and c.ab.1 as the constants file (if you are
    using the default filename templates).

    Type commandRunnerLoadName('name') load all files with base 'name'.
    This does the same thing as running
    commandRunnerLoadName(e='name',c='name,s='name',h='name').
    \section*{Aliases}
ld load \medskip\hrule\end{minipage}\subsection{commandRunnerPrintFort2} \label{sec:clui_ref_commandRunnerPrintFort2}\begin{minipage}{6in}\hrule\medskip\section*{Purpose}
Print continuation parameters.\section*{Description}

    Type commandRunnerPrintFort2() to print all the parameters.
    Type commandRunnerPrintFort2('xxx') to return the parameter 'xxx'.
    \section*{Aliases}
pc pr printconstant \medskip\hrule\end{minipage}\subsection{commandShell} \label{sec:clui_ref_commandShell}\begin{minipage}{6in}\hrule\medskip\section*{Purpose}
Run a shell command.\section*{Description}
        
    Type 'shell xxx' to run the command 'xxx' in the Unix shell and display
    the results in the AUTO command line user interface.
    \section*{Aliases}
shell \medskip\hrule\end{minipage}\subsection{commandTriple} \label{sec:clui_ref_commandTriple}\begin{minipage}{6in}\hrule\medskip\section*{Purpose}
Triple a solution.\section*{Description}

    Type commandTriple() to triple the solution in 'fort.7' and 'fort.8'.

    Type commandTriple('xxx') to triple the solution in b.xxx and s.xxx (if you
    are using the default filename templates).
    \section*{Aliases}
tr triple \medskip\hrule\end{minipage}\subsection{commandUserData} \label{sec:clui_ref_commandUserData}\begin{minipage}{6in}\hrule\medskip\section*{Purpose}
Covert user-supplied data files.\section*{Description}

    Type commandUserData('xxx') to convert a user-supplied data file 'xxx.dat' to
    AUTO format. The converted file is called 's.dat'.  The original
    file is left unchanged.  AUTO automatically sets the period in
    PAR(10).  Other parameter values must be set in 'stpnt'. (When
    necessary, PAR(10) may also be redefined there.)  The
    constants-file file 'c.xxx' must be present, as the AUTO-constants
    'NTST' and 'NCOL' are used to define the new mesh.  For examples
    of using the 'userData' command see demos 'lor' and 'pen' (where
    it has the old name 'fc').
    \section*{Aliases}
us userdata \medskip\hrule\end{minipage}\subsection{commandWait} \label{sec:clui_ref_commandWait}\begin{minipage}{6in}\hrule\medskip\section*{Purpose}
Wait for the user to enter a key.\section*{Description}

    Type 'commandWait' to have the AUTO interface wait
    until the user hits any key (mainly used in scripts).
    \section*{Aliases}
wait \medskip\hrule\end{minipage}





                                % Local Variables:
                                % tex-main-file: "auto.tex"
                                % End:











